% CORRECTION : Problème du Casting
% ==================================================================================================

\documentclass[10pt,a4paper]{article}

\usepackage{stmaryrd} % to use \llbracket and \rrbracket
\usepackage{tikz} % to draw the graph of the last question
\usepackage{pgfplots} % to enable axis coordinate system
\pgfplotsset{compat=1.18} % set compatibility mode for pgfplots

% Set the root path
\providecommand{\rootpath}{../../..}
% Fonts
\usepackage[utf8]{inputenc} % for accents
\usepackage[T1]{fontenc} % for accents
\usepackage[french]{babel} % for french language
\usepackage{helvet} % sans serif font family
\renewcommand*\familydefault{\sfdefault} % sans serif font family

% Mathematics
\usepackage{amsmath,amsfonts,amssymb} % for math symbols
\usepackage{array} % for tabular


\usepackage{parskip} % no indent, space between paragraphs

\usepackage{geometry} % margin
\geometry{
    a4paper,
    left=15mm,
    right=15mm,
    top=20mm,
    bottom=20mm
}

\usepackage{circledsteps} % to draw circles around numbers

\usepackage{fancyhdr} % for headers and footers

\usepackage{enumitem} % for customizing lists
\setlist[enumerate]{itemsep=1em} % space between items only in enumerate environment (not itemize)
\setlist[itemize]{label=--} % set itemize label to em-dash

% Command: \customPageLayout{#1}{#2}{#3}
% --------------------------------------
% Description: Custom page layout with header and footer content.
% Arguments:
% #1: Header and footer content
% #2: Left header content
% #3: Right header content
% Example:
% \customPageLayout{Title}{Lycée Henri IV}{2024}
% Required Packages: fancyhdr
\newcommand{\customPageLayout}[3]{
    \pagestyle{fancy} % set page style to fancy, i.e. header and footer
    \fancyhf{#1} % set header and footer content
    \lhead{#2} % set left header content
    \rhead{#3} % set right header content
    \fancyfoot{} % clear footer content
    \rfoot{\thepage} % set page number in footer
}

% Counter: \q
% -----------
% Description: Display a question number in a circle.
\newcounter{q}
\setcounter{q}{0} % set initial value of counter
\newcommand{\q}{
    \bigskip
    \addtocounter{q}{1}
    \par
    \Circled{\textbf{\theq}} \space
}


\title{Probabilités dans un univers fini - Problème du casting}
\author{Esther Poniatowski}
\date{2024-2025}

\customPageLayout{Correction}{Lycée Henri IV}{2024}

% ==================================================================================================
\begin{document}

\q Probabilité que le meilleur candidat se présente à la \( k^\text{ème} \) position

L'ensemble des ordres possibles de passage est modélisé par l'ensemble des permutations de \(
\llbracket 1, n \rrbracket \), muni de la probabilité uniforme.

Le nombre de permutations dans lesquelles le candidat classé premier (numéro \( 1 \)) apparaît en \(
k^\text{ème} \) position est :
\[
(n-1)!
\]

car les \( n-1 \) autres candidats peuvent être placés librement sur les autres positions. D'où :

\[
\mathbb{P}(A_k) = \frac{(n-1)!}{n!} = \boxed{\dfrac{1}{n}}
\]

% ------

\q Probabilité que le meilleur candidat se présente en \( k^\text{ème} \) position et soit
sélectionné
% ------
Cet événement suppose :
\begin{itemize}
 \item que le candidat classé premier arrive en position \( k \) ;
 \item que parmi les candidats en positions \( 1 \) à \( k-1 \), le meilleur d'entre eux figure dans
 les \( m \) premiers.
\end{itemize}

Dénombrement :
\begin{itemize}
 \item Choix des \( k-1 \) candidats avant la position \( k \) : \( \binom{n-1}{k-1} \) ;
 \item Choix de la position du meilleur d'entre eux dans les \( m \) premières : \( m \)
 possibilités ;
 \item Ordre des \( k-2 \) autres candidats sur les \( k-2 \) premières places restantes : \( (k-2)!
 \) ;
 \item Ordre des \( n - k \) candidats suivants : \( (n - k)! \).
\end{itemize}

Nombre total de permutations favorables :

\[
|B_k| = \binom{n-1}{k-1} \cdot m \cdot (k-2)! \cdot (n-k)! = \dfrac{(n-1)! \cdot m}{k - 1}
\]

D'où la probabilité cherchée :

\[
\mathbb{P}(B_k) = \dfrac{|B_k|}{n!} = \boxed{\dfrac{m}{n(k-1)}}, \quad \text{pour } k \in \llbracket m+1, n \rrbracket
\]

% ------

\q Probabilité que le meilleur candidat soit sélectionné
% ------
Il s'agit de sommer les probabilités précédentes pour toutes les positions admissibles \( k \in
\llbracket m+1, n \rrbracket \) :

\[
p_n(m) = \sum_{k = m+1}^{n} \dfrac{m}{n(k-1)} = \dfrac{m}{n} \sum_{k = m+1}^{n} \dfrac{1}{k - 1}
\]

\[
\boxed{p_n(m) = \dfrac{m}{n} \sum_{j = m}^{n - 1} \dfrac{1}{j}}
\]

% ------

\q Encadrement de la somme
% ------
On utilise les encadrements classiques de la fonction logarithme :

\[
\dfrac{1}{k+1} \leq \ln(k+1) - \ln(k) \leq \dfrac{1}{k}, \quad \forall k \in \mathbb{N}^*
\]

En sommant de \( k = m+1 \) à \( k = n \), on obtient :

\[
\ln(n+1) - \ln(m+1) \leq \sum_{k = m+1}^{n} \dfrac{1}{k} \leq \ln(n) - \ln(m)
\]

D'où :

\[
\boxed{
\dfrac{m}{n} \ln\left( \dfrac{n+1}{m+1} \right) \leq p_n(m) \leq \dfrac{m}{n} \ln\left( \dfrac{n}{m} \right)
}
\]

% ------

\q Comportement asymptotique
% ------
Soit \( x = \dfrac{m}{n} \in (0, 1) \). L'encadrement devient :

\[
x \ln\left( \dfrac{1 + \frac{1}{n}}{x + \frac{1}{n}} \right) \leq p_n(m) \leq x \ln\left( \dfrac{1}{x} \right)
\]

Or :

\[
\lim_{n \to +\infty} x \ln\left( \dfrac{1 + \frac{1}{n}}{x + \frac{1}{n}} \right) = x \ln\left( \dfrac{1}{x} \right)
\]

Donc, par le théorème des gendarmes :

\[
\boxed{\lim_{n \to +\infty} p_n(m) = x \ln\left( \dfrac{1}{x} \right)}
\]

% ------

\q Valeur optimale
% ------
On considère la fonction :

\[
f : x \mapsto x \ln\left( \dfrac{1}{x} \right) = -x \ln(x)
\]

Cette fonction admet un maximum en \( x = \dfrac{1}{e} \), car :

\[
f'(x) = -\ln(x) - 1 = 0 \iff x = \dfrac{1}{e}
\]

Et :

\[
f\left( \dfrac{1}{e} \right) = \dfrac{1}{e}
\]

Donc la valeur optimale du rapport \( \dfrac{m}{n} \) est :

\[
\boxed{\dfrac{m}{n} = \dfrac{1}{e}} \quad \text{et} \quad \boxed{\max p_n(m) \to \dfrac{1}{e}}
\]

% ------

\begin{center}
    \begin{tikzpicture}
        \begin{axis}[
            domain=0.01:1, samples=300, ymax=0.5,
            axis x line=middle, axis y line=middle,
            xlabel={$x = \frac{m}{n}$}, ylabel={$x \ln\left( \frac{1}{x} \right)$},
            width=10cm, height=6cm, grid=major
            ]
            \addplot[thick, blue] {x*ln(1/x)};
            \addplot[only marks, mark=*, blue] coordinates {(1/e, 1/e)};
            \draw[dashed] (axis cs:1/e,0) -- (axis cs:1/e,1/e);
            \draw[dashed] (axis cs:0,1/e) -- (axis cs:1/e,1/e);
        \end{axis}
    \end{tikzpicture}
\end{center}

\end{document}
% ==================================================================================================
