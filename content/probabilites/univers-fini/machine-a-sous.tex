% PROBLÈME : Machine à sous
% ==================================================================================================
%
% Buts
% ----
% - formule des probabilités totales
% - probabilités conditionnelles
% - raisonnements par récurrence
% - comportement asymptotique
% ==================================================================================================

\documentclass[10pt,a4paper]{article}

% Set the root path
\providecommand{\rootpath}{../../..}
% Fonts
\usepackage[utf8]{inputenc} % for accents
\usepackage[T1]{fontenc} % for accents
\usepackage[french]{babel} % for french language
\usepackage{helvet} % sans serif font family
\renewcommand*\familydefault{\sfdefault} % sans serif font family

% Mathematics
\usepackage{amsmath,amsfonts,amssymb} % for math symbols
\usepackage{array} % for tabular


\usepackage{parskip} % no indent, space between paragraphs

\usepackage{geometry} % margin
\geometry{
    a4paper,
    left=15mm,
    right=15mm,
    top=20mm,
    bottom=20mm
}

\usepackage{circledsteps} % to draw circles around numbers

\usepackage{fancyhdr} % for headers and footers

\usepackage{enumitem} % for customizing lists
\setlist[enumerate]{itemsep=1em} % space between items only in enumerate environment (not itemize)
\setlist[itemize]{label=--} % set itemize label to em-dash

% Command: \customPageLayout{#1}{#2}{#3}
% --------------------------------------
% Description: Custom page layout with header and footer content.
% Arguments:
% #1: Header and footer content
% #2: Left header content
% #3: Right header content
% Example:
% \customPageLayout{Title}{Lycée Henri IV}{2024}
% Required Packages: fancyhdr
\newcommand{\customPageLayout}[3]{
    \pagestyle{fancy} % set page style to fancy, i.e. header and footer
    \fancyhf{#1} % set header and footer content
    \lhead{#2} % set left header content
    \rhead{#3} % set right header content
    \fancyfoot{} % clear footer content
    \rfoot{\thepage} % set page number in footer
}

% Counter: \q
% -----------
% Description: Display a question number in a circle.
\newcounter{q}
\setcounter{q}{0} % set initial value of counter
\newcommand{\q}{
    \bigskip
    \addtocounter{q}{1}
    \par
    \Circled{\textbf{\theq}} \space
}


\title{Domaine - Thème}
\author{Esther Poniatowski}
\date{2024-2025}

\customPageLayout{Sujets d'interrogation orale}{Lycée Henri IV}{2024}

% ==================================================================================================
\begin{document}

\textbf{Objectifs}

Modéliser un processus stochastique  dépendant d'une stratégie adaptative dans un univers
fini.

\bigskip
\textbf{Contexte}

Les stratégies de décision adaptatives ("à renforcement") consistent à modifier les choix futurs en
fonction de l'expérience passée, afin d'optimiser les gains à long terme dans des contextes où
l'information est partielle. Ce type de comportement s'apparente à des modèles élémentaires
d'apprentissage en environnement incertain, rencontrés en algorithmique ou en théorie des jeux.

\bigskip
Ce problème considère un joueur explorant le fonctionnement de deux machines à sous.

La probabilité de gagner sur chaque machine est inconnue du joueur :
\begin{itemize}
 \item Sur la machine \( A \) : \( p_A = \frac{1}{5} \) ;
 \item Sur la machine \( B \) : \( p_B = \frac{1}{10} \).
\end{itemize}

Puisque le joueur ignore quelle machine est la plus favorable, il adopte une stratégie de décision
adaptative, en jouant sur l'une ou l'autre machine à sous en fonction de ses gains et pertes
successifs :
\begin{itemize}
 \item Il commence par choisir une machine au hasard.
 \item Après chaque partie :
   \begin{itemize}
    \item en cas de perte, il change de machine ;
    \item en cas de gain, il rejoue sur la même machine.
   \end{itemize}
\end{itemize}

\bigskip
Notations : pour tout \( k \geq 1 \),
\begin{itemize}
 \item \( G_k \) désigne l'événement "le joueur gagne la \( k^\text{ème} \) partie" ;
 \item \( A_k \) désigne l'événement "le joueur joue la \( k^\text{ème} \) partie sur la machine \(
 A \)" ;
 \item \( B_k \) désigne l'événement "le joueur joue la \( k^\text{ème} \) partie sur la machine \(
 B \)".
\end{itemize}

% --------------------------------------------------------------------------------------------------
\bigskip

\q Déterminer la probabilité de gagner la première partie.
% But : Calculer une probabilité totale sur un événement initial aléatoire. Méthode : Utiliser la
% formule des probabilités totales en fonction de \( A_1 \) et \( B_1 \), avec choix initial
% aléatoire.

\q Déterminer la probabilité de gagner la deuxième partie.
% But : Étudier la dépendance de la stratégie sur les gains ou pertes successifs. Méthode :
% Considérer les cas issus du premier coup (gain ou perte sur chaque machine) et appliquer les lois
% conditionnelles.

\q Déterminer \( \mathbb{P}(A_1 \mid G_2) \).
% But : Introduire une probabilité conditionnelle inverse. Méthode : Utiliser la formule de Bayes
% sur les événements \( A_1 \) et \( G_2 \).

\q Exprimer \( \mathbb{P}(G_k) \) en fonction de \( \mathbb{P}(A_k) \).
% But : Réduire l'étude de la probabilité de gain à la connaissance de la machine utilisée. Méthode
% : Appliquer la formule des probabilités totales conditionnelles selon les machines.

\q Exprimer \( \mathbb{P}(A_{k+1}) \) en fonction de \( \mathbb{P}(A_k) \).
% But : Mettre en place une récurrence sur l'évolution de la probabilité d'utilisation de la machine
% \( A \). Méthode : Étudier les cas gain/perte conditionnés à l'utilisation de la machine \( A \)
% ou \( B \) à l'instant \( k \).

\q Déterminer \( \mathbb{P}(A_k) \), puis \( \mathbb{P}(G_k) \), en fonction de \( k \).
% But : Résoudre la récurrence obtenue à la question précédente. Méthode : Identifier une suite
% géométrique ou arithmético-géométrique ; utiliser les valeurs initiales.

\q Déterminer, pour \( n \in \mathbb{N}^* \), la valeur de
\[
S_n = \sum_{k = 1}^{n} \mathbb{P}(G_k)
\]
puis déterminer la limite de \( \dfrac{S_n}{n} \) lorsque \( n \to +\infty \). Interpréter ce
résultat.
% But : Étudier le gain moyen à long terme. Méthode : Étudier la convergence de la moyenne des gains
% en s'appuyant sur la limite de \( \mathbb{P}(G_k) \).

\end{document}
% ==================================================================================================
