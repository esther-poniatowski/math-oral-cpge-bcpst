% CORRECTION : Titre
% ==================================================================================================

\documentclass[10pt,a4paper]{article}

% Set the root path
\providecommand{\rootpath}{../../..}
% Fonts
\usepackage[utf8]{inputenc} % for accents
\usepackage[T1]{fontenc} % for accents
\usepackage[french]{babel} % for french language
\usepackage{helvet} % sans serif font family
\renewcommand*\familydefault{\sfdefault} % sans serif font family

% Mathematics
\usepackage{amsmath,amsfonts,amssymb} % for math symbols
\usepackage{array} % for tabular


\usepackage{parskip} % no indent, space between paragraphs

\usepackage{geometry} % margin
\geometry{
    a4paper,
    left=15mm,
    right=15mm,
    top=20mm,
    bottom=20mm
}

\usepackage{circledsteps} % to draw circles around numbers

\usepackage{fancyhdr} % for headers and footers

\usepackage{enumitem} % for customizing lists
\setlist[enumerate]{itemsep=1em} % space between items only in enumerate environment (not itemize)
\setlist[itemize]{label=--} % set itemize label to em-dash

% Command: \customPageLayout{#1}{#2}{#3}
% --------------------------------------
% Description: Custom page layout with header and footer content.
% Arguments:
% #1: Header and footer content
% #2: Left header content
% #3: Right header content
% Example:
% \customPageLayout{Title}{Lycée Henri IV}{2024}
% Required Packages: fancyhdr
\newcommand{\customPageLayout}[3]{
    \pagestyle{fancy} % set page style to fancy, i.e. header and footer
    \fancyhf{#1} % set header and footer content
    \lhead{#2} % set left header content
    \rhead{#3} % set right header content
    \fancyfoot{} % clear footer content
    \rfoot{\thepage} % set page number in footer
}

% Counter: \q
% -----------
% Description: Display a question number in a circle.
\newcounter{q}
\setcounter{q}{0} % set initial value of counter
\newcommand{\q}{
    \bigskip
    \addtocounter{q}{1}
    \par
    \Circled{\textbf{\theq}} \space
}


\title{Domaine - Sujet}
\author{Esther Poniatowski}
\date{2024-2025}

\customPageLayout{Correction}{Lycée Henri IV}{2024}

% ==================================================================================================
\begin{document}

\textbf{Sans état absorbant}

\q Probabilité de retour à l'origine après \( k \) sauts

Pour un nombre impair de sauts : si \( k = 2n + 1 \), la particule ne peut pas revenir à son point
de départ, donc :
\[
\boxed{p_{2n+1} = 0}
\]

Pour un nombre pair de sauts : si \( k = 2n \), la particule effectue autant de sauts vers la gauche que vers la droite.

\medskip

\emph{Méthode 1 - Dénombrement} :

L'univers est l'ensemble des mots de longueur \( k \) sur l'alphabet \( \{D, G\} \) :

\[
|\Omega_k| = 2^k
\]

L'ensemble favorable correspond aux mots contenant exactement \( \dfrac{k}{2} \) lettres \( D \) et \(
\dfrac{k}{2} \) lettres \( G \) :

\[
|E_k| = \binom{k}{k/2}
\]

La probabilité cherchée est donc :

\[
\boxed{p_k = \dfrac{\binom{k}{k/2}}{2^k} = \dfrac{k!}{2^k \left( \dfrac{k}{2}! \right)^2}}
\]

\medskip

\emph{Méthode 2 - Loi binomiale} :

La probabilité d'obtenir exactement \( \dfrac{k}{2} \) déplacements vers la droite (ou gauche) est :

\[
p_k = \binom{k}{k/2} \left( \dfrac{1}{2} \right)^k = \dfrac{k!}{2^k \left( \dfrac{k}{2}! \right)^2}
\]

\q Probabilité que la particule se trouve à une distance \( y \) de l'origine après \( k \) sauts

Si \( k \) et \( y \) sont de parité différente, la probabilité est nulle (car en un nombre pair
(resp. impair) de sauts, particule atteint nécessairement une abscisse paire (resp. impaire)), donc
:

\[
\boxed{p_k(y) = 0 \quad \text{si } k \not\equiv y \mod 2}
\]

Si \( k \equiv y \mod 2 \), alors la particule a effectué $d$ sauts vers la droite et $g$ sauts vers
la gauche, tels que :
\[
\begin{cases}
d + g = k\\
d - g = y
\end{cases} \implies
\begin{cases}
d = \dfrac{k + y}{2} \\
g = \dfrac{k - y}{2}
\end{cases}
\]

En dénombrant comme précédemment :

\[
\boxed{p_k(y) = \dfrac{\binom{k}{\frac{k - y}{2}}}{2^k}}
\]

\q Retour à l'origine avec immobilité autorisée

Contrairement au cas précédent, la particule peut revenir à l'origine en un nombre impair de sauts.
Le retour à l'origine suppose autant de sauts à gauche qu'à droite (soit \( m \in [[ 0,
\frac{k'}{2} ]] \) de chaque), le reste étant des sauts sur place (\( k - m \) sauts), tels que :
\[
\begin{cases}
k = 2n +1 \implies k' = 2n = k - 1\\
k = 2n \implies k' = 2n = k
\end{cases}
\]

La situation peut désormais être modélisée de la manière suivante :
\begin{itemize}
 \item L'univers est représenté par l'ensemble des mots de $k$ lettres comportant 'D', 'G' ou 'I' : $\Omega_k = \lbrace D, G, I \rbrace^k$.
 \item Le cardinal de l'univers est : $|\Omega_k| = 3^k$.
 \item Le cardinal de l'ensemble d'intérêt est obtenu comme le nombre de manières d'agencer $m$
 lettres 'D' et $m$ lettres 'G' (les positions des $k - m$ lettres 'I' étant alors fixées) : $|E_k|
 = \sum_{m=0}^{\frac{k'}{2}}\binom{k}{m}\binom{k-m}{m}$.
\end{itemize}

\[
\boxed{p_k = \dfrac{ \sum_{m=0}^{\lfloor k/2 \rfloor} \binom{k}{m} \binom{k - m}{m}}{3^k}}
\]

\q Passage à une position \( y \) avec immobilité autorisée

Pour dénombrer les événements d'intérêt, on considère \( i \in \{0, \dots, k\} \) sauts immobiles,
le reste étant réparti entre droite et gauche. La position \( y \) est atteignable si \( k - i \) et
\( y \) sont de même parité.

Dans ce cas, la particule effectue :
\begin{itemize}
 \item $i \in [[ 0, k ]]$ sauts immobiles,
 \item $d = \frac{k-i+y}{2}$ vers la droite,
 \item $g = \frac{k-i-y}{2}$ vers la gauche.
\end{itemize}

Le cardinal des événements d'intérêt est donc, en choisissant d'abord les positions des lettres 'I'
puis celles des 'G' :
\[
\boxed{p_k(y) = \dfrac{\displaystyle \sum_{i = 0}^{k} \binom{k}{i} \binom{k - i}{\frac{k - i - y}{2}}}{3^k}}
\]




\bigskip

\textbf{Avec états absorbants}

\q Conditions aux bords :

\begin{itemize}
\item $u_0 = 1$ : lorsque le processus commence en 0, il s'arrête immédiatement en 0.
\item $u_N = 0$ : lorsque le processus commence en N, il s'arrête immédiatement en N.
\end{itemize}

\q Relation de récurrence :

Pour $i \notin [[ 0, N ]]$, la particule effectue au moins un déplacement. \\
Lors de ce premier déplacement, la particule ne peut atteindre que deux abscisses :
\begin{itemize}
 \item $i+1$, avec probabilité $p$,
 \item $i-1$, avec probabilité $q = 1-p$,
\end{itemize}
Notations des événements :
\begin{itemize}
 \item $Z_i$ : partant de $i$, le processus s'arrête en 0,
 \item $A_i$ : partant de $i$, la particule se trouve à l'abscisse $i+1$ à l'instant 1 (i.e. avance),
 \item $R_i$ : partant de $i$, la particule se trouve à l'abscisse $i-1$ à l'instant 1 (i.e. recule).
\end{itemize}

Les deux événements $(A_i, R_i)$ forment un système complet d'événements, donc par la formule des
probabilités totales :
\[
\begin{aligned}
   u_i = \mathbb{P}(Z_i) & = \mathbb{P}(A_i)\mathbb{P}(Z_i|A_i) + \mathbb{P}(R_i)\mathbb{P}(Z_i|R_i) \\
  & = p.\mathbb{P}(Z_i|A_i) + q.\mathbb{P}(Z_i|R_i)
\end{aligned}
\]
Or, par translation dans le temps, l'abscisse au temps 1 peut être considérée comme nouvelle origine
du processus :
\[
\mathbb{P}(Z_i|A_i) = u_{i+1}, \quad \mathbb{P}(Z_i|R_i) = u_{i-1}
\]
Conclusion :
\[
\boxed{u_i = p u_{i+1} + q u_{i-1} \quad \text{pour } i \in \{1, \dots, N-1\}}
\]

\q Résolution de l'équation

L'équation caractéristique associée est :

\[
X^2 - \dfrac{1}{p} X + \dfrac{q}{p} = 0
\]

Les racines sont \( r_1 = 1 \), \( r_2 = \dfrac{q}{p} \), donc la solution générale est :

\[
u_i = \lambda + \mu \left( \dfrac{q}{p} \right)^i
\]

Conditions initiales :

\[
\begin{cases}
u_0 = 1 = \lambda + \mu \\
u_N = 0 = \lambda + \mu \left( \dfrac{q}{p} \right)^N
\end{cases}
\]

Résolution :

\[
\lambda = \dfrac{q^N}{q^N - p^N} \quad ; \quad \mu = \dfrac{p^N}{p^N - q^N}
\]

D'où :

\[
\boxed{u_i = \dfrac{1}{q^N - p^N} \left( q^N - p^N \left( \dfrac{q}{p} \right)^i \right)}
\]

\q Probabilité d'absorption en \( N \)

Par symétrie et complémentarité :

\[
\boxed{v_i = \dfrac{1}{p^N - q^N} \left( p^N - q^N \left( \dfrac{q}{p} \right)^i \right)}
\]

\q Somme des deux probabilités

\[
\boxed{u_i + v_i = 1}
\]

Cela signifie que le processus s'arrête presque sûrement en \( 0 \) ou \( N \).

\end{document}
% ==================================================================================================
