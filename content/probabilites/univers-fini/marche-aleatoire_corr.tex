% CORRECTION : Titre
% ==================================================================================================

\documentclass[10pt,a4paper]{article}

% Set the root path
\providecommand{\rootpath}{../../..}
% Fonts
\usepackage[utf8]{inputenc} % for accents
\usepackage[T1]{fontenc} % for accents
\usepackage[french]{babel} % for french language
\usepackage{helvet} % sans serif font family
\renewcommand*\familydefault{\sfdefault} % sans serif font family

% Mathematics
\usepackage{amsmath,amsfonts,amssymb} % for math symbols
\usepackage{array} % for tabular


\usepackage{parskip} % no indent, space between paragraphs

\usepackage{geometry} % margin
\geometry{
    a4paper,
    left=15mm,
    right=15mm,
    top=20mm,
    bottom=20mm
}

\usepackage{circledsteps} % to draw circles around numbers

\usepackage{fancyhdr} % for headers and footers

\usepackage{enumitem} % for customizing lists
\setlist[enumerate]{itemsep=1em} % space between items only in enumerate environment (not itemize)
\setlist[itemize]{label=--} % set itemize label to em-dash

% Command: \customPageLayout{#1}{#2}{#3}
% --------------------------------------
% Description: Custom page layout with header and footer content.
% Arguments:
% #1: Header and footer content
% #2: Left header content
% #3: Right header content
% Example:
% \customPageLayout{Title}{Lycée Henri IV}{2024}
% Required Packages: fancyhdr
\newcommand{\customPageLayout}[3]{
    \pagestyle{fancy} % set page style to fancy, i.e. header and footer
    \fancyhf{#1} % set header and footer content
    \lhead{#2} % set left header content
    \rhead{#3} % set right header content
    \fancyfoot{} % clear footer content
    \rfoot{\thepage} % set page number in footer
}

% Counter: \q
% -----------
% Description: Display a question number in a circle.
\newcounter{q}
\setcounter{q}{0} % set initial value of counter
\newcommand{\q}{
    \bigskip
    \addtocounter{q}{1}
    \par
    \Circled{\textbf{\theq}} \space
}


\title{Domaine - Sujet}
\author{Esther Poniatowski}
\date{2024-2025}

\customPageLayout{Correction}{Lycée Henri IV}{2024}

% ==================================================================================================
\begin{document}

\textbf{Sans état absorbant}

\q Probabilité de retour à l'origine après \( k \) sauts
% ------
Démonstration de la nullité de la probabilité pour un nombre impair de sauts : si \( k = 2n + 1 \),
la particule ne peut pas revenir à son point de départ.

\[
\boxed{p_{2n+1} = 0}
\]

Pour \( k = 2n \) pair, la particule effectue autant de sauts vers la gauche que vers la droite.

\medskip

\emph{Méthode 1 - Dénombrement} :

L'univers est l'ensemble des mots de longueur \( k \) sur l'alphabet \( \{D, G\} \) :

\[
|\Omega_k| = 2^k
\]

L'ensemble favorable correspond aux mots avec exactement \( \dfrac{k}{2} \) lettres \( D \) et \(
\dfrac{k}{2} \) lettres \( G \) :

\[
|E_k| = \binom{k}{k/2}
\]

La probabilité cherchée est donc :

\[
\boxed{p_k = \dfrac{\binom{k}{k/2}}{2^k} = \dfrac{k!}{2^k \left( \dfrac{k}{2}! \right)^2}}
\]

\medskip

\emph{Méthode 2 - Loi binomiale} :

La probabilité d'obtenir exactement \( \dfrac{k}{2} \) déplacements vers la droite (ou gauche) est :

\[
p_k = \binom{k}{k/2} \left( \dfrac{1}{2} \right)^k = \dfrac{k!}{2^k \left( \dfrac{k}{2}! \right)^2}
\]

\q Probabilité d'être à une distance \( y \) de l'origine après \( k \) sauts
% ------
Analyse des parités : si \( k \) et \( y \) sont de parité différente, la probabilité est nulle :

\[
\boxed{p_k(y) = 0 \quad \text{si } k \not\equiv y \mod 2}
\]

Si \( k \equiv y \mod 2 \), alors la particule a effectué :

\[
\begin{cases}
d = \dfrac{k + y}{2} \\
g = \dfrac{k - y}{2}
\end{cases}
\]

D'où :

\[
\boxed{p_k(y) = \dfrac{\binom{k}{\frac{k - y}{2}}}{2^k}}
\]

\q Retour à l'origine avec immobilité autorisée
% ------
Contrairement au cas précédent, la particule peut revenir à l'origine en un nombre impair de sauts.
Le retour à l'origine suppose qu'il y ait autant de sauts à gauche qu'à droite (soit \( m \) de
chaque), le reste étant des pas immobiles.

\[
\boxed{p_k = \dfrac{\displaystyle \sum_{m=0}^{\lfloor k/2 \rfloor} \binom{k}{m} \binom{k - m}{m}}{3^k}}
\]

\q Passage à une position \( y \) avec immobilité autorisée
% ------
On considère \( i \in \{0, \dots, k\} \) sauts immobiles, le reste étant réparti entre droite et
gauche. La position \( y \) est atteignable si \( k - i \) et \( y \) sont de même parité.

\[
\boxed{p_k(y) = \dfrac{\displaystyle \sum_{i = 0}^{k} \binom{k}{i} \binom{k - i}{\frac{k - i - y}{2}}}{3^k}}
\]

\bigskip

\textbf{Avec états absorbants}

\q Conditions aux bords
% ------
\[
\boxed{u_0 = 1 \quad ; \quad u_N = 0}
\]

\q Relation de récurrence
% ------
Par la formule des probabilités totales conditionnelles :

\[
\boxed{u_i = p u_{i+1} + q u_{i-1} \quad \text{pour } i \in \{1, \dots, N-1\}}
\]

\q Résolution de l'équation
% ------
L'équation caractéristique associée est :

\[
X^2 - \dfrac{1}{p} X + \dfrac{q}{p} = 0
\]

Les racines sont \( r_1 = 1 \), \( r_2 = \dfrac{q}{p} \), donc la solution générale est :

\[
u_i = \lambda + \mu \left( \dfrac{q}{p} \right)^i
\]

Conditions initiales :

\[
\begin{cases}
u_0 = 1 = \lambda + \mu \\
u_N = 0 = \lambda + \mu \left( \dfrac{q}{p} \right)^N
\end{cases}
\]

Résolution :

\[
\lambda = \dfrac{q^N}{q^N - p^N} \quad ; \quad \mu = \dfrac{p^N}{p^N - q^N}
\]

D'où :

\[
\boxed{u_i = \dfrac{1}{q^N - p^N} \left( q^N - p^N \left( \dfrac{q}{p} \right)^i \right)}
\]

\q Probabilité d'absorption en \( N \)
% ------
Par symétrie et complémentarité :

\[
\boxed{v_i = \dfrac{1}{p^N - q^N} \left( p^N - q^N \left( \dfrac{q}{p} \right)^i \right)}
\]

\q Somme des deux probabilités
% ------
\[
\boxed{u_i + v_i = 1}
\]

Cela signifie que le processus s'arrête presque sûrement en \( 0 \) ou \( N \).

\end{document}
% ==================================================================================================
