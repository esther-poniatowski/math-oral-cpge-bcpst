% PROBLÈME : Problème du Casting
% ==================================================================================================
%
% Buts
% ----
% - combinatoire des permutations
% - probabilités conditionnelles
% - encadrements d'intégrales
% - passage à la limite
% ==================================================================================================

\documentclass[10pt,a4paper]{article}

% Set the root path
\providecommand{\rootpath}{../../..}
% Fonts
\usepackage[utf8]{inputenc} % for accents
\usepackage[T1]{fontenc} % for accents
\usepackage[french]{babel} % for french language
\usepackage{helvet} % sans serif font family
\renewcommand*\familydefault{\sfdefault} % sans serif font family

% Mathematics
\usepackage{amsmath,amsfonts,amssymb} % for math symbols
\usepackage{array} % for tabular


\usepackage{parskip} % no indent, space between paragraphs

\usepackage{geometry} % margin
\geometry{
    a4paper,
    left=15mm,
    right=15mm,
    top=20mm,
    bottom=20mm
}

\usepackage{circledsteps} % to draw circles around numbers

\usepackage{fancyhdr} % for headers and footers

\usepackage{enumitem} % for customizing lists
\setlist[enumerate]{itemsep=1em} % space between items only in enumerate environment (not itemize)
\setlist[itemize]{label=--} % set itemize label to em-dash

% Command: \customPageLayout{#1}{#2}{#3}
% --------------------------------------
% Description: Custom page layout with header and footer content.
% Arguments:
% #1: Header and footer content
% #2: Left header content
% #3: Right header content
% Example:
% \customPageLayout{Title}{Lycée Henri IV}{2024}
% Required Packages: fancyhdr
\newcommand{\customPageLayout}[3]{
    \pagestyle{fancy} % set page style to fancy (add header and footer)
    \fancyhf{} % clear all header and footer content
    \lhead{#2} % left header content
    \rhead{#3} % right header content
    \chead{\textbf{#1}} % center header content in bold (if needed)
    \rfoot{\thepage} % page number in the footer
}


% Counter: \q
% -----------
% Description: Display a question number in a circle.
% Usage:
% - Create a new question: add \q followed by the question content.
% - Reset the question counter: add \setcounter{q}{0} before the first question.
\newcounter{q}
\setcounter{q}{0} % set initial value of the counter
\newcommand{\q}{
    \bigskip
    \addtocounter{q}{1}
    \par
    \Circled{\textbf{\theq}} \space
}


% Counter: \ql
% ------------
% Description: Display a question letter in a round box with indentation (lowercase and not bold).
% Usage:
% - Create a new question: add \ql followed by the question content.
% - Reset the question counter: add \setcounter{ql}{0} before the first question.
\newcounter{ql}
\setcounter{ql}{0} % set initial value of the counter
\newcommand{\ql}{
    \addtocounter{ql}{1}
    \par
    \hspace{1.5em} % indentation before the circled letter
    \textcolor{gray}{\Circled{\alph{ql}}} \space % gray color
}


\title{Probabilités dans un univers fini - Problème du casting}
\author{Esther Poniatowski}
\date{2024-2025}

\customPageLayout{Sujets d'interrogation orale}{Lycée Henri IV}{2024}

% ==================================================================================================
\begin{document}

\textbf{Objectifs}

Modéliser une stratégie probabiliste optimale dans un processus de décision séquentiel. Illustrer
une optimisation sous contrainte dans un cadre probabiliste discret.

\bigskip
\textbf{Contexte}

Le \textit{problème d'arrêt optimal}  modélise une prise de décision en temps réel, dont le but est
d'optimiser un choix parmi une séquence d'éléments révélés successivement et de manière
irréversible. De telles modélisations apparaissent en économie, en théorie du choix, et en
algorithmique décisionnelle.

% --------------------------------------------------------------------------------------------------
\bigskip

Ce sujet considère une version classique de ce problème. Dans un studio de cinéma, un casteur doit
sélectionner rapidement le meilleur candidat parmi \( n \) postulants qui se présentent dans un
ordre aléatoire.

Afin d'éviter d'auditionner l'ensemble des postulants, le casteur adopte la stratégie suivante :
\begin{itemize}
 \item Il fixe un nombre de candidats à auditionner
  sans sélection : \( m \in \{1, \dots, n-1\} \).
 \item Parmi les candidats suivants, il sélectionne le premier qui est meilleur que tous les \( m \)
 premiers auditionnés.
\end{itemize}

\bigskip
Notations :
\begin{itemize}
 \item \( C_k \) : Événement "le meilleur candidat se présente à la position \( k \)" ;
 \item \( S_k \) : Événement "le meilleur candidat est sélectionné" ;
 \item \( p_n(m) \) : Probabilité que le meilleur candidat soit sélectionné lorsque le nombre de
  auditionnés est \( m \) et que le nombre total de candidats est \( n \).
\end{itemize}

\q Préciser la probabilité que le meilleur candidat se présente à la \( k^\text{ème} \) position,
pour tout \( k \in \{1, \dots, n\} \).\\ Faire le lien avec le nombre de permutations des candidats,
et le nombre de permutations où le meilleur candidat est en position \( k \).
% But : Identifier une loi uniforme sur les positions. Méthode : Considérer que toutes les
% permutations sont équiprobables.

\q Montrer que, pour tout \( k \in \{m+1, \dots, n\} \), la probabilité que le meilleur candidat se
présente à la position \( k \) et soit sélectionné vaut :
\[
\mathbb{P}(C_k \cap S_k) = \frac{m}{n(k-1)}
\]
% But : Évaluer une probabilité conditionnelle dans une stratégie sélective. Méthode : Étudier le
% cas favorable : le meilleur est en \( k \), et aucun meilleur que les m premiers ne survient
% entre m+1 et k-1.

\q Exprimer la probabilité \( p_n(m) \) sous la forme d'une somme.
% But : Exprimer la probabilité totale de réussite de la stratégie. Méthode : Somme des probabilités
% conditionnelles pour \( k = m+1 \) à \( n \).

\q Montrer que \( p_n(m) \) satisfait les encadrements :
\[
\dfrac{m}{n} \ln\left( \dfrac{n+1}{m+1} \right) \leq p_n(m) \leq \dfrac{m}{n} \ln\left( \dfrac{n}{m} \right)
\]
% But : Encadrer la somme obtenue précédemment. Méthode : Utiliser une inégalité classique
% d'encadrement d'une somme par une intégrale ou des différences de logarithmes.

\q Montrer que la probabilité \( p_n(m) \) admet une limite lorsque \( n \to +\infty \) avec un
rapport \( \dfrac{m}{n} \) fixé.
% But : Étudier une asymptotique continue de la stratégie. Méthode : Appliquer le passage à la
% limite sur l'encadrement logarithmique.

\q Représenter graphiquement la limite obtenue en fonction de \( \dfrac{m}{n} \). Proposer une
valeur optimale de ce rapport pour maximiser la probabilité de sélection du meilleur candidat.
% But : Interpréter graphiquement et optimiser le rendement de la stratégie. Méthode : Tracer la
% fonction limite, repérer le maximum et le rapport optimal correspondant.

\end{document}
% ==================================================================================================
