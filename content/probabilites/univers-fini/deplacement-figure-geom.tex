% PROBLÈME : Déplacement sur une figure géométrique
% ==================================================================================================
%
% Buts
% ----
% - formule des probabilités totales
% - probabilités conditionnelles
% - propriétés de symétrie
% - raisonnements par récurrence
%   ==================================================================================================

\documentclass[10pt,a4paper]{article}

% Set the root path
\providecommand{\rootpath}{../../..}
% Fonts
\usepackage[utf8]{inputenc} % for accents
\usepackage[T1]{fontenc} % for accents
\usepackage[french]{babel} % for french language
\usepackage{helvet} % sans serif font family
\renewcommand*\familydefault{\sfdefault} % sans serif font family

% Mathematics
\usepackage{amsmath,amsfonts,amssymb} % for math symbols
\usepackage{array} % for tabular


\usepackage{parskip} % no indent, space between paragraphs

\usepackage{geometry} % margin
\geometry{
    a4paper,
    left=15mm,
    right=15mm,
    top=20mm,
    bottom=20mm
}

\usepackage{circledsteps} % to draw circles around numbers

\usepackage{fancyhdr} % for headers and footers

\usepackage{enumitem} % for customizing lists
\setlist[enumerate]{itemsep=1em} % space between items only in enumerate environment (not itemize)
\setlist[itemize]{label=--} % set itemize label to em-dash

% Command: \customPageLayout{#1}{#2}{#3}
% --------------------------------------
% Description: Custom page layout with header and footer content.
% Arguments:
% #1: Header and footer content
% #2: Left header content
% #3: Right header content
% Example:
% \customPageLayout{Title}{Lycée Henri IV}{2024}
% Required Packages: fancyhdr
\newcommand{\customPageLayout}[3]{
    \pagestyle{fancy} % set page style to fancy, i.e. header and footer
    \fancyhf{#1} % set header and footer content
    \lhead{#2} % set left header content
    \rhead{#3} % set right header content
    \fancyfoot{} % clear footer content
    \rfoot{\thepage} % set page number in footer
}

% Counter: \q
% -----------
% Description: Display a question number in a circle.
\newcounter{q}
\setcounter{q}{0} % set initial value of counter
\newcommand{\q}{
    \bigskip
    \addtocounter{q}{1}
    \par
    \Circled{\textbf{\theq}} \space
}


\title{Probabilités dans un univers fini - Déplacement sur une figure géométrique}
\author{Esther Poniatowski}
\date{2024-2025}

\customPageLayout{Sujets d'interrogation orale}{Lycée Henri IV}{2024}

% ==================================================================================================
\begin{document}

\textbf{Objectifs}

Étudier un processus aléatoire évolutif dans un univers fini, modélisé par une marche aléatoire.

\bigskip
\textbf{Contexte}

Les marches aléatoires sur des graphes finis sont des modèles probabilistes qui étudient des
processus stochastiques évoluant dans un univers discret et fini. Ces modèles trouvent des
applications en physique (mouvement brownien simplifié), en informatique (algorithmes de navigation
aléatoire), ou en théorie des graphes (diffusion sur réseaux).


\bigskip

Ce problème considère les déplacements aléatoires d'un mobile sur les sommets de figures
géométriques régulières (spécifiées dans l'énoncé).


Dans tout l'exercice, les notations suivantes sont utilisées :
\begin{itemize}
\item \( X_n \) désigne l'événement \textit{"le mobile se trouve au sommet \( X \) à l'instant \( n
\)"} ;
\item \( x_n \) désigne la probabilité associée à l'événement \( X_n \).
\end{itemize}


% --------------------------------------------------------------------------------------------------
\bigskip
\textbf{Déplacement dans un triangle}

La figure géométrique considérée est un triangle \( ABC \). Initialement, le mobile se trouve au
sommet \( A \).

À chaque instant \( n \), le mobile est situé sur un sommet et peut se déplacer vers l'un des deux
autres sommets avec une probabilité \( \dfrac{1}{6} \) pour chacun.

\q Déterminer les expressions de \( a_{n+1} \), \( b_{n+1} \) et \( c_{n+1} \) en fonction de \( a_n
\), \( b_n \) et \( c_n \), pour tout \( n \in \mathbb{N} \).
% But : Construire le système de récurrence issu de la dynamique probabiliste. Méthode : Appliquer
% la formule des probabilités totales à partir des lois de transition.

\q Donner des expressions explicites de \( a_n \), \( b_n \) et \( c_n \) pour tout \( n \in
\mathbb{N} \).
% But : Résoudre le système de récurrence à l'aide de la symétrie du triangle. Méthode : Introduire
% les différences \( a_n - b_n \) et \( a_n - c_n \), puis utiliser la condition de normalisation.

% ---------------------------------------------------------------------------------------------------
\bigskip
\textbf{Déplacement dans un carré}

La figure géométrique considérée est un carré \( ABCD \) de centre \( O \). Initialement, le mobile
est situé au sommet \( A \).

À chaque instant, le mobile se déplace vers un sommet voisin de sa position actuelle, choisi de
manière équiprobable.

\q Déterminer la probabilité que le mobile soit au sommet \( A \), \( B \), \( C \), \( D \) ou \( O
\) après deux déplacements.
% But : Étudier les premiers termes d'une chaîne de Markov sur un graphe géométrique. Méthode :
% Appliquer deux fois la formule des probabilités totales et exploiter les symétries.

\q Déterminer une relation de récurrence liant \( o_{n+1} \) à \( o_n \) pour tout \( n \in
\mathbb{N} \). En déduire une expression explicite de \( o_n \).
% But : Étudier la dynamique probabiliste du sommet central. Méthode : Utiliser la conservation de
% la probabilité totale et reconnaître une suite arithmético-géométrique.

\q Déterminer les probabilités conditionnelles suivantes :
\begin{itemize}
 \item \( \mathbb{P}(O_3 \mid B_2) \) ;
 \item \( \mathbb{P}(O_3 \mid B_1) \).
\end{itemize}
% But : Introduire et manipuler des probabilités conditionnelles dans un processus markovien.
% Méthode : Exploiter les invariances temporelles du processus.

\q Déterminer la probabilité que le mobile soit passé par \( B \) sachant qu'il est en \( O \) après
trois déplacements.
% But : Calculer une probabilité conditionnelle inverse. Méthode : Utiliser la formule de Bayes avec
% disjonction des événements \( B_1 \) et \( B_2 \).

\end{document}
% ==================================================================================================
