% CORRECTION : Déplacement sur une figure géométrique
% ==================================================================================================

\documentclass[10pt,a4paper]{article}

% Set the root path
\providecommand{\rootpath}{../../..}
% Fonts
\usepackage[utf8]{inputenc} % for accents
\usepackage[T1]{fontenc} % for accents
\usepackage[french]{babel} % for french language
\usepackage{helvet} % sans serif font family
\renewcommand*\familydefault{\sfdefault} % sans serif font family

% Mathematics
\usepackage{amsmath,amsfonts,amssymb} % for math symbols
\usepackage{array} % for tabular


\usepackage{parskip} % no indent, space between paragraphs

\usepackage{geometry} % margin
\geometry{
    a4paper,
    left=15mm,
    right=15mm,
    top=20mm,
    bottom=20mm
}

\usepackage{circledsteps} % to draw circles around numbers

\usepackage{fancyhdr} % for headers and footers

\usepackage{enumitem} % for customizing lists
\setlist[enumerate]{itemsep=1em} % space between items only in enumerate environment (not itemize)
\setlist[itemize]{label=--} % set itemize label to em-dash

% Command: \customPageLayout{#1}{#2}{#3}
% --------------------------------------
% Description: Custom page layout with header and footer content.
% Arguments:
% #1: Header and footer content
% #2: Left header content
% #3: Right header content
% Example:
% \customPageLayout{Title}{Lycée Henri IV}{2024}
% Required Packages: fancyhdr
\newcommand{\customPageLayout}[3]{
    \pagestyle{fancy} % set page style to fancy (add header and footer)
    \fancyhf{} % clear all header and footer content
    \lhead{#2} % left header content
    \rhead{#3} % right header content
    \chead{\textbf{#1}} % center header content in bold (if needed)
    \rfoot{\thepage} % page number in the footer
}


% Counter: \q
% -----------
% Description: Display a question number in a circle.
% Usage:
% - Create a new question: add \q followed by the question content.
% - Reset the question counter: add \setcounter{q}{0} before the first question.
\newcounter{q}
\setcounter{q}{0} % set initial value of the counter
\newcommand{\q}{
    \bigskip
    \addtocounter{q}{1}
    \par
    \Circled{\textbf{\theq}} \space
}


% Counter: \ql
% ------------
% Description: Display a question letter in a round box with indentation (lowercase and not bold).
% Usage:
% - Create a new question: add \ql followed by the question content.
% - Reset the question counter: add \setcounter{ql}{0} before the first question.
\newcounter{ql}
\setcounter{ql}{0} % set initial value of the counter
\newcommand{\ql}{
    \addtocounter{ql}{1}
    \par
    \hspace{1.5em} % indentation before the circled letter
    \textcolor{gray}{\Circled{\alph{ql}}} \space % gray color
}


\title{Probabilités dans un univers fini - Déplacement sur une figure géométrique}
\author{Esther Poniatowski}
\date{2024-2025}

\customPageLayout{Correction}{Lycée Henri IV}{2024}

% ==================================================================================================
\begin{document}

\textbf{Déplacement dans un triangle}

\q Les événements \( A_n, B_n, C_n \) forment un système complet d'événements. D'après la formule
des probabilités totales :
\[
\mathbb{P}(A_{n+1}) = \mathbb{P}(A_{n+1} \mid A_n) \mathbb{P}(A_n) + \mathbb{P}(A_{n+1} \mid B_n) \mathbb{P}(B_n) + \mathbb{P}(A_{n+1} \mid C_n) \mathbb{P}(C_n)
\]
On en déduit :
\[
\begin{cases}
a_{n+1} = \dfrac{2}{3} a_n + \dfrac{1}{6} b_n + \dfrac{1}{6} c_n \\
b_{n+1} = \dfrac{1}{6} a_n + \dfrac{2}{3} b_n + \dfrac{1}{6} c_n \\
c_{n+1} = \dfrac{1}{6} a_n + \dfrac{1}{6} b_n + \dfrac{2}{3} c_n
\end{cases}
\]
% But : Introduire un système probabiliste évolutif dans un cadre fini. Méthode : Appliquer la
% formule des probabilités totales avec une matrice de transition implicite.

\q Montrer que les suites \( (a_n - b_n)_{n \in \mathbb{N}} \) et \( (a_n - c_n)_{n \in \mathbb{N}}
\) sont géométriques de raison \( \dfrac{1}{2} \)
% But : Exploiter la symétrie des équations de récurrence. Méthode : Calculer les différences et
% vérifier la forme géométrique.

\q Déterminer des expressions explicites de \( a_n, b_n \) et \( c_n \)
% But : Déduire les expressions générales en utilisant les suites géométriques précédentes et la
% condition \( a_n + b_n + c_n = 1 \). Méthode : Résolution d’un système par substitution en
% exploitant les résultats précédents.


\bigskip
\textbf{Déplacement dans un carré}

\q \emph{Étude des premiers déplacements}
% But : Comprendre les premières itérations du processus probabiliste. Méthode : Utiliser la formule
% des probabilités totales et les symétries.

Après un déplacement, le mobile ne peut rester en \( A \), ni atteindre \( C \). Il atteint alors \(
B \), \( D \), ou \( O \) avec équiprobabilité :
\[
\begin{cases}
a_1 = c_1 = 0 \\
b_1 = d_1 = o_1 = \dfrac{1}{3}
\end{cases}
\]

Après deux déplacements, en appliquant la formule des probabilités totales et en tenant compte des
symétries :
\[
\begin{cases}
a_2 = c_2 = \dfrac{11}{36} \\
b_2 = d_2 = \dfrac{1}{12} \\
o_2 = \dfrac{2}{9}
\end{cases}
\]

\q Déterminer une expression explicite de \( o_n \)
% But : Étudier la suite associée à la probabilité d’être en \( O \). Méthode : Mise en place d’une
% relation de récurrence et résolution d’une suite arithmético-géométrique.

La relation de récurrence est donnée par :
\[
o_{n+1} = \dfrac{1}{3} (1 - o_n)
\]
Il s'agit d'une suite arithmético-géométrique. Son point fixe est \( \ell = \dfrac{1}{4} \) et sa
raison est \( -\dfrac{1}{3} \). Comme \( o_1 = \dfrac{1}{3} \), on obtient :
\[
o_n = \dfrac{1}{4} + \dfrac{1}{12} \left( -\dfrac{1}{3} \right)^{n-1}
\]

\q Déterminer \( \mathbb{P}(O_3 \mid B_2) \) et \( \mathbb{P}(O_3 \mid B_1) \)
% But : Introduire et exploiter l’invariance temporelle du processus. Méthode : Utiliser les
% propriétés de stationnarité du processus markovien.

Par invariance temporelle :
\[
\mathbb{P}(O_3 \mid B_2) = \mathbb{P}(O_1 \mid B_0) = \dfrac{1}{3}
\]
Par symétrie et invariance :
\[
\mathbb{P}(O_3 \mid B_1) = \mathbb{P}(O_3 \mid A_1) = \mathbb{P}(O_2 \mid A_0) = \dfrac{2}{9}
\]

\q Calculer \( \mathbb{P}_{O_3}(B_1 \cup B_2) \)
% But : Étudier une probabilité conditionnelle inverse par la formule de Bayes. Méthode : Utiliser
% la disjonction des événements et la formule de Bayes.

On a :
\[
\mathbb{P}_{O_3}(B_1 \cup B_2) = \mathbb{P}(B_1 \mid O_3) + \mathbb{P}(B_2 \mid O_3)
\]

Par la formule de Bayes :
\[
\begin{cases}
\mathbb{P}(B_2 \mid O_3) = \dfrac{\mathbb{P}(O_3 \mid B_2) \mathbb{P}(B_2)}{\mathbb{P}(O_3)} = \dfrac{\dfrac{1}{3} \times \dfrac{1}{12}}{\dfrac{7}{27}} = \dfrac{3}{28} \\
\mathbb{P}(B_1 \mid O_3) = \dfrac{\mathbb{P}(O_3 \mid B_1) \mathbb{P}(B_1)}{\mathbb{P}(O_3)} = \dfrac{\dfrac{2}{9} \times \dfrac{1}{3}}{\dfrac{7}{27}} = \dfrac{2}{7}
\end{cases}
\]

Il vient :
\[
\mathbb{P}_{O_3}(B_1 \cup B_2) = \dfrac{11}{28}
\]

\end{document}
% ==================================================================================================
