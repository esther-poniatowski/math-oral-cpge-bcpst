% CORRECTION : Déplacement sur une figure géométrique
% ==================================================================================================

\documentclass[10pt,a4paper]{article}

% Set the root path
\providecommand{\rootpath}{../../..}
% Fonts
\usepackage[utf8]{inputenc} % for accents
\usepackage[T1]{fontenc} % for accents
\usepackage[french]{babel} % for french language
\usepackage{helvet} % sans serif font family
\renewcommand*\familydefault{\sfdefault} % sans serif font family

% Mathematics
\usepackage{amsmath,amsfonts,amssymb} % for math symbols
\usepackage{array} % for tabular


\usepackage{parskip} % no indent, space between paragraphs

\usepackage{geometry} % margin
\geometry{
    a4paper,
    left=15mm,
    right=15mm,
    top=20mm,
    bottom=20mm
}

\usepackage{circledsteps} % to draw circles around numbers

\usepackage{fancyhdr} % for headers and footers

\usepackage{enumitem} % for customizing lists
\setlist[enumerate]{itemsep=1em} % space between items only in enumerate environment (not itemize)
\setlist[itemize]{label=--} % set itemize label to em-dash

% Command: \customPageLayout{#1}{#2}{#3}
% --------------------------------------
% Description: Custom page layout with header and footer content.
% Arguments:
% #1: Header and footer content
% #2: Left header content
% #3: Right header content
% Example:
% \customPageLayout{Title}{Lycée Henri IV}{2024}
% Required Packages: fancyhdr
\newcommand{\customPageLayout}[3]{
    \pagestyle{fancy} % set page style to fancy, i.e. header and footer
    \fancyhf{#1} % set header and footer content
    \lhead{#2} % set left header content
    \rhead{#3} % set right header content
    \fancyfoot{} % clear footer content
    \rfoot{\thepage} % set page number in footer
}

% Counter: \q
% -----------
% Description: Display a question number in a circle.
\newcounter{q}
\setcounter{q}{0} % set initial value of counter
\newcommand{\q}{
    \bigskip
    \addtocounter{q}{1}
    \par
    \Circled{\textbf{\theq}} \space
}


\title{Probabilités dans un univers fini - Déplacement sur une figure géométrique}
\author{Esther Poniatowski}
\date{2024-2025}

\customPageLayout{Correction}{Lycée Henri IV}{2024}

% ==================================================================================================
\begin{document}

\textbf{Déplacement dans un triangle}

\q Les événements \( A_n, B_n, C_n \) forment un système complet d'événements. D'après la formule
des probabilités totales :
\[
\mathbb{P}(A_{n+1}) = \mathbb{P}(A_{n+1} \mid A_n) \mathbb{P}(A_n) + \mathbb{P}(A_{n+1} \mid B_n) \mathbb{P}(B_n) + \mathbb{P}(A_{n+1} \mid C_n) \mathbb{P}(C_n)
\]
On en déduit :
\[
\begin{cases}
a_{n+1} = \frac{2}{3} a_n + \frac{1}{6} b_n + \frac{1}{6} c_n \\
b_{n+1} = \frac{1}{6} a_n + \frac{2}{3} b_n + \frac{1}{6} c_n \\
c_{n+1} = \frac{1}{6} a_n + \frac{1}{6} b_n + \frac{2}{3} c_n
\end{cases}
\]

\q Relation sur la somme des probabilités :

Pour tout \( \) :
\[
\forall  n \in \mathbb{N}, \quad  a_n + b_n + c_n = 1
\]

\q Expressions explicites :

Les suites \( (a_n - b_n)_{n \in \mathbb{N}} \) et \( (a_n - c_n)_{n \in \mathbb{N}}
\) sont géométriques de raison \( \frac{1}{2} \) :
\[
\begin{cases}
  a_{n+1} - c_{n+1} = \frac{1}{2}(a_n - c_n)\\
  a_{n+1} - b_{n+1} = \frac{1}{2}(a_n - b_n)\\
\end{cases}
\]
Terme général :
\[
\begin{cases}
  a_n - c_n = \frac{1}{2^n}(a_0 - c_0) = \frac{1}{2^n}\\
  a_n - b_n = \frac{1}{2^n}(a_0 - b_0) = \frac{1}{2^n}\\
 \end{cases}
\]
(car $a_0 = 1$ et $b_0 = c_0 = 0$), d'où $  b_n = c_n = \frac{1}{2^n} - a_n$\\

Or, $a_n + b_n + c_n = 1 \implies 3 a_n - \frac{1}{2^{n-1}} = 1 \implies a_n = \frac{1}{3}(1 -
\frac{1}{2^{n-1}})$, donc :
\[
\begin{cases}
  a_n = \frac{1}{3}(1 - \frac{1}{2^{n-1}})\\
  b_n = c_n = \frac{1}{3}(1 - \frac{1}{2^n})
 \end{cases}
 \]


\bigskip
\textbf{Déplacement dans un carré}

\q \emph{Étude des premiers déplacements}

Après un déplacement, le mobile ne peut rester en \( A \), ni atteindre \( C \). Il atteint alors \(
B \), \( D \), ou \( O \) avec équiprobabilité :
\[
\begin{cases}
a_1 = c_1 = 0 \\
b_1 = d_1 = o_1 = \frac{1}{3}
\end{cases}
\]

Après deux déplacements, par la formule des probabilités totales, pour tout sommet $X$ :\\
$x_2 = \mathbb{P}(X_2|A_1)a_1 + \mathbb{P}(X_2|B_1)b_1 + \mathbb{P}(X_2|C_1)c_1 +
\mathbb{P}(X|D_1)d_1 + \mathbb{P}(X_2|O_1)o_1$.

De plus :
\begin{itemize}
 \item le mobile ne reste pas sur le même sommet : $\mathbb{P}(X_2|X_1) = 0$,
 \item à partir des sommets A, B, C, et D, trois sommets peuvent être atteints avec probabilité $\frac{1}{3}$,
 \item à partir du sommet O, quatre sommets peuvent être atteints avec probabilité $\frac{1}{4}$.
 \end{itemize}
 D'autre part, certains sommets jouent un rôle symétrique :
 \begin{itemize}
 \item B et D,
 \item A et C, à partir du premier déplacement.
\end{itemize}
Ainsi :
\[
\begin{cases}
  a_2 = \frac{1}{3}b_1 + 0.c_1 + \frac{1}{3}d_1 + \frac{1}{4}o_1 = \frac{1}{3}\frac{1}{3} + \frac{1}{3}\frac{1}{3} + \frac{1}{4}\frac{1}{3} = \frac{1}{3}\mathbb{P}(2 \times \frac{1}{3} + \frac{1}{4}) = \frac{11}{36}\\
  b_2 = \frac{1}{3}a_1 + \frac{1}{3} c_1 + 0.d_1 + \frac{1}{4}o_1 = \frac{1}{3}.0 + \frac{1}{3}.0 + \frac{1}{4}\frac{1}{3} = \frac{1}{12}\\
  o_2 = \frac{1}{3}a_1 + \frac{1}{3}b_1 + \frac{1}{3}c_1 + \frac{1}{3}d_1 = \frac{1}{3}.0 + \frac{1}{3}\frac{1}{3} + \frac{1}{3}.0 + \frac{1}{3}\frac{1}{3} = \frac{2}{9}
 \end{cases}
\]

Conclusion :
\[
\begin{cases}
  a_2 = c_2 = \frac{11}{36}\\
  b_2 = d_2 = \frac{1}{12}\\
  o_2 = \frac{2}{9}
 \end{cases}
\]

\q Expression explicite de \( o_n \) :

La relation de récurrence est donnée par :
\[
o_{n+1} = \frac{1}{3}(a_n + b_n + c_n + d_n) = \frac{1}{3} (1 - o_n)
\]
Il s'agit d'une suite arithmético-géométrique. Son point fixe est \( \ell = \frac{1}{4} \) et sa
raison est \( -\frac{1}{3} \). Comme \( o_1 = \frac{1}{3} \) :
\[
o_n = \frac{1}{4} + \frac{1}{12} \left( -\frac{1}{3} \right)^{n-1}
\]

\q Probabilités conditionnelles de présence en O :

Par invariance temporelle :
\[
\mathbb{P}(O_3 \mid B_2) = \mathbb{P}(O_1 \mid B_0) = \frac{1}{3}
\]
Par symétrie et invariance :
\[
\mathbb{P}(O_3 \mid B_1) = \mathbb{P}(O_3 \mid A_1) = \mathbb{P}(O_2 \mid A_0) = \frac{2}{9}
\]

\q Probabilités conditionnelles de passage en B :

Comme les événements $B_1$ et $B_2$ sont disjoints, la probabilité recherchée est :

\[
\mathbb{P}_{O_3}(B_1 \cup B_2) = \mathbb{P}(B_1 \mid O_3) + \mathbb{P}(B_2 \mid O_3)
\]

Par la formule de Bayes :
\[
\begin{cases}
\mathbb{P}(B_2 \mid O_3) = \frac{\mathbb{P}(O_3 \mid B_2) \mathbb{P}(B_2)}{\mathbb{P}(O_3)} = \frac{\frac{1}{3} \times \frac{1}{12}}{\frac{7}{27}} = \frac{3}{28} \\
\mathbb{P}(B_1 \mid O_3) = \frac{\mathbb{P}(O_3 \mid B_1) \mathbb{P}(B_1)}{\mathbb{P}(O_3)} = \frac{\frac{2}{9} \times \frac{1}{3}}{\frac{7}{27}} = \frac{2}{7}
\end{cases}
\]

Avec :
\begin{itemize}
    \item $\mathbb{P}(O_3) = o_3 = \frac{1}{4} + \frac{1}{12}(-\frac{1}{3})^{2} = \frac{7}{27}$ selon la question ?,
    \item $\mathbb{P}(O_3|B_2) = \frac{1}{3}$ et $\mathbb{P}(O_3|B_1) = \frac{2}{9}$ selon la question ?,
    \item $\mathbb{P}(B_1) = b_1 = \frac{1}{3}$ et $\mathbb{P}(B_2) = b_2 = \frac{1}{12}$ selon la
    question ?. \\
\end{itemize}

Application numérique :
\[
\mathbb{P}_{O_3}(B_1 \cup B_2) = \frac{11}{28}
\]

\end{document}
% ==================================================================================================
