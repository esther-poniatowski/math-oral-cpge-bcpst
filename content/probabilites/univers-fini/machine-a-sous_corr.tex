% CORRECTION : Machine à sous
% ==================================================================================================

\documentclass[10pt,a4paper]{article}

% Set the root path
\providecommand{\rootpath}{../../..}
% Fonts
\usepackage[utf8]{inputenc} % for accents
\usepackage[T1]{fontenc} % for accents
\usepackage[french]{babel} % for french language
\usepackage{helvet} % sans serif font family
\renewcommand*\familydefault{\sfdefault} % sans serif font family

% Mathematics
\usepackage{amsmath,amsfonts,amssymb} % for math symbols
\usepackage{array} % for tabular


\usepackage{parskip} % no indent, space between paragraphs

\usepackage{geometry} % margin
\geometry{
    a4paper,
    left=15mm,
    right=15mm,
    top=20mm,
    bottom=20mm
}

\usepackage{circledsteps} % to draw circles around numbers

\usepackage{fancyhdr} % for headers and footers

\usepackage{enumitem} % for customizing lists
\setlist[enumerate]{itemsep=1em} % space between items only in enumerate environment (not itemize)
\setlist[itemize]{label=--} % set itemize label to em-dash

% Command: \customPageLayout{#1}{#2}{#3}
% --------------------------------------
% Description: Custom page layout with header and footer content.
% Arguments:
% #1: Header and footer content
% #2: Left header content
% #3: Right header content
% Example:
% \customPageLayout{Title}{Lycée Henri IV}{2024}
% Required Packages: fancyhdr
\newcommand{\customPageLayout}[3]{
    \pagestyle{fancy} % set page style to fancy, i.e. header and footer
    \fancyhf{#1} % set header and footer content
    \lhead{#2} % set left header content
    \rhead{#3} % set right header content
    \fancyfoot{} % clear footer content
    \rfoot{\thepage} % set page number in footer
}

% Counter: \q
% -----------
% Description: Display a question number in a circle.
\newcounter{q}
\setcounter{q}{0} % set initial value of counter
\newcommand{\q}{
    \bigskip
    \addtocounter{q}{1}
    \par
    \Circled{\textbf{\theq}} \space
}


\title{Probabilités dans un univers fini - Machine à sous}
\author{Esther Poniatowski}
\date{2024-2025}

\customPageLayout{Correction}{Lycée Henri IV}{2024}

% ==================================================================================================
\begin{document}

\q Probabilité de gagner la première partie :

Par la formule des probabilités totales :
\[
\mathbb{P}(G_1) = \mathbb{P}(G_1 \mid A_1) \mathbb{P}(A_1) + \mathbb{P}(G_1 \mid B_1) \mathbb{P}(B_1)
\]

Comme \( \mathbb{P}(A_1) = \mathbb{P}(B_1) = \dfrac{1}{2} \) :
\[
\boxed{\mathbb{P}(G_1) = \dfrac{1}{2}(p_A + p_B)}
\quad \Rightarrow \quad \boxed{\mathbb{P}(G_1) = \dfrac{3}{20}}
\]

% ------

\q Probabilité de gagner la deuxième partie :

Étude par cas selon la machine choisie au premier coup et l'issue (gain ou perte) :
\[
\mathbb{P}(G_2) = \dfrac{1}{2} \left( p_A^2 + (1 - p_A)p_B \right) + \dfrac{1}{2} \left( p_B^2 + (1 - p_B)p_A \right)
\]

Application numérique :
\[
\boxed{\mathbb{P}(G_2) = \dfrac{31}{200}}
\]

% ------

\q Probabilité conditionnelle de la machine sachant la victoire :

Par la formule de Bayes :
\[
\mathbb{P}(A_1 \mid G_2) = \frac{\mathbb{P}(A_1\cap G_2)}{\mathbb{P}(G_2)} = \dfrac{\mathbb{P}(G_2 \mid A_1) \mathbb{P}(A_1)}{\mathbb{P}(G_2)}
\]

Avec :
\[
\mathbb{P}(G_2 \mid A_1) = p_A^2 + (1 - p_A)p_B
\quad ; \quad
\mathbb{P}(A_1) = \dfrac{1}{2}
\]
La probabilité que la deuxième partie soit gagnée sachant que la première partie se déroule sur la
machine A se calcule de même avec l'arbre.

Ainsi :
\[
\boxed{\mathbb{P}(A_1 \mid G_2) = \dfrac{p_A^2 + (1 - p_A)p_B}{p_A^2 + (1 - p_A)p_B + p_B^2 + (1 - p_B)p_A}}
\quad \Rightarrow \quad \boxed{\mathbb{P}(A_1 \mid G_2) = \dfrac{12}{31}}
\]

% ------

\q Probabilité de gagner sachant la machine :

Par la formule des probabilités totales conditionnelles :

\[
\mathbb{P}(G_k) = p_A \mathbb{P}(A_k) + p_B (1 - \mathbb{P}(A_k))
= p_B + (p_A - p_B) \mathbb{P}(A_k)
\]

\[
\boxed{\mathbb{P}(G_k) = p_B + (p_A - p_B) \mathbb{P}(A_k)}
\quad \Rightarrow \quad
\boxed{\mathbb{P}(G_k) = \dfrac{1}{10}(1 + \mathbb{P}(A_k))}
\]

% ------

\q Relation de récurrence :

Étude des transitions selon la stratégie :
\[
\mathbb{P}(A_{k+1}) = \mathbb{P}(G_k \mid A_k)\mathbb{P}(A_k) + \mathbb{P}(\overline{G_k} \mid B_k) \mathbb{P}(B_k)
\]

\[
\mathbb{P}(A_{k+1}) = p_A \mathbb{P}(A_k) + (1 - p_B)(1 - \mathbb{P}(A_k))
\]

\[
\boxed{\mathbb{P}(A_{k+1}) = (1 - p_B) + (p_A + p_B - 1) \mathbb{P}(A_k)}
\quad \Rightarrow \quad
\boxed{\mathbb{P}(A_{k+1}) = \dfrac{9}{10} - \dfrac{7}{10} \mathbb{P}(A_k)}
\]

% ------

\q Expression explicite :

La suite \( (A_k) \) est arithmético-géométrique. Elle admet un point fixe \( \ell \) tel que :

\[
\ell = \dfrac{9}{10} - \dfrac{7}{10} \ell \quad \Rightarrow \quad \boxed{\ell = \dfrac{9}{17}}
\]

La raison de la suite homogène est \( r = -\dfrac{7}{10} \), et \( A_1 = \dfrac{1}{2} \). D'où :

\[
\boxed{\mathbb{P}(A_k) = \dfrac{9}{17} - \dfrac{1}{34} \left( -\dfrac{7}{10} \right)^{k - 1}}
\]

Par substitution dans l'expression de \( \mathbb{P}(G_k) \) :

\[
\mathbb{P}(G_k) = \dfrac{1}{10} \left( 1 + \mathbb{P}(A_k) \right)
= \dfrac{1}{10} \left( 1 + \dfrac{9}{17} - \dfrac{1}{34} \left( -\dfrac{7}{10} \right)^{k - 1} \right)
\]

\[
\boxed{\mathbb{P}(G_k) = \dfrac{13}{85} - \dfrac{1}{340} \left( -\dfrac{7}{10} \right)^{k - 1}}
\]

% ------

\q Proportion de parties gagnées :

La proportion de parties gagnées après \( n \) parties est donnée par :

\[
S_n = \sum_{k = 1}^{n} \mathbb{P}(G_k)
= \sum_{k = 1}^{n} \left( \dfrac{13}{85} - \dfrac{1}{340} \left( -\dfrac{7}{10} \right)^{k - 1} \right)
\]

Cette quantité représente la \fbox{proportion de parties gagnées au cours de $n$ tentatives},
puisque $S_n$ est une somme des variables de Bernouilli (la $k$\up{ème} partie étant remportée avec
probabilité $\mathbb{P}(G_k)$).

Le premier terme donne :

\[
\sum_{k = 1}^{n} \dfrac{13}{85} = \dfrac{13}{85} n
\]

La somme géométrique donne :

\[
\sum_{k = 1}^{n} \left( -\dfrac{7}{10} \right)^{k - 1}
= \dfrac{1 - \left( -\dfrac{7}{10} \right)^n}{1 + \dfrac{7}{10}} = \dfrac{10}{17} \left( 1 - \left( -\dfrac{7}{10} \right)^n \right)
\]

D'où :

\[
\boxed{S_n = \dfrac{13}{85} n - \dfrac{1}{578} \left( 1 - \left( -\dfrac{7}{10} \right)^n \right)}
\]

\medskip

En divisant par \( n \), on obtient :

\[
\boxed{\dfrac{S_n}{n} = \dfrac{13}{85} - \dfrac{1}{578 n} \left( 1 - \left( -\dfrac{7}{10} \right)^n \right) \longrightarrow \dfrac{13}{85}}
\quad \text{lorsque } n \to +\infty
\]

Ce résultat représente la \emph{proportion asymptotique de parties gagnées}. Il correspond à la
limite moyenne de gain du joueur à long terme.

\end{document}
% ==================================================================================================
