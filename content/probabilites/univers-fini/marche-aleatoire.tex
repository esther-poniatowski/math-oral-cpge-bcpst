% PROBLÈME : Marche aléatoire unidimensionnelle
% ==================================================================================================
%
% Buts
% ----
% - formule des probabilités totales
% - probabilités conditionnelles
% - raisonnements combinatoires
% - suites définies par récurrence
% - étude de chaînes de Markov finies et infinies
%   ==================================================================================================

\documentclass[10pt,a4paper]{article}

% Set the root path
\providecommand{\rootpath}{../../..}
% Fonts
\usepackage[utf8]{inputenc} % for accents
\usepackage[T1]{fontenc} % for accents
\usepackage[french]{babel} % for french language
\usepackage{helvet} % sans serif font family
\renewcommand*\familydefault{\sfdefault} % sans serif font family

% Mathematics
\usepackage{amsmath,amsfonts,amssymb} % for math symbols
\usepackage{array} % for tabular


\usepackage{parskip} % no indent, space between paragraphs

\usepackage{geometry} % margin
\geometry{
    a4paper,
    left=15mm,
    right=15mm,
    top=20mm,
    bottom=20mm
}

\usepackage{circledsteps} % to draw circles around numbers

\usepackage{fancyhdr} % for headers and footers

\usepackage{enumitem} % for customizing lists
\setlist[enumerate]{itemsep=1em} % space between items only in enumerate environment (not itemize)
\setlist[itemize]{label=--} % set itemize label to em-dash

% Command: \customPageLayout{#1}{#2}{#3}
% --------------------------------------
% Description: Custom page layout with header and footer content.
% Arguments:
% #1: Header and footer content
% #2: Left header content
% #3: Right header content
% Example:
% \customPageLayout{Title}{Lycée Henri IV}{2024}
% Required Packages: fancyhdr
\newcommand{\customPageLayout}[3]{
    \pagestyle{fancy} % set page style to fancy (add header and footer)
    \fancyhf{} % clear all header and footer content
    \lhead{#2} % left header content
    \rhead{#3} % right header content
    \chead{\textbf{#1}} % center header content in bold (if needed)
    \rfoot{\thepage} % page number in the footer
}


% Counter: \q
% -----------
% Description: Display a question number in a circle.
% Usage:
% - Create a new question: add \q followed by the question content.
% - Reset the question counter: add \setcounter{q}{0} before the first question.
\newcounter{q}
\setcounter{q}{0} % set initial value of the counter
\newcommand{\q}{
    \bigskip
    \addtocounter{q}{1}
    \par
    \Circled{\textbf{\theq}} \space
}


% Counter: \ql
% ------------
% Description: Display a question letter in a round box with indentation (lowercase and not bold).
% Usage:
% - Create a new question: add \ql followed by the question content.
% - Reset the question counter: add \setcounter{ql}{0} before the first question.
\newcounter{ql}
\setcounter{ql}{0} % set initial value of the counter
\newcommand{\ql}{
    \addtocounter{ql}{1}
    \par
    \hspace{1.5em} % indentation before the circled letter
    \textcolor{gray}{\Circled{\alph{ql}}} \space % gray color
}


\title{Probabilités dans un univers infini - Marche aléatoire unidimensionnelle}
\author{Esther Poniatowski}
\date{2024-2025}

\customPageLayout{Sujets d'interrogation orale}{Lycée Henri IV}{2024}

% ==================================================================================================
\begin{document}

\textbf{Objectifs}

Modéliser et analyser des marches aléatoires sur des graphes linéaires, avec ou sans états
absorbants.

\bigskip
\textbf{Contexte}

La marche aléatoire unidimensionnelle est un modèle probabiliste discret introduit pour décrire le
mouvement d'une particule dans l'espace. Ce modèle trouve des applications en physique statistique
(diffusion), en algorithmique (random walk), ou encore en théorie des files d'attente.

% --------------------------------------------------------------------------------------------------
\bigskip
\textbf{Marche aléatoire sans état absorbant}

Ce problème considère les déplacements aléatoires d'une particule sur l'axe des entiers relatifs. La
particule se trouve initialement à la position d'abscisse nulle.

À chaque instant, la particule effectue un saut dans une direction aléatoire : elle se déplace soit
d'une unité vers la droite (+1) avec une probabilité \( p \in [0, 1] \), doit d'une unité vers la
gauche (-1).

\q Calculer la probabilité \( p_k \) que la particule se trouve à nouveau à l'origine après \( k \)
déplacements, pour tout entier \( k \geq 0 \).
% But : Étudier le retour à l'origine pour une marche aléatoire simple symétrique. Méthode :
% Décompte combinatoire des chemins de longueur \( k \) ramenant à 0, en utilisant le binôme.

\q Déterminer la probabilité \( p_k(y) \) que la particule se trouve à une distance \( y \) de
l'origine après \( k \) déplacements, pour tout entier \( y \in \mathbb{Z} \).
% But : Généraliser l'étude à une position fixée à l'instant \( k \). Méthode : Comptage des chemins
% menant à \( y \), avec conditions de parité.

\bigskip

Désormais, la particule adopte une dynamique qui l'autorise à rester sur place. A chaque instant, la
particule peut avec équiprobabilité :
\begin{itemize}
 \item se déplacer d'une unité vers la gauche,
 \item se déplacer d'une unité vers la droite,
 \item rester immobile.
\end{itemize}

\q Reprendre la question précédente relative à la probabilité de retour à l'origine.
% But : Adapter la méthode précédente à une marche avec trois options équiprobables. Méthode :
% Introduction d'un coefficient trinomial ou raisonnement itératif.

\q Reprendre la question précédente relative à la probabilité d'être à une distance \( y \) de
l'origine.
% But : Étendre le raisonnement combinatoire à une marche avec immobilité possible. Méthode : Étude
% des configurations permettant d'atteindre une position donnée.

\bigskip

\textbf{Marche aléatoire avec états absorbants}

\medskip

Désormais, la particule se déplace dans un segment borné de \( 0 \) à \( N \), avec \( N \in
\mathbb{N}^* \).

La position initiale peut être n'importe quel entier \( i \) tel que \( 0 \leq i \leq N \).

Le processus s'arrête lorsque la particule atteint l'un des bords \( 0 \) ou \( N \), qui sont
qualifiés d'\emph{états absorbants}.

\medskip

Notations :
\begin{itemize}
 \item \( p \in  ]0, \frac{1}{2}[ \) : Probabilité (asymétrique) de déplacement vers la droite.
 \item \( u_i \) : Probabilité que le processus s'arrête en \( 0 \) lorsque la particule débute à l'abscisse \( i \).
 \item \( v_i \) : Probabilité que le processus s'arrête en \( N \) lorsque la particule débute à l'abscisse \( i \).
\end{itemize}

\q Déterminer les valeurs de \( u_0 \) et \( u_N \).
% But : Identifier les probabilités aux bords absorbants. Méthode : Exploiter les définitions mêmes
% de ces états.

\q Établir la relation \( u_i = p u_{i+1} + q u_{i-1} \) pour tout \( i \in \{1, \dots, N-1\} \),
% But : Mettre en place l'équation de récurrence satisfaite par \( u_i \). Méthode : Définir les
% événements et utiliser la formule des probabilités totales conditionnelles.

\q En déduire l'expression explicite de \( u_i \) en fonction de \( i \), \( p \), \( q \) et \( N
\).
% But : Résoudre l'équation de récurrence précédente. Méthode : Utiliser les techniques classiques
% (homogène + conditions aux bords).

\q Reprendre les questions précédentes pour déterminer les probabilités \( v_i \).
% But : Étendre l'étude aux probabilités d'absorption en \( N \). Méthode : Utiliser la
% complémentarité des événements ou refaire la résolution.

\q Déterminer la valeur de \( u_i + v_i \). En déduire une propriété sur la somme des deux
probabilités.
% But : Vérifier la cohérence du modèle. Méthode : Utiliser l'exhaustivité des issues absorbantes.


\end{document}
% ==================================================================================================
