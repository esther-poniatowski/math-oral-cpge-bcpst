% PROBLÈME : Système de recommandations
% ==================================================================================================
%
% But
% ---
% Utiliser le dénombrement dans le contexte d'un système de recommandation, aboutissant à la
% modélisation de sa diversité et de sa personnalisation.
% ==================================================================================================

\documentclass[10pt,a4paper]{article}

% Set the root path
\providecommand{\rootpath}{../../..}
% Fonts
\usepackage[utf8]{inputenc} % for accents
\usepackage[T1]{fontenc} % for accents
\usepackage[french]{babel} % for french language
\usepackage{helvet} % sans serif font family
\renewcommand*\familydefault{\sfdefault} % sans serif font family

% Mathematics
\usepackage{amsmath,amsfonts,amssymb} % for math symbols
\usepackage{array} % for tabular


\usepackage{parskip} % no indent, space between paragraphs

\usepackage{geometry} % margin
\geometry{
    a4paper,
    left=15mm,
    right=15mm,
    top=20mm,
    bottom=20mm
}

\usepackage{circledsteps} % to draw circles around numbers

\usepackage{fancyhdr} % for headers and footers

\usepackage{enumitem} % for customizing lists
\setlist[enumerate]{itemsep=1em} % space between items only in enumerate environment (not itemize)
\setlist[itemize]{label=--} % set itemize label to em-dash

% Command: \customPageLayout{#1}{#2}{#3}
% --------------------------------------
% Description: Custom page layout with header and footer content.
% Arguments:
% #1: Header and footer content
% #2: Left header content
% #3: Right header content
% Example:
% \customPageLayout{Title}{Lycée Henri IV}{2024}
% Required Packages: fancyhdr
\newcommand{\customPageLayout}[3]{
    \pagestyle{fancy} % set page style to fancy, i.e. header and footer
    \fancyhf{#1} % set header and footer content
    \lhead{#2} % set left header content
    \rhead{#3} % set right header content
    \fancyfoot{} % clear footer content
    \rfoot{\thepage} % set page number in footer
}

% Counter: \q
% -----------
% Description: Display a question number in a circle.
\newcounter{q}
\setcounter{q}{0} % set initial value of counter
\newcommand{\q}{
    \bigskip
    \addtocounter{q}{1}
    \par
    \Circled{\textbf{\theq}} \space
}



\title{Dénombrement - Système de recommandations d'articles}
\author{}
\date{2024}

\customPageLayout{Sujets d'interrogation orale}{Lycée Henri IV}{2024}

\begin{document}

\textbf{Contexte}

Un système de recommandation est un outil informatique qui vise à recommander des éléments
(produits, services, informations, etc.) à des utilisateurs en fonction de leurs préférences.

Pour garantir des recommandations de qualité, un système de recommandation cherche à remplir
plusieurs objectifs :
\begin{itemize}
    \item Personnalisation : Adapter les recommandations aux préférences de chaque utilisateur.
    \item Diversité : Eviter de recommander des éléments trop similaires à un même utilisateur.
\end{itemize}
\bigskip

\textbf{Objectifs}

Employer les outils du dénombrement pour quantifier les possibilités de diversité et de
personnalisation des recommandations.

\bigskip
Ce problème considère un système de recommandation comprenant un ensemble de \( n \) articles et de
\( m \) utilisateurs.


\q \textbf{Sélection personnalisée d'articles}

    Le système recommande un unique article à chaque utilisateur. Combien de configurations
    différentes de recommandations peut-il générer dans ce cas le plus général ? Interpréter le
    résultat en termes d'applications entre deux ensembles.

    Le système recommande exactement \( p \) articles à chaque utilisateur.

   \begin{enumerate}
      \item Si l'ordre des recommandations compte, combien de recommandations distinctes peuvent
      être générées pour un utilisateur donné ?
      \item Si seul l'ensemble des articles recommandés compte (et non leur ordre), donner la formule
      correspondante.
    \end{enumerate}


\q \textbf{Notes et Profils d'utilisateurs}

   Chaque utilisateur note les \( n \) articles sur une échelle de \( 1 \) à \( 5 \).

   Deux utilisateurs sont considérés comme "similaires" s'ils ont attribué exactement la même note à
   au moins \( h \) articles.

    \begin{enumerate}
        \item Calculer le nombre total de configurations de notes possibles pour l'ensemble des \( m \)
        utilisateurs.
        \item Soient deux utilisateurs quelconques. Déterminer le nombre de configurations possibles
        de leurs notes respectives pour qu'ils soient considérés comme similaires.
        \item  Justifier que si \( m > 5^n \), alors il existe nécessairement deux utilisateurs
        ayant exactement les mêmes notes sur les \( k \) articles qu'ils ont notés.
    \end{enumerate}


\q \textbf{Catégories d'articles thématiques}

    Le système organise les \( n \) articles en \( m \) catégories thématiques, de sorte que chaque
    article appartienne à exactement un groupe.

    \begin{enumerate}
        \item Justifier que ce problème revient à compter le nombre de façons de partitionner un
        ensemble de \( n \) éléments en \( m \) sous-ensembles non vides.
        \item On note \( S(n, m) \) ce nombre. Établir sa relation de récurrence :
        \[
        S(n, m) = m S(n-1, m) + S(n-1, m-1)
        \]
        \item Si chaque élément peut être placé librement dans l'un des \( m \) sous-ensembles (y
        compris des sous-ensembles vides), combien y a-t-il de telles répartitions ?
        \item Quel principe permettrait de corriger cette quantité pour ne conserver que les
        répartitions où chaque sous-ensemble est non vide ?
        \item Montrer que le nombre de répartitions où au moins $k$ des \( m \) sous-ensembles est
        vide est donné par :
        \item En déduire que le nombre total de répartitions possibles est :
        \[
        \sum_{k=1}^{m} (-1)^k \binom{m}{k} (m-k)^n
        \]
        \item Expliquer la division par \( m! \) pour obtenir la formule finale :
        \[
        S(n, m) = \frac{1}{m!} \sum_{k=0}^{m} (-1)^k \binom{m}{k} (m-k)^n
        \]


    \end{enumerate}

\end{document}
