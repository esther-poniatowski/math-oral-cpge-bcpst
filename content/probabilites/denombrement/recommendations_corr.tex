% CORRECTION : Système de recommandations
% ==================================================================================================

\documentclass[10pt,a4paper]{article}

% Set the root path
\providecommand{\rootpath}{../../..}
% Fonts
\usepackage[utf8]{inputenc} % for accents
\usepackage[T1]{fontenc} % for accents
\usepackage[french]{babel} % for french language
\usepackage{helvet} % sans serif font family
\renewcommand*\familydefault{\sfdefault} % sans serif font family

% Mathematics
\usepackage{amsmath,amsfonts,amssymb} % for math symbols
\usepackage{array} % for tabular


\usepackage{parskip} % no indent, space between paragraphs

\usepackage{geometry} % margin
\geometry{
    a4paper,
    left=15mm,
    right=15mm,
    top=20mm,
    bottom=20mm
}

\usepackage{circledsteps} % to draw circles around numbers

\usepackage{fancyhdr} % for headers and footers

\usepackage{enumitem} % for customizing lists
\setlist[enumerate]{itemsep=1em} % space between items only in enumerate environment (not itemize)
\setlist[itemize]{label=--} % set itemize label to em-dash

% Command: \customPageLayout{#1}{#2}{#3}
% --------------------------------------
% Description: Custom page layout with header and footer content.
% Arguments:
% #1: Header and footer content
% #2: Left header content
% #3: Right header content
% Example:
% \customPageLayout{Title}{Lycée Henri IV}{2024}
% Required Packages: fancyhdr
\newcommand{\customPageLayout}[3]{
    \pagestyle{fancy} % set page style to fancy, i.e. header and footer
    \fancyhf{#1} % set header and footer content
    \lhead{#2} % set left header content
    \rhead{#3} % set right header content
    \fancyfoot{} % clear footer content
    \rfoot{\thepage} % set page number in footer
}

% Counter: \q
% -----------
% Description: Display a question number in a circle.
\newcounter{q}
\setcounter{q}{0} % set initial value of counter
\newcommand{\q}{
    \bigskip
    \addtocounter{q}{1}
    \par
    \Circled{\textbf{\theq}} \space
}



\title{Dénombrement - Système de recommandations d'articles}
\author{}
\date{2024}

\customPageLayout{Correction}{Lycée Henri IV}{2024}

\begin{document}

% --------------------------------------------------------------------------------------------------

% --------------------------------------------------------------------------------------------------
\q \textbf{Sélection personnalisée d'articles}

    \textbf{Nombre total de recommandations possibles}

    Pour chaque utilisateur, il y a $n$ choix possibles d'articles. Comme il y a $m$ utilisateurs, le
    nombre total de configurations différentes de recommandations est :
    $$ n^m $$

    Ce résultat correspond au nombre de fonctions possibles de l'ensemble des utilisateurs vers
    l'ensemble des articles.

   \begin{enumerate}

      \item  Si l'ordre des recommandations compte, il s'agit d'un arrangement de $p$ articles parmi
      $n$. Nombre de recommandations distinctes pour un utilisateur :
      $$ A_n^p = \frac{n!}{(n-p)!} $$

      \item Si l'ordre ne compte pas, il s'agit d'une combinaison de $p$ articles parmi $n$.
      Nombre de recommandations distinctes pour un utilisateur :
      $$ C_n^p = \binom{n}{p} = \frac{n!}{p!(n-p)!} $$
    \end{enumerate}

% --------------------------------------------------------------------------------------------------
\q \textbf{Notes et Profils d'utilisateurs}
    \begin{enumerate}

        \item Nombre de configurations de notes pour un utilisateur :

        Chaque utilisateur note exactement \( n \) articles, et chaque note peut prendre \( 5 \)
        valeurs, donc le nombre de configurations possibles est :
        $$ 5^n $$

        Nombre total de configurations de notes pour \( m \) utilisateurs :
        $$ (5^n)^m = 5^{n \cdot m} $$

        \item Nombre de configurations possibles de notes pour deux utilisateurs similaires :

        Choix des notes attribuées par les utilisateurs \(i\) et \(j\) pour coincider sur exactement
        \(r\) articles (avec \(0 \le r \le n\)) :

        \begin{itemize}
            \item Choisir d'abord le vecteur de notes de l'utilisateur \(i\) : il y a \(5^n\) façons
            de procéder.
            \item Choisir les \(r\) articles où les deux utilisateurs auront la même note : il y a
            \(\displaystyle \binom{n}{r}\) façons de choisir ces articles.
            \item Sur chacun de ces \(r\) articles, la note de \(j\) est déjà imposée par la note de
            \(i\). Il n'y a donc pas de choix supplémentaire.
            \item Sur les \(n - r\) autres articles, il faut que \(j\) diffère de \(i\), ce qui
            laisse 4 autres valeurs possibles : il y a donc \(4^{\,n-r}\) possibilités pour ces
            \(n-r\) articles.
            \item Le nombre de façons de rendre les notes de \(i\) et \(j\) exactement concordantes
            sur \(r\) articles (et différentes sur les autres) est donc :
            \[
            5^n \;\times\; \binom{n}{r}\,4^{n-r}
            \]
        \end{itemize}

        Coicidence sur au moins \(h\) articles : il suffit de sommer le résultat précédent pour \(r
        = h, h+1, \ldots, n\). Le nombre de manières de choisir les vecteurs de \(i\) et \(j\)
        satisfaisant cette condition est
        \[
        5^n \;\sum_{r=h}^{n} \binom{n}{r}\,4^{\,n-r}
        \]

        \item Si $m > 5^n$, alors par le principe des tiroirs, il existe nécessairement deux
        utilisateurs ayant exactement les mêmes notes sur les $n$ articles. En effet, il y a $m$
        utilisateurs (objects) et $5^n$ configurations de notes possibles (tiroirs), donc il y a
        plus d'utilisateurs que de combinaisons de notes possibles.
        \end{enumerate}

% --------------------------------------------------------------------------------------------------
\q \textbf{Catégories d'articles thématiques}
    \begin{enumerate}

        \item Ce problème revient à compter le nombre de façons de partitionner un ensemble de $n$
        éléments en $m$ sous-ensembles non vides car :
        \begin{itemize}
            \item Chaque article doit être assigné à exactement une catégorie (partition).
            \item Chaque catégorie doit contenir au moins un article (sous-ensembles non vides).
            \item L'ordre des articles dans chaque catégorie n'importe pas.
        \end{itemize}

        \item Pour former une partition de $n$ éléments en $m$ sous-ensembles, considérons les deux
        cas possibles lors de l'ajout du n-ème élément :
        \begin{itemize}
            \item Le n-ème élément est placé dans l'un des $m$ sous-ensembles existants : $ m \times S(n-1,m)$ possibilités.
            \begin{itemize}
                \item Il y a $m$ façons de choisir le sous-ensemble.
                \item Pour chacun de ces choix, les $n-1$ éléments précédents doivent être répartis
                en $m$ sous-ensembles.
            \end{itemize}

            \item Le n-ème élément forme un nouveau sous-ensemble à lui seul : $ S(n-1,m-1)$
            possibilités.
            \begin{itemize}
                \item Dans ce cas, les $n-1$ éléments précédents doivent être répartis en $m-1$
                sous-ensembles.
            \end{itemize}
        \end{itemize}

        La somme de ces deux cas donne la relation de récurrence :
        $$ S(n, m) = m \cdot S(n-1, m) + S(n-1, m-1) $$

        \item Si chaque élément est librement attribué à l'un des \( m \) sous-ensembles, alors chaque élément a
        \( m \) choix, donc il y a :
        \[
        m^n
        \]
        façons d'affecter les \( n \) éléments.

        \item Le principe d'inclusion-exclusion permet de corriger cette quantité pour ne conserver
        que les répartitions où chaque sous-ensemble est non vide :
        \begin{itemize}
            \item On commence par compter toutes les répartitions (\( m^n \)).
            \item On exclut les répartitions où au moins un sous-ensemble est vide.
            \item On réintroduit les répartitions où au moins deux sous-ensembles sont vides.
            \item Et ainsi de suite, jusqu'à inclure les répartitions où tous les sous-ensembles
            sont non vides.
        \end{itemize}

        \item Si on force au moins un sous-ensemble à être vide, il reste \( m-k \) sous-ensembles
        disponibles pour placer les \( n \) éléments. Chaque élément peut être placé librement
        dans l'un de ces \( m-k \) sous-ensembles, donc il y a \( (m-k)^n \) façons de placer les
        éléments.

        On doit compter toutes les façons de choisir \( k \) sous-ensembles à rendre vides : il y a
        \( \binom{m}{k} \) façons de choisir les \( k \) sous-ensembles à supprimer.

        \item En appliquant le principe d'inclusion-exclusion, nous obtenons :
        \[
        \sum_{k=1}^{m} (-1)^k \binom{m}{k} (m-k)^n
        \]
        La somme est sur tous les \( k \) sous-ensembles à rendre vides.

        \item Le raisonnement précédent compte les répartitions où les sous-ensembles sont
        identifiables (c'est-à-dire qu'on distingue quel sous-ensemble est quel groupe).

        Cependant, on cherche à ne compter que le nombre de partitions, sans prendre en compte l'ordre des
        sous-ensembles.

        Comme il y a \( m! \) façons d'ordonnancer \( m \) sous-ensembles, il faut donc diviser par \( m! \)
        pour obtenir le nombre réel de partitions :
        \[
        S(n, m) = \frac{1}{m!} \sum_{k=0}^{m} (-1)^k \binom{m}{k} (m-k)^n
        \]

    \end{enumerate}


\end{document}
