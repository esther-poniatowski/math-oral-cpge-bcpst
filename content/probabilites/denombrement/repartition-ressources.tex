% PROBLÈME : Répartition des ressources
% ==================================================================================================
%
% But
% ---
%
%
% Objectifs spécifiques
% ---------------------
% -
% ==================================================================================================

\documentclass[10pt,a4paper]{article}

% Set the root path
\providecommand{\rootpath}{../../..}
% Fonts
\usepackage[utf8]{inputenc} % for accents
\usepackage[T1]{fontenc} % for accents
\usepackage[french]{babel} % for french language
\usepackage{helvet} % sans serif font family
\renewcommand*\familydefault{\sfdefault} % sans serif font family

% Mathematics
\usepackage{amsmath,amsfonts,amssymb} % for math symbols
\usepackage{array} % for tabular


\usepackage{parskip} % no indent, space between paragraphs

\usepackage{geometry} % margin
\geometry{
    a4paper,
    left=15mm,
    right=15mm,
    top=20mm,
    bottom=20mm
}

\usepackage{circledsteps} % to draw circles around numbers

\usepackage{fancyhdr} % for headers and footers

\usepackage{enumitem} % for customizing lists
\setlist[enumerate]{itemsep=1em} % space between items only in enumerate environment (not itemize)
\setlist[itemize]{label=--} % set itemize label to em-dash

% Command: \customPageLayout{#1}{#2}{#3}
% --------------------------------------
% Description: Custom page layout with header and footer content.
% Arguments:
% #1: Header and footer content
% #2: Left header content
% #3: Right header content
% Example:
% \customPageLayout{Title}{Lycée Henri IV}{2024}
% Required Packages: fancyhdr
\newcommand{\customPageLayout}[3]{
    \pagestyle{fancy} % set page style to fancy, i.e. header and footer
    \fancyhf{#1} % set header and footer content
    \lhead{#2} % set left header content
    \rhead{#3} % set right header content
    \fancyfoot{} % clear footer content
    \rfoot{\thepage} % set page number in footer
}

% Counter: \q
% -----------
% Description: Display a question number in a circle.
\newcounter{q}
\setcounter{q}{0} % set initial value of counter
\newcommand{\q}{
    \bigskip
    \addtocounter{q}{1}
    \par
    \Circled{\textbf{\theq}} \space
}



\title{Dénombrement - Répartition optimale des ressources}
\author{}
\date{2024}

\customPageLayout{Sujets d'interrogation orale}{Lycée Henri IV}{2024}

\begin{document}

\textbf{Contexte}

Une organisation dispose de \textbf{\( n \)} unités de ressources (budget, matériel, temps,
personnel) et doit les répartir entre \textbf{\( k \)} projets stratégiques. Chaque projet nécessite
un minimum d'investissement pour être viable, et certaines priorités imposent des allocations
minimales spécifiques.

\textbf{Objectifs}

Modéliser et analyser un problème d'allocation de ressources limitées en utilisant les outils du
dénombrement. Étudier les effets des contraintes minimales et de plafonnements sur les possibilités
de répartition.

\q \textbf{Modélisation du problème sans contrainte spécifique}

   Justifier que, sans contrainte, la répartition des \( n \) ressources entre \( k \) projets est
   équivalente au dénombrement des solutions de l'équation :
   \[
   x_1 + x_2 + \dots + x_k = n
   \]
   avec \( x_i \geq 1 \) pour tout \( i \).

\q \textbf{Dénombrement des répartitions possibles}

   Démontrer que le nombre total de répartitions possibles dans ce cas est donné par :
   \[
   \binom{n-1}{k-1}
   \]

\q \textbf{Allocation avec priorités}

   Désormais, certains projets nécessitent une allocation minimale. Par exemple, le projet \( P_1 \)
   doit recevoir au moins $m$ unités.

   \begin{enumerate}
       \item Modifier l'équation précédente pour prendre en compte cette contrainte.
       \item Montrer que ce nouveau problème peut se réduire au cas général via un changement de
       variables.
       \item En déduire le nombre de répartitions valides.
       \item Généraliser au cas où plusieurs projets ont des allocations minimales différentes.
   \end{enumerate}

\q \textbf{Contraintes de plafonnement}

   Désormais, chaque projet $P_i$ ne peut pas recevoir plus de $M_i$ ressources.

   \begin{enumerate}
      \item Soit $A_i$ l'ensemble des répartitions où le projet $P_i$ reçoit plus de $M_i$
      ressources. Exprimer le nombre de répartitions valides en utilisant ces ensembles et la
      formule d'inclusion-exclusion.
      \item Calculer $|A_i|$ et $|A_i \cap A_j|$ en fonction de $n$, $k$, $M_i$ et $M_j$.
      \item En déduire une formule explicite pour le nombre de répartitions valides dans le cas où
      $k=3$.
   \end{enumerate}


\q \textbf{Ressources discernables}

Dans cette partie, on suppose que les $n$ ressources allouées à chaque projet sont distinguables.
Chaque unité de ressource est numérotée (ou identifiable).

Ici, l'état final de l'allocation est caractérisé par la partition des $n$ ressources dans les $k$
projets. Toutefois, la différence d'ordre des ressources à l'intérieur d'un même projet ne compte
pas. Autrement dit, si un projet $P_i$ contient les ressources $R2,R5,R10$, toute permutation de
$R2,R5,R10$ au sein de ce projet est considérée comme la même affectation.

\begin{enumerate}
   \item Expliquer pourquoi leur affectation peut être vue comme une permutation des \( n \)
   ressources avec répétitions, où chaque projet \( P_i \) reçoit \( x_i \) ressources.
   \item Exprimer le nombre de telles permutations en fonction de \( x_1, x_2, \dots, x_k \) et $n$.
   \item Démontrer que la somme de toutes les répartitions valides en faisant varier \( x_1, x_2,
   \dots, x_k \) sous la contrainte \( x_1 + x_2 + \dots + x_k = n \) donne exactement :
   \[
   \sum_{x_1 + x_2 + \dots + x_k = n} \frac{n!}{x_1! x_2! \dots x_k!} = k^n.
   \]
   \item Résumer le raisonnement et établir le coefficient multinomial classique :
   \[
   \binom{n}{x_1, x_2, \dots, x_k} = \frac{n!}{x_1! x_2! \dots x_k!}.
   \]
   \item Expliquer en quoi ce résultat généralise le coefficient binomial et pourquoi il est utilisé
   pour compter les répartitions de \( n \) objets entre \( k \) groupes distincts.
   \item Expliquer pourquoi cette somme correspond au fait qu'il existe \( k^n \) façons d'attribuer
   indépendamment chaque ressource à l'un des \( k \) projets.
\end{enumerate}

\end{document}

% ==================================================================================================
\q \textbf{Dénombrement des répartitions avec contraintes de plafonnement par bijection combinatoire}

Précédemment, contrainte $1 \leq x_i \leq M_i, \quad \forall i \in \{1, \dots, k\}$ a été traitée
par inclusion-exclusion, qui repose sur une soustraction exhaustive des répartitions invalides où
une ou plusieurs variables dépassent \( M_i \). Cependant, cette méthode devient lourde pour un
grand nombre de projets.

Une approche alternative consiste à établir une bijection combinatoire directe avec une autre
structure, de sorte à compter uniquement les répartitions admissibles dès le départ. En effet, le
problème peut être reformulé comme la construction d'un mot de longueur \( n \) composé de \( k \)
blocs distincts, chaque bloc représentant une affectation de ressources à un projet et contenant
entre \( 1 \) et \( M_i \) ressources.

Le comptage peut alors être vu comme le nombre de mots de longueur \( n \) formés par \( k \)
sous-séquences, dont chacune est bornée en taille.

Dans cette partie, on suppose que les $n$ ressources allouées à chaque projet doivent être
distinguables.

On note cette quantité par :
\[
T(n, k, M_1, M_2, \dots, M_k)
\]

qui peut être calculée récursivement en comptant les affectations valides :
\[
T(n, k, M_1, M_2, \dots, M_k) = \sum_{x_1=1}^{M_1} \sum_{x_2=1}^{M_2} \dots \sum_{x_k=1}^{M_k} \mathbb{1}(x_1 + x_2 + \dots + x_k = n).
\]

Ce qui donne une formule directe sous forme de coefficient multinomial tronqué :
\[
T(n, k, M_1, \dots, M_k) = \sum_{\substack{x_1 + x_2 + \dots + x_k = n \\ 1 \leq x_i \leq M_i}} \frac{n!}{x_1! x_2! \dots x_k!}.
\]

\begin{enumerate}
   \item Justifier la bijection entre une répartition valide des ressources et un mot construit
   selon les règles énoncées.
   \item Exprimer le nombre de manières de répartir $n$ objets distincts en $k$ groupes de tailles
   respectives \( x_1, x_2, \dots, x_k \) fixées, sans contraintes supérieures.
   \item Justifier pourquoi la quantité $T(n, k, M_1, M_2, \dots, M_k)$ peut être définie
   récursivement en sous-problèmes de taille inférieure. Indication : Considérer les différentes
   possibilités d'allocation pour le premier projet.
   \item Justifier les conditions initiales : $T(0,0)=1$ et $T(n, k, M_1, M_2, \dots, M_k) = 0$.

   \item Pourquoi doit-on sommer sur toutes les combinaisons possibles de $$ x_1, x_2, \dots, x_k $$
   satisfaisant certaines conditions ? Quelles sont ces conditions et comment les exprimer
   mathématiquement ?
\end{enumerate}
