% PROBLÈME : Cryptographie
% ==================================================================================================
%
% But
% ---
%
%
% Objectifs spécifiques
% ---------------------
% -
% ==================================================================================================

\documentclass[10pt,a4paper]{article}

% Set the root path
\providecommand{\rootpath}{../../..}
% Fonts
\usepackage[utf8]{inputenc} % for accents
\usepackage[T1]{fontenc} % for accents
\usepackage[french]{babel} % for french language
\usepackage{helvet} % sans serif font family
\renewcommand*\familydefault{\sfdefault} % sans serif font family

% Mathematics
\usepackage{amsmath,amsfonts,amssymb} % for math symbols
\usepackage{array} % for tabular


\usepackage{parskip} % no indent, space between paragraphs

\usepackage{geometry} % margin
\geometry{
    a4paper,
    left=15mm,
    right=15mm,
    top=20mm,
    bottom=20mm
}

\usepackage{circledsteps} % to draw circles around numbers

\usepackage{fancyhdr} % for headers and footers

\usepackage{enumitem} % for customizing lists
\setlist[enumerate]{itemsep=1em} % space between items only in enumerate environment (not itemize)
\setlist[itemize]{label=--} % set itemize label to em-dash

% Command: \customPageLayout{#1}{#2}{#3}
% --------------------------------------
% Description: Custom page layout with header and footer content.
% Arguments:
% #1: Header and footer content
% #2: Left header content
% #3: Right header content
% Example:
% \customPageLayout{Title}{Lycée Henri IV}{2024}
% Required Packages: fancyhdr
\newcommand{\customPageLayout}[3]{
    \pagestyle{fancy} % set page style to fancy, i.e. header and footer
    \fancyhf{#1} % set header and footer content
    \lhead{#2} % set left header content
    \rhead{#3} % set right header content
    \fancyfoot{} % clear footer content
    \rfoot{\thepage} % set page number in footer
}

% Counter: \q
% -----------
% Description: Display a question number in a circle.
\newcounter{q}
\setcounter{q}{0} % set initial value of counter
\newcommand{\q}{
    \bigskip
    \addtocounter{q}{1}
    \par
    \Circled{\textbf{\theq}} \space
}



\title{Dénombrement - }
\author{}
\date{2024}

\customPageLayout{Sujets d'interrogation orale}{Lycée Henri IV}{2024}

\begin{document}

\textbf{Objectif}

\end{document}


Ce problème explore le dénombrement dans le contexte d'un système de cryptographie, aboutissant à la
modélisation d'un protocole de partage de secret.

1. Un message secret est composé de n caractères, chacun pouvant être une lettre majuscule (26
possibilités) ou un chiffre (10 possibilités). Calculer le nombre de messages possibles.

2. Pour renforcer la sécurité, on décide que deux caractères consécutifs doivent être différents.
Combien de messages valides peut-on former?

3. On souhaite partager le secret entre m personnes de telle sorte que k personnes soient
nécessaires pour le reconstituer. Chaque personne reçoit un fragment de n/m caractères. Calculer le
nombre de façons de distribuer ces fragments.

4. Démontrer que le nombre de sous-ensembles de k personnes parmi m est égal à $\binom{m}{k}$.

5. On utilise un polynôme de degré k-1 pour encoder le secret. Combien de tels polynômes existent
dans un corps fini à q éléments?

6. Calculer le nombre de points distincts qu'on peut générer à partir de ce polynôme pour distribuer
aux m participants.

7. Démontrer que n'importe quel sous-ensemble de k points permet de reconstituer le polynôme de
manière unique.

8. Calculer la probabilité qu'un groupe de k-1 personnes puisse deviner le secret par hasard.

9. On souhaite ajouter de la redondance en distribuant r fragments supplémentaires. Combien de
distributions différentes peut-on réaliser?

10. Utiliser le lemme des tiroirs pour démontrer qu'il existe nécessairement deux participants ayant
au moins un caractère en commun dans leurs fragments si m > 36^(n/m).


### Problème 3 : Cryptographie et génération sécurisée de codes
#### Objectif final : Déterminer le nombre de façons de générer des codes sécurisés respectant des règles de sécurité avancées.

#### Contexte :
Un système de sécurité utilise des codes à usage unique composés de \( n \) caractères pris parmi 36 symboles (26 lettres et 10 chiffres). Pour des raisons de sécurité, les codes doivent respecter plusieurs contraintes :
- Contenir au moins une lettre et un chiffre.
- Ne pas contenir deux fois le même caractère consécutivement.
- Éviter certaines combinaisons interdites (par exemple, des motifs connus "1234", "AAAA", etc.).

#### Questions :
1. Déterminer le nombre total de codes possibles sans contrainte.
2. Introduire la contrainte "au moins une lettre et un chiffre" et montrer que l'on peut utiliser la formule du crible.
3. Calculer combien de codes contiennent uniquement des chiffres ou uniquement des lettres.
4. Montrer que le nombre de codes valides est donné par une inclusion-exclusion.
5. Introduire la contrainte "pas de répétition consécutive". Montrer que le dénombrement devient un problème de comptage de chemins dans un graphe d'états.
6. Décrire ce graphe et démontrer que le nombre de codes valides suit une relation de récurrence.
7. Ajouter la contrainte des motifs interdits et discuter de son effet sur la combinatoire.
8. Pour un ensemble \( M \) de motifs interdits, formuler une estimation du nombre de codes valides en fonction de la taille de \( M \).
9. Discuter une application concrète en cryptographie.
10. Étendre le raisonnement à la génération de mots de passe robustes en tenant compte de critères modernes de cybersécurité.
