% PROBLÈME : Arbres binaires
% ==================================================================================================
%
% But
% ---
%
% ==================================================================================================

\documentclass[10pt,a4paper]{article}

% Set the root path
\providecommand{\rootpath}{../../..}
% Fonts
\usepackage[utf8]{inputenc} % for accents
\usepackage[T1]{fontenc} % for accents
\usepackage[french]{babel} % for french language
\usepackage{helvet} % sans serif font family
\renewcommand*\familydefault{\sfdefault} % sans serif font family

% Mathematics
\usepackage{amsmath,amsfonts,amssymb} % for math symbols
\usepackage{array} % for tabular


\usepackage{parskip} % no indent, space between paragraphs

\usepackage{geometry} % margin
\geometry{
    a4paper,
    left=15mm,
    right=15mm,
    top=20mm,
    bottom=20mm
}

\usepackage{circledsteps} % to draw circles around numbers

\usepackage{fancyhdr} % for headers and footers

\usepackage{enumitem} % for customizing lists
\setlist[enumerate]{itemsep=1em} % space between items only in enumerate environment (not itemize)
\setlist[itemize]{label=--} % set itemize label to em-dash

% Command: \customPageLayout{#1}{#2}{#3}
% --------------------------------------
% Description: Custom page layout with header and footer content.
% Arguments:
% #1: Header and footer content
% #2: Left header content
% #3: Right header content
% Example:
% \customPageLayout{Title}{Lycée Henri IV}{2024}
% Required Packages: fancyhdr
\newcommand{\customPageLayout}[3]{
    \pagestyle{fancy} % set page style to fancy (add header and footer)
    \fancyhf{} % clear all header and footer content
    \lhead{#2} % left header content
    \rhead{#3} % right header content
    \chead{\textbf{#1}} % center header content in bold (if needed)
    \rfoot{\thepage} % page number in the footer
}


% Counter: \q
% -----------
% Description: Display a question number in a circle.
% Usage:
% - Create a new question: add \q followed by the question content.
% - Reset the question counter: add \setcounter{q}{0} before the first question.
\newcounter{q}
\setcounter{q}{0} % set initial value of the counter
\newcommand{\q}{
    \bigskip
    \addtocounter{q}{1}
    \par
    \Circled{\textbf{\theq}} \space
}


% Counter: \ql
% ------------
% Description: Display a question letter in a round box with indentation (lowercase and not bold).
% Usage:
% - Create a new question: add \ql followed by the question content.
% - Reset the question counter: add \setcounter{ql}{0} before the first question.
\newcounter{ql}
\setcounter{ql}{0} % set initial value of the counter
\newcommand{\ql}{
    \addtocounter{ql}{1}
    \par
    \hspace{1.5em} % indentation before the circled letter
    \textcolor{gray}{\Circled{\alph{ql}}} \space % gray color
}


\usepackage{tikz}
\usepackage{subcaption} % for subfigures
\usetikzlibrary{trees} % for tree layout
\usetikzlibrary{decorations.pathreplacing} % for curly braces


\title{Dénombrement - Arbres Binaires}
\author{Esther Poniatowski}
\date{2024-2025}

\customPageLayout{Sujets d'interrogation orale}{Lycée Henri IV}{2024}

\begin{document}

\textbf{Contexte}

Un \textbf{arbre binaire} est une structure récursive utilisée en informatique et en combinatoire
pour représenter des problèmes variés.

\textbf{Propriétés fondamentales :}
\begin{itemize}
    \item Chaque noeud a  \textit{au plus deux} enfants (enfant gauche et enfant droit).
    \item Un noeud sans enfant est appelé une \textbf{feuille}.
    \item Un noeud avec au moins un enfant est appelé un \textbf{noeud interne}.
    \item Le noeud au sommet de l'arbre est appelé la \textbf{racine}.
    \item La \textbf{hauteur} d'un arbre est la longueur du plus long chemin de la racine à une
    feuille.
\end{itemize}

%\input{figures/arbres_exemples.tex}

\bigskip
\textbf{Objectifs}

Employer les  outils de dénombrement pour compter les structures d'arbres binaires en utilisant
des relations de récurrence et des bijections combinatoires.

\q \textbf{Nombre de noeuds}

Le nombre de noeuds internes et de feuilles dans un arbre binaire suit des relations simples dans
certains types d'arbres particuliers :
\begin{itemize}
     \item Un \textbf{arbre binaire plein} est un arbre où chaque noeud interne a
    \textit{exactement} deux enfants.
    \item Un \textbf{arbre binaire parfait} est un arbre où tous les niveaux sont complètement
    remplis.
\end{itemize}
On note \( N \) le nombre total de noeuds, \( I \) le nombre de noeuds internes et \( F \) le nombre
de feuilles.

\begin{enumerate}
    \item En toute généralité, donner une relation entre le nombre total de noeuds, le nombre de
    noeuds internes et le nombre de feuilles dans un arbre binaire.
    \item Dans un arbre binaire \textit{plein}, montrer que le nombre de feuilles est lié au nombre
    de noeuds internes par la relation \( F = I + 1 \). Justifier par récurrence sur le nombre total
    de noeuds.
    \item En déduire le nombre de feuilles et le nombre de noeuds internes en fonction du nombre de
    noeuds total.
    \item Dans un arbre binaire \textit{parfait}, déterminer le nombre total de noeuds \( N \) en
    fonction de la hauteur \( h \) de l'arbre.
\end{enumerate}

\q \textbf{Dénombrement des structures d'arbres binaires}

La \textbf{structure} d'un arbre binaire est définie par la manière dont les noeuds sont disposés et
connectés entre eux, sans tenir compte des valeurs spécifiques contenues dans ces noeuds. Elle
représente la "forme" de l'arbre.

\begin{enumerate}
    \item Proposer une méthode récursive pour générer systématiquement toutes les structures
    d'arbres binaires ayant \( n \) feuilles, en partant de la racine.
    Pour cela, considérer une partition de l'arbre en deux sous-arbres gauche et droit.
    \item Illustrer cette méthode en construisant tous les arbres binaires ayant 4 feuilles.
    \item Par analogie avec cette méthode, établir une relation de récurrence reliant le
    nombre d'arbres binaires ayant \( n \) feuilles à ceux ayant \( k \) feuilles, pour \( k < n \).
\end{enumerate}

\q \textbf{Chemins de Dyck}

Les arbres binaires peuvent être mis en correspondance avec d'autres objets combinatoires, tels que
les \textbf{chemins de Dyck}.

On considère une grille cartésienne de points de taille $2n \times 2n$. Un chemin de Dyck est une
suite de pas vers le Nord-Est $(1,1)$ et de pas vers le Sud-Est $(1,-1)$ qui relie le point $(0,0)$
au point $(2n,0)$ sans jamais en dessous de l'axe horizontal.


\begin{tikzpicture}[scale=0.5]

    % Dessin des chemins de Dyck pour n = 3, alignés horizontalement

    % Chemin 1
    \begin{scope}[shift={(0,0)}]
        \draw[->,thick] (-0.5,0) -- (6.5,0); % Axe horizontal
        \draw[->,thick] (0,-0.5) -- (0,3.5); % Axe vertical
        \draw[thick, blue] (0,0) -- (1,1) -- (2,2) -- (3,1) -- (4,2) -- (5,1) -- (6,0);
    \end{scope}

    % Chemin 2
    \begin{scope}[shift={(8,0)}]
        \draw[->,thick] (-0.5,0) -- (6.5,0); % Axe horizontal
        \draw[->,thick] (0,-0.5) -- (0,3.5); % Axe vertical
        \draw[thick, red] (0,0) -- (1,1) -- (2,2) -- (3,1) -- (4,0) -- (5,1) -- (6,0);
    \end{scope}

    % Chemin 3
    \begin{scope}[shift={(16,0)}]
        \draw[->,thick] (-0.5,0) -- (6.5,0); % Axe horizontal
        \draw[->,thick] (0,-0.5) -- (0,3.5); % Axe vertical
        \draw[thick, green] (0,0) -- (1,1) -- (2,0) -- (3,1) -- (4,0) -- (5,1) -- (6,0);
    \end{scope}

    % Chemin 4
    \begin{scope}[shift={(24,0)}]
        \draw[->,thick] (-0.5,0) -- (6.5,0); % Axe horizontal
        \draw[->,thick] (0,-0.5) -- (0,3.5); % Axe vertical
        \draw[thick, orange] (0,0) -- (1,1) -- (2,0) -- (3,1) -- (4,2) -- (5,1) -- (6,0);
    \end{scope}


\end{tikzpicture}


\begin{enumerate}
    \item Déterminer le nombre total de chemins reliant \( (0,0) \) à \( (2n,0) \).
    \item Montrer qu'un chemin qui passe sous l'axe peut être mis en correspondance bijective avec
    un chemin allant de \( (0,0) \) à \( (2n,-2) \). Pour cela, considérer le premier point auquel
    le chemin passe sous l'axe horizontal, et considérer la réflexion de la portion restante par
    rapport à l'axe $y = -1$.
    \item Déterminer le nombre de ces chemins qui passent sous l'axe horizontal.
    \item En déduire une expression pour le nombre de chemins de Dyck de longueur \( 2n \).
    \item Par un raisonnement combinatoire, prouver l'identité :
    $$ \forall n \in \mathbb{N}, (n+1)\binom{2n}{n-1} = n\binom{2n}{n} $$
    Indication : Considérer deux méthodes équivalentes pour former une équipe de \( n + 1 \)
    membres parmi \( 2n \) ainsi que le capitaine de l'équipe.
    \item En déduire que le nombre de chemins de Dyck se réécrit :
    $$C_n = \frac{1}{n+1}\binom{2n}{n}$$
    \item Établir une correspondance bijective entre les chemins de Dyck de longueur \( 2n \) et les
    arbres binaires pleins ayant \( n \) noeuds internes.
    Pour cela, considérer le parcours en profondeur d'un arbre binaire avec deux types d'événements :
    \begin{itemize}
        \item Descente dans l'arbre (ajout d'un noeud interne) : représentée par un pas montant \(
        (+1,+1) \).
        \item Remontée après avoir traité un sous-arbre : représentée par un pas descendant \(
        (+1,-1) \).
    \end{itemize}
    \item En déduire une formule explicite pour le nombre d'arbres binaires à \( n \) noeuds
    internes.
\end{enumerate}


\end{document}

% ==================================================================================================
\q \textbf{Parenthésage des expressions mathématiques}

Les arbres binaires peuvent être utilisés pour représenter des expressions mathématiques impliquant
des parenthésages.

Un parenthésage est une syntaxe qui permet de regrouper les opérations dans une expression
mathématique. Par exemple, l'expression $a + b \times c$ peut être parenthésée de deux manières
différentes : $(a + b) \times c$ ou $a + (b \times c)$.

\begin{enumerate}
    \item Démontrer que le nombre de façons de parenthéser une expression de $n+1$ termes est égal au
    $n$-ième nombre de Catalan.
\end{enumerate}

\begin{enumerate}
    \item Montrer comment un arbre binaire peut être utilisé pour représenter une expression
    parenthésée.
    \item Démontrer que le nombre de façons de parenthéser une expression de \( n+1 \) termes est
    donné par le \( n \)-ième nombre de Catalan.
\end{enumerate}
