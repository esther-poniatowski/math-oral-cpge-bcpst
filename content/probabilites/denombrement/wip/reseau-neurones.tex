% PROBLÈME : Réseau de neurones
% ==================================================================================================
%
% But
% ---
%
%
% Objectifs spécifiques
% ---------------------
% -
% ==================================================================================================

\documentclass[10pt,a4paper]{article}

% Set the root path
\providecommand{\rootpath}{../../..}
% Fonts
\usepackage[utf8]{inputenc} % for accents
\usepackage[T1]{fontenc} % for accents
\usepackage[french]{babel} % for french language
\usepackage{helvet} % sans serif font family
\renewcommand*\familydefault{\sfdefault} % sans serif font family

% Mathematics
\usepackage{amsmath,amsfonts,amssymb} % for math symbols
\usepackage{array} % for tabular


\usepackage{parskip} % no indent, space between paragraphs

\usepackage{geometry} % margin
\geometry{
    a4paper,
    left=15mm,
    right=15mm,
    top=20mm,
    bottom=20mm
}

\usepackage{circledsteps} % to draw circles around numbers

\usepackage{fancyhdr} % for headers and footers

\usepackage{enumitem} % for customizing lists
\setlist[enumerate]{itemsep=1em} % space between items only in enumerate environment (not itemize)
\setlist[itemize]{label=--} % set itemize label to em-dash

% Command: \customPageLayout{#1}{#2}{#3}
% --------------------------------------
% Description: Custom page layout with header and footer content.
% Arguments:
% #1: Header and footer content
% #2: Left header content
% #3: Right header content
% Example:
% \customPageLayout{Title}{Lycée Henri IV}{2024}
% Required Packages: fancyhdr
\newcommand{\customPageLayout}[3]{
    \pagestyle{fancy} % set page style to fancy (add header and footer)
    \fancyhf{} % clear all header and footer content
    \lhead{#2} % left header content
    \rhead{#3} % right header content
    \chead{\textbf{#1}} % center header content in bold (if needed)
    \rfoot{\thepage} % page number in the footer
}


% Counter: \q
% -----------
% Description: Display a question number in a circle.
% Usage:
% - Create a new question: add \q followed by the question content.
% - Reset the question counter: add \setcounter{q}{0} before the first question.
\newcounter{q}
\setcounter{q}{0} % set initial value of the counter
\newcommand{\q}{
    \bigskip
    \addtocounter{q}{1}
    \par
    \Circled{\textbf{\theq}} \space
}


% Counter: \ql
% ------------
% Description: Display a question letter in a round box with indentation (lowercase and not bold).
% Usage:
% - Create a new question: add \ql followed by the question content.
% - Reset the question counter: add \setcounter{ql}{0} before the first question.
\newcounter{ql}
\setcounter{ql}{0} % set initial value of the counter
\newcommand{\ql}{
    \addtocounter{ql}{1}
    \par
    \hspace{1.5em} % indentation before the circled letter
    \textcolor{gray}{\Circled{\alph{ql}}} \space % gray color
}


\title{Dénombrement - Réseau de Neurones Artificiel}
\author{}
\date{2024}

\customPageLayout{Sujets d'interrogation orale}{Lycée Henri IV}{2024}

\begin{document}

\textbf{Objectif}

\end{document}


Ce problème explore le dénombrement dans le contexte d'un réseau de neurones artificiels,
aboutissant à la modélisation de sa capacité d'apprentissage.

1. Un réseau de neurones est composé de n neurones d'entrée, m neurones cachés et p neurones de
sortie. Combien de connexions possibles y a-t-il entre les couches?

2. Chaque connexion peut avoir un poids entier entre -k et k. Calculer le nombre de configurations
possibles pour les poids du réseau.

3. On souhaite que la somme des poids entrants pour chaque neurone soit comprise entre -s et s.
Combien de configurations valides existe-t-il?

4. Démontrer que le nombre de façons de répartir s unités de poids entre n connexions est égal à
$\binom{s+n-1}{n-1}$.

5. On considère maintenant que les poids sont des nombres réels avec une précision de d décimales.
Combien de configurations distinctes peut-on avoir?

6. Le réseau doit apprendre à partir d'un ensemble de t exemples d'entraînement. Calculer le nombre
de façons de choisir ces exemples parmi un ensemble de N données.

7. On souhaite que le réseau puisse classifier correctement au moins r exemples parmi les t.
Utiliser la formule du crible pour calculer le nombre de configurations satisfaisantes.

8. Démontrer que si le nombre de configurations possibles est supérieur à 2^t, alors il existe
toujours une configuration qui classe correctement les t exemples.

9. Calculer la probabilité qu'une configuration choisie au hasard classe correctement tous les
exemples.

10. Utiliser le lemme des tiroirs pour démontrer qu'il existe nécessairement deux configurations
donnant les mêmes sorties sur l'ensemble d'entraînement si le nombre de configurations est supérieur
à p^t.
