% CORRECTION : Tournoi
% ==================================================================================================

\documentclass[10pt,a4paper]{article}

% Set the root path
\providecommand{\rootpath}{../../..}
% Fonts
\usepackage[utf8]{inputenc} % for accents
\usepackage[T1]{fontenc} % for accents
\usepackage[french]{babel} % for french language
\usepackage{helvet} % sans serif font family
\renewcommand*\familydefault{\sfdefault} % sans serif font family

% Mathematics
\usepackage{amsmath,amsfonts,amssymb} % for math symbols
\usepackage{array} % for tabular


\usepackage{parskip} % no indent, space between paragraphs

\usepackage{geometry} % margin
\geometry{
    a4paper,
    left=15mm,
    right=15mm,
    top=20mm,
    bottom=20mm
}

\usepackage{circledsteps} % to draw circles around numbers

\usepackage{fancyhdr} % for headers and footers

\usepackage{enumitem} % for customizing lists
\setlist[enumerate]{itemsep=1em} % space between items only in enumerate environment (not itemize)
\setlist[itemize]{label=--} % set itemize label to em-dash

% Command: \customPageLayout{#1}{#2}{#3}
% --------------------------------------
% Description: Custom page layout with header and footer content.
% Arguments:
% #1: Header and footer content
% #2: Left header content
% #3: Right header content
% Example:
% \customPageLayout{Title}{Lycée Henri IV}{2024}
% Required Packages: fancyhdr
\newcommand{\customPageLayout}[3]{
    \pagestyle{fancy} % set page style to fancy, i.e. header and footer
    \fancyhf{#1} % set header and footer content
    \lhead{#2} % set left header content
    \rhead{#3} % set right header content
    \fancyfoot{} % clear footer content
    \rfoot{\thepage} % set page number in footer
}

% Counter: \q
% -----------
% Description: Display a question number in a circle.
\newcounter{q}
\setcounter{q}{0} % set initial value of counter
\newcommand{\q}{
    \bigskip
    \addtocounter{q}{1}
    \par
    \Circled{\textbf{\theq}} \space
}


\usepackage{tikz}
\usepackage{subcaption} % for subfigures
\usetikzlibrary{trees} % for tree layout
\usetikzlibrary{positioning} % for relative positioning


\title{Dénombrement - Tournoi avec contraintes}
\author{}
\date{2024}

\customPageLayout{Correction}{Lycée Henri IV}{2024}

\begin{document}

% --------------------------------------------------------------------------------------------------
\q \textbf{Nombre de parties}
Un tournoi à élimination directe avec $n$ joueurs nécessite exactement $n-1$ parties.

Démonstration par récurrence :
\begin{itemize}
     \item Initialisation : Pour $n = 2$, il faut 1 partie. La propriété est vraie.
     \item Hérédité : Supposons la propriété vraie pour $n$ joueurs et considérons un tournoi à
     $n+1$ joueurs.
     \begin{itemize}
          \item Le premier tour élimine 1 joueur, laissant $n$ joueurs.
          \item Par hypothèse de récurrence, il faut $n-1$ parties pour ces $n$ joueurs.
          \item Donc au total : $1 + (n-1) = n$ parties pour $n+1$ joueurs.
     \end{itemize}
     \item Conclusion : La propriété est vraie pour tout $n \geq 2$.
\end{itemize}

Raisonnement combinatoire :

Chaque partie élimine un joueur. Pour passer de $n$ joueurs à 1 vainqueur, il faut éliminer $n-1$
joueurs, donc $n-1$ parties.

% --------------------------------------------------------------------------------------------------

\q \textbf{Modélisation par un arbre binaire}
\begin{enumerate}

    \item Construction d'un arbre binaire :


    \item Bijection par le lemme des bergers :
    \begin{itemize}
        \item Un ordre de parties peut être décrit comme une séquence d'élimination de \( n-1 \)
        joueurs.
        \item Réciproquement, un arbre binaire définit un unique ordre de parties en le parcourant
        en profondeur.
    \end{itemize}
\end{enumerate}

Pour construire méthodiquement tous les arbres binaires complets à \( n \) feuilles :

\begin{itemize}
     \item A partir de la racine, partitionner l'arbre en deux sous-arbres : \( k \) feuilles pour
     le sous-arbre gauche, \( n - k \) feuilles pour celui de droite, où \( k \) varie de 1 à
     \( n - 1 \).
     \item Pour chaque partition \( (k, n - k) \), créer un nouveau noeud racine et générer
     récursivement tous les couples possibles de sous-arbres gauches et droits.
\end{itemize}

Pour \( n = 4 \) feuilles, les partitions possibles sont  \( (1, 3) \), \( (2, 2) \), \( (3, 1) \).
\begin{itemize}
     \item Pour \( k = 1 \) et \( n - k = 3 \) : Le sous-arbre gauche est une feuille unique. Le
  sous-arbre droit est un arbre binaire complet avec 3 feuilles, qui peut être construit en
  appliquant récursivement la même méthode.
  \item  Pour \( k = 2 \) et \( n - k = 2 \) : Les deux
  sous-arbres sont des arbres binaires complets avec 2 feuilles chacun, qui sont des arbres simples
  avec une racine et deux feuilles.
  \item Pour \( k = 3 \) et \( n - k = 1 \) : Le sous-arbre gauche
  est un arbre binaire complet avec 3 feuilles. Le sous-arbre droit est une feuille unique.
\end{itemize}



\begin{tikzpicture}[scale=0.7]
  % Arbre 1 : Partition (1, 3)
  \node {F}
    child { node {J 1} }
    child { node {1/2}
      child { node {J 2} }
      child { node {1/4}
        child { node {J 3} }
        child { node {J 4} }
      }
    };
  \node[below=1cm of current bounding box.south] {(1, 3)};
\end{tikzpicture}
\hspace{1cm}
\begin{tikzpicture}[scale=0.7]
  % Arbre 2 : Partition (1, 3)
  \node {F}
    child { node {J 1} }
    child { node {1/2}
      child { node {1/4}
        child { node {J 2} }
        child { node {J 3} }
      }
      child { node {J 4} }
    };
  \node[below=1cm of current bounding box.south] {(1, 3)};
\end{tikzpicture}
\hspace{1cm}
\begin{tikzpicture}[scale=0.7]
  % Arbre 3 : Partition (2, 2)
  \node {F}
    child { node {1/2}
      child { node {J 1} }
      child { node {J 2} }
    }
    child { node {1/2}
      child { node {J 3} }
      child { node {J 4} }
    };
  \node[below=1cm of current bounding box.south] {(2, 2)};
\end{tikzpicture}
\hspace{1cm}
\begin{tikzpicture}[scale=0.7]
  % Arbre 4 : Partition (3, 1)
  \node {F}
    child { node {1/2}
      child { node {J 1} }
      child { node {1/4}
        child { node {J 2} }
        child { node {J 3} }
      }
    }
    child { node {J 4} };
  \node[below=1cm of current bounding box.south] {(3, 1)};
\end{tikzpicture}
\hspace{1cm}
\begin{tikzpicture}[scale=0.7]
  % Arbre 5 : Partition (3, 1)
  \node {F}
    child { node {1/2}
      child { node {1/4}
        child { node {J 1} }
        child { node {J 2} }
      }
      child { node {J 3} }
    }
    child { node {J 4} };
  \node[below=1cm of current bounding box.south] {(3, 1)};
\end{tikzpicture}


% --------------------------------------------------------------------------------------------------
\q \textbf{Organisation des parties sans contrainte}

\begin{enumerate}
    \item Dans un arbre de tournoi pré-rempli, le nombre de façons d'affecter les $n$ joueurs est
    $n!$. Il s'agit simplement de permuter $n$ éléments dans $n$ positions fixes.
    Dans un arbre pré-rempli, les structures des matchs sont fixées.
    Il ne reste qu'à assigner les \( n \) joueurs aux \( n \) feuilles de l'arbre.
    Il y a \( n! \) façons de les placer (permutation des \( n \) joueurs).

    \item Décompte des structures possibles, sans tenir compte de l'identité des joueurs :
    $$C_{n-1} = \frac{1}{n} \binom{2(n-1)}{n-1}$$
    Justification combinatoire :

    \item Placement des joueurs dans un arbre donné : Une fois que la structure d'arbre est fixée,
    il reste \( n! \) façons d'affecter les joueurs aux feuilles.

    Nombre total de configurations dans un arbre donné :
    \[
    C_{n-1} \times n!
    \]

    \item Comparaison
     \begin{itemize}
          \item En tenant compte uniquement de la structure : $C_{n-1}$ configurations.
          \item En tenant compte du placement des joueurs : $n! \times C_{n-1}$ configurations.
     \end{itemize}
     Le facteur $n!$ représente toutes les façons possibles d'étiqueter un arbre donné avec $n$ joueurs
     distincts. Cette comparaison illustre que pour chaque structure de tournoi possible, il y a n!
     façons différentes de placer les joueurs dans cette structure.

\end{enumerate}

% --------------------------------------------------------------------------------------------------
\q \textbf{Séparation des fédérations}

\begin{enumerate}

        \item Explication : Cette contrainte impose une structure particulière à l'arbre binaire car
        les joueurs d'une même fédération doivent être dans des sous-arbres distincts jusqu'à une
        certaine profondeur.

        \item Nombre de façons de répartir les joueurs de manière équilibrée :
        $$\prod_{i=1}^k \binom{n/k}{p_i}$$
        où $n/k$ est le nombre de positions disponibles pour chaque fédération.
        % TODO: Expliquer le raisonnement

        \item Nombre d'arbres de tournoi valides :
        $$\left(\prod_{i=1}^k p_i!\right) \cdot \left(\prod_{i=1}^k \binom{n/k}{p_i}\right)$$ Le
        premier produit représente les permutations au sein de chaque fédération, le second la
        répartition entre les fédérations.

        \item Comparaison avec la formule du crible :
          \begin{itemize}
               \item Sans contrainte : $n!$
               \item Avec contrainte :
               $\left(\prod_{i=1}^k p_i!\right) \cdot \left(\prod_{i=1}^k \binom{n/k}{p_i}\right)$
          \end{itemize}
        La différence représente le nombre de configurations interdites.

        \item Comparaison : L'espace des possibilités est plus réduit avec la contrainte car :
        $$\left(\prod_{i=1}^k p_i!\right) \cdot \left(\prod_{i=1}^k \binom{n/k}{p_i}\right) < n!$$
        Cette inégalité est due aux restrictions imposées sur le placement des joueurs.

\end{enumerate}

% --------------------------------------------------------------------------------------------------
\q \textbf{Qualification automatique}
\begin{enumerate}

        \item Nombre de façons d'organiser le premier tour : $$(n-r-1)!$$
        Car il reste $n-r$ joueurs à organiser en un tournoi à élimination directe.

        \item Généralisation : Avec $r_i$ joueurs qualifiés pour le i-ème tour, le nombre total de parties est :
        $$n - 1 - \sum_{i=1}^{\log_2 n} r_i$$
        Car chaque joueur qualifié réduit d'un le nombre de parties nécessaires.


        \item Nombre de façons d'organiser un tel tournoi :
        $$\prod_{i=1}^{\log_2 n} (n_i-1)!$$
          où $n_i$ est le nombre de joueurs au i-ème tour.
          Preuve : À chaque tour, on a un mini-tournoi à organiser, dont le nombre de façons est
          $(n_i-1)!$.

\end{enumerate}

% --------------------------------------------------------------------------------------------------
\q \textbf{Têtes de séries}
\begin{enumerate}

        \item Nombre d'arbres valides avec $s$ têtes de série fixées est :
        $$(n-s)!$$
        Car il reste $n-s$ joueurs à placer librement.

        \item Comparaison avec le principe d'inclusion-exclusion :
          \begin{itemize}
               \item Sans têtes de série fixées : $n!$
               \item Avec têtes de série fixées : $(n-s)!$
          \end{itemize}
        La différence représente le nombre de configurations où au moins une tête de série n'est pas
        à sa place prédéterminée.


        \item Limite du rapport :
        $$\lim_{n \to \infty} \frac{(n-s)!}{n!} =
        \lim_{n \to \infty} \frac{1}{n(n-1)...(n-s+1)} =
        e^{-s}$$

        Cette limite s'obtient en utilisant la définition de $e$ comme limite de $(1+1/n)^n$ quand
        $n$ tend vers l'infini.

\end{enumerate}

\end{document}
