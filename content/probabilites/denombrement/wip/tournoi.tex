% PROBLÈME : Tournoi
% ==================================================================================================
%
% But
% ---
%
% ==================================================================================================

\documentclass[10pt,a4paper]{article}

% Set the root path
\providecommand{\rootpath}{../../..}
% Fonts
\usepackage[utf8]{inputenc} % for accents
\usepackage[T1]{fontenc} % for accents
\usepackage[french]{babel} % for french language
\usepackage{helvet} % sans serif font family
\renewcommand*\familydefault{\sfdefault} % sans serif font family

% Mathematics
\usepackage{amsmath,amsfonts,amssymb} % for math symbols
\usepackage{array} % for tabular


\usepackage{parskip} % no indent, space between paragraphs

\usepackage{geometry} % margin
\geometry{
    a4paper,
    left=15mm,
    right=15mm,
    top=20mm,
    bottom=20mm
}

\usepackage{circledsteps} % to draw circles around numbers

\usepackage{fancyhdr} % for headers and footers

\usepackage{enumitem} % for customizing lists
\setlist[enumerate]{itemsep=1em} % space between items only in enumerate environment (not itemize)
\setlist[itemize]{label=--} % set itemize label to em-dash

% Command: \customPageLayout{#1}{#2}{#3}
% --------------------------------------
% Description: Custom page layout with header and footer content.
% Arguments:
% #1: Header and footer content
% #2: Left header content
% #3: Right header content
% Example:
% \customPageLayout{Title}{Lycée Henri IV}{2024}
% Required Packages: fancyhdr
\newcommand{\customPageLayout}[3]{
    \pagestyle{fancy} % set page style to fancy, i.e. header and footer
    \fancyhf{#1} % set header and footer content
    \lhead{#2} % set left header content
    \rhead{#3} % set right header content
    \fancyfoot{} % clear footer content
    \rfoot{\thepage} % set page number in footer
}

% Counter: \q
% -----------
% Description: Display a question number in a circle.
\newcounter{q}
\setcounter{q}{0} % set initial value of counter
\newcommand{\q}{
    \bigskip
    \addtocounter{q}{1}
    \par
    \Circled{\textbf{\theq}} \space
}


\usepackage{tikz}
\usepackage{subcaption} % for subfigures
\usetikzlibrary{trees} % for tree layout

\title{Dénombrement - Tournoi avec contraintes}
\author{}
\date{2024}

\customPageLayout{Sujets d'interrogation orale}{Lycée Henri IV}{2024}

\begin{document}

\textbf{Contexte}

Un tournoi de \( n \) joueurs est organisé selon un format \textbf{à élimination directe} : chaque
partie oppose deux joueurs, et le perdant est éliminé. Le tournoi continue ainsi jusqu'à ce qu'il ne
reste qu'un seul vainqueur.

Lors de l'organisation d'un tournoi, différentes contraintes peuvent être imposées pour garantir une
certaine équité de niveau dans les affrontements ou augmenter le suspense. Ainsi, la répartition des
joueurs dans les parties des premiers tours ne sont pas complètement aléatoires.

\textbf{Objectifs}

Modéliser et analyser la structure d'un tournoi à élimination directe en utilisant les outils du
dénombrement et des arbres binaires. Évaluer l'impact de contraintes organisationnelles sur le
nombre de configurations possibles du tournoi.


\q \textbf{Nombre de parties}

Justifier qu'un tournoi à élimination directe avec \( n \) joueurs nécessite exactement \( n-1 \)
parties. Procéder par un raisonnement combinatoire ou une démonstration par récurrence.

\q \textbf{Modélisation par un arbre binaire}

Un tournoi par élimination directe peut être modélisé par un arbre binaire :
\begin{itemize}
    \item Les feuilles correspondent aux joueurs participant au tournoi.
    \item Les noeuds internes représentent les parties entre deux joueurs.
    \item Les niveaux de l'arbre correspondent aux tours du tournoi.
    \item La racine de l'arbre correspond à la finale du tournoi.
\end{itemize}

\begin{figure}[htbp]
    \begin{subfigure}{0.49\textwidth}
    \centering
      \begin{tikzpicture}[
        every node/.style = {circle, draw, minimum size=0.6cm},
        level/.style = {sibling distance=3.5cm/#1}
    ]
    \node {F}
        child {node {1/2}
            child {node {1/4}
                child {node {1}}
                child {node {2}}
            }
            child {node {1/4}
                child {node {3}}
                child {node {4}}
            }
        }
        child {node {1/2}
            child {node {1/4}
                child {node {5}}
                child {node {6}}
            }
            child {node {1/4}
                child {node {7}}
                child {node {8}}
            }
        };
    \end{tikzpicture}
    \caption{Arbre complet}
    \label{fig:arbre_complet}
    \end{subfigure}%
    \hfill
    \begin{subfigure}{0.49\textwidth}
    \centering
      \begin{tikzpicture}[
        every node/.style = {circle, draw, minimum size=0.6cm},
        level/.style = {sibling distance=3.5cm/#1}
    ]
    \node {F}
        child {node {1/2}
            child {node {1/4}
                child {node {1}}
                child {node {2}}
            }
            child {node {1/4}
                child {node {3}}
                child {node {4}}
            }
        }
        child {node {1/2}
            child {node {1/4}
                child {node {5}}
                child {node {6}}
            }
            child {node {1/4}
                child {node {7}}
                child[missing]
            }
        };
    \end{tikzpicture}
    \caption{Arbre incomplet}
    \label{fig:arbre_incomplet}
    \end{subfigure}
\end{figure}


\begin{enumerate}
    \item Utiliser le lemme des bergers pour établir une bijection entre les ordres de parties
    possibles et les arbres binaires à $ n $ feuilles.
    \item Une structure d'arbre binaire se définit par la disposition des noeuds et leurs
    connections, sans tenir compte des valeurs spécifiques contenues dans ces noeuds. Elle
    représente la "forme" de l'arbre. Proposer une méthode pour générer systématiquement toutes les
    structures d'arbres binaires de manière récursive en partant de la racine.
    \item Illustrer cette méthode en construisant tous les arbres binaires complets à 4 feuilles.
    \item Par analogie avec cette méthode, exprimer le nombre de structures d'arbres binaires
    complets à $n$ feuilles en fonction des nombres de structures d'arbres binaires complets à $k$
    feuilles, pour $k < n$.
\end{enumerate}

\q \textbf{Organisation des parties sans contrainte}

\begin{enumerate}
    \item Déterminer le nombre de façons d'affecter les \( n \) joueurs à cet arbre (en supposant
    que leur placement est aléatoire).
    \item Décompte des structures possibles :
    Montrer que le nombre de façons possibles d'organiser la structure du tournoi, en ne tenant pas
    compte de l'identité des joueurs, est donné par le \( (n-1) \)-ième nombre de Catalan :
    \[
    C_{n-1} = \frac{1}{n} \binom{2(n-1)}{n-1}
    \]
    \item Placement des joueurs dans un arbre donné : Déterminer le nombre de façons d'affecter les
    \( n \) joueurs à cet arbre.
    \item Comparer le nombre total de configurations obtenues dans deux cas :
    \begin{itemize}
        \item En tenant compte uniquement de la structure de l'arbre.
        \item En tenant compte du placement des joueurs.
    \end{itemize}
    Expliquer pourquoi cette comparaison met en évidence la distinction entre les arbres étiquetés
    et non étiquetés.
\end{enumerate}


\q \textbf{Séparation des fédérations}

Le tournoi accueille \( k \) fédérations ayant respectivement \( p_1, p_2, \dots, p_k \) joueurs.

L'organisateur du tournoi impose la contrainte que les joueurs appartenant à la même fédération ne
peuvent pas s'affronter avant une certaine phase du tournoi (par exemple, les quarts de finale).

\begin{enumerate}
    \item  Expliquer pourquoi cette contrainte impose une structure particulière à l'arbre binaire.
    \item En supposant que les joueurs appartenant à la même fédération doivent être répartis de
    manière équilibrée dans l'arbre, déterminer le nombre de façons possibles de les disposer
    tout en respectant la contrainte de séparation.
    \item  Déterminer une formule générale pour le nombre d'arbres de tournoi valides
    respectant la contrainte de séparation des fédérations.
    \item Utiliser la formule du crible pour comparer le nombre de configurations avec et sans
    contrainte.
    \item Dans quel cas l'espace des possibilités est-il le plus réduit ? Justifier mathématiquement
    cette différence.
\end{enumerate}


\q \textbf{Têtes de séries}

Les "têtes de série" sont des joueurs considérés comme favoris dans un tournoi en raison de leur
classement élevé lors des rencontres précédentes. Elles sont placées stratégiquement dans l'arbre
pour éviter de s'affronter trop tôt dans la compétition.

Désormais, \( s \) joueurs sont désignés comme \textbf{têtes de série}, c'est-à-dire que leur
position dans l'arbre est fixe avant le début du tournoi.
\begin{enumerate}
    \item Combien d'arbres valides peuvent être construits si les \( s \) joueurs sont placés à des
    positions prédéterminées et les \( n-s \) autres joueurs sont placés librement ?
    \item Comparer ce nombre avec le cas où les \( s \) joueurs ne sont pas préassignés en utilisant
    le principe d'inclusion-exclusion.
    \item Démontrer que le rapport entre ces deux nombres tend vers $ e^{-s} $ lorsque $ n $
    tend vers l'infini, pour $s$ fixé.
\end{enumerate}

\q \textbf{Exempts}

Les "exempts" sont des joueurs bénéficiant d'une qualification directe pour un tour avancé (par
exemple, en huitièmes de finale). Cette faveur est parfois accordée aux têtes de série les mieux
classées, qui sont alors dispensées de disputer le premier tour.

\begin{enumerate}
    \item Dans une première organisation, $r$ joueurs sont directement qualifiés pour le deuxième
    tour. Calculer le nombre de façons d'organiser le premier tour.
    \item Généraliser ce raisonnement à un tournoi où des qualifications automatiques existent à
    plusieurs niveaux. Soit $ r_i $ le nombre de joueurs qualifiés pour le i-ème tour. Exprimer le
    nombre total de parties en fonction de $ n $ et des $ r_i $.
    \item Démontrer que le nombre de façons d'organiser un tel tournoi est donné par
    $$\prod_{i=1}^{\log_2 n} (n_i-1)! $$, où $$ n_i $$ est le nombre de joueurs au i-ème tour.
\end{enumerate}



\end{document}
