% CORRECTION : Répartition des ressources
% ==================================================================================================

\documentclass[10pt,a4paper]{article}

% Set the root path
\providecommand{\rootpath}{../../..}
% Fonts
\usepackage[utf8]{inputenc} % for accents
\usepackage[T1]{fontenc} % for accents
\usepackage[french]{babel} % for french language
\usepackage{helvet} % sans serif font family
\renewcommand*\familydefault{\sfdefault} % sans serif font family

% Mathematics
\usepackage{amsmath,amsfonts,amssymb} % for math symbols
\usepackage{array} % for tabular


\usepackage{parskip} % no indent, space between paragraphs

\usepackage{geometry} % margin
\geometry{
    a4paper,
    left=15mm,
    right=15mm,
    top=20mm,
    bottom=20mm
}

\usepackage{circledsteps} % to draw circles around numbers

\usepackage{fancyhdr} % for headers and footers

\usepackage{enumitem} % for customizing lists
\setlist[enumerate]{itemsep=1em} % space between items only in enumerate environment (not itemize)
\setlist[itemize]{label=--} % set itemize label to em-dash

% Command: \customPageLayout{#1}{#2}{#3}
% --------------------------------------
% Description: Custom page layout with header and footer content.
% Arguments:
% #1: Header and footer content
% #2: Left header content
% #3: Right header content
% Example:
% \customPageLayout{Title}{Lycée Henri IV}{2024}
% Required Packages: fancyhdr
\newcommand{\customPageLayout}[3]{
    \pagestyle{fancy} % set page style to fancy, i.e. header and footer
    \fancyhf{#1} % set header and footer content
    \lhead{#2} % set left header content
    \rhead{#3} % set right header content
    \fancyfoot{} % clear footer content
    \rfoot{\thepage} % set page number in footer
}

% Counter: \q
% -----------
% Description: Display a question number in a circle.
\newcounter{q}
\setcounter{q}{0} % set initial value of counter
\newcommand{\q}{
    \bigskip
    \addtocounter{q}{1}
    \par
    \Circled{\textbf{\theq}} \space
}


\usepackage{tikz}
\usepackage{tikz-3dplot}

\title{Dénombrement - Répartition optimale des ressources}
\author{}
\date{2024}

\customPageLayout{Correction}{Lycée Henri IV}{2024}

\begin{document}

% --------------------------------------------------------------------------------------------------
\q \textbf{Modélisation du problème sans contrainte spécifique}

La répartition des $ n $ ressources entre $ k $ projets est équivalente au dénombrement des
solutions de l'équation :
$$
x_1 + x_2 + \dots + x_k = n
$$
avec $ x_i \geq 1 $ pour tout $ i $.

En effet, chaque solution de cette équation correspond à une répartition possible des ressources, où
$ x_i $ représente le nombre de ressources allouées au projet $ i $.

La condition $ x_i \geq 1$ assure que chaque projet reçoit au moins une unité de ressource.

% --------------------------------------------------------------------------------------------------
\q \textbf{Dénombrement des répartitions possibles}

Le nombre total de répartitions possibles est donné par :
$$
\binom{n-1}{k-1}
$$

Considérons $ n $ unités alignées. Former $ k $ groupes revient à placer $ k-1 $ séparateurs
entre ces unités. Il y a $ n-1 $ espaces entre les unités où placer ces séparateurs.

Choisir $ k-1 $ positions parmi ces $ n-1 $ espaces revient à choisir $ k-1$ éléments parmi
$n-1 $, d'où la formule $ \binom{n-1}{k-1} $.

% --------------------------------------------------------------------------------------------------
\q \textbf{Allocation avec priorités}

   \begin{enumerate}

      \item Pour prendre en compte la contrainte que le projet $ P_1 $ doit recevoir au moins $m$
      unités, on modifie l'équation :
      $$
      (x_1 + m) + x_2 + \dots + x_k = n
      $$
      qui devient :
      $$
      x_1 + x_2 + \dots + x_k = n - m
      $$
      avec $ x_1 \geq 0 $ et $ x_i \geq 1 $ pour $ i \geq 2 $.

      En effet, on attribute automatiquement $ m $ unités pour le projet $ P_1 $.

      \item Réduction au cas général :

      Pour garantir que toutes les nouvelles variables sont bien supérieures ou égales à 1, posons :
      \begin{align*}
      y_1 &= x_1 + 1 \\
      y_i &= x_i \quad \forall i \geq 2
      \end{align*}

      L'équation devient :
      $$
      (y_1 - 1) + y_2 + ... + y_k = n - m
      $$
      soit :
      $$
      y_1 + y_2 + \dots + y_k = n - m + 1
      $$

      Ce problème se ramène au cas général avec $ n - m + 1 $ ressources à répartir.

      \item En utilisant la formule précédente, le nombre de répartitions possibles est :
      $$
      \binom{(n-m+1)-1}{k-1} = \binom{n-m}{k-1}
      $$

      \item Extension aux priorités multiples :

      Pour des allocations minimales $ m_1, m_2, \dots, m_k $ pour chaque projet, l'équation
      devient :
      $$
      (x_1 - m_1) + (x_2 - m_2) + \dots + (x_k - m_k) = n - (m_1 + m_2 + \dots + m_k)
      $$
      avec $ x_i - m_i \geq 0 $ pour tout $ i $.

      En posant $ y_i = x_i - m_i + 1 $, on se ramène au cas général avec $ n - \sum_{i=1}^k m_i + k $
      ressources à répartir.

   \end{enumerate}

% --------------------------------------------------------------------------------------------------
\q \textbf{Contraintes de plafonnement}
   \begin{enumerate}

      \item Soit $S$ l'ensemble de toutes les répartitions possibles et $A_i$ l'ensemble des
      répartitions où $P_i$ reçoit plus de $m_i$ ressources. Le nombre de répartitions valides est
      donné par :
      $$
      |S \setminus (A_1 \cup A_2 \cup \dots \cup A_k)|
      $$
      D'après la formule d'inclusion-exclusion :
      $$
      |S \setminus (A_1 \cup A_2 \cup \dots \cup A_k)| = |S| - \sum_i |A_i| + \sum_{i<j} |A_i \cap A_j| - \sum_{i<j<l} |A_i \cap A_j \cap A_l| + \dots
      $$

      \item Calcul de $|A_i|$ et $|A_i \cap A_j|$ :

      Reformulons le problème en tenant compte des contraintes et du changement de variable associé.

      Cardinal de \( |A_i | \) :

      L'ensemble \( A_i \) représente les répartitions où \( x_i > m_i \), ou encore
      \( x_i \geq m_i+1 \).
      Effectuons un changement de variable pour garantir que toutes les nouvelles variables sont
      supérieures ou égales à 1 :
      \[
      y_i = x_i - m_i \quad \Rightarrow \quad y_i \geq 1
      \]

      Ce changement entraîne une réécriture de l'équation de répartition :
      \[
      x_1 + \dots + x_{i-1} + y_i + x_{i+1} + \dots + x_k = n - m_i
      \]

      Le nombre de solutions en entiers naturels de cette équation est donné par la formule
      classique du nombre de solutions de :
      \[
      |A_i| = \binom{n - m_i - 1}{k-1}
      \]

      Cardinal de \( |A_i \cap A_j| \) :

      L'intersection \( A_i \cap A_j \) représente les répartitions où deux projets \( P_i \) et \(
      P_j \) reçoivent plus que leur plafond respectif, c'est-à-dire :
      \[
      x_i > m_i \quad \text{et} \quad x_j > m_j
      \]
      En introduisant les nouvelles variables, l'équation devient :
      \[
      x_1 + \dots + x_{i-1} + y_i + x_{i+1} + \dots + x_{j-1} + y_j + x_{j+1} + \dots + x_k = n - (m_i + m_j)
      \]
      Le nombre de solutions en entiers naturels de cette nouvelle équation est :
      \[
      |A_i \cap A_j| = \binom{n - (m_i + m_j) - 1}{k-1}.
      \]

      \item Pour $k=3$, la formule explicite est :
      \begin{align*}
      &\binom{n-1}{2} - \binom{n-m_1-1}{2} - \binom{n-m_2-1}{2} - \binom{n-m_3-1}{2} \\
      &+ \binom{n-m_1-m_2-1}{2} + \binom{n-m_1-m_3-1}{2} + \binom{n-m_2-m_3-1}{2}
      \end{align*}
      Cette formule compte le nombre total de répartitions, puis soustrait les répartitions violant
      chaque contrainte individuellement, et ajoute les répartitions violant deux contraintes
      simultanément pour éviter de les compter deux fois.

   \end{enumerate}

% --------------------------------------------------------------------------------------------------
\q \textbf{Ressources discernables}

   \begin{enumerate}

      \item Modélisation du problème comme une permutation avec répétitions :

      Une "permutation avec répétitions" décrit une suite de longueur $n$ dans laquelle certains
      "symboles" apparaissent plusieurs fois. Dans l'allocation de $n$ ressources à
      $k$ projets, ces symboles correspondent aux projets $P_i$.

      Au lieu d'énumérer directement toutes les fonctions qui attribuent chaque ressource au projet
      de son choix (ce qui donne $k^n$ affectations), il est possible d'envisager chaque
      distribution comme une permutation de $n$ positions, mais où le symbole "i" apparaît
      exactement $x_i$ fois, ce qui motive l'expression "permutation avec répétitions".

      \item Expression du nombre de permutations avec répétitions :

      Le nombre total de permutations des \( n \) ressources est $ n! $.

      Cependant, au sein de chaque projet \( P_i \), les \( x_i \) ressources sont interchangeables
      (elles ne sont pas distinguables à l'intérieur du groupe). Il faut donc diviser par le nombre
      de permutations internes à chaque groupe, soit \( x_1! \), \( x_2! \), …, \( x_k! \). Ainsi,
      le nombre total d'affectations possibles est donné par :
      \[
      \frac{n!}{x_1! x_2! \dots x_k!}
      \]

      Ce résultat est un coefficient multinomial, qui compte le nombre de façons de répartir des
      objets discernables en groupes indiscernables.


      \item Somme des répartitions valides sous la contrainte \( x_1 + x_2 + \dots + x_k = n \) :

      Interprétation combinatoire :
      \begin{itemize}
         \item Chaque ressource peut être affectée à l'un des \( k \) projets de manière indépendante.
         \item Puisqu'il y a \( n \) ressources, le nombre total de façons d'attribuer ces ressources est simplement \( k^n \).
      \end{itemize}
      Ainsi :
      \[
      \sum_{x_1 + x_2 + \dots + x_k = n} \frac{n!}{x_1! x_2! \dots x_k!} = k^n
      \]

   \end{enumerate}

\end{document}
% ==================================================================================================

% ==================================================================================================
\q \textbf{Dénombrement des répartitions avec contraintes de plafonnement par bijection combinatoire}

   \begin{enumerate}
      \item Établissement de la bijection : Cette relation est bijective car à chaque répartition
      valide correspond un unique mot construit selon les règles, et réciproquement, chaque mot
      construit selon les règles définit une unique répartition valide des ressources.

      \item Décomposition récursive en sous-problèmes :

      On peut construire une répartition valide en affectant un certain nombre \( x_1 \) de
      lettres au premier bloc, puis en répartissant les \( n - x_1 \) lettres restantes
      entre les \( k-1 \) autres projets. Puisque \( x_1 \) doit être compris entre \( 1 \) et \(
      M_1 \) :
      \[
      T(n, k, M_1, M_2, \dots, M_k) = \sum_{x_1=1}^{M_1} T(n - x_1, k-1, M_2, \dots, M_k)
      \]

      \item Nombre de répartitions de $n$ objects distincts en $k$ groupes de tailles données par
      $x_1, x_2, \dots, x_k$ :

      \item Somme générale :
      Il faut sommer sur toutes les valeurs possibles de $x_1$ entre 1 et $M_1$, puis répéter le
      même processus pour les autres blocs. La somme sur toutes les combinaisons possibles est
      nécessaire car chaque combinaison valide des $x_1i$ représente une répartition possible.

      Conditions à respecter :
      \begin{align*}
      x_1 + x_2 + \dots + x_k &= n \\
      1 \leq x_i &\leq M_i
      \end{align*}

      La solution finale  correspond au comptage des affectations valides en utilisant le
      coefficient multinomial, qui compte le nombre de façons d'organiser \( n \) ressources en \( k
      \) groupes de tailles données. La somme assure que seules les répartitions respectant \( 1
      \leq x_i \leq M_i \) sont prises en compte :
      \[
      T(n, k, M_1, \dots, M_k) = \sum_{\substack{x_1 + x_2 + \dots + x_k = n \\ 1 \leq x_i \leq M_i}} \frac{n!}{x_1! x_2! \dots x_k!}
      \]

   \end{enumerate}
