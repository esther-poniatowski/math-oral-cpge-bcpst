% CORRECTION : Arbres Binaires
% ==================================================================================================

\documentclass[10pt,a4paper]{article}

% Set the root path
\providecommand{\rootpath}{../../..}
% Fonts
\usepackage[utf8]{inputenc} % for accents
\usepackage[T1]{fontenc} % for accents
\usepackage[french]{babel} % for french language
\usepackage{helvet} % sans serif font family
\renewcommand*\familydefault{\sfdefault} % sans serif font family

% Mathematics
\usepackage{amsmath,amsfonts,amssymb} % for math symbols
\usepackage{array} % for tabular


\usepackage{parskip} % no indent, space between paragraphs

\usepackage{geometry} % margin
\geometry{
    a4paper,
    left=15mm,
    right=15mm,
    top=20mm,
    bottom=20mm
}

\usepackage{circledsteps} % to draw circles around numbers

\usepackage{fancyhdr} % for headers and footers

\usepackage{enumitem} % for customizing lists
\setlist[enumerate]{itemsep=1em} % space between items only in enumerate environment (not itemize)
\setlist[itemize]{label=--} % set itemize label to em-dash

% Command: \customPageLayout{#1}{#2}{#3}
% --------------------------------------
% Description: Custom page layout with header and footer content.
% Arguments:
% #1: Header and footer content
% #2: Left header content
% #3: Right header content
% Example:
% \customPageLayout{Title}{Lycée Henri IV}{2024}
% Required Packages: fancyhdr
\newcommand{\customPageLayout}[3]{
    \pagestyle{fancy} % set page style to fancy, i.e. header and footer
    \fancyhf{#1} % set header and footer content
    \lhead{#2} % set left header content
    \rhead{#3} % set right header content
    \fancyfoot{} % clear footer content
    \rfoot{\thepage} % set page number in footer
}

% Counter: \q
% -----------
% Description: Display a question number in a circle.
\newcounter{q}
\setcounter{q}{0} % set initial value of counter
\newcommand{\q}{
    \bigskip
    \addtocounter{q}{1}
    \par
    \Circled{\textbf{\theq}} \space
}


\usepackage{tikz}
\usepackage{subcaption} % for subfigures
\usetikzlibrary{trees} % for tree layout
\usetikzlibrary{positioning} % for relative positioning
\usetikzlibrary{patterns,calc} % for patterns
\usetikzlibrary{trees,arrows} % for tree layout

\title{Dénombrement - Tournoi avec contraintes}
\author{Esther Poniatowski}
\date{2024-2025}

\customPageLayout{Correction}{Lycée Henri IV}{2024}

\begin{document}

% --------------------------------------------------------------------------------------------------
\q \textbf{Nombre de noeuds}

\begin{enumerate}
    \item En toute généralité : \( N = I + F \).

    \item Pour un arbre binaire plein :

    Initialisation (cas de \( N = 1 \)) :

    Si l'arbre ne contient qu'un seul noeud, alors ce noeud
    est une feuille.
    \begin{itemize}
        \item Il n'y a aucun noeud interne, donc \( I = 0 \).
        \item Il y a exactement une feuille, donc \( F = 1 \).
    \end{itemize}
    On vérifie que \( F = I + 1 \), ce qui est vrai puisque \( 1 = 0 + 1 \).

    Hypothèse de récurrence :

    Supposons que pour un arbre binaire plein ayant \( N \) noeuds, la relation \( F = I + 1 \) est
    vérifiée.

    Considérons un arbre binaire plein avec \( N+2 \) noeuds.
    \begin{itemize}
        \item Ajouter un nouveau niveau à l'arbre signifie que l'on convertit une feuille existante
        en un noeud interne ayant deux nouveaux enfants.
        \item Ainsi, le nombre de noeuds internes augmente de 1 : \( I' = I + 1 \).
        \item Le nombre de feuilles augmente de 1 (car une feuille disparaît et deux nouvelles
        apparaissent) : \( F' = F + 1 \).
    \end{itemize}
    La relation est toujours vérifiée après cette transformation :
    \[
    F' = F + 1 = (I + 1) + 1 = I' + 1
    \]

    Conclusion : Par récurrence, la relation \( F = I + 1 \) est vraie pour tout arbre binaire
    plein.

    \item Nombre de noeuds internes et de feuilles dans un arbre binaire plein :

    $ N = I + F = I + (I + 1) = 2I + 1 $ donc $ I = \frac{N - 1}{2} $

    $ F = N - I = N - \frac{N - 1}{2} = \frac{N + 1}{2} $

    \item Pour un arbre binaire parfait de hauteur \(h\) :

    Soit \(N_k\) le nombre de noeuds au niveau \(k\).

    Puisque l'arbre est complet, tous les niveaux sont remplis, donc :
    \[ N_k = 2^k \text{ pour } 0 \leq k \leq h \]

    Le nombre total de noeuds \(N\) est la somme des noeuds à tous les niveaux :
    \[ N = \sum_{k=0}^h N_k = \sum_{k=0}^h 2^k = 2^{h+1} - 1 \]

\end{enumerate}

% --------------------------------------------------------------------------------------------------
\q \textbf{Dénombrement des structures d'arbres binaires}

\begin{enumerate}
    \item Méthode récursive pour construire tous les arbres binaires complets à \( n \) feuilles :
    \begin{itemize}
         \item A partir de la racine, partitionner l'arbre en deux sous-arbres : \( k \) feuilles
         pour le sous-arbre gauche, \( n - k \) feuilles pour celui de droite, où \( k \) varie de 1
         à \( n - 1 \).
         \item Pour chaque partition \( (k, n - k) \), créer un nouveau noeud racine et générer
         récursivement tous les couples possibles de sous-arbres gauches et droits.
    \end{itemize}

    \item Illustration pour \( n = 4 \) feuilles :

    Les partitions possibles sont  \( (1, 3) \), \( (2, 2) \), \( (3, 1) \).
    \begin{itemize}
         \item Pour \( k = 1 \) et \( n - k = 3 \) : Le sous-arbre gauche est une feuille unique. Le
      sous-arbre droit est un arbre binaire complet avec 3 feuilles, qui peut être construit en
      appliquant récursivement la même méthode.
      \item  Pour \( k = 2 \) et \( n - k = 2 \) : Les deux
      sous-arbres sont des arbres binaires complets avec 2 feuilles chacun, qui sont des arbres simples
      avec une racine et deux feuilles.
      \item Pour \( k = 3 \) et \( n - k = 1 \) : Le sous-arbre gauche
      est un arbre binaire complet avec 3 feuilles. Le sous-arbre droit est une feuille unique.
    \end{itemize}

    \begin{tikzpicture}[scale=0.7]
  % Arbre 1 : Partition (1, 3)
  \node {R}
    child { node {G} }
    child { node {D}
      child { node {G} }
      child { node {D}
        child { node {G} }
        child { node {D} }
      }
    };
  \node[below=1cm of current bounding box.south] {(1, 3)};
\end{tikzpicture}
\hspace{1cm}
\begin{tikzpicture}[scale=0.7]
  % Arbre 2 : Partition (1, 3)
  \node {R}
    child { node {G} }
    child { node {D}
      child { node {G}
        child { node {G} }
        child { node {D} }
      }
      child { node {D} }
    };
  \node[below=1cm of current bounding box.south] {(1, 3)};
\end{tikzpicture}
\hspace{1cm}
\begin{tikzpicture}[scale=0.7]
  % Arbre 3 : Partition (2, 2)
  \node {R}
    child { node {G}
      child { node {G} }
      child { node {D} }
    }
    child { node {D}
      child { node {G} }
      child { node {D} }
    };
  \node[below=1cm of current bounding box.south] {(2, 2)};
\end{tikzpicture}
\hspace{1cm}
\begin{tikzpicture}[scale=0.7]
  % Arbre 4 : Partition (3, 1)
  \node {R}
    child { node {G}
      child { node {G} }
      child { node {D}
        child { node {G} }
        child { node {D} }
      }
    }
    child { node {D} };
  \node[below=1cm of current bounding box.south] {(3, 1)};
\end{tikzpicture}
\hspace{1cm}
\begin{tikzpicture}[scale=0.7]
  % Arbre 5 : Partition (3, 1)
  \node {R}
    child { node {G}
      child { node {G}
        child { node {G} }
        child { node {D} }
      }
      child { node {D} }
    }
    child { node {D} };
  \node[below=1cm of current bounding box.south] {(3, 1)};
\end{tikzpicture}


    \item Relation de récurrence :

    Soit \(T_n\) le nombre d'arbres binaires avec \(n\) feuilles. On a :
    \[ T_n = \sum_{k=1}^{n-1} T_k T_{n-k} \]
\end{enumerate}

% --------------------------------------------------------------------------------------------------
\q \textbf{Chemins de Dyck}

\begin{enumerate}
    \item Nombre total de chemins de $(0,0)$ à $(2n,0)$ :

    Un chemin allant de \( (0,0) \) à \( (2n,0) \) consiste en \( n \) montées \( (+1, +1) \) et \(
    n \) descentes \( (+1, -1) \). Le nombre total de tels chemins (sans contrainte) est donné par :
    \[ \binom{2n}{n} \]
    car il faut choisir $n$ pas parmi $2n$ pour les pas montants.

    \item Bijection entre les chemins passant sous l'axe et les chemins de $(0,0)$ à $(2n,-2)$ :

    Soit un chemin qui passe sous l'axe \( y=0 \), soit \( P(2k, -1) \) le premier point où il
    atteint l'axe \( y=-1 \), où \( k \) est le plus petit entier tel que le chemin atteint \( -1
    \).

    Réflexion du chemin par rapport à l'axe \( y=-1 \) : Après ce premier passage sous l'axe en
    \( P \), chaque montée \( (+1,+1) \) est échangée avec une descente \( (+1,-1) \) et
    inversement.

    Dans le chemin initial, la portion restante après $ P $ réalise un déplacement vertical net de 1
    unité vers le haut, pour remonter de l'ordonnée -1 au point finale \( (2n,0) \). Dans le chemin
    réfléchi, ce déplacement vertical est inversé à partir de l'ordonnée -1, de sorte qu'il aboutit
    en \( (2n,-2) \). Cette réflexion transforme donc un chemin qui allait de \( (0,0) \) à \(
    (2n,0) \) en un chemin allant de \( (0,0) \) à \( (2n-2) \).

    Réciprocité : Soit un chemin de \( (0,0) \) à \( (2n,-2) \), la réflexion inverse fournit un
    chemin passant sous l'axe.

    Conclusion : Chaque chemin passant sous l'axe est donc mis en correspondance bijective avec un
    chemin de \( (0,0) \) à \( (2n-2,2) \).

    
\begin{tikzpicture}[scale=0.8]
    % Axes
    \draw[->] (-0.5,0) -- (10,0) node[below] {\( x \)};
    \draw[->] (0,-3) -- (0,3.5) node[left] {\( y \)};

    % Grille
    \draw[help lines, dashed] (0,-4) grid (10,3);

    % Chemin initial (en noir)
    \draw[thick] (0,0) -- (1,1) -- (2,2) -- (3,1) -- (4,0) -- (5,-1) -- (6,0) -- (7,-1) -- (8,0) --
    (9,1) -- (10,0);

    % Chemin réfléchi (en rouge)
    \draw[thick, red] (5,-1) -- (6,-2) -- (7,-1) -- (8,-2) -- (9,-3) -- (10,-2);

    % Point de réflexion
    \fill[black] (5,-1) circle (3pt) node[below left] {\( P \)};

    % Axe de réflexion
    \draw[dashed, blue] (-0.5,-1) -- (9.5,-1);

    % Légende
    %\node[right, red] at (9,-1) {Réflexion du chemin};
    \node[right, black] at (10,0) {\( (2n,0) \)};
    %\node[right, blue] at (9,-1) {Axe \( y=-1 \)};
\end{tikzpicture}


    \item Le nombre de chemins passant sous l'axe  est égal au nombre de chemins allant de \( (0,0)
    \) à \( (2n,-2) \). Pour aboutir à ce point en $2n$ étapes, il faut effectuer \( 1 \) descente
    de plus que de montées. Il faut donc choisir \( n-1 \) pas parmi \( 2n \) pour les pas montants.
    Ainsi, le nombre de chemins passant sous l'axe est donné par :
    \[
    \binom{2n}{n-1}
    \]

    \item Nombre de chemins de Dyck :

    \begin{itemize}
        \item Le nombre total de chemins de \( (0,0) \) à \( (2n,0) \) est donné par :
        \[
        \binom{2n}{n}
        \]
        \item  Le nombre de chemins passant sous l'axe est égal à :
        \[
        \binom{2n}{n-1}
        \]
        \item Le nombre chemins chemins de Dyck (qui ne passent jamais sous l'axe) est obtenu par
        soustraction :
        \[
        C_n = \binom{2n}{n} - \binom{2n}{n-1}
        \]
    \end{itemize}

    \item Identité de Darboux :

    Pour former une équipe de \( n + 1 \) membres parmi \( 2n \) ainsi que le capitaine de l'équipe,
    deux méthodes équivalentes sont possibles :
    \begin{itemize}
        \item Choisir les \( n + 1 \) membres parmi les \( 2n \) puis choisir le capitaine parmi les
        \( n + 1 \) membres : \(\binom{2n}{n+1} \times (n+1)\).
        \item Choisir les \( n \) membres "non-capitaines"  parmi les \( 2n \) puis ajouter le
        capitaine parmi les \( n \) personnes restantes : \(\binom{2n}{n} \times n\).
    \end{itemize}
    Conclusion : $$ (n+1)\binom{2n}{n+1} = n\binom{2n}{n} $$

    \item Exprimons $\binom{2n}{n-1}$ en fonction de $\binom{2n}{n}$ en utilisant la formule
    précédente :
    $$\binom{2n}{n-1} = \binom{2n}{n+1} = \frac{n}{n+1}\binom{2n}{n}$$

    Substituons cette expression dans l'équation originale :
    $$\binom{2n}{n} - \binom{2n}{n-1} = \binom{2n}{n} - \frac{n}{n+1}\binom{2n}{n}$$

    Factorisons et simplifions :
    $$\binom{2n}{n}\left(1 - \frac{n}{n+1}\right) = \frac{1}{n+1}\binom{2n}{n}$$

    Ainsi, le nombre de chemins de Dyck est donné par le $n$-ième nombre de Catalan :
    \[ C_n = \frac{1}{n+1} \binom{2n}{n} \]

    \item Bijection entre chemins de Dyck et arbres binaires pleins :

    Construction d'un chemin de Dyck à partir d'un arbre binaire plein :
    \begin{itemize}
        \item Partir de la racine de l'arbre.
        \item À chaque visite d'un noeud interne pour la première fois, ajouter un pas montant \(
        (+1,+1) \).
        \item À chaque fois que l'on termine l'exploration d'un sous-arbre et que l'on remonte à son
        parent, ajouter un pas descendant \( (+1,-1) \).
        \item Continuer jusqu'à ce que l'arbre soit entièrement exploré.
    \end{itemize}

    L'arbre ayant \( n \) noeuds internes, il possède exactement \( n \) montées et \( n \)
    descentes, donnant ainsi un chemin de \( 2n \) étapes qui ne passe jamais sous l'axe.

    Reconstruction réciproque d'un arbre binaire à partir d'un chemin de Dyck :
    \begin{itemize}
        \item Commencer avec une racine.
        \item Chaque montée \( (+1,+1) \) ajoute un nouveau noeud interne qui devient temporairement
        le noeud courant.
        \item Chaque descente \( (+1,-1) \) signifie que l'on remonte vers le parent du noeud
        courant.
    \end{itemize}
    Une fois l'arbre entièrement reconstruit, chaque noeud interne a exactement deux
    enfants.

    Exemple d'un arbre binaire et de son chemin de Dyck associé :
    \begin{center}
        \begin{tikzpicture}[scale=0.7,level distance=1.2cm,sibling distance=2cm]
  % Arbre binaire
  \node[circle,draw] (root) {} child { node[circle,draw] (left) {} child { node[circle,draw]
      {} } child { node[circle,draw] {} } } child { node[circle,draw] (right) {} };

  % Chemin de Dyck correspondant
  \draw[->,thick] (0,-3) -- (8,-3); \draw[thick] (0,-3) -- (1,-2) -- (2,-1) -- (3,-2) --
  (4,-1) -- (5,-2) -- (6,-3) -- (7,-2) -- (8,-3);

  % Labels des montées et descentes
  \node at (1,-3.5) {\(\uparrow\)}; \node at (2,-3.5) {\(\uparrow\)}; \node at (3,-3.5)
  {\(\downarrow\)}; \node at (4,-3.5) {\(\uparrow\)}; \node at (5,-3.5) {\(\downarrow\)};
  \node at (6,-3.5) {\(\downarrow\)}; \node at (7,-3.5) {\(\uparrow\)}; \node at (8,-3.5)
  {\(\downarrow\)};
\end{tikzpicture}

    \end{center}

    Parcours en profondeur :
    \begin{itemize}
        \item Commencer à la racine (montée) : \( \uparrow \).
        \item Descendre dans le sous-arbre gauche (montée) : \( \uparrow \).
        \item Visiter un noeud feuille (descente) : \( \downarrow \).
        \item Remonter au parent et visiter l'enfant droit (montée) : \( \uparrow \).
        \item Terminer ce sous-arbre (descente) : \( \downarrow \).
        \item Remonter à la racine et visiter l'enfant droit (montée) : \( \uparrow \).
        \item Ce dernier noeud est une feuille (descente) : \( \downarrow \).
        \item Enfin, terminer (descente) : \( \downarrow \).
    \end{itemize}

    Chemin de Dyck correspondant :
    \(
    \uparrow \uparrow \downarrow \uparrow \downarrow \downarrow \uparrow \downarrow
    \)

    \item Nombre d'arbres binaires à \( n \) noeuds internes :

    Par la bijection précédente, le nombre d'arbres binaires à \( n \) noeuds internes est donné par
    le \( n \)-ième nombre de Catalan : \( C_n \).

\end{enumerate}

\end{document}

% ==================================================================================================

% --------------------------------------------------------------------------------------------------
\q \textbf{Parenthésage des expressions mathématiques}

\begin{enumerate}
    \item Représentation d'une expression parenthésée par un arbre binaire :

    \begin{itemize}
        \item Les feuilles représentent les termes.
        \item Les noeuds internes représentent les opérations.
        \item Le sous-arbre gauche représente la partie gauche de l'opération.
        \item Le sous-arbre droit représente la partie droite de l'opération.
    \end{itemize}

    \item Démonstration que le nombre de parenthésages est le $n$-ième nombre de Catalan :

    Soit $P_n$ le nombre de parenthésages pour $n+1$ termes. Pour parenthéser $n+1$ termes, on
    choisit un opérateur principal qui divise l'expression en deux parties. On a donc :
    \[ P_n = \sum_{k=1}^n P_{k-1} P_{n-k} \] Cette relation est identique à celle des nombres de
    Catalan. Donc $P_n = C_n$.
\end{enumerate}
