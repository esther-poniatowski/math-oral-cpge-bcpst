% PROBLÈME : Polynômes Orthogonaux
% ==================================================================================================

\documentclass[10pt,a4paper]{article}

% Set the root path
\providecommand{\rootpath}{../../..}
% Fonts
\usepackage[utf8]{inputenc} % for accents
\usepackage[T1]{fontenc} % for accents
\usepackage[french]{babel} % for french language
\usepackage{helvet} % sans serif font family
\renewcommand*\familydefault{\sfdefault} % sans serif font family

% Mathematics
\usepackage{amsmath,amsfonts,amssymb} % for math symbols
\usepackage{array} % for tabular


\usepackage{parskip} % no indent, space between paragraphs

\usepackage{geometry} % margin
\geometry{
    a4paper,
    left=15mm,
    right=15mm,
    top=20mm,
    bottom=20mm
}

\usepackage{circledsteps} % to draw circles around numbers

\usepackage{fancyhdr} % for headers and footers

\usepackage{enumitem} % for customizing lists
\setlist[enumerate]{itemsep=1em} % space between items only in enumerate environment (not itemize)
\setlist[itemize]{label=--} % set itemize label to em-dash

% Command: \customPageLayout{#1}{#2}{#3}
% --------------------------------------
% Description: Custom page layout with header and footer content.
% Arguments:
% #1: Header and footer content
% #2: Left header content
% #3: Right header content
% Example:
% \customPageLayout{Title}{Lycée Henri IV}{2024}
% Required Packages: fancyhdr
\newcommand{\customPageLayout}[3]{
    \pagestyle{fancy} % set page style to fancy (add header and footer)
    \fancyhf{} % clear all header and footer content
    \lhead{#2} % left header content
    \rhead{#3} % right header content
    \chead{\textbf{#1}} % center header content in bold (if needed)
    \rfoot{\thepage} % page number in the footer
}


% Counter: \q
% -----------
% Description: Display a question number in a circle.
% Usage:
% - Create a new question: add \q followed by the question content.
% - Reset the question counter: add \setcounter{q}{0} before the first question.
\newcounter{q}
\setcounter{q}{0} % set initial value of the counter
\newcommand{\q}{
    \bigskip
    \addtocounter{q}{1}
    \par
    \Circled{\textbf{\theq}} \space
}


% Counter: \ql
% ------------
% Description: Display a question letter in a round box with indentation (lowercase and not bold).
% Usage:
% - Create a new question: add \ql followed by the question content.
% - Reset the question counter: add \setcounter{ql}{0} before the first question.
\newcounter{ql}
\setcounter{ql}{0} % set initial value of the counter
\newcommand{\ql}{
    \addtocounter{ql}{1}
    \par
    \hspace{1.5em} % indentation before the circled letter
    \textcolor{gray}{\Circled{\alph{ql}}} \space % gray color
}


\title{Polynômes - Polynômes de Legendre}
\author{Esther Poniatowski}
\date{2024-2025}

\customPageLayout{Sujets d'interrogation orale}{Lycée Henri IV}{2024}

\begin{document}


% ========

\bigskip
\textbf{Contexte}

Les polynômes de Legendre sont une famille de polynômes utilisés pour l'approximation numérique.

Étant donné une fonction $f(x)$ définie sur un intervalle $[-1,1]$ (sans perte de généralité), une
question mathématique consiste à approximer $f$ par un polynôme de degré $n$, car les polynômes sont
faciles à manipuler analytiquement et numériquement.

L'idée de Legendre est de construire une approximation polynomiale en combinant des polynômes de
référence. Ce cadre est naturellement analogue à celui des espaces vectoriels, où chaque élément
peut être décomposé sur une à l'aide de combinaisons linéaires. Cette approche permet
d'utiliser des outils d'algèbre linéaire pour garantir l'existence et l'unicité du polynôme optimal
d'approximation.


\bigskip
\textbf{Objectifs}

Étudier les propriétés des polynômes de Legendre, en particulier leur orthogonalité, leur degré et
la structure de leurs racines.

\bigskip
\textbf{Décomposition des Polynômes}

\q Justifier que tout polynôme de degré inférieur ou égal à $n$ s'écrit de manière \textit{unique}
comme combinaison linéaire des polynômes de la famille $\{1, X, X^2, \dots, X^n\}$.
\textit{Indication : Démontrer d'une par l'existence et d'autre par l'unicité de la décomposition.}
% Objectif : Introduire la notion de base d'un espace vectoriel.

\bigskip
\textbf{Orthogonalité}

Dans un problème d'approximation, il est nécessaire de quantifier la "similarité" entre deux
fonctions. Une mesure usuelle consiste à agréger leurs valeurs pour un ensemble de points de
l'intervalle considéré. Ainsi, la valeur maximale est obtenue lorsque les fonctions sont égales.

Par exemple, en considérant trois points de l'intervalle $[-1,1]$ :
\[
\langle P, Q \rangle = P(-1)Q(-1) + P(0)Q(0) + P(1)Q(1)
\]
Cette expression est appelée "produit scalaire (discret)" car elle vérifie les propriétés
essentielles d'un produit scalaire : bilinéarité, symétrie et positivité définie.

Pour un produit scalaire donné, deux polynômes $P$ et $Q$ sont dits orthogonaux si :
$$\langle P, Q \rangle = 0$$.

\q Vérifier que les polynômes \( 1, X, X^2 \) ne sont pas orthogonaux pour ce produit scalaire.
% Objectif : Fixer les idées et illustrer la notion d'orthogonalité dans un cas concret.

\bigskip
Lorsque l'on a défini un produit scalaire, il est souvent avantageux travailler à partir de
polynômes de référence orthogonaux, ce qui facilite les calculs et l'interprétation des résultats.

La méthode de Gram-Schmidt permet de construire une famille de polynômes orthonormaux à partir de la
famille ${1, X, X^2}$. Cette méthode procède itérativement :
\begin{itemize}
   \item Initialisation : Inclure le premier polynôme $P_0(X) = 1$.
   \item Pour chaque $k \in \{0, 1, 2, \dots, n\}$ :
   \begin{itemize}
      \item Construire un polynôme orthogonal à $P_0, P_1, \dots, P_{k-1}$ à partir du polynôme
      $X^K$ et d'une combinaison linéaire des polynômes $P_0, P_1, \dots, P_{k-1}$ déjà présents.
      \item Ajuster le polynôme obtenu de sorte à satisfaire $P_k(1) = 1$.
   \end{itemize}
\end{itemize}

\q Déterminer le degré des polynômes de Legendre $P_n(X)$.
% Objectif : Comprendre la structure des polynômes de Legendre.

\q Déterminer le polynôme $P_1(X)$ à partir de la méthode exposée précédemment.

\q Déterminer le polynôme $P_2(X)$ à partir de la méthode exposée précédemment.

\bigskip \q Proposer une généralisation de la méthode obtenue à une famille ${1, X, X^2, \dots,
X^n}$.


\bigskip
\textbf{Interlacement des racines des Polynômes de Legendre}
% Objectif : Étudier une propriété fondamentale de l'orthogonalité des polynômes. Cette propriété
% renforce l'idée que les polynômes orthogonaux sont “bien séparés” dans l'espace des fonctions.

Pour définir les polynômes de Legendre, le produit scalaire usuel généralise le produit scalaire
discret en sommant sur tous les points de l'intervalle, ce qui revient à calculer une intégrale sur
$[-1,1]$ :
\[
\langle P, Q \rangle = \int_{-1}^{1} P(X) Q(X) \, dX
\]
\footnotesize NB : Cette formule n'intervient pas pour la suite de l'exercice. \normalsize

Avec ce produit scalaire, les polynômes de Legendre vérifient alors la relation de récurrence :
\[
(n+1) P_{n+1}(X) = (2n+1) X P_n(X) - n P_{n-1}(X)
\]

Cette relation donne une méthode pour construire les polynômes de Legendre de manière récursive, en
partant des polynômes de degré 0 et 1.

\q Polynômes consécutifs\\
Démontrer que si deux polynômes consécutifs $ P_n $ et $ P_{n+1} $ partagent une racine
commune, alors tous les polynômes $ P_k , \quad k \leq n-1 $ partagent cette racine.
En déduire une contradiction.

\q Polynômes non consécutifs\\
Démontrer que si deux polynômes non consécutifs $ P_m $ et $ P_n $ (avec $m < n$) partagent une
racine commune, alors on aboutit aussi à une contradiction.

\q Conclure en expliquant pourquoi l'hypothèse initiale (existence d'une racine commune) est fausse.


\end{document}


% ==================================================================================================

\bigskip
\textit{Factorisation des polynômes de Legendre}

\q Démontrer que les racines des polynômes de Legendre $P_n(X)$ sont toutes réelles, distinctes et
situées dans $[-1,1]$.
% Objectif : Exploiter la répartition des racines en vue de factoriser intégralement sur R.
% Démarche : Comment l'élève doit-il procéder pour répondre ?

\q Justifier qu'un polynôme de Legendre admet une factorisation sur $\mathbb{R}$.
% Objectif : Utiliser la répartition des racines pour factoriser un polynôme de Legendre.

\bigskip
\textit{Divisabilité des polynômes de Legendre}

\q Soient $P_n(X)$ et $P_m(X)$ deux polynômes de Legendre distincts ($n\neq m$).
Montrer que si un polynôme de degré $d$ est divisible par deux polynômes de degrés $n$ et $m$, alors
son degré est au moins $n+m$.
En déduire qu'un polynôme de Legendre ne peut pas être divisible par un autre polynôme de Legendre
distinct.
% Objectifs : Mobiliser le lien entre racines et divisibilité. Utiliser le fait qu'un polynôme de
% degré n admet au plus n racines. Mettre en pratique les méthodes de factorisation d'un polynôme.
% Démarche : Comment l'élève doit-il procéder pour répondre ?

Une approche classique consiste à interpréter la relation de récurrence comme une équation
fonctionnelle sur les polynômes, et à en déduire toutes les solutions possibles.

\q Soit $P(X)$ un polynôme satisfaisant l'équation fonctionnelle issue de la relation de récurrence.
Montrer qu'il est nécessairement un multiple d'un polynôme de Legendre.
% Objectif : Faire intervenir la technique du “passage au degré” pour contraindre la forme de P(X).
% Utiliser la récurrence pour structurer l'espace des solutions.
% Démarche : Comment l'élève doit-il procéder pour répondre ?
