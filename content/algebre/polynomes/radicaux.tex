% PROBLÈME : Titre
% ==================================================================================================

\documentclass[10pt,a4paper]{article}

% Set the root path
\providecommand{\rootpath}{../../..}
% Fonts
\usepackage[utf8]{inputenc} % for accents
\usepackage[T1]{fontenc} % for accents
\usepackage[french]{babel} % for french language
\usepackage{helvet} % sans serif font family
\renewcommand*\familydefault{\sfdefault} % sans serif font family

% Mathematics
\usepackage{amsmath,amsfonts,amssymb} % for math symbols
\usepackage{array} % for tabular


\usepackage{parskip} % no indent, space between paragraphs

\usepackage{geometry} % margin
\geometry{
    a4paper,
    left=15mm,
    right=15mm,
    top=20mm,
    bottom=20mm
}

\usepackage{circledsteps} % to draw circles around numbers

\usepackage{fancyhdr} % for headers and footers

\usepackage{enumitem} % for customizing lists
\setlist[enumerate]{itemsep=1em} % space between items only in enumerate environment (not itemize)
\setlist[itemize]{label=--} % set itemize label to em-dash

% Command: \customPageLayout{#1}{#2}{#3}
% --------------------------------------
% Description: Custom page layout with header and footer content.
% Arguments:
% #1: Header and footer content
% #2: Left header content
% #3: Right header content
% Example:
% \customPageLayout{Title}{Lycée Henri IV}{2024}
% Required Packages: fancyhdr
\newcommand{\customPageLayout}[3]{
    \pagestyle{fancy} % set page style to fancy, i.e. header and footer
    \fancyhf{#1} % set header and footer content
    \lhead{#2} % set left header content
    \rhead{#3} % set right header content
    \fancyfoot{} % clear footer content
    \rfoot{\thepage} % set page number in footer
}

% Counter: \q
% -----------
% Description: Display a question number in a circle.
\newcounter{q}
\setcounter{q}{0} % set initial value of counter
\newcommand{\q}{
    \bigskip
    \addtocounter{q}{1}
    \par
    \Circled{\textbf{\theq}} \space
}


\title{Polynômes - Résolubilité des équations polynomiales par radicaux}

\author{Esther Poniatowski}
\date{2024-2025}

\customPageLayout{Sujets d'interrogation orale}{Lycée Henri IV}{2024}

\begin{document}

\textbf{Contexte}

L'étude des solutions des équations polynomiales a occupé une place centrale dans l'histoire des
mathématiques.

Dès l'Antiquité, les Babyloniens savaient résoudre certaines équations quadratiques (de degré 2),
mais ce n'est qu'à la Renaissance que des formules générales ont été établies pour les équations
cubiques (de degré 3, formules de Cardan) et quartiques (de degré 4, méthode de Ferrari). Bien que
ces méthodes s'avèrent en général complexes, elles garantissent qu'il est toujours possible
\textit{en principe} d'aboutir à une expression \textit{explicite} des solutions de n'importe quelle
équation de degré au plus 4.

Cs formules illustrent un phénomène fondamental : les équations quadratiques, cubiques et quartiques
sont résolubles par \textit{radicaux}. Un radical, dans ce contexte, désigne une expression
construite uniquement à partir d'opérations "élémentaires": des nombres rationnels, des additions,
soustractions, multiplications, divisions, et des racines (carrées, cubiques, ou d'ordre supérieur).
Par exemple : \( x = \sqrt[3]{2 + \sqrt{5}} \), \( y = \frac{1}{4}\sqrt{3} \), etc.

Toutefois, ces méthodes se heurtent à leurs limites pour les équations de degré supérieur à 4. À
partir du XIXe siècle, les travaux d'Abel et de Galois ont démontré que certaines équations
polynomiales de degré 5 ne sont pas résolubles par radicaux. En d'autres termes, il n'existe pas de
formule générale exprimant leurs solutions à l'aide d'opérations élémentaires.

\bigskip
\textbf{Objectifs}

Comprendre les principes fondamentaux permettant la résolution explicite des équations quadratiques,
cubiques et quartiques par radicaux. \\
Mettre en évidence les limites des méthodes classiques pour les équations de degré supérieur à 4. \\
Explorer des exemples concrets illustrant l'existence de polynômes dont les solutions ne peuvent pas
être obtenues par radicaux.

\bigskip
\textbf{Résolubilité par radicaux}

\bigskip
\textit{Equations Quadratiques (Degré 2)}

\q Soit une équation quadratique générale de forme \( X^2 + pX + q = 0 \) avec \( p, q \in \mathbb{C} \).
Justifier brièvement pourquoi une équation quadratique est toujours résoluble par radicaux.
% Objectif : Introduire la notion de résolution par radicaux dans un cas simple.
% Méthode : Utiliser la formule du discriminant pour exhiber explicitement les racines sous forme de
% radicaux.

\bigskip
\textit{Equations Cubiques (Degré 3)}

\q Soit \( P(X) = X^3 - 3X + 2 \).
\begin{itemize}
    \item Vérifier que \( P \) est factorisable sur \( \mathbb{C} \).
    \item En déduire une méthode élémentaire de résolution par radicaux valable pour les équations
    cubiques.
\end{itemize}
% Objectif : Appliquer concrètement la résolution par radicaux à un cas simple de degré 3.
% Méthode : Déterminer les racines réelles en factorisant à la main et exprimer la résolution sous
% forme de radicaux.

\q Soit \( Q(X) = X^3 + 2 \).
\begin{itemize}
    \item Déterminer ses racines.
    \item Discuter la nature des radicaux impliqués dans l'expression des solutions.
\end{itemize}
% Objectif : Introduire le lien entre résolution par radicaux et nombres complexes.
% Méthode : Utiliser la forme exponentielle et les racines de l'unité pour exprimer les solutions
% sous une forme exploitable.

\q (Passer cette question) \textit{Méthode de Cadran} Soit une équation cubique générale de la forme \( X^3 + aX^2 + bX + c = 0 \).
\begin{itemize}
    \item Introduire la substitution \( X = Y - \frac{a}{3} \) et montrer qu'elle permet de l'étude
    aux termes de degré 3 et inférieur.
    \item En posant \( Y = z + \frac{m}{z} \) pour un certain paramètre \( m \), montrer que
    cette substitution conduit à une équation du second degré en \( z^3 \).
    \item Résoudre cette équation en \( z^3 \), en exprimant \( z \) à l'aide de radicaux.
    \item En déduire la forme explicite des solutions de l'équation cubique initiale.
\end{itemize}
% Objectif : Guider la démonstration de la résolubilité des équations cubiques par radicaux.
% Méthode :
% 1. Utiliser la substitution X = Y - a/3 pour obtenir une équation de la forme Y^3 + pY + q = 0.
% 2. Introduire une substitution de la forme Y = z + m/z, qui permet d'obtenir une équation en z^3.
% 3. Résoudre cette équation par la méthode classique et exprimer Y à l'aide de radicaux.
% 4. Revenir à la variable initiale X.

\q (Passer cette question) Soit l'équation quartique $ X^4 - 10X^2 + 9 = 0 $.
\begin{itemize}
    \item Résoudre cette équation en effectuant un changement de variable.
    \item En déduire une méthode de résolution par radicaux.
\end{itemize}
% Objectif : Montrer que certaines équations quartiques peuvent être réduites à des quadratiques.
% Méthode : Introduire une substitution qui transforme l'équation en un cas élémentaire, puis
% exhiber les solutions sous forme de radicaux.

\bigskip
\textbf{Limites des méthodes classiques}

\q Soit l'équation quartique \( P(X) = X^4 + X^3 + X^2 + X + 1 \).
\begin{itemize}
    \item Vérifier si \( P(X) \) est décomposable sur \( \mathbb{R} \) en deux équations
    quadratiques par les méthodes classiques.
    \item En observant une relation avec le polynôme $X^5 - 1$, en déduire une factorisation de
    $P(X)$.
    \item Comparer avec l'équation quartique résolue à la question précédente, et expliquer
    pourquoi les méthodes élémentaires de résolution ne s'appliquent pas directement (cette
    équation est résoluble par radicaux par des méthodes plus avancées).
\end{itemize}
% Objectif : Introduire une équation quartique qui ne se factorise pas simplement et qui ne se prête
% pas immédiatement à une résolution par radicaux.
% Méthode : Utiliser le critère d'irréductibilité et poser la question de l'existence d'une
% factorisation en produits de polynômes de degré inférieur.

\q Soit l'équation quintique $ P(X) = X^5 - 3X^4 + 3X^3 - X^2 + X - 1 $.
\begin{itemize}
    \item Donner l'expression factorisée complète de $ P(X) $, si possible.
    \item Discuter si toutes les racines peuvent être trouvées par des méthodes élémentaires.
\end{itemize}
% Objectifs : Mettre en évidence qu'un polynôme de degré 5 peut ne pas être complètement
% factorisable avec des outils élémentaires.
% Méthodes : Utiliser la divisibilité par une racine entière évidente (1), factoriser
% successivement.

\q Soit l'équation quintique \( P(X) = X^5 - X - 1 \). Pour ce polynôme, il a été démontré qu'il
n'existe pas de solution sous forme de radicaux.\\
Justifier par un argument de degré que \( P(X) \) admet néanmoins au moins une racine réelle.
% Objectif : Aborder une équation polynomiale de degré 5 qui n'est pas résoluble par radicaux.
% Méthode : Utiliser le critère d'Eisenstein modulo 2 ou 3, ou une analyse sur ses racines
% rationnelles potentielles.

\q À partir des exemples précédents, proposer une conjecture sur les conditions nécessaires pour
qu'une équation soit résoluble par radicaux.
% Objectif : Formuler une intuition sur la résolubilité par radicaux avant d'introduire
% rigoureusement les critères issus de la théorie de Galois.
% Méthode : Comprendre qu'une équation polynomiale est résoluble par radicaux si sa structure permet
% une factorisation successive en sous-polynômes plus simples.

\bigskip
\textbf{Relations Symétriques}

La résolubilité des équations polynomiales par radicaux est intimement liée à la structure des
racines du polynôme, et plus précisément aux symétries qui existent entre ces racines.

Les racines d'un polynôme sont liées entre elles par des relations impliquant les coefficients du
polynôme, connues sous le nom de \textit{relations de Viète}. Ces relations sont
\textit{symétriques}, c'est-à-dire qu'elles ne dépendent pas de l'ordre des racines. Par exemple,
pour l'équation $x^2 - 1 = 0$, la somme des racines ($1 + (-1)$) est égale à $0$, et leur produit
($1 \times (-1)$) est égal à $-1$, indépendamment de l'ordre des racines dans ces expressions.

\q Démontrer que le nombre de relations symétriques entre les racines est égal au degré du polynôme
et exprimer ces relations en fonction des coefficients du polynôme.

\bigskip
La méthode de résolution par radicaux repose sur deux principes fondamentaux :
\begin{itemize}
    \item Réduire progressive des relations entre les racines à des expressions plus simples.
    \item Exploiter des symétries de ces relations pour construire explicitement les racines.
\end{itemize}

Or, la complexité de ces contraintes de symétrie augmente avec le degré du polynôme. À partir du
degré 5, il peut exister des relations entre les racines qui ne peuvent pas être "réduites" à des
opérations élémentaires.


\end{document}
