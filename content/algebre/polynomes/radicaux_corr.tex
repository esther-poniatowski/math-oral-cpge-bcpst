% CORRECTION :
% ==================================================================================================

\documentclass[10pt,a4paper]{article}

% Set the root path
\providecommand{\rootpath}{../../..}
% Fonts
\usepackage[utf8]{inputenc} % for accents
\usepackage[T1]{fontenc} % for accents
\usepackage[french]{babel} % for french language
\usepackage{helvet} % sans serif font family
\renewcommand*\familydefault{\sfdefault} % sans serif font family

% Mathematics
\usepackage{amsmath,amsfonts,amssymb} % for math symbols
\usepackage{array} % for tabular


\usepackage{parskip} % no indent, space between paragraphs

\usepackage{geometry} % margin
\geometry{
    a4paper,
    left=15mm,
    right=15mm,
    top=20mm,
    bottom=20mm
}

\usepackage{circledsteps} % to draw circles around numbers

\usepackage{fancyhdr} % for headers and footers

\usepackage{enumitem} % for customizing lists
\setlist[enumerate]{itemsep=1em} % space between items only in enumerate environment (not itemize)
\setlist[itemize]{label=--} % set itemize label to em-dash

% Command: \customPageLayout{#1}{#2}{#3}
% --------------------------------------
% Description: Custom page layout with header and footer content.
% Arguments:
% #1: Header and footer content
% #2: Left header content
% #3: Right header content
% Example:
% \customPageLayout{Title}{Lycée Henri IV}{2024}
% Required Packages: fancyhdr
\newcommand{\customPageLayout}[3]{
    \pagestyle{fancy} % set page style to fancy (add header and footer)
    \fancyhf{} % clear all header and footer content
    \lhead{#2} % left header content
    \rhead{#3} % right header content
    \chead{\textbf{#1}} % center header content in bold (if needed)
    \rfoot{\thepage} % page number in the footer
}


% Counter: \q
% -----------
% Description: Display a question number in a circle.
% Usage:
% - Create a new question: add \q followed by the question content.
% - Reset the question counter: add \setcounter{q}{0} before the first question.
\newcounter{q}
\setcounter{q}{0} % set initial value of the counter
\newcommand{\q}{
    \bigskip
    \addtocounter{q}{1}
    \par
    \Circled{\textbf{\theq}} \space
}


% Counter: \ql
% ------------
% Description: Display a question letter in a round box with indentation (lowercase and not bold).
% Usage:
% - Create a new question: add \ql followed by the question content.
% - Reset the question counter: add \setcounter{ql}{0} before the first question.
\newcounter{ql}
\setcounter{ql}{0} % set initial value of the counter
\newcommand{\ql}{
    \addtocounter{ql}{1}
    \par
    \hspace{1.5em} % indentation before the circled letter
    \textcolor{gray}{\Circled{\alph{ql}}} \space % gray color
}


\title{Polynômes - Résolubilité des équations polynomiales par radicaux}
\author{Esther Poniatowski}
\date{2024-2025}

\customPageLayout{Correction}{Lycée Henri IV}{2024}

\begin{document}

\bigskip
\textbf{Résolubilité par radicaux}

% --------------------------------------------------------------------------------------------------
\bigskip
\textit{Equations Quadratiques (Degré 2)}

\q Résolution d'une équation quadratique par radicaux de forme \( X^2 + pX + q = 0 \)

La formule du discriminant permet d'exprimer les solutions sous forme de radicaux.
\[
\Delta = p^2 - 4q
\]

Si \( \Delta \neq 0 \), l'équation admet deux solutions distinctes données par :
\[
X = \frac{-p \pm \sqrt{\Delta}}{2}
\]
Si \( \Delta = 0 \), il y a une unique solution :
\[
X = \frac{-p}{2}
\]
Si \( \Delta < 0 \), c'est-à-dire si \( p^2 - 4q \) est un nombre négatif, alors l'extraction de \(
\sqrt{\Delta} \) se fait dans l'ensemble des nombres complexes, en posant :
\[
\sqrt{\Delta} = i \sqrt{|\Delta|}
\]

L'expression des solutions montre qu'elles peuvent toujours être obtenues par additions,
multiplications et extraction de racines carrées, ce qui correspond à la définition de la
résolubilité par radicaux.
L'opération principale est l'extraction de racine carrée de \( \Delta \), ce qui reste dans le
cadre des radicaux.

Conclusion : L'équation quadratique est donc toujours résoluble par radicaux, quels que soient \( p
\) et \( q \).

% --------------------------------------------------------------------------------------------------
\bigskip
\textit{Equations Cubiques (Degré 3)}

\q Soit \( P(X) = X^3 - 3X + 2 \).

Vérification de la factorisation de \( P(X) \) sur \( \mathbb{C} \)

Cherchons des racines rationnelles en utilisant le théorème des racines rationnelles. Les diviseurs
du terme constant \( 2 \) sont \( \pm 1 \) et \( \pm 2 \).
\[
P(1) = 1^3 - 3(1) + 2 = 1 - 3 + 2 = 0
\]
Ainsi, \( X = 1 \) est une racine de \( P(X) \). On peut donc factoriser \( P(X) \) par \( (X - 1)
\) en effectuant une division euclidienne :
\[
P(X) = (X - 1)(X^2 + X - 2)
\]
Factorisons maintenant \( X^2 + X - 2 \), qui est un polynôme quadratique, en recherchant ses
racines :
\[
\Delta = 1^2 - 4(-2) = 1 + 8 = 9
\]
Solutions :
\[
X = \frac{-1 \pm \sqrt{9}}{2} = \frac{-1 \pm 3}{2}.
\]
On obtient donc \( X = 1 \) et \( X = -2 \), d'où la factorisation complète :
\[
P(X) = (X - 1)(X + 2)(X - 1) = (X - 1)^2 (X + 2)
\]

Résolution de l'équation \( P(X) = 0 \) par radicaux

L'équation \( P(X) = 0 \) possède trois racines réelles : \( 1 \) (de multiplicité 2) et \( -2 \).
Ces racines ont été obtenues par décomposition en facteurs linéaires, ce qui prouve que l'équation
est résoluble par radicaux.

% --------------------------------------------------------------------------------------------------

\q Soit \( Q(X) = X^3 + 2 \).

L'équation \( Q(X) = 0 \) revient à résoudre $X^3 = 2$.

Les solutions de cette équation sont les racines cubiques de \( -2 \). Tout nombre complexe non nul
admet exactement $n$ racines $n$-ièmes distinctes dans $\mathbb{C}$ (théorème fondamental de
l'algèbre), donc on cherche donc trois solutions.

Réécriture exponentielle de $-2$ :
$$
-2 = 2 \cdot e^{i\pi} \quad (\text{module } 2,\ \text{argument } \pi)
$$

On cherche le module et les arguments tels que
$$X_k^3 = 2e^{i(\pi + 2k\pi)} \quad \text{pour } k = 0, 1, 2$$
Conclusion :
$$X_k = \sqrt[3]{2} e^{i(\pi + 2k\pi)/3}$$

\[
X_k = \sqrt[3]{2} e^{i \frac{2k\pi}{3}}, \quad k \in \{0,1,2\}
\]

Calcul explicite de ces trois racines :

Pour \( k = 0 \) : \( X_0 = \sqrt[3]{2} e^{i\pi/3} = \sqrt[3]{2} \left(\frac{1}{2} +
i\frac{\sqrt{3}}{2}\right) \)\\
Pour \( k = 1 \) : \( X_1 = \sqrt[3]{2} e^{i\pi} = -\sqrt[3]{2} \)\\
Pour \( k = 2 \) : \( X_2 = \sqrt[3]{2} e^{i5\pi/3} = \sqrt[3]{2} \left(\frac{1}{2} - i\frac{\sqrt{3}}{2}\right) \)

Nature des radicaux impliqués

Les racines trouvées contiennent une racine cubique réelle \( \sqrt[3]{2} \) et des facteurs
complexes impliquant des racines de l'unité. Ces solutions sont donc obtenues en combinant des
extractions de racines cubiques et l'utilisation de nombres complexes.

% --------------------------------------------------------------------------------------------------

\q Soit l'équation cubique générale \( X^3 + aX^2 + bX + c = 0 \).

\begin{enumerate}
    \item Substitution pour éliminer le terme quadratique :
    \[
    X = Y - \frac{a}{3}
    \]

    Cette transformation repose sur l'idée de centrer les racines de l'équation autour de l'origine
    pour simplifier l'expression du polynôme. En substituant \( X \) dans l'équation initiale, on
    calcule les nouveaux coefficients en développant :
    \[
    \left(X + \frac{a}{3}\right)^3 + a\left(X + \frac{a}{3}\right)^2 + b\left(X + \frac{a}{3}\right) + c = 0
    \]

    Développement de chaque terme :
    \[
    \left(X + \frac{a}{3}\right)^3 = X^3 + 3 \frac{a}{3} X^2 + 3 \left(\frac{a}{3}\right)^2 X + \left(\frac{a}{3}\right)^3
    \]
    \[
    a \left(X + \frac{a}{3}\right)^2 = a \left(X^2 + 2 \frac{a}{3} X + \left(\frac{a}{3}\right)^2 \right)
    \]
    \[
    b \left(X + \frac{a}{3}\right) = bX + b \frac{a}{3}
    \]

    En réarrangeant les termes et en développant complètement, on obtient une nouvelle équation de
    la forme :
    \[
    Y^3 + pY + q = 0
    \]
    où les nouveaux coefficients sont donnés par :
    \[
    p = b - \frac{a^2}{3}, \quad q = c - \frac{ab}{3} + \frac{2a^3}{27}
    \]

    L'équation obtenue ne comporte plus de terme en \( Y^2 \), ce qui simplifie la résolution de
    l'équation cubique.

    \item Substitution \( Y = z + \frac{m}{z} \) où \( m \) est un paramètre à déterminer.
    L'objectif est de transformer l'équation en une équation quadratique en \( z^3 \).

    En remplaçant \( Y \) dans l'équation réduite, on obtient :
    \[
    \left(z + \frac{m}{z}\right)^3 + p\left(z + \frac{m}{z}\right) + q = 0
    \]

    En développant le cube : $\left(z + \frac{m}{z}\right)^3 = z^3 + \frac{m^3}{z^3} + 3m\left(z +
    \frac{m}{z}\right)$, donc :
    \[
      z^3 + \frac{m^3}{z^3} + 3m\left(z + \frac{m}{z}\right) + p\left(z + \frac{m}{z}\right) + q = 0
    \]

    Pour éliminer les termes linéaires ($z$ et $\frac{1}{z}$), on impose :
    $$
    3m + p = 0 \quad \Rightarrow \quad m = -\frac{p}{3}
    $$
    Ainsi :
    \[
    z^3 + \frac{m^3}{z^3} + q = 0.
    \]

    En posant \( u = z^3 \), on obtient une équation quadratique en \( u \) :
    \[
    u^2 + qu - m^3 = 0
    \]

    \item Résolution en \( z^3 \)

    Discriminant :
    \[
    \Delta = q^2 + 4m^3
    \]

    Solutions :
    \[
    z^3 = \frac{-q \pm \sqrt{q^2 + 4m^3}}{2}.
    \]

    On en déduit \( z \) en prenant les racines cubiques des deux solutions possibles.

    \item Expression des solutions finales

    Les valeurs de \( z \) étant obtenues par extraction de racines cubiques, on en déduit les valeurs
    de \( Y \), puis celles de \( X \) en réintroduisant la transformation initiale :
    \(
    X = Y - \frac{a}{3}
    \)

    \item Conclusion : Les solutions de l'équation cubique s'expriment sous forme de radicaux, en
    combinant une extraction de racines carrées et une extraction de racines cubiques.

    Cette méthode permet d'obtenir les racines d'une équation cubique en utilisant uniquement des
    opérations algébriques et des radicaux.
\end{enumerate}

% --------------------------------------------------------------------------------------------------
\bigskip
\textbf{Limites des méthodes classiques}

\q Soit l'équation quartique \( X^4 - 10X^2 + 9 = 0 \).

\begin{enumerate}
    \item Changement de variable :
    \[
    Y = X^2
    \]
    En remplaçant dans l'équation initiale, on obtient une équation quadratique en \( Y \) :
    \[
    Y^2 - 10Y + 9 = 0
    \]

    Discriminant :
    \[
    \Delta = (-10)^2 - 4 \times 9 = 100 - 36 = 64
    \]

    Solutions :
    \[
    Y = \frac{10 \pm \sqrt{64}}{2} = \frac{10 \pm 8}{2}
    \]
    \[
    Y = \frac{10 + 8}{2} = \frac{18}{2} = 9, \quad Y = \frac{10 - 8}{2} = \frac{2}{2} = 1
    \]

    En revenant à la variable \( X \), on obtient deux équations :
    \[
    X^2 = 9 \quad \text{et} \quad X^2 = 1
    \]

    \item Résolution par radicaux

    Les solutions de ces équations sont directement obtenues en extrayant la racine carrée :
    \[
    X = \pm \sqrt{9} = \pm 3, \quad X = \pm \sqrt{1} = \pm 1
    \]

    Ainsi, les solutions de l'équation initiale sont :
    \[
    X \in \{-3, -1, 1, 3\}.
    \]

    \item Conclusion : L'équation quartique a été réduite à une équation quadratique par la substitution \( Y = X^2 \),
    puis résolue en extrayant successivement des racines carrées. Cette méthode montre que l'équation
    est résoluble par radicaux.
\end{enumerate}

% --------------------------------------------------------------------------------------------------

\q Soit \( P(X) = X^4 + X + 1 \).

\begin{enumerate}
    \item Vérification de la décomposabilité sur \( \mathbb{R} \) en deux équations quadratiques.

   On cherche à factoriser $ P(X) = X^4 + X^3 + X^2 + X + 1 $ en un produit de deux polynômes
   quadratiques réels :
   $$
   P(X) = (X^2 + aX + b)(X^2 + cX + d) \quad \text{avec } a, b, c, d \in \mathbb{R}
   $$

   Développement :
   $$
   X^4 + (a + c)X^3 + (ac + b + d)X^2 + (ad + bc)X + bd = X^4 + X^3 + X^2 + X + 1
   $$

   Par identification, on obtient un système d'équations :
   $$\begin{cases}
      a + c = 1 & \text{(coefficient de \( X^3 \))} \\
      ac + b + d = 1 & \text{(coefficient de \( X^2 \))} \\
      ad + bc = 1 & \text{(coefficient de \( X \))} \\
      bd = 1 & \text{(terme constant)}
   \end{cases}$$

   Résolution :\\
   - De (1) : $ c = 1 - a $\\
   - De (4) : $ d = \frac{1}{b} $

   Substitution dans (2) :
   $$
   a(1 - a) + b + \frac{1}{b} = 1 \quad \Rightarrow \quad -a^2 + a + b + \frac{1}{b} = 1
   $$

   Substitution dans (3) :
   $$
   a \cdot \frac{1}{b} + b \cdot (1 - a) = 1 \quad \Rightarrow \quad \frac{a}{b} + b - ab = 1
   $$

   De l'équation (3), isolons $a$ :
   $$a = \frac{1-b}{b-b^2}$$

   Substituons cette expression de $a$ dans l'équation (2) :
   $$-\left(\frac{1-b}{b-b^2}\right)^2 + \frac{1-b}{b-b^2} + b + \frac{1}{b} = 1$$

   Multiplions tous les termes par $(b-b^2)^2b$ :
   $$-b(1-b)^2 + (1-b)(b-b^2)b + b^2(b-b^2)^2 + (b-b^2)^2 = (b-b^2)^2b$$

   Développons et simplifions:
      $$-b(1-2b+b^2) + (b-b^2-b^2+b^3)b + b^2(b^2-2b^3+b^4) + (b^2-2b^3+b^4) = b^3-2b^4+b^5$$
      $$-b+2b^2-b^3 + b^2-b^3-b^3+b^4 + b^4-2b^5+b^6 + b^2-2b^3+b^4 = b^3-2b^4+b^5$$
      $$b^6 - 2b^5 + 2b^4 + 3b^2 - 4b^3 - b = b^3 - 2b^4 + b^5$$
      $$b^6 - 3b^5 + 4b^4 - 5b^3 + 3b^2 - b = 0$$
      $$b(b^5 - 3b^4 + 4b^3 - 5b^2 + 3b - 1) = 0$$

   Les solutions de cette équation sont $b = 0$ (qui n'est pas valide car $b \neq 0$ d'après
   l'équation (4) $bd = 1$) et les racines du polynôme de degré 5 :
      $$b^5 - 3b^4 + 4b^3 - 5b^2 + 3b - 1 = 0$$

   Ce polynôme de degré 5 n'a pas de solution simple.

   Conclusion : La résolution algébrique directe du système d'équations mène à une équation de degré
   5 qui n'a pas de solution simple. Cela montre que l'approche directe par résolution du système
   d'équations est impraticable, justifiant ainsi l'utilisation de la méthode des racines de l'unité
   pour factoriser le polynôme.

    \item Décomposition par les complexes

    Le polynôme $X^4 + X + 1$ est le quotient du polynôme $X^5 - 1$ par $X - 1$ :
    $$X^5 - 1 = (X - 1)(X^4 + X^3 + X^2 + X + 1)$$

    Les racines de $X^5 - 1 = 0$ sont les racines 5-ièmes de l'unité :
    $$e^{2i\pi k/5}, \quad k \in \{0,1,2,3,4\}$$ L'une d'entre elles est la racine triviale $X = 1$,
    qui annule $X - 1$, ce qui signifie que les racines de $X^4 + X^3 + X^2 + X + 1$ sont les quatre
    racines non triviales de l'unité :
    $$\alpha = e^{2i\pi/5}, \quad \alpha^2 = e^{4i\pi/5}, \quad \alpha^3 = e^{6i\pi/5}, \quad
    \alpha^4 = e^{8i\pi/5}$$

    Pour factoriser $P(X)$ sur $\mathbb{R}$, il faut associer les racines en paires conjuguées :
    $\alpha = e^{2i\pi/5}$ et $\alpha^4 = e^{-2i\pi/5}$, $\alpha^2 = e^{4i\pi/5}$ et $\alpha^3 =
    e^{-4i\pi/5}$

    Ces paires conjuguées permettent de construire deux polynômes quadratiques réels en utilisant
    les relations trigonométriques :
    $$e^{i\theta} + e^{-i\theta} = 2\cos\theta, \quad e^{i\theta} e^{-i\theta} = 1$$ En appliquant
    ces relations aux racines :
    $$e^{2i\pi/5} + e^{-2i\pi/5} = 2\cos(2\pi/5)$$
    $$e^{4i\pi/5} + e^{-4i\pi/5} = 2\cos(4\pi/5)$$

    Ainsi, les polynômes quadratiques sont :
    $$(X^2 - 2\cos(2\pi/5) X + 1) \quad \text{et} \quad (X^2 - 2\cos(4\pi/5) X + 1)$$

    En exprimant en termes de radicaux :
    $$\cos(2\pi/5) = \frac{\sqrt{5} - 1}{4}, \quad \cos(4\pi/5) = -\frac{\sqrt{5} + 1}{4}$$

    On obtient alors la factorisation exacte :
    $$P(X) = \left(X^2 - \frac{\sqrt{5} - 1}{2} X + 1\right) \left(X^2 + \frac{\sqrt{5} + 1}{2} X +
    1\right)$$

    Les coefficients $a,b,c,d$ recherchés dans le système initial sont donc :
    $$a = -\frac{\sqrt{5} - 1}{2}, \quad b = 1, \quad c = \frac{\sqrt{5} + 1}{2}, \quad d = 1$$

    Vérification des valeurs avec le système d'équations initial :\\
      - Condition (1) : $$ a + c = -\frac{\sqrt{5} - 1}{2} + \frac{\sqrt{5} + 1}{2} = 1 $$ \\
      - Condition (4) : $$ b \cdot d = 1 \cdot 1 = 1 $$ \\
      - Condition (2) :
      $$
      ac + b + d = \left(-\frac{\sqrt{5} - 1}{2}\right)\left(\frac{\sqrt{5} + 1}{2}\right) + 1 + 1 = -\frac{4}{4} + 2 = 1 \\
      $$
      - Condition (3) :
      $$
      ad + bc = \left(-\frac{\sqrt{5} - 1}{2}\right)(1) + (1)\left(\frac{\sqrt{5} + 1}{2}\right) = \frac{-\sqrt{5} + 1 + \sqrt{5} + 1}{2} = 1
      $$

    \item Comparaison avec l'équation quadratique précédente

    L'équation quartique a pu être résolue en effectuant un changement de variable qui la ramenait à
    une équation quadratique, puis en extrayant des racines carrées.

    Ici, \( P(X) \) ne peut pas être transformé de manière similaire car ils ne se décompose pas
    simplement  en polynômes de degré inférieur.

\end{enumerate}

% --------------------------------------------------------------------------------------------------

\q Soit le polynôme \( P(X) = X^5 - 3X^4 + 3X^3 - X^2 + X - 1 \).
\begin{enumerate}
    \item Étude de la divisibilité

    Tentons de factoriser \( P(X) \) en utilisant le théorème du reste pour vérifier si \( X - 1 \)
    est un facteur de \( P(X) \).
    \[
    P(1) = 1 - 3 + 3 - 1 + 1 - 1 = 0
    \]
    Donc, \( X - 1 \) divise \( P(X) \).

    \item Division euclidienne de $ P(x) $ par $ x - 1 $ :

    Divisons le terme dominant :
    $$ \frac{x^5}{x} = x^4 $$

    Multiplions :
    $$ (x - 1)(x^4) = x^5 - x^4 $$

    Soustrayons :
    $$ (x^5 - 3x^4 + \dots) - (x^5 - x^4) = -2x^4 + 3x^3 $$

    Recommençons avec le nouveau terme dominant :
    $$ \frac{-2x^4}{x} = -2x^3 $$

    Multiplions :
    $$ (x - 1)(-2x^3) = -2x^4 + 2x^3 $$

    Soustrayons :
    $$ (-2x^4 + 3x^3) - (-2x^4 + 2x^3) = x^3 $$

    Continuer ainsi jusqu'à obtenir un reste nul.

    Le quotient obtenu est :
    $$
    Q(x) = x^4 - 2x^3 + x^2 - x + 1
    $$

    Factorisation partielle :
    $$
    P(x) = (x - 1)(Q(x)),
    \quad \text{où } Q(x) = x^4 - 2x^3 + x^2 - x + 1
    $$

    \item Étude des racines de $ Q(x) $

    Recherchons des racines évidentes parmi les entiers ($$ \pm 1, \pm 2, \dots $$), par exemple :
    $$
    Q(1) = (1)^4 - 2(1)^3 + (1)^2 - (1) + 1 = 1 - 2 + 1 - 1 + 1 = 0
    $$
    Aindi $ Q(x) $ est également divisible par $ x - 1 $.

    \item Division de $ Q(x) $ par $ x - 1 $ :

    En répétant la méthode précédente, on obtient le quotient suivant :
    $$
    Q(x) = (x-1)(x^3-x^2+x-1)
    $$

    Factorisation partielle :
    $$
    P(x) = (x-1)^2(x^3-x^2+x-1)
    $$

    \item Étude du polynôme restant : $ R(x) = x^3-x^2+x-1 $

    On recherche à nouveau les racines évidentes, par exemple :
    $$
    R(1) = (1)^3-(1)^2+(1)-1 = 0
    $$
    Ainsi, $ R(x) $ est également divisible par $ x-1 $.

    Division finale : :
    $$
    R(x) = (x-1)(x^2+0\cdot x+1)
    $$

    \item Factorisation complète de $ P(x) $ :
    $$
    P(x) = (x - 1)^3(x^2 + 1)
    $$

    \item Conclusion : Le polynôme est factorisé en deux parties distinctes.
    \begin{itemize}
        \item $ (x-1)^3 $, qui correspond à trois racines réelles égales à $ x=1 $.
        \item $ x^2+1 $, qui n'a pas de racines réelles mais deux racines complexes données par
        $ x = i $ et $ x = -i $.
    \end{itemize}
\end{enumerate}

% --------------------------------------------------------------------------------------------------
\q Soit le polynôme \( P(X) = X^5 - X - 1 \).

Existence d'au moins une racine réelle :

Tout polynôme réel peut se décomposer en un produit de facteurs irréductibles réels. Ces
facteurs sont soit :
\begin{itemize}
    \item Des polynômes de degré 1 (ayant une racine réelle),
    \item Des polynômes quadratiques irréductibles (ayant des racines complexes conjuguées).
\end{itemize}
Si un polynôme est de degré impair, alors le nombre total de ses racines (réelles ou complexes)
est impair. Cela implique qu'il doit y avoir au moins un facteur de degré 1 dans la
décomposition, donc au moins une racine réelle.

% --------------------------------------------------------------------------------------------------
\q Conjecture

À partir de ces observations, on remarque que les équations de degré 2, 3 et 4 peuvent être résolues
par radicaux, mais qu'à partir du degré 5, certaines équations ne le sont plus. Cela suggère que la
résolubilité par radicaux ne dépend pas uniquement du degré de l'équation, mais aussi de la manière
dont ses racines sont liées entre elles.

% --------------------------------------------------------------------------------------------------
\bigskip
\textbf{Relations Symétriques}

\q Relations de Viete

Soit $P(x)$ un polynôme de degré $n$ :
$$P(x) = a_nx^n + a_{n-1}x^{n-1} + \cdots + a_1x + a_0$$
où $a_n \neq 0$, et soient $r_1, r_2, \ldots, r_n$ ses racines (pas nécessairement distinctes).

D'après le théorème fondamental de l'algèbre, ce polynôme peut s'écrire :
$$P(x) = a_n(x-r_1)(x-r_2)\cdots(x-r_n)$$

Développement du produit :

Terme de degré $n$ : $$a_nx^n$$
Terme de degré $n-1$ : $$-a_n(r_1 + r_2 + \cdots + r_n)x^{n-1}$$
Terme de degré $n-2$ : $$a_n(r_1r_2 + r_1r_3 + \cdots + r_{n-1}r_n)x^{n-2}$$
\dots
Terme constant : $$(-1)^na_n(r_1r_2\cdots r_n)$$

Comparaison des termes avec la forme initiale du polynôme :
$$a_{n-1} = -a_n(r_1 + r_2 + \cdots + r_n)$$
$$a_{n-2} = a_n(r_1r_2 + r_1r_3 + \cdots + r_{n-1}r_n)$$
$$a_{n-k} = (-1)^ka_ne_k \quad \text{pour } k = 1, 2, \ldots, n \quad \text{avec } e_k = \sum_{1\leq
i_1<i_2<\cdots<i_k\leq n} r_{i_1}r_{i_2}\cdots r_{i_k}$$
$$a_0 = (-1)^na_n(r_1r_2\cdots r_n)$$

Ces $n$ relations correspondent aux $n$ coefficients du polynôme (excluant $a_n$), et elles relient
les coefficients aux sommes symétriques élémentaires des racines.

\end{document}

% ==================================================================================================
Pour déterminer la valeur exacte de $$ \cos\left(\frac{2\pi}{5}\right) $$ en termes de radicaux, analysons les étapes clés à partir de différentes approches mathématiques :

---

### 1. **Équation polynomiale et identités trigonométriques**
Posons $$ \theta = \frac{2\pi}{5} $$. En exploitant l'égalité $$ \cos(2\theta) = \cos(3\theta) $$ (car $$ 2\theta + 3\theta = 2\pi $$), on développe :
$$
\cos(2\theta) = 2\cos^2\theta - 1, \quad \cos(3\theta) = 4\cos^3\theta - 3\cos\theta.
$$
En égalant les deux expressions :
$$
4\cos^3\theta - 2\cos^2\theta - 3\cos\theta + 1 = 0.
$$
En factorisant par $$ \cos\theta - 1 $$ (solution évidente $$ \theta = 0 $$), on obtient :
$$
(4\cos^2\theta + 2\cos\theta - 1)(\cos\theta - 1) = 0.
$$
Éliminant la solution triviale $$ \cos\theta = 1 $$, il reste :
$$
4\cos^2\theta + 2\cos\theta - 1 = 0.
$$
**Résolution quadratique** :
$$
\cos\theta = \frac{-1 \pm \sqrt{5}}{4}.
$$
Comme $$ \cos\left(\frac{2\pi}{5}\right) > 0 $$, on retient :
$$
\cos\left(\frac{2\pi}{5}\right) = \frac{\sqrt{5} - 1}{4}.
$$

---

### 2. **Racines cinquièmes de l'unité**
Les solutions de $$ z^5 = 1 $$ sont $$ z_k = e^{i\frac{2k\pi}{5}} $$, pour $$ k \in \{0,1,2,3,4\} $$. La somme des racines donne :
$$
1 + z_1 + z_2 + z_3 + z_4 = 0.
$$
En séparant parties réelles et imaginaires :
$$
1 + 2\cos\left(\frac{2\pi}{5}\right) + 2\cos\left(\frac{4\pi}{5}\right) = 0.
$$
Sachant que $$ \cos\left(\frac{4\pi}{5}\right) = -\cos\left(\frac{\pi}{5}\right) $$, on substitue :
$$
1 + 2\cos\left(\frac{2\pi}{5}\right) - 2\cos\left(\frac{\pi}{5}\right) = 0.
$$
En utilisant $$ \cos\left(\frac{\pi}{5}\right) = \frac{\sqrt{5} + 1}{4} $$, on retrouve :
$$
\cos\left(\frac{2\pi}{5}\right) = \frac{\sqrt{5} - 1}{4}.
$$

---
