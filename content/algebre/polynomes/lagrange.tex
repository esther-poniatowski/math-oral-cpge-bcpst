% PROBLÈME : Polynômes Orthogonaux
% ==================================================================================================

\documentclass[10pt,a4paper]{article}

% Set the root path
\providecommand{\rootpath}{../../..}
% Fonts
\usepackage[utf8]{inputenc} % for accents
\usepackage[T1]{fontenc} % for accents
\usepackage[french]{babel} % for french language
\usepackage{helvet} % sans serif font family
\renewcommand*\familydefault{\sfdefault} % sans serif font family

% Mathematics
\usepackage{amsmath,amsfonts,amssymb} % for math symbols
\usepackage{array} % for tabular


\usepackage{parskip} % no indent, space between paragraphs

\usepackage{geometry} % margin
\geometry{
    a4paper,
    left=15mm,
    right=15mm,
    top=20mm,
    bottom=20mm
}

\usepackage{circledsteps} % to draw circles around numbers

\usepackage{fancyhdr} % for headers and footers

\usepackage{enumitem} % for customizing lists
\setlist[enumerate]{itemsep=1em} % space between items only in enumerate environment (not itemize)
\setlist[itemize]{label=--} % set itemize label to em-dash

% Command: \customPageLayout{#1}{#2}{#3}
% --------------------------------------
% Description: Custom page layout with header and footer content.
% Arguments:
% #1: Header and footer content
% #2: Left header content
% #3: Right header content
% Example:
% \customPageLayout{Title}{Lycée Henri IV}{2024}
% Required Packages: fancyhdr
\newcommand{\customPageLayout}[3]{
    \pagestyle{fancy} % set page style to fancy (add header and footer)
    \fancyhf{} % clear all header and footer content
    \lhead{#2} % left header content
    \rhead{#3} % right header content
    \chead{\textbf{#1}} % center header content in bold (if needed)
    \rfoot{\thepage} % page number in the footer
}


% Counter: \q
% -----------
% Description: Display a question number in a circle.
% Usage:
% - Create a new question: add \q followed by the question content.
% - Reset the question counter: add \setcounter{q}{0} before the first question.
\newcounter{q}
\setcounter{q}{0} % set initial value of the counter
\newcommand{\q}{
    \bigskip
    \addtocounter{q}{1}
    \par
    \Circled{\textbf{\theq}} \space
}


% Counter: \ql
% ------------
% Description: Display a question letter in a round box with indentation (lowercase and not bold).
% Usage:
% - Create a new question: add \ql followed by the question content.
% - Reset the question counter: add \setcounter{ql}{0} before the first question.
\newcounter{ql}
\setcounter{ql}{0} % set initial value of the counter
\newcommand{\ql}{
    \addtocounter{ql}{1}
    \par
    \hspace{1.5em} % indentation before the circled letter
    \textcolor{gray}{\Circled{\alph{ql}}} \space % gray color
}


\title{Polynômes - Polynômes de Lagrange}
\author{Esther Poniatowski}
\date{2024-2025}

\customPageLayout{Sujets d'interrogation orale}{Lycée Henri IV}{2024}

\begin{document}

% ========

\bigskip
\textbf{Contexte}

Les polynômes de Lagrange sont une famille de polynômes qui interviennent dans d'autres domaines
mathématiques, notamment en analyse, pour aborder des problèmes d'interpolation :

Étant donnée une fonction $f$ passant par un ensemble de $n+1$ points $ (x_i, y_i) $ avec des $x_i$
distincts, existe-t-il un polynôme qui passe par chacun de ces points, c'est-à-dire tel que $P(x_i)
= y_i$ pour tout $i$ ?


\bigskip
\textbf{Objectifs}

Étudier la construction et les propriétés des polynômes de Lagrange.

% --------------------------------------------------------------------------------------------------
\bigskip
\textbf{Existence et Unicité}

%\q Quel est le degré minimal d'un polynôme passant par tous les points $ (x_i, y_i) $ ?
% But : Comprendre la relation entre le nombre de points et le degré du polynôme interpolateur.
% Méthode : Raisonner sur le nombre de coefficients à déterminer dans un polynôme de degré n.

\q Soit $P$ un polynôme de degré $n$. Montrer que si $P$ admet $n+1$ racines, alors il est
identiquement nul.
% Objectif : Introduire une propriété fondamentale des polynômes qui sera utilisée pour justifier
% l'unicité du polynôme interpolateur de Lagrange.
% Démarche : Raisonner par récurrence ou par division euclidienne pour montrer que si un polynôme de
% degré n a plus de n racines, alors il est identiquement nul.

\q En déduire que s'il existe un polynôme $P_n$ de degré au plus $n$ qui satisfait $P_n(x_i) = y_i$
pour tout $i$, alors il est unique.
% Objectif : Formaliser la construction du polynôme d'interpolation. Préparer l'introduction de la
% forme de Lagrange.
% Démarche : Appliquer la propriété précédente à un polynôme P−Q  qui annule n+1 valeurs.

\bigskip

Pour construire un tel polynôme explicitement, la méthode de Lagrange consiste à :
\begin{itemize}
   \item Construire $n + 1$ polynômes de "base" \( L_k(X) \) tels que pour $k \in \{0, ..., n\}$ :
   $$L_k(X) = \prod_{i=0, i\neq k}^n \frac{X-x_k}{x_k-x_i}$$
   \item Écrire le polynôme interpolant \( P_n(X) \) comme une somme pondérée des polynômes \(
   L_k(X) \) :
   $$P_n(X) = \sum_{k=0}^n \alpha_k L_k(X)$$
   où les coefficients \( \alpha_k \) sont déterminés par les conditions d'interpolation.
\end{itemize}

\q Propriétés des polynômes de base de Lagrange

\begin{enumerate}
   \item Montrer que le polynôme \( L_j \) est de degré \( n \).
   \item Évaluer \( L_j(x_k) \) pour \( k \in \{0, ..., n\} \).
\end{enumerate}
% But : Introduire les polynômes de base de Lagrange et étudier leurs propriétés fondamentales.
% Méthode : a) Développement du produit et analyse du degré. b) Évaluation directe en distinguant
% les cas k=j et k≠j.

\q Polynôme interpolateur de Lagrange

\begin{enumerate}
   \item Déterminer les coefficients de \( P_n(X) \) en utilisant les conditions d'interpolation.
   \item En déduire une expression explicite de \( P_n(X) \).
\end{enumerate}
% Objectif : Appliquer concrètement la formule des polynômes de Lagrange et comprendre la
% construction effective du polynôme interpolant.

% --------------------------------------------------------------------------------------------------
\bigskip
\textbf{Applications à l'Interpolation}

\q Soit la fonction $f : \mathbb{R} \rightarrow \mathbb{R}$ définie par $f(x) = \frac{1}{1 + x^2}$.
\begin{enumerate}
   \item Construire explicitement le polynôme de Lagrange $P_2$ interpolant $f$ aux points $x_0 = -1$,
   $x_1 = 0$ et $x_2 = 1$.
   \item Calculer $|f(0.5) - P_2(0.5)|$. Commenter la qualité de l'approximation.
   \item Effectuer une illustration graphique pour illustrer ce phénomène.
\end{enumerate}
% But : Appliquer la méthode de Lagrange à un exemple concret et évaluer la qualité de
% l'approximation.
% Méthode : Calcul numérique et interprétation.

% --------------------------------------------------------------------------------------------------
\bigskip
\textbf{Forme de Newton vs Lagrange}

Les polynômes d'interpolation de Lagrange et de Newton sont deux représentations différentes d'un
même polynôme interpolateur, mais ils sont construits selon des approches distinctes.

Le polynôme interpolateur de Lagrange est exprimé via les polynômes de base \( L_k(X) \) définis de
sorte à valoir 1 en \( x_k \) et 0 en tous les autres points :
\[
P_n(X) = \sum_{i=0}^{n} y_i L_i(X), \quad \text{où} \quad L_i(X) = \prod_{j \neq i} \frac{X - x_j}{x_i - x_j}
\]
Cette forme met explicitement en évidence que \( P_n(X) \) prend les valeurs imposées aux \( n+1 \)
points d'interpolation. Toutefois, elle nécessite de recalculer complètement le polynôme si un point
d'interpolation est ajouté ou modifié.

Le polynôme interpolateur de Newton est exprimé sous forme d'une somme pondérée de produits
successifs :
\[
P_n(X) = a_0 + a_1 (X - x_0) + a_2 (X - x_0)(X - x_1) + \dots + a_n (X - x_0)(X - x_1) \cdots (X - x_{n-1})
\]
%où les coefficients \( a_k \) sont déterminés à l'aide des différences divisées :
%\[
%a_k = f[x_0, x_1, ..., x_k] = \frac{f[x_1, ..., x_k] - f[x_0, ..., x_{k-1}]}{x_k - x_0}
%\]
Cette forme est plus efficace pour ajouter progressivement des points (méthode récursive), utile en
calcul numérique.

Soit $P_n$ le polynôme de Lagrange interpolant $n+1$ points $(x_i, y_i)$, $i = 0, \ldots, n$.

\q Exprimer le coefficient $a_n$ en fonction des $y_i$ et des $x_i$.

\q Pour trois points d'interpolation $(x_0, y_0)$, $(x_1, y_1)$ et $(x_2, y_2)$, construire le
polynôme de Newton itérativement en ajoutant progressivement des termes correspondant à des points
d'interpolation successifs. Conjecturer la forme générale des coefficients du polynôme de Newton.


\end{document}


% ==================================================================================================


\q \textbf{Extensions aux fractions rationnelles}

Si $y_i = f(x_i)$ où $f$ est rationnelle, comment $P_n(X)$ se relie-t-il à $f$ ? \\
% Lien : Prépare le terrain pour les décompositions en éléments simples et l'approximation
% rationnelle

\q \textbf{Généralizations}

\q Montrer que pour tout polynôme $Q$ de degré au plus $n$, on a :

$Q(X) = \sum_{j=0}^n Q(x_j) L_j(X)$

% But : Établir une propriété fondamentale des polynômes de Lagrange, qui généralise la construction
% du polynôme interpolateur. Méthode : Considérer la différence entre Q et la somme donnée, montrer
% qu'elle s'annule en tous les x_i, puis utiliser l'unicité.


\bigskip
\textbf{Extensions aux Complexes}

\q Interpoler des points $z_k \in \mathbb{C}$. Comment les propriétés d'unicité sont-elles
préservées ?
% Objectif : Généralisation naturelle mettant en jeu la factorialité dans C[X].

\q Déterminer $P_2(X)$ pour $z_k = e^{ik\pi/2}$. Analyser les symétries.
% Lien : Utilise les racines de l'unité pour illustrer des structures polynomiales remarquables


Soit $n \in \mathbb{N}^*$ et $(z_k, w_k) \in \mathbb{C}^2$, $k = 0, \ldots, n$, avec les $z_k$
distincts.

\q Montrer que le polynôme interpolateur de Lagrange $P \in \mathbb{C}[X]$ vérifiant $P(z_k) = w_k$
pour tout $k$ existe et est unique.

\q Donner l'expression de ce polynôme. En quoi diffère-t-elle du cas réel ?
% But : Généraliser l'interpolation de Lagrange aux nombres complexes et mettre en évidence les
% similitudes et différences avec le cas réel.
% Méthode : Adapter la preuve d'unicité du cas réel au cas complexe, puis construire explicitement
% le polynôme interpolateur.

\bigskip
On considère maintenant les racines n-ièmes de l'unité : $z_k = e^{2ik\pi/n}$, $k = 0, \ldots, n-1$.

\q Déterminer le polynôme interpolateur $P_n$ vérifiant $P_n(z_k) = z_k^m$ pour un $m \in
\mathbb{N}$ fixé.

\q Que se passe-t-il si $m \equiv 0 \pmod{n}$ ? Si $m \equiv 1 \pmod{n}$ ?

\q Montrer que si $m \not\equiv 0 \pmod{n}$, alors $P_n(X) = X^m$.
% But : Utiliser les propriétés des racines de l'unité pour mettre en évidence des structures
% polynomiales remarquables.
% Méthode : Exploiter les symétries des racines de l'unité et les propriétés de l'interpolation de
% Lagrange.

\bigskip
Soit $f(z) = \frac{1}{1-z}$ définie sur $\mathbb{C} \setminus \{1\}$.

\q Déterminer le polynôme interpolateur $P_n$ de $f$ aux points $z_k = e^{2ik\pi/n}$, $k = 0,
\ldots, n-1$.

\q Montrer que $P_n(z) = 1 + z + z^2 + \cdots + z^{n-1}$ pour tout $z \in \mathbb{C}$.

\q En déduire une preuve de la formule de la somme géométrique : $1 + z + z^2 + \cdots + z^{n-1} =
\frac{1-z^n}{1-z}$ pour $z \neq 1$.
% But : Utiliser l'interpolation complexe pour démontrer un résultat classique d'une manière
% originale.
% Méthode : Construire le polynôme interpolateur, l'identifier à la somme des puissances, puis
% utiliser les propriétés de l'interpolation pour conclure.
