% CORRECTION : Polynômes de Lagrange
% ==================================================================================================

\documentclass[10pt,a4paper]{article}

% Set the root path
\providecommand{\rootpath}{../../..}
% Fonts
\usepackage[utf8]{inputenc} % for accents
\usepackage[T1]{fontenc} % for accents
\usepackage[french]{babel} % for french language
\usepackage{helvet} % sans serif font family
\renewcommand*\familydefault{\sfdefault} % sans serif font family

% Mathematics
\usepackage{amsmath,amsfonts,amssymb} % for math symbols
\usepackage{array} % for tabular


\usepackage{parskip} % no indent, space between paragraphs

\usepackage{geometry} % margin
\geometry{
    a4paper,
    left=15mm,
    right=15mm,
    top=20mm,
    bottom=20mm
}

\usepackage{circledsteps} % to draw circles around numbers

\usepackage{fancyhdr} % for headers and footers

\usepackage{enumitem} % for customizing lists
\setlist[enumerate]{itemsep=1em} % space between items only in enumerate environment (not itemize)
\setlist[itemize]{label=--} % set itemize label to em-dash

% Command: \customPageLayout{#1}{#2}{#3}
% --------------------------------------
% Description: Custom page layout with header and footer content.
% Arguments:
% #1: Header and footer content
% #2: Left header content
% #3: Right header content
% Example:
% \customPageLayout{Title}{Lycée Henri IV}{2024}
% Required Packages: fancyhdr
\newcommand{\customPageLayout}[3]{
    \pagestyle{fancy} % set page style to fancy (add header and footer)
    \fancyhf{} % clear all header and footer content
    \lhead{#2} % left header content
    \rhead{#3} % right header content
    \chead{\textbf{#1}} % center header content in bold (if needed)
    \rfoot{\thepage} % page number in the footer
}


% Counter: \q
% -----------
% Description: Display a question number in a circle.
% Usage:
% - Create a new question: add \q followed by the question content.
% - Reset the question counter: add \setcounter{q}{0} before the first question.
\newcounter{q}
\setcounter{q}{0} % set initial value of the counter
\newcommand{\q}{
    \bigskip
    \addtocounter{q}{1}
    \par
    \Circled{\textbf{\theq}} \space
}


% Counter: \ql
% ------------
% Description: Display a question letter in a round box with indentation (lowercase and not bold).
% Usage:
% - Create a new question: add \ql followed by the question content.
% - Reset the question counter: add \setcounter{ql}{0} before the first question.
\newcounter{ql}
\setcounter{ql}{0} % set initial value of the counter
\newcommand{\ql}{
    \addtocounter{ql}{1}
    \par
    \hspace{1.5em} % indentation before the circled letter
    \textcolor{gray}{\Circled{\alph{ql}}} \space % gray color
}


\usepackage{pgfplots}
\pgfplotsset{compat=1.18}
%\usepgfplotslibrary{math}

\title{Polynômes - Polynômes de Lagrange}
\author{}
\date{2024}

\customPageLayout{Correction}{Lycée Henri IV}{2024}

\begin{document}
% --------------------------------------------------------------------------------------------------
\bigskip
\textbf{Existence et Unicité}


% --------------------------------------------------------------------------------------------------
\q Démonstration : \( n+1 \) racines et polynôme nul

Soit \( P \) un polynôme de degré \( n \) ayant \( n+1 \) racines distinctes \( x_0, x_1, \dots, x_n
\), c'est-à-dire $ P(x_i) = 0 $ pour tout \( i \in \{0, \dots, n\} \).

Par le théorème fondamental de l'algèbre, tout polynôme \( P \) ayant \( n+1 \) racines distinctes
est divisible par le polynôme produit des facteurs linéaires associés, donc il existe un polynôme \(
Q(X) \) tel que :
\[
P(X) = (X - x_0)(X - x_1) \dots (X - x_n) Q(X)
\]

Par l'absurde, supposons que $P$ ne soit pas nul. Cette condition impose que \( Q(X) \) soit au
moins un polynôme constant non nul. Alors, le degré de \( P \) serait \( n+1 \), ce qui contredirait
l'hypothèse sur le degré de \( P \).

Il en résulte que \( Q(X) \) est nécessairement nul, ce qui entraîne $ P(X) = 0 $ pour tout \( X \).

Conclusion : Le polynôme \( P \) est identiquement nul.

% --------------------------------------------------------------------------------------------------
\q Unicité

Supposons qu'il existe deux polynômes \( P \) et \( Q \) de degré au plus \( n \) vérifiant les
conditions d'interpolation :
\[
P(x_i) = y_i, \quad Q(x_i) = y_i, \quad \forall i \in \{0, \dots, n\}
\]
Considérons alors le polynôme \( R \) défini par :
\[
R(X) = P(X) - Q(X)
\]
Ce polynôme est également de degré au plus \( n \), et possède \( n+1 \) racines distinctes :
\[
R(x_i) = P(x_i) - Q(x_i) = y_i - y_i = 0, \quad \forall i \in \{0, \dots, n\}
\]
D'après la question précédente, un polynôme de degré au plus \( n \) ayant \( n+1 \) racines est
nécessairement le polynôme nul, donc $ R(X) \equiv 0 $.

Donc \( P(X) = Q(X) \) pour tout \( X \), ce qui prouve l'unicité du polynôme interpolateur.

% --------------------------------------------------------------------------------------------------
\q Propriétés des polynômes de base de Lagrange
\begin{enumerate}
    \item Degré : Le polynôme \( L_j \) est de degré \( n \) car il est le produit de \( n \)
    facteurs de degré 1.

    \item Evaluation en chaque point d'interpolation :

    Pour \( i \neq j \) :
    \(
    L_j(x_i) = \prod_{\substack{k=0 \\ k\neq j}}^{n} \frac{x_i - x_k}{x_j - x_k} = 0
    \)
    car le numérateur contient le facteur \( x_i - x_i = 0 \).

    Pour \( i = j \) :
    \(
    L_j(x_j) = \prod_{\substack{k=0 \\ k\neq j}}^{n} \frac{x_j - x_k}{x_j - x_k} = 1
    \)
    car le numérateur et le dénominateur sont égaux.

    Conclusion : Les polynômes de base de Lagrange satisfont
    \(
    L_k(x_i) =
    \begin{cases}
        1 & \text{si } i = k \\
        0 & \text{sinon}
    \end{cases}
    \)
\end{enumerate}

\q Polynôme interpolateur de Lagrange
\begin{enumerate}
    \item Détermination des coefficients du polynôme interpolateur

    Les conditions d'interpolation imposent que $P(x_i) = y_i$ pour tout $i \in \{0, ..., n\}$.
    Or en $x_i$, le polynôme interpolateur s'évalue :
    $$y_i = P(x_i) = \sum_{k=0}^n \alpha_k L_k(x_i) = \sum_{k=0}^n \alpha_k \delta_{ik} = \alpha_i$$

    Conclusion : Les coefficients du polynôme interpolateur valent $\alpha_k = y_k$, et le polynôme
    interpolateur s'écrit sous la forme :
    \[
    P(X) = \sum_{k=0}^{n} y_k L_k(X)
    \]

    \item Expression explicite du polynôme interpolateur :
    \[
    P_n(X) = \sum_{i=0}^n y_i \prod_{j \neq i} \frac{X - x_j}{x_i - x_j}
    \]
\end{enumerate}

% --------------------------------------------------------------------------------------------------
\bigskip
\textbf{Applications à l'Interpolation}

\q Soit la fonction \( f : \mathbb{R} \to \mathbb{R} \) définie par \( f(x) = \frac{1}{1 + x^2} \).

\begin{enumerate}
   \item Construction du polynôme de Lagrange \( P_2(X) \)
   Selon la formule du polynôme interpolant de Lagrange :
   \[
    P_2(X) = \sum_{k=0}^{2} f(x_k) L_k(X)
    \]
    où les polynômes de base \( L_k(X) \) sont donnés par :
    \[
    L_k(X) = \prod_{\substack{j=0 \\ j\neq k}}^{2} \frac{X - x_j}{x_k - x_j}
    \]

    Calcul des valeurs de \( f(x_k) \) :
    \[
    f(-1) = \frac{1}{1 + (-1)^2} = \frac{1}{2}, \quad
    f(0) = \frac{1}{1 + 0^2} = 1, \quad
    f(1) = \frac{1}{1 + 1^2} = \frac{1}{2}
    \]

    Polynômes de base :

    \[
    L_0(X) = \frac{(X - 0)(X - 1)}{(-1 - 0)(-1 - 1)} = \frac{(X)(X - 1)}{2}
    \]
    \[
    L_1(X) = \frac{(X + 1)(X - 1)}{(0 + 1)(0 - 1)} = -(X+1)(X-1)
    \]
    \[
    L_2(X) = \frac{(X + 1)(X - 0)}{(1 + 1)(1 - 0)} = \frac{(X + 1)(X)}{2}
    \]

    Ainsi :
    \[
    P_2(X) = \frac{1}{2} \cdot \frac{X(X-1)}{2} + 1 \cdot (-(X+1)(X-1)) + \frac{1}{2} \cdot \frac{(X+1)X}{2}
    \]
    \[
    P_2(X) = \frac{1}{4} (X^2 - X) - (X^2 - 1) + \frac{1}{4} (X^2 + X)
    \]
    \[
    P_2(X) = \frac{1}{4}X^2 - \frac{1}{4}X - X^2 + 1 + \frac{1}{4}X^2 + \frac{1}{4}X
    \]
    \[
    P_2(X) = \left(\frac{1}{4}X^2 - X^2 + \frac{1}{4}X^2 \right) + \left(-\frac{1}{4}X + \frac{1}{4}X \right) + 1
    \]
    \[
    P_2(X) = -\frac{1}{2}X^2 + 1
    \]

    Forme finale du polynôme interpolant  :
    \[
    P_2(X) = 1 - \frac{1}{2}X^2
    \]

    \item Calcul de l'erreur en \( x = 0.5 \) :
    \[
    f(0.5) = \frac{1}{1 + (0.5)^2} = \frac{1}{1.25} = 0.8
    \]
    \[
    P_2(0.5) = 1 - \frac{1}{2} (0.5)^2 = 1 - \frac{1}{2} \times 0.25 = 1 - 0.125 = 0.875
    \]
    Erreur absolue :
    \[
    |f(0.5) - P_2(0.5)| = |0.8 - 0.875| = 0.075
    \]

    \item Commentaire sur la qualité de l'approximation

    L'erreur \( 0.075 \) reste relativement faible, indiquant que l'approximation quadratique capture
    assez bien la variation de \( f(x) \) autour des points d'interpolation. Toutefois, l'erreur est
    significative, ce qui montre que l'interpolation polynomiale d'ordre 2 ne permet pas de représenter
    exactement \( f(x) \), qui est une fonction rationnelle.

    Améliorations possibles :
    \begin{itemize}
        \item Augmenter le degré du polynôme interpolant en ajoutant plus de points.
        \item Utiliser des bases d'interpolation mieux adaptées, comme les polynômes de Chebyshev, pour
        minimiser l'erreur d'approximation sur un intervalle donné.
    \end{itemize}
\end{enumerate}

\bigskip
% --- Illustration graphique ---
\begin{center}
\begin{tikzpicture}
    \begin{axis}[
        axis lines = middle,
        samples = 100,
        domain = -2:2,
        xlabel = $x$,
        ylabel = $y$,
        legend pos = south east,
        enlargelimits = true,
        grid = major
    ]
        % Fonction f(x)
        \addplot[blue, thick] {1/(1 + x^2)};
        \addlegendentry{$f(x) = \frac{1}{1+x^2}$}

        % Polynôme interpolateur P_2(x)
        \addplot[red, dashed, thick] {1 - 0.5*x^2};
        \addlegendentry{$P_2(x) = 1 - \frac{1}{2}x^2$}

        % Points d'interpolation
        \addplot[only marks, mark=*, black] coordinates {(-1,1/2) (0,1) (1,1/2)};
    \end{axis}
\end{tikzpicture}
\end{center}
% ------------------------------

% --------------------------------------------------------------------------------------------------
\bigskip
\textbf{Forme de Newton vs Lagrange}

\q Détermination du coefficient dominant

Soit \( P(x) \) le polynôme interpolateur de Lagrange des points \( (x_0, y_0), \dots, (x_n, y_n)
\), donné par :
\[
P(x) = \sum_{i=0}^{n} y_i L_i(x), \quad \text{avec} \quad
L_i(x) = \prod_{\substack{j=0 \\ j \neq i}}^{n} \frac{x - x_j}{x_i - x_j}
\]

Chaque \(L_i(x)\) est un polynôme de degré \( n \) et son terme dominant provient du produit :
\[
\prod_{\substack{j=0 \\ j \neq i}}^{n} (x - x_j)
\]
Ainsi, le terme dominant de \( L_i(x) \) est :
\[
\frac{x^n}{\prod_{\substack{j=0 \\ j \neq i}}^{n} (x_i - x_j)}
\]

Donc, le terme dominant du polynôme de Lagrange s'écrit :
\[
a_n x^n = \sum_{i=0}^{n} y_i \frac{x^n}{\prod_{\substack{j=0 \\ j \neq i}}^{n} (x_i - x_j)}.
\]

Conclusion : Le coefficient de \( x^n \) dans \( P(x) \) est donné par :

\[
a_n = \sum_{i=0}^{n} \frac{y_i}{\prod_{\substack{j=0 \\ j \neq i}}^{n} (x_i - x_j)}.
\]

Ce coefficient est aussi le coefficient dominant de la forme de Newton, par unicité du polynôme
interpolateur.

% --------------------------------------------------------------------------------------------------
\q Construction itérative de la forme de Newton

Soient trois points d'interpolation \( (x_0, y_0), (x_1, y_1), (x_2, y_2) \)

Étape 1 : Construction de \( P_0(x) \)

Le premier polynôme est simplement la valeur en \( x_0 \) :
\[
P_0(x) = y_0 = a_0
\]

Étape 2 : Construction de \( P_1(x) \)

Ajout du terme linéaire :
\[
P_1(x) = P_0(x) + a_1 (x - x_0) = y_0 + a_1 (x - x_0)
\]
Le coefficient \( a_1 \) est alors imposé par la condition d'interpolation en \( x_1 \) :
\[
P_1(x_1) = y_1 \quad \Rightarrow \quad y_0 + a_1 (x_1 - x_0) = y_1
\quad \Rightarrow \quad
a_1 = \frac{y_1 - y_0}{x_1 - x_0}
\]
Ainsi :
\[
P_1(x) = y_0 + \frac{y_1 - y_0}{x_1 - x_0} (x - x_0)
\]

Étape 3 : Construction de \( P_2(x) \)

Ajoute du terme quadratique :
\[
P_2(x) = P_1(x) + a_2 (x - x_0)(x - x_1) = y_0 + \frac{y_1 - y_0}{x_1 - x_0} (x - x_0) + a_2 (x - x_0)(x - x_1)
\]
Le coefficient \( a_2 \) est alors imposé par la condition d'interpolation en \( x_2 \) :
\[
P_2(x_2) = y_2 \quad \Rightarrow \quad y_0 + \frac{y_1 - y_0}{x_1 - x_0} (x_2 - x_0) + a_2 (x_2 - x_0)(x_2 - x_1) = y_2
\]
En explicitant :
\[
a_2 = \frac{\frac{y_2 - y_1}{x_2 - x_1} - \frac{y_1 - y_0}{x_1 - x_0}}{x_2 - x_0}
\]
Ainsi :
\[
P_2(x) = y_0 + \frac{y_1 - y_0}{x_1 - x_0} (x - x_0) + \left( \frac{\frac{y_2 - y_1}{x_2 - x_1} - \frac{y_1 - y_0}{x_1 - x_0}}{x_2 - x_0} \right) (x - x_0)(x - x_1)
\]

Conjecture : Les coefficients \( a_i \) sont donnés par les différences divisées,
définies récursivement par :
\[
a_0 = y_0
\]
\[
a_1 = \frac{y_1 - y_0}{x_1 - x_0} = \frac{y_1 - a_0}{x_1 - x_0}
\]
\[
a_2 = \frac{\frac{y_2 - y_1}{x_2 - x_1} - \frac{y_1 - y_0}{x_1 - x_0}}{x_2 - x_0} =
\frac{\frac{y_2 - y_1}{x_2 - x_1} - a_1}{x_2 - x_0}
\]
et plus généralement,
\[
a_n = \frac{\frac{y_{n} - y_{n-1}}{x_{n} - x_{n-1}} - a_{n-1}}{x_{n} - x_{0}}
\]

% --------------------------------------------------------------------------------------------------

\end{document}
