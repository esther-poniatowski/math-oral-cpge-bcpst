% PROBLÈME : Endomorphismes nilpotents
% ==================================================================================================
\documentclass[10pt,a4paper]{article}

\usepackage{amsmath, amssymb}
\usepackage{mathrsfs}

\newcommand{\Mat}[2]{\mathrm{Mat}_{#1}(#2)}

\providecommand{\rootpath}{../../..}
% Fonts
\usepackage[utf8]{inputenc} % for accents
\usepackage[T1]{fontenc} % for accents
\usepackage[french]{babel} % for french language
\usepackage{helvet} % sans serif font family
\renewcommand*\familydefault{\sfdefault} % sans serif font family

% Mathematics
\usepackage{amsmath,amsfonts,amssymb} % for math symbols
\usepackage{array} % for tabular


\usepackage{parskip} % no indent, space between paragraphs

\usepackage{geometry} % margin
\geometry{
    a4paper,
    left=15mm,
    right=15mm,
    top=20mm,
    bottom=20mm
}

\usepackage{circledsteps} % to draw circles around numbers

\usepackage{fancyhdr} % for headers and footers

\usepackage{enumitem} % for customizing lists
\setlist[enumerate]{itemsep=1em} % space between items only in enumerate environment (not itemize)
\setlist[itemize]{label=--} % set itemize label to em-dash

% Command: \customPageLayout{#1}{#2}{#3}
% --------------------------------------
% Description: Custom page layout with header and footer content.
% Arguments:
% #1: Header and footer content
% #2: Left header content
% #3: Right header content
% Example:
% \customPageLayout{Title}{Lycée Henri IV}{2024}
% Required Packages: fancyhdr
\newcommand{\customPageLayout}[3]{
    \pagestyle{fancy} % set page style to fancy (add header and footer)
    \fancyhf{} % clear all header and footer content
    \lhead{#2} % left header content
    \rhead{#3} % right header content
    \chead{\textbf{#1}} % center header content in bold (if needed)
    \rfoot{\thepage} % page number in the footer
}


% Counter: \q
% -----------
% Description: Display a question number in a circle.
% Usage:
% - Create a new question: add \q followed by the question content.
% - Reset the question counter: add \setcounter{q}{0} before the first question.
\newcounter{q}
\setcounter{q}{0} % set initial value of the counter
\newcommand{\q}{
    \bigskip
    \addtocounter{q}{1}
    \par
    \Circled{\textbf{\theq}} \space
}


% Counter: \ql
% ------------
% Description: Display a question letter in a round box with indentation (lowercase and not bold).
% Usage:
% - Create a new question: add \ql followed by the question content.
% - Reset the question counter: add \setcounter{ql}{0} before the first question.
\newcounter{ql}
\setcounter{ql}{0} % set initial value of the counter
\newcommand{\ql}{
    \addtocounter{ql}{1}
    \par
    \hspace{1.5em} % indentation before the circled letter
    \textcolor{gray}{\Circled{\alph{ql}}} \space % gray color
}


\title{Applications linéaires et Matrices -- Endomorphismes cycliques}
\author{Esther Poniatowski}
\date{2024-2025}

\customPageLayout{Sujets d'interrogation orale}{Lycée Henri IV}{2024}

% ==================================================================================================
\begin{document}

\textbf{Contexte}

\textit{Définition} -- Soit $E$ un espace vectoriel de dimension $n \geq 2$. Un endomorphisme $f \in
\mathrm{End}(E)$ est dit \textbf{cyclique} s'il existe un vecteur $\mathbf{x}_0 \in E$ tel que :
\[
E = \mathrm{Vect}(\mathbf{x}_0, f(\mathbf{x}_0), f^2(\mathbf{x}_0), \dots, f^{n-1}(\mathbf{x}_0))
\]

Les endomorphismes cycliques permettent une représentation matricielle simplifiée (matrice
compagnon), facilitant le calcul des polynômes caractéristiques et minimaux. Ils interviennent dans
la résolution de systèmes linéaires et l'étude des sous-espaces stables par un endomorphisme.

\bigskip
\textbf{Exemples d'endomorphismes cycliques et non cycliques}

\q Soit $\dim(E) = 4$, $\mathscr{B} = (e_1, e_2, e_3, e_4)$, et un endomorphisme $f$ tel que :
\[
\Mat{\mathscr{B}}{f} =
\begin{pmatrix}
0 & -1 & 0 & 0\\
1 & 0 & 0 & 0\\
0 & 0 & 0 & -1\\
0 & 0 & 1 & 0
\end{pmatrix}
\]
Montrer que l'endomorphisme $f$ n'est pas cyclique.
% But : Étudier un exemple d'endomorphisme non cyclique à l'aide du critère de généricité. Méthode :
% Calcul du sous-espace engendré par les itérés d'un vecteur.

\q Soit $\dim(E) = 3$, $\mathscr{B} = (e_1, e_2, e_3)$, et un endomorphisme $f$ tel que :
\[
\Mat{\mathscr{B}}{f} =
\begin{pmatrix}
0 & 0 & \alpha\\
1 & 0 & \beta\\
0 & 1 & \gamma
\end{pmatrix}
\quad \text{où } (\alpha, \beta, \gamma) \in \mathbb{R}^3
\]
Montrer que l'endomorphisme $f$ est cyclique.
% But : Vérifier l'existence d'un vecteur générateur cyclique. Méthode : Calcul des itérés
% successifs et vérification de la généricité.

\q Soit $\dim(E) = n \in \mathbb{N}^*$, $\mathscr{B} = (e_1, \dots, e_n)$, et un endomorphisme $f$
tel que :
\[
\Mat{\mathscr{B}}{f} =
\begin{pmatrix}
0 & \cdots & & & 0\\
1 & 0 & & & \vdots\\
0 & 1 & \ddots & & \\
\vdots & & \ddots & 0 & 0\\
0 & \cdots & 0 & 1 & 1
\end{pmatrix}
\]
Montrer que l'endomorphisme $f$ est cyclique. Déterminer son rang et indiquer s'il est inversible.
% But : Analyser un cas générique construit de manière cyclique explicite. Méthode : Vérification de
% la définition et analyse des propriétés matricielles.

\bigskip
\textbf{Base adaptée et matrice compagnon}

Soit $f$ est un endomorphisme cyclique dans un espace $E$ de dimension $n \geq 2$. Il
existe donc un vecteur $\mathbf{x}_0$ tel que :
\[
E = \mathrm{Vect}(\mathbf{x}_0, f(\mathbf{x}_0), \dots, f^{n-1}(\mathbf{x}_0))
\]

\q Justifier qu'il existe certains réels $a_0, \dots, a_{n-1}$ tels que :
\[
f^n(\mathbf{x}_0) = a_{n-1}f^{n-1}(\mathbf{x}_0) + \dots + a_1 f(\mathbf{x}_0) + a_0 \mathbf{x}_0
\]
% But : Utiliser la dimension finie de l'espace pour établir une relation de dépendance linéaire.
% Méthode : Existence de polynôme annulateur minimal.

\q En déduire la matrice de $f$ dans la base $(\mathbf{x}_0, f(\mathbf{x}_0), \dots,
f^{n-1}(\mathbf{x}_0))$.
% But : Mettre en évidence la matrice compagnon associée au polynôme minimal. Méthode : Appliquer
% directement l'action de $f$ sur la base.

\bigskip
\textbf{Commutant d'un endomorphisme cyclique}

\textit{Définition} -- Le \textbf{commutant} d'un endomorphisme $f \in \mathrm{End}(E)$ est
l'ensemble des endomorphismes qui commutent avec $f$ :
\[
\mathscr{C}(f) = \{ g \in \mathrm{End}(E) \mid g \circ f = f \circ g \}
\]

Le but de cette partie est de prouver le théorème suivant :

\textit{Théorème} -- Si $f$ est cyclique, alors le commutant de $f$ est l'espace vectoriel $\mathbb{R}_{n-1}[f]$ engendré
par les puissances de $f$ jusqu'à l'ordre $n - 1$ :
\[
\mathscr{C}(f) = \mathbb{R}_{n-1}[f] := \left\{ a_0 \mathrm{Id}_E + a_1 f + \dots + a_{n-1} f^{n-1}, \; (a_0, \dots, a_{n-1}) \in \mathbb{R}^n \right\}
\]

\q Justifier que $\mathscr{C}(f)$ est un sous-espace vectoriel de \(\mathrm{End}(E)\) (espace
vectoriel des endomorphismes de $E$).
% But : Montrer que le commutant est stable par combinaison linéaire. Méthode : Utiliser la
% linéarité et la stabilité par composition.

\q Soit $\mathbf{x}_0 \in E$ un vecteur générateur cyclique de $f$.
Montrer que, pour tous $(v, w) \in \mathscr{C}(f)^2$ :
\[
v(\mathbf{x}_0) = w(\mathbf{x}_0) \iff v = w
\]
% But : Identifier l'unicité d'un endomorphisme à partir de l'image d'un générateur cyclique.
% Méthode : Exploiter la propriété de généricité de $\mathbf{x}_0$.

\q Démontrer par double inclusion que $\mathscr{C}(f) = \mathbb{R}_{n-1}[f]$.
% But : Établir l'égalité du commutant avec l'algèbre des polynômes de degré $\leq n-1$. Méthode :
% Inclusion directe par calcul et réciproque via unicité de la représentation.

\q Déterminer une base de $\mathscr{C}(f)$ et en déduire sa dimension.
% But : Caractériser le commutant comme espace vectoriel. Méthode : Identifier les générateurs
% naturels et compter les dimensions.

\bigskip
\textbf{Lien avec la nilpotence}

\textit{Définition} -- Soit $E$ un espace vectoriel de dimension $n \in \mathbb{N}^*$. Un
endomorphisme $f$ est dit \textbf{nilpotent d'indice $p \leq n$} si :
\[
f^{p-1} \neq 0 \quad \text{et} \quad f^p = 0
\]
Il est dit \textbf{nilpotent maximal} lorsque $p = n$.

\q Montrer que, dans un espace vectoriel de dimension $n \geq 2$, un endomorphisme nilpotent est
cyclique si et seulement s'il est nilpotent maximal.
% But : Caractériser les nilpotents cycliques parmi tous les nilpotents. Méthode : Montrer
% l'équivalence entre l'existence d'un vecteur générateur et la maximalité de l'indice.

\end{document}
% ==================================================================================================
