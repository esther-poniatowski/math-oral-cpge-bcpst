% CORRECTION : Titre
% ==================================================================================================

\documentclass[10pt,a4paper]{article}

\usepackage{amsmath,amssymb,mathrsfs,stmaryrd,enumitem}

% Set the root path
\providecommand{\rootpath}{../../..}
% Fonts
\usepackage[utf8]{inputenc} % for accents
\usepackage[T1]{fontenc} % for accents
\usepackage[french]{babel} % for french language
\usepackage{helvet} % sans serif font family
\renewcommand*\familydefault{\sfdefault} % sans serif font family

% Mathematics
\usepackage{amsmath,amsfonts,amssymb} % for math symbols
\usepackage{array} % for tabular


\usepackage{parskip} % no indent, space between paragraphs

\usepackage{geometry} % margin
\geometry{
    a4paper,
    left=15mm,
    right=15mm,
    top=20mm,
    bottom=20mm
}

\usepackage{circledsteps} % to draw circles around numbers

\usepackage{fancyhdr} % for headers and footers

\usepackage{enumitem} % for customizing lists
\setlist[enumerate]{itemsep=1em} % space between items only in enumerate environment (not itemize)
\setlist[itemize]{label=--} % set itemize label to em-dash

% Command: \customPageLayout{#1}{#2}{#3}
% --------------------------------------
% Description: Custom page layout with header and footer content.
% Arguments:
% #1: Header and footer content
% #2: Left header content
% #3: Right header content
% Example:
% \customPageLayout{Title}{Lycée Henri IV}{2024}
% Required Packages: fancyhdr
\newcommand{\customPageLayout}[3]{
    \pagestyle{fancy} % set page style to fancy, i.e. header and footer
    \fancyhf{#1} % set header and footer content
    \lhead{#2} % set left header content
    \rhead{#3} % set right header content
    \fancyfoot{} % clear footer content
    \rfoot{\thepage} % set page number in footer
}

% Counter: \q
% -----------
% Description: Display a question number in a circle.
\newcounter{q}
\setcounter{q}{0} % set initial value of counter
\newcommand{\q}{
    \bigskip
    \addtocounter{q}{1}
    \par
    \Circled{\textbf{\theq}} \space
}


\title{Applications linéaires -- Décomposition d'un endomorphisme en une somme finie de projections}
\author{Esther Poniatowski}
\date{2024-2025}

\customPageLayout{Correction}{Lycée Henri IV}{2024}

% ==================================================================================================
\begin{document}

\bigskip
\textbf{Généralités sur les projections}

\q Projection associée :

L'endomorphisme $p' = \mathrm{Id}_E - p$ est aussi une projection, car :
\[
p' \circ p' = (\mathrm{Id}_E - p) \circ (\mathrm{Id}_E - p) = \mathrm{Id}_E - 2p + p^2
\]
or $p^2 = p$ par définition de la projection, donc :
\[
p' \circ p' = \mathrm{Id}_E - 2p + p = \mathrm{Id}_E - p = p'
\]

Relations entre les deux projections (but, à montrer) :
\[
\mathrm{Im}(p) = \mathrm{Ker}(\mathrm{Id}_E - p) \quad \text{et} \quad \mathrm{Im}(\mathrm{Id}_E - p) = \mathrm{Ker}(p)
\]

\emph{Par double inclusion} :
\begin{itemize}
    \item Soit $\mathbf{x} \in \mathrm{Im}(\mathrm{Id}_E - p)$, alors $\exists \mathbf{y} \in E, \; \mathbf{y} - p(\mathbf{y}) = \mathbf{x}$. \\
    En appliquant $p$ : $p(\mathbf{x}) = p(\mathbf{y} - p(\mathbf{y})) = p(\mathbf{y}) - p(p(\mathbf{y})) = p(\mathbf{y}) - p(\mathbf{y}) = \mathbf{0}$.\\
    Donc $\mathbf{x} \in \ker(p)$.
    \item Inversement, soit $\mathbf{x} \in \ker(p)$. On cherche un antécédent $\mathbf{y}$ tel que $\mathbf{y} - p(\mathbf{y}) = \mathbf{x}$.\\
    En particulier : $p(\mathbf{x}) = 0$ donc $\mathbf{x} - p(\mathbf{x}) = \mathbf{x}$. Ainsi, un antécédent est $\mathbf{y} = \mathbf{x}$.\\
    Donc $\mathbf{x} \in \mathrm{Im}(\mathrm{Id}_E - p)$.
    \item Soit $\mathbf{x} \in \mathrm{Im}(p)$, alors $\exists \mathbf{y} \in E, \; p(\mathbf{y}) = \mathbf{x}$. \\
    En appliquant $p'$ : $(\mathrm{Id}_E - p)(\mathbf{x}) = (\mathrm{Id}_E - p)(p(\mathbf{y})) = p(\mathbf{y}) - p(p(\mathbf{y})) =  \mathbf{0}$.\\
    Donc $\mathbf{x} \in \ker(\mathrm{Id}_E - p)$.
    \item Inversement, soit $\mathbf{x} \in \ker(\mathrm{Id}_E - p)$. On cherche un antécédent $\mathbf{y}$ tel que $p(\mathbf{y}) = \mathbf{x}$.\\
    En particulier : $(\mathrm{Id}_E - p)(\mathbf{x}) = 0$ donc $\mathbf{x} - p(\mathbf{x}) = \mathbf{x}$. Ainsi, un antécédent est $\mathbf{y} = \mathbf{x}$.\\
    Donc $\mathbf{x} \in \mathrm{Im}(p)$. \\
\end{itemize}

Conclusion : \fbox{$\begin{cases}\mathrm{Im}(\mathrm{Id}_E - p) = \ker(p)\\
\mathrm{Im}(p) = \ker(\mathrm{Id}_E - p)
\end{cases}$}

% --------------------------------------------------------------------------------------------------
\q Exemples de matrices :
\[
P =
\begin{pmatrix}
1 & 0\\
1 & 0
\end{pmatrix}
\]

Calcul de $P^2$ :
\[
P^2 =
\begin{pmatrix}
1 & 0\\
1 & 0
\end{pmatrix}
\begin{pmatrix}
1 & 0\\
1 & 0
\end{pmatrix}
=
\begin{pmatrix}
1 & 0\\
1 & 0
\end{pmatrix}
= P
\]

La matrice $P$ est idempotente : il s'agit d'une projection.

Image : engendrée par $\begin{pmatrix}1\\1\end{pmatrix}$.

Noyau : constitué des vecteurs de la forme $\begin{pmatrix}0\\y\end{pmatrix}$.

Interprétation géométrique : projection sur la droite $\operatorname{Vect}\left(\begin{pmatrix}1\\1\end{pmatrix}\right)$ parallèlement à l'axe des ordonnées.

% ------

\[
P = \frac{1}{2}
\begin{pmatrix}
1 & 1\\
1 & 1
\end{pmatrix}
\]

Calcul de $P^2$ :
\[
P^2 = \frac{1}{4}
\begin{pmatrix}
1 & 1\\
1 & 1
\end{pmatrix}
\begin{pmatrix}
1 & 1\\
1 & 1
\end{pmatrix}
= \frac{1}{4}
\begin{pmatrix}
2 & 2\\
2 & 2
\end{pmatrix}
=
\frac{1}{2}
\begin{pmatrix}
1 & 1\\
1 & 1
\end{pmatrix}
= P
\]

La matrice $P$ est idempotente : il s'agit d'une projection.

Image : $\operatorname{Vect}\left(\begin{pmatrix}1\\1\end{pmatrix}\right)$.

Noyau : $\operatorname{Vect}\left(\begin{pmatrix}1\\-1\end{pmatrix}\right)$.

Interprétation géométrique : projection orthogonale sur la droite d'équation $x = y$ dans $\mathbb{R}^2$.

% ------
\[
P =
\begin{pmatrix}
1 & 0 & 0\\
0 & 0 & 0\\
0 & 0 & 0
\end{pmatrix}
\]

Vérification :
\[
P^2 =
\begin{pmatrix}
1 & 0 & 0\\
0 & 0 & 0\\
0 & 0 & 0
\end{pmatrix}
= P
\]

Il s'agit d'une projection.

Image : $\operatorname{Vect}(e_1)$.

Noyau : sous-espace $\operatorname{Vect}(e_2, e_3)$.

Interprétation géométrique : projection parallèle sur l'axe $x$, le long du plan $(yOz)$.

% ------
\[
P =
\begin{pmatrix}
1 & 0 & 0\\
0 & 1 & 0\\
0 & 0 & 0
\end{pmatrix}
\]

Vérification :
\[
P^2 = P
\]

Image : plan $\operatorname{Vect}(e_1, e_2)$.

Noyau : $\operatorname{Vect}(e_3)$.

Interprétation géométrique : projection orthogonale sur le plan $(xOy)$ dans $\mathbb{R}^3$.

% ------
\[
P =
\begin{pmatrix}
1 & 0 & 0\\
a & 0 & 0\\
b & 0 & 0
\end{pmatrix}
\]

Calcul de $P^2$ :
\[
P^2 =
\begin{pmatrix}
1 & 0 & 0\\
a & 0 & 0\\
b & 0 & 0
\end{pmatrix}
\begin{pmatrix}
1 & 0 & 0\\
a & 0 & 0\\
b & 0 & 0
\end{pmatrix}
=
\begin{pmatrix}
1 & 0 & 0\\
a & 0 & 0\\
b & 0 & 0
\end{pmatrix}
= P
\]

La condition $P^2 = P$ est vérifiée pour tout $(a, b) \in \mathbb{R}^2$.

Image : $\operatorname{Vect}\left(\begin{pmatrix}1\\a\\b\end{pmatrix}\right)$.

Noyau : plan d'équation $x = 0$.

Interprétation géométrique : projection sur la droite dirigée par $\begin{pmatrix}1\\a\\b\end{pmatrix}$ parallèlement au plan $x = 0$.

% --------------------------------------------------------------------------------------------------
\q Décomposition de $E$ en somme directe : $E = \mathrm{Im}(p) \oplus \ker(p)$

\textit{Méthode 1 : Décomposition explicite}

Existence :

Soit un vecteur $\mathbf{x} \in E$. Il peut être décomposé ainsi : $\mathbf{x} = (\mathbf{x} - \mathbf{p(x)}) + \mathbf{p(x)}$.
\begin{itemize}
 \item $\mathbf{p(x)}$ est la projection sur l'image de $p$, donc $\mathbf{p(x)} \in \mathrm{Im}(p)$.
 \item $\mathbf{x} - \mathbf{p(x)} \in \ker(p)$, puisque : $p(\mathbf{x} - \mathbf{p(x)}) =  \mathbf{p(x)} - \mathbf{p(p(x))} = \mathbf{p(x)} - \mathbf{p(x)} = \mathbf{0}$.
\end{itemize}

Unicité :

Supposons qu'il existe deux décompositions telles que $\mathbf{x} = \mathbf{x}_1 + \mathbf{x}_2 =
\mathbf{y}_1 + \mathbf{y}_2$, avec $\mathbf{x}_1, \mathbf{y}_1 \in \mathrm{Im}(p)$ et $\mathbf{x}_2,
\mathbf{y}_2 \in \ker(p)$.

Alors : $$\underbrace{\mathbf{x}_2 - \mathbf{y}_2}_{\in \ker(p)} = \underbrace{\mathbf{x}_1 -
\mathbf{y}_1}_{\in \mathrm{Im}(p)}$$

En appliquant $p$ :
$$p(\mathbf{x}_2 - \mathbf{y}_2) = p(\mathbf{x}_1 - \mathbf{y}_1) \implies 0 = \mathbf{x}_1 -
\mathbf{y}_1 \implies \mathbf{x}_1 = \mathbf{y}_1$$

Et en réinjectant dans l'expression de $\mathbf{x}_2$ :
$$\mathbf{x}_2 = \mathbf{x} - \mathbf{x}_1 = \mathbf{y} - \mathbf{y}_1 = \mathbf{y}_2$$

Donc $\mathbf{x}_2 = \mathbf{y}_2$.

\textit{Méthode 2 : Sous-espaces}

Construisons une base de $E$ à partir de bases de $\mathrm{Im}(p)$ et $\ker(p)$.

D'une part, les dimensions sont compatibles, car par le théorème du rang :
$$\dim(\ker(p)) + \dim(\mathrm{Im}(p)) = \dim(E)$$

Il reste à montrer que l'intersection de ces sous-espaces est réduite au vecteur nul : $\ker(p) \cap
\mathrm{Im}(p) = \{\mathbf{0}\}$.

Soit $\mathbf{x} \in \ker(p) \cap \mathrm{Im}(p)$, alors :
$$
\begin{cases}
    p(\mathbf{x}) = \mathbf{0}\\
    \exists \mathbf{y} \in E, \; p(\mathbf{y}) = \mathbf{x}
\end{cases}
$$
Pour montrer qu'il s'agit du vecteur nul, il suffit d'appliquer $p$ à la seconde relation :
$$p(p(\mathbf{y})) = p(\mathbf{x}) = \mathbf{0}$$

Or $p(p(\mathbf{y})) = p(\mathbf{y}) = \mathbf{x}$, donc :
$$\mathbf{x} = \mathbf{0}$$

Conclusion : \fbox{$E = \mathrm{Im}(p) \oplus \ker(p)$}

% --------------------------------------------------------------------------------------------------
\q Représentation matricielle en base 'adaptée'

Soit $r = \dim(\mathrm{Im}(p))$ le rang de $p$.

Selon la question précédente, tout vecteur de $E$ se décompose de manière unique comme la somme d'un vecteur de l'image et d'un vecteur du noyau de $p$.
Il est donc possible de construire une base de $E$ en joignant une base de $\mathrm{Im}(p)$ et une
base de $\ker(p)$, formée de :

\begin{itemize}
 \item $r$ vecteurs $(\mathbf{x}_i^{(j)})_{1\leq j\leq r} \in \mathrm{Im}(p)$, qui vérifient $p(\mathbf{x}_i^{(j)}) = \mathbf{x}_i^{(j)}$ par définition de l'image.
 \item $n-r$ vecteurs $(\mathbf{x}_k^{(\ell)})_{1\leq \ell \leq n-r} \in \ker(p)$, qui vérifient $p(\mathbf{x}_k^{(\ell)}) = \mathbf{0}$ par définition du noyau.
\end{itemize}

Ainsi, la représentation matricielle de $p$ dans la base $\mathcal{B} = (\mathbf{x}_i^{(1)}, ..., \mathbf{x}_i^{(r)}, \mathbf{x}_k^{(1)}, ..., \mathbf{x}_k^{(n-r)})$ est :\\
\centerline{$\mathrm{Mat}_{\mathcal{B}}(p) = \bordermatrix{
& \mathbf{x}_i^{(1)} & \mathbf{x}_i^{(2)} & \cdots & \mathbf{x}_i^{(r)} & \mathbf{x}_k^{(1)} & \cdots & \mathbf{x}_k^{(n-r)} \cr
\mathbf{x}_i^{(1)} & 1 & 0 & \cdots & 0 & 0 & \cdots & 0 \cr
\mathbf{x}_i^{(2)} & 0 & 1 &  & 0 & 0 &  & 0 \cr
\vdots & \vdots &  & \ddots & \vdots & \vdots &  & \vdots \cr
\mathbf{x}_i^{(r)} & 0 & 0 & \cdots & 1 & 0 & \cdots & 0 \cr
\mathbf{x}_k^{(1)} & 0 & 0 & \cdots & 0 & 0 & \cdots & 0 \cr
\vdots & \vdots &  &  & \vdots & \vdots &  & \vdots \cr
\mathbf{x}_k^{(n-r)} & 0 & 0 & \cdots & 0 & 0 & \cdots & 0 \cr}$}\\

Conclusion : \fbox{$\mathrm{Mat}_{\mathcal{B}}(p) =
\begin{pmatrix}
I_r & 0\\
0 & 0
\end{pmatrix}$}

% --------------------------------------------------------------------------------------------------
\q Trace d'une projection :

Dans la base précédente, la représentation matricielle de $p$ a $r$ coefficients diagonaux à $1$ et les autres à $0$, donc :
\[
\mathrm{Tr}(p) = r = \mathrm{rg}(p)
\]

% --------------------------------------------------------------------------------------------------
\bigskip
\textbf{Démonstration du théorème -- Sens direct}

\q Linéarité de la trace :

Soient $u$ et $v$ deux endomorphismes de $E$, et $\mathcal{B}$ une base de $E$.
\begin{itemize}
 \item La matrice représentative de l'endomorphisme $u + v$ dans $\mathcal{B}$ est : $\mathrm{Mat}_{\mathcal{B}}(u + v)
= \mathrm{Mat}_{\mathcal{B}}(u)
+ \mathrm{Mat}_{\mathcal{B}}(v)$, les coefficients diagonaux s'obtiennent en sommant ceux des deux matrices.
\item La matrice représentative de l'endomorphisme $\lambda u $ dans $\mathcal{B}$ est : $\mathrm{Mat}_{\mathcal{B}}(\lambda u)
= \lambda\mathrm{Mat}_{\mathcal{B}}(u)$, la somme des coefficients diagonaux est multipliée par $\lambda$.
\end{itemize}
Conclusion : \fbox{La trace est un opérateur linéaire.}

Trace entière pour l'endomorphisme $u$ :

L'endomorphisme $u$ est une somme finie de projections, donc par linéarité de la trace, la trace de
$u$ est la somme des traces des projections :
\[
\mathrm{Tr}(u) = \mathrm{Tr}(p_1) + \dots + \mathrm{Tr}(p_m)
\]

D'autre part, puisque la trace d'une projection est égale à son rang :
$$\sum_{i=1}^{m}\mathrm{Tr}(p_i) = \sum_{i=1}^{m}\mathrm{rg}(p_i)$$

Chaque $\mathrm{rg}(p_i)$ est un entier, donc \fbox{$\mathrm{Tr}(u) \in \mathbb{N}$}.

% --------------------------------------------------------------------------------------------------
\q Dimension d'une somme de deux sous-espaces :

La dimension de $F + G$ est inférieure ou égale au nombre des vecteurs de n'importe quelle famille
génératrice de $F + G$. Une telle famille peut être obtenue en concaténant une base de $F$ et une
base de $G$, et est formée de $\dim(F) +\dim(G)$ vecteurs.

Conclusion : \boxed{\dim(F + G) \leq \dim(F) + \dim(G)}.

Majoration de la trace :

Par récurrence sur le nombre de projections dans la somme :

\textit{Initialisation} : $u = p_1$ est une projection donc $\mathrm{rg}(u) = \mathrm{Tr}(u)$ selon la propriété de la trace
et du rang.

\textit{Hérédité :} Supposons la propriété vraie au rang $m-1$ : $\mathrm{Tr}\left(\sum_{i=1}^{m-1}p_i\right) \leq
\mathrm{rg}\left(\sum_{i=1}^{m-1}p_i\right)$.

Par définition : $u = \sum_{i=1}^{m-1}p_i + p_m$. Chaque vecteur de l'image de $u$ s'obtient par la somme d'un vecteur de l'image de $\sum_{i=1}^{m-1}p_i$ et d'un vecteur de l'image de $p_m$, donc en termes de sous-espaces : $\mathrm{Im}(u) = \mathrm{Im}\left(\sum_{i=1}^{m-1}p_i\right) + \mathrm{Im}(p_m)$.\\
En appliquant la propriété sur la dimension d'une somme de sous-espaces avec
$\begin{cases}
F = \mathrm{Im}\left(\sum_{i=1}^{m-1}p_i\right) \\
G = \mathrm{Im}(p_m)\\
F + G = \mathrm{Im}(u)
\end{cases}$ :
$$\dim(u) \leq \dim\left(\sum_{i=1}^{m-1}p_i\right) + \dim(\mathrm{Im}(p_m))$$

Or :
$$\begin{cases}
\dim(u) = \mathrm{rg}(u)\\
\dim\left(\sum_{i=1}^{m-1}p_i\right) =
\mathrm{rg}\left(\sum_{i=1}^{m-1}p_i\right) \leq \mathrm{Tr}\left(\sum_{i=1}^{m-1}p_i\right) = \sum_{i=1}^{m-1}\mathrm{Tr}(p_i)\; \text{par hypothèse de récurrence et linéarité de la trace}\\
\dim(\mathrm{Im}(p_m)) : \mathrm{rg}(p_m) = \mathrm{Tr}(p_m)\; \text{selon l'initialisation}
\end{cases}$$

Donc :
$$\mathrm{rg}(u) \leq \sum_{i=1}^{m-1}\mathrm{Tr}(p_i) + \mathrm{Tr}(p_m) =
\sum_{i=1}^{m}\mathrm{Tr}(p_i) = \mathrm{Tr}(u)$$

Conclusion : \fbox{$\mathrm{Tr}(u) \geq \mathrm{rg}(u)$}.

% --------------------------------------------------------------------------------------------------

\bigskip
\textbf{Démonstration du théorème -- Sens réciproque}

TODO: Finaliser la correction.

\end{document}
% ==================================================================================================
