% CORRECTION : Endomorphismes cycliques
% ==================================================================================================

\documentclass[10pt,a4paper]{article}

% Set the root path
\providecommand{\rootpath}{../../..}
% Fonts
\usepackage[utf8]{inputenc} % for accents
\usepackage[T1]{fontenc} % for accents
\usepackage[french]{babel} % for french language
\usepackage{helvet} % sans serif font family
\renewcommand*\familydefault{\sfdefault} % sans serif font family

% Mathematics
\usepackage{amsmath,amsfonts,amssymb} % for math symbols
\usepackage{array} % for tabular


\usepackage{parskip} % no indent, space between paragraphs

\usepackage{geometry} % margin
\geometry{
    a4paper,
    left=15mm,
    right=15mm,
    top=20mm,
    bottom=20mm
}

\usepackage{circledsteps} % to draw circles around numbers

\usepackage{fancyhdr} % for headers and footers

\usepackage{enumitem} % for customizing lists
\setlist[enumerate]{itemsep=1em} % space between items only in enumerate environment (not itemize)
\setlist[itemize]{label=--} % set itemize label to em-dash

% Command: \customPageLayout{#1}{#2}{#3}
% --------------------------------------
% Description: Custom page layout with header and footer content.
% Arguments:
% #1: Header and footer content
% #2: Left header content
% #3: Right header content
% Example:
% \customPageLayout{Title}{Lycée Henri IV}{2024}
% Required Packages: fancyhdr
\newcommand{\customPageLayout}[3]{
    \pagestyle{fancy} % set page style to fancy (add header and footer)
    \fancyhf{} % clear all header and footer content
    \lhead{#2} % left header content
    \rhead{#3} % right header content
    \chead{\textbf{#1}} % center header content in bold (if needed)
    \rfoot{\thepage} % page number in the footer
}


% Counter: \q
% -----------
% Description: Display a question number in a circle.
% Usage:
% - Create a new question: add \q followed by the question content.
% - Reset the question counter: add \setcounter{q}{0} before the first question.
\newcounter{q}
\setcounter{q}{0} % set initial value of the counter
\newcommand{\q}{
    \bigskip
    \addtocounter{q}{1}
    \par
    \Circled{\textbf{\theq}} \space
}


% Counter: \ql
% ------------
% Description: Display a question letter in a round box with indentation (lowercase and not bold).
% Usage:
% - Create a new question: add \ql followed by the question content.
% - Reset the question counter: add \setcounter{ql}{0} before the first question.
\newcounter{ql}
\setcounter{ql}{0} % set initial value of the counter
\newcommand{\ql}{
    \addtocounter{ql}{1}
    \par
    \hspace{1.5em} % indentation before the circled letter
    \textcolor{gray}{\Circled{\alph{ql}}} \space % gray color
}


\title{Applications linéaires -- Endomorphismes cycliques}
\author{Esther Poniatowski}
\date{2024-2025}

\customPageLayout{Correction}{Lycée Henri IV}{2024}

% ==================================================================================================
\begin{document}

\textbf{Exemples d'endomorphismes cycliques et non cycliques}

Le but est de calculer le sous-espace engendré par les itérés d'un vecteur, et de vérifier s'il est
égal à \(E\).

\q Endomorphisme non cyclique en dimension 4 :

En calculant le carré de la matrice : \(f^2 = -\mathrm{Id}\).

Donc pour tout vecteur \(\mathbf{x} \in E,\; \begin{cases} f^2(\mathbf{x}) = -\mathbf{x}\\
f^3(\mathbf{x}) = -f(\mathbf{x})\end{cases}\).

La famille \((\mathbf{x}, f(\mathbf{x}),f^2(\mathbf{x}), f^3(\mathbf{x}))\) est liée, donc n'est pas
une base de \(E\), et ceci vaut pour tout vecteur de \(E\).

Conclusion : \fbox{\(f\) n'est pas cyclique.}

% --------------------------------------------------------------------------------------------------
\q Endomorphisme cyclique en dimension 3 :

Par calcul : \(\begin{cases} f(e_1) = e_2\\
f^2(e_1) = f(e_2) = e_3\end{cases}\).

Donc \(\mathrm{Span}(e_1, f(e_1),f^2(e_1)) = \mathrm{Span}(e_1,e_2,e_3)=E\).

Conclusion : \fbox{\(f\) est cyclique}.

% --------------------------------------------------------------------------------------------------
\q Endomorphisme cyclique en dimension \(n\) -- Inversibilité :

D'après la forme de la matrice : \(\forall k \in \{ 1,\dots,n-1 \},\; f(e_k) = e_{k+1}\).

Par récurrence immédiate : \(\forall k \in \{ 1,\dots,n-1 \},\; f^k(e_1) = e_{k+1}\).

Ainsi : \((e_1, f(e_1), ..., f^{n-1}(e_1)) = (e_1, e_2, ..., e_n)\), qui est la base canonique de
\(E\).

Conclusion : \fbox{\(f\) est cyclique}.

La matrice de \(f\) possède \(n-1\) premières colonnes formant une famille libre, et la dernière
colonne identique à l'avant dernière.

Donc \fbox{\(\mathrm{rank}(f) = n-1\), et \(f\) n'est pas inversible}.

% ------------------------------------------------------------------------------------------------
\bigskip
\textbf{Base adaptée et matrice compagnon}

\q Décomposition de $f^n(\mathbf{x_0})$ :

Puisque la famille \((\mathbf{x_0}, f(\mathbf{x_0}), ..., f^{n-1}(\mathbf{x_0}))\) est une base de
\(E\), il existe un unique n-uplet de coordonnées \((a_0, a_1, ..., a_{n-1}) \in \mathbb{R}^n\) tel
que :
\[
f^n(\mathbf{x_0}) = a_{n-1}f^{n-1}(\mathbf{x_0}) + \cdots + a_1 f(\mathbf{x_0}) + a_0 \mathbf{x_0}
\]

% --------------------------------------------------------------------------------------------------
\q Représentation matricielle de \(f\) :

Calculs des images des vecteurs de base \((\mathbf{x_0}, f(\mathbf{x_0}), ...,
f^{n-1}(\mathbf{x_0}))\) par \(f\) :
\[
\begin{cases}
\forall k \in \{ 1, \dots, n-2 \},\; f(f^k(\mathbf{x_0})) = f^{k+1}(\mathbf{x_0}) \\
f(f^{n-1}(\mathbf{x_0})) = f^n(\mathbf{x_0}) = a_{n-1}f^{n-1}(\mathbf{x_0}) + \cdots + a_1 f(\mathbf{x_0}) + a_0 \mathbf{x_0}
\end{cases}
\]

Donc la matrice de \(f\) s'écrit dans la base \(\mathcal{B} = (\mathbf{x_0}, f(\mathbf{x_0}), ...,
f^{n-1}(\mathbf{x_0}))\) :
\[
\mathrm{Mat}_{\mathcal{B}}(f) = \begin{bmatrix}
0 & 0 & \cdots & 0 & a_0 \\
1 & 0 & \cdots & 0 & a_1 \\
0 & 1 & \cdots & 0 & a_2 \\
\vdots & \vdots & \ddots & \vdots & \vdots \\
0 & 0 & \cdots & 1 & a_{n-1}
\end{bmatrix}
\]

% ------------------------------------------------------------------------------------------------
\bigskip
\textbf{Commutant d'un endomorphisme cyclique}

\q Sous-espace vectoriel :

L'endomorphisme nul appartient à \(\mathcal{C}(f)\) car l'endomorphisme nul commute avec tout
endomorphisme.

Soient \((g, h) \in \mathcal{C}(f)\) :

\((\lambda g + \mu h) \circ f = \lambda (g \circ f) + \mu (h\circ f) = \lambda (f \circ g) + \mu
(f\circ h) = f \circ (\lambda g + \mu h)\), donc \((\lambda g + \mu h) \in \mathcal{C}(f)\).

Conclusion : \fbox{\(\mathcal{C}(f)\) est un sous-espace vectoriel de \(\mathrm{End}(E)\)}.

% ------------------------------------------------------------------------------------------------
\q Egalité entre endomorphismes - Egalité vectorielle : Montrons que  \(\forall (u, v) \in \mathcal{C}(f)^2, \; u(\mathbf{x_0}) = v(\mathbf{x_0}) \iff u =
v\).

Sens \((\impliedby)\) : Si \(u = v\) pour tout vecteur, alors en particulier \(u(\mathbf{x_0}) =
v(\mathbf{x_0})\).

Sens \((\implies)\) : Puisqu'un endomorphisme est uniquement déterminé par les images des vecteurs
d'une base, il suffit de vérifier que \(u\) et \(v\) coïncident pour tous les vecteurs de la base
\(\mathcal{B} = (\mathbf{x_0}, f(\mathbf{x_0}), ..., f^{n-1}(\mathbf{x_0}))\).

Egalité vectorielle : \(u(f(\mathbf{x_0})) = f(u(\mathbf{x_0})) = f(v(\mathbf{x_0})) =
v(f(\mathbf{x_0}))\).

Par itération : \(\forall k \in \{ 0, \dots, n-1 \}, \; u(f^k(\mathbf{x_0})) = v(f^k(\mathbf{x_0}))\).

Cette égalité valant pour tous les vecteurs de base, elle vaut aussi pour tous les vecteurs de
\(E\).

Conclusion : \fbox{\(v(\mathbf{x_0}) = w(\mathbf{x_0}) \iff v = w\)}.

% ------------------------------------------------------------------------------------------------
\q Double inclusion :

\fbox{\(\mathbb{R}_{n-1}[f] \subseteq \mathcal{C}(f)\)} -- Soit \(g \in \mathbb{R}_{n-1}[f]\),
montrons que  $g$ appartient au commutant de $f$, i.e. que $g \circ f = f \circ
g$.

Par définition, \(g\) est un polynôme de degré au plus \(n-1\) en \(f\), donc il existe un unique
n-uplet de réels \((a_0, a_1, ..., a_{n-1}) \in \mathbb{R}^n\) tel que :
\[
g = a_0 \mathrm{Id}_E + a_1 f + ... + a_{n-1} f^{n-1}
\]

En appliquant \(f\) à cette égalité:
\[
\begin{aligned}
g \circ f & = (a_0 \mathrm{Id}_E + a_1 f + ... + a_{n-1} f^{n-1}) \circ f\\
& = a_0 f + a_1 f^2 + ... + a_{n-1} f^n\\
& = f \circ (a_0 \mathrm{Id}_E + a_1 f + ... + a_{n-1} f^{n-1})\\
& = f \circ g
\end{aligned}
\]
car deux polynômes d'un même endomorphisme commutent entre eux.

\fbox{\(\mathcal{C}(f) \subseteq \mathbb{R}_{n-1}[f]\)} -- Soit \(g \in \mathcal{C}(f)\), montrons que \(g\) s'écrit sous la forme d'un polynôme de \(f\).

Le \emph{vecteur} \(g(\mathbf{x_0})\) s'écrit sous la forme \(g(\mathbf{x_0}) = a_0 \mathbf{x_0} +
a_1 f(\mathbf{x_0}) + ... + a_{n-1} f^{n-1}(\mathbf{x_0})\) avec un unique n-uplet \((a_0, ...,
a_{n-1}) \in \mathbb{R}^n\), puisque \(\mathcal{B} = (\mathbf{x_0},f(\mathbf{x_0}),...,
f^{n-1}(\mathbf{x_0}))\) est une base de \(E\).

Cette égalité vectorielle implique l'égalité d'endomorphismes \(g = a_0 \mathrm{Id}_E + a_1 f +
... + a_{n-2} f^{n-2} + a_{n-1} f^{n-1}\), par la question précédente.

Ainsi, \(g\) s'écrit sous la forme d'un polynôme de \(f\), donc \(g \in \mathbb{R}_{n-1}[f]\).

Conclusion : Par double inclusion, \fbox{\(\mathcal{C}(f) = \mathbb{R}_{n-1}[f]\)}.

% ------------------------------------------------------------------------------------------------
\q Base de \(\mathcal{C}(f)\) :

Selon la question précédente, une famille génératrice de \(\mathcal{C}(f) = \mathbb{R}_{n-1}[f]\)
est donnée par \((\mathrm{Id}_E, f, ..., f^{n-1})\).

Montrons que cette famille d'endomorphismes est libre dans \(\mathrm{End}(E)\).

Soient \((\lambda_0, ...\lambda_{n-1}) \in \mathbb{R}^n\) tel que \(\lambda_0\mathrm{Id}_E +
\lambda_1 f + ... + \lambda_{n-1} f^{n-1} = 0_{\mathrm{End}(E)}\).

Évaluons cette égalité en \(\mathbf{x_0}\) : \(\lambda_0\mathbf{x_0} + \lambda_1 f(\mathbf{x_0}) +
... + \lambda_{n-2} f^{n-2}(\mathbf{x_0}) + \lambda_{n-1} f^{n-1}(\mathbf{x_0}) = \mathbf{0}\).

Or, la famille \((\mathbf{x_0}, f(\mathbf{x_0}), ..., f^{n-1}(\mathbf{x_0}))\) est libre \emph{dans
\(E\)}, ce qui implique \(\lambda_0 = \lambda_1 = ... =\lambda_{n-1} = 0\).

Donc la famille \((\mathrm{Id}_E, f, ..., f^{n-1})\) est aussi libre \emph{dans
\(\mathrm{End}(E)\)}.

Conclusion : \fbox{Une base de \(\mathcal{C}(f)\) est \((\mathrm{Id}_E, f, ..., f^{n-1})\),
\(\mathcal{C}(f)\) est donc de dimension \(n\)}.

% ---------------------------------------------------------------------------------------------------
\bigskip
\textbf{Lien avec la nilpotence}

\q Endomorphisme nilpotent maximal :

Soit $f$ un endomorphisme nilpotent.

\textit{Condition nécessaire} -- Montrons que si $f$ est cyclique, alors il est maximal.

Puisque $f$ est cyclique, il existe un vecteur \(\mathbf{x_0} \neq \mathbf{0}\) telle que la famille
\((\mathbf{x_0},f(\mathbf{x_0}), ..., f^{n-1}(\mathbf{x_0}))\) soit une base. Ceci implique que tous
ses vecteurs sont non nuls.

Or, par l'absurde, si \(f\) était nilpotent d'indice \(p < n\), alors pour tout vecteur
\(\mathbf{x_0} \neq \mathbf{0}\), \(f^p(\mathbf{x_0}) = 0\).

Ainsi, pour qu'il existe un vecteur tel que \(f^{n-1}(\mathbf{x_0}) \neq \mathbf{0}\), il faut que
l'indice de nilpotence soit \(p = n\), donc que \(f\) soit maximal.

\textit{Condition suffisante} -- Montrons que si \(f\) est maximal, alors il est cyclique.

Soit un vecteur \(\mathbf{x_0} \in E\). Puisque $f$ est nilpotent maximal, alors la famille
\((\mathbf{x_0},f(\mathbf{x_0}), ..., f^{n-1}(\mathbf{x_0}))\) contient \(n\) vecteurs et \(\dim(E)
= n\), il suffit donc de montrer qu'elle est libre.

Soient \((\lambda_0, ...\lambda_{n-1}) \in \mathbb{R}^n\) tel que \(\lambda_0\mathbf{x_0} +
\lambda_1 f(\mathbf{x_0}) + ... + \lambda_{n-1} f^{n-1}(\mathbf{x_0}) = \mathbf{0}\).

En appliquant successivement \(f\), \(f^2\), ..., \(f^{n-1}\) :

\[
\begin{aligned}
\lambda_0\mathbf{x_0} & + \lambda_1 f(\mathbf{x_0}) & + ... & + \lambda_{n-2} f^{n-2} & + \lambda_{n-1} f^{n-1}(\mathbf{x_0}) & = \mathbf{0}\\
\lambda_0f(\mathbf{x_0}) & + \lambda_1 f^2(\mathbf{x_0}) & + ... & + \lambda_{n-2} f^{n-1} & & = \mathbf{0}\\
\vdots \\
\lambda_0 f^{n-2}(\mathbf{x_0}) &+ \lambda_1 f^{n-1}(\mathbf{x_0}) & & & & = \mathbf{0}\\
\lambda_0 f^{n-1}(\mathbf{x_0}) & & & & & = \mathbf{0}
\end{aligned}
\]

Puisque \(f^{n-1}(\mathbf{x_0}) \neq \mathbf{0}\), la dernière équation implique \(\lambda_0 = 0\).

En remontant ce système d'équations vectorielles triangulaire inférieur, on obtient successivement
\(\lambda_0 = \lambda_1 = ... = \lambda_{n-1} = 0\), donc la famille est libre.

Ainsi, il existe un vecteur \(\mathbf{x_0}\) tel que \((\mathbf{x_0},f(\mathbf{x_0}),...,
f^{n-1}(\mathbf{x_0}))\) est une base de \(E\), et donc \(f\) est cyclique.

Conclusion : \fbox{Un endomorphisme nilpotent est cyclique si et seulement si il est maximal.}

\end{document}
% ==================================================================================================
