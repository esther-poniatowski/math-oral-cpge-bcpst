% PROBLÈME : Décomposition d'un endomorphisme en une somme finie de projections
% ==================================================================================================
\documentclass[10pt,a4paper]{article}

\usepackage{amsmath, amssymb}
\usepackage{mathrsfs}

\newcommand{\Mat}[2]{\mathrm{Mat}_{#1}(#2)}

% Set the root path
\providecommand{\rootpath}{../../..}
% Fonts
\usepackage[utf8]{inputenc} % for accents
\usepackage[T1]{fontenc} % for accents
\usepackage[french]{babel} % for french language
\usepackage{helvet} % sans serif font family
\renewcommand*\familydefault{\sfdefault} % sans serif font family

% Mathematics
\usepackage{amsmath,amsfonts,amssymb} % for math symbols
\usepackage{array} % for tabular


\usepackage{parskip} % no indent, space between paragraphs

\usepackage{geometry} % margin
\geometry{
    a4paper,
    left=15mm,
    right=15mm,
    top=20mm,
    bottom=20mm
}

\usepackage{circledsteps} % to draw circles around numbers

\usepackage{fancyhdr} % for headers and footers

\usepackage{enumitem} % for customizing lists
\setlist[enumerate]{itemsep=1em} % space between items only in enumerate environment (not itemize)
\setlist[itemize]{label=--} % set itemize label to em-dash

% Command: \customPageLayout{#1}{#2}{#3}
% --------------------------------------
% Description: Custom page layout with header and footer content.
% Arguments:
% #1: Header and footer content
% #2: Left header content
% #3: Right header content
% Example:
% \customPageLayout{Title}{Lycée Henri IV}{2024}
% Required Packages: fancyhdr
\newcommand{\customPageLayout}[3]{
    \pagestyle{fancy} % set page style to fancy (add header and footer)
    \fancyhf{} % clear all header and footer content
    \lhead{#2} % left header content
    \rhead{#3} % right header content
    \chead{\textbf{#1}} % center header content in bold (if needed)
    \rfoot{\thepage} % page number in the footer
}


% Counter: \q
% -----------
% Description: Display a question number in a circle.
% Usage:
% - Create a new question: add \q followed by the question content.
% - Reset the question counter: add \setcounter{q}{0} before the first question.
\newcounter{q}
\setcounter{q}{0} % set initial value of the counter
\newcommand{\q}{
    \bigskip
    \addtocounter{q}{1}
    \par
    \Circled{\textbf{\theq}} \space
}


% Counter: \ql
% ------------
% Description: Display a question letter in a round box with indentation (lowercase and not bold).
% Usage:
% - Create a new question: add \ql followed by the question content.
% - Reset the question counter: add \setcounter{ql}{0} before the first question.
\newcounter{ql}
\setcounter{ql}{0} % set initial value of the counter
\newcommand{\ql}{
    \addtocounter{ql}{1}
    \par
    \hspace{1.5em} % indentation before the circled letter
    \textcolor{gray}{\Circled{\alph{ql}}} \space % gray color
}


\title{Applications linéaires et Matrices - Décomposition d'un endomorphisme en une somme finie de projections}
\author{Esther Poniatowski}
\date{2024-2025}

\customPageLayout{Sujets d'interrogation orale}{Lycée Henri IV}{2024}

% ==================================================================================================
\begin{document}

\textbf{Contexte}

\textit{Définitions} -- Une \textbf{projection} $p$ d'un espace vectoriel $E$ est un endomorphisme
tel que :
\[
p \circ p = p
\]

\bigskip

Les projections sont des endomorphismes intéressants qui permettent de décomposer un espace
vectoriel en somme directe de deux sous-espaces.

Pour déterminer si un endomorphisme peut se décomposer en somme de projections, une condition
nécessaire et suffisante est donnée par la trace de cet endomorphisme.

\bigskip

\textbf{Objectifs}

Démontrer le théorème suivant :

\textit{Théorème} -- Soit $u$ un endomorphisme non nul de $E$. Il se décompose comme une somme finie
de projections si et seulement si sa trace est un entier naturel supérieur ou égal à son rang :
\[
u = p_1 + \dots + p_m \quad \iff \quad
\begin{cases}
\mathrm{Tr}(u) \in \mathbb{N}\\
\mathrm{Tr}(u) \geq \mathrm{rg}(u)
\end{cases}
\]

\bigskip
\textbf{Généralités sur les projections}

Soit un espace vectoriel $E$ de dimension $n \geq 2$, une projection $p$ de $E$.

\q Montrer que l'endomorphisme \(
p' = \mathrm{Id}_E - p
\)
associé à $p$ est également une projection, et qu'il vérifie :
\[
\mathrm{Im}(p) = \mathrm{Ker}(\mathrm{Id}_E - p) \quad \text{et} \quad \mathrm{Im}(\mathrm{Id}_E - p) = \mathrm{Ker}(p)
\]
% But : Comprendre la structure de l'image et du noyau associés à une projection et sa projection
% complémentaire. Méthode : Utiliser la définition d'une projection pour relier image et noyau.

\q Justifier que les matrices suivantes représentent des projections sur $\mathbb{R}^2$ ou
$\mathbb{R}^3$ en base canonique. Identifier leurs images et projections associées, et proposer des
interprétations géométriques :
\[
\begin{pmatrix}
1 & 0\\
1 & 0
\end{pmatrix},æ
\quad
\frac{1}{2}
\begin{pmatrix}
1 & 1\\
1 & 1
\end{pmatrix},
\quad
\begin{pmatrix}
1 & 0 & 0\\
0 & 0 & 0\\
0 & 0 & 0
\end{pmatrix},
\quad
\begin{pmatrix}
1 & 0 & 0\\
0 & 1 & 0\\
0 & 0 & 0
\end{pmatrix},
\quad
\begin{pmatrix}
1 & 0 & 0\\
* & 0 & 0\\
* & 0 & 0
\end{pmatrix}
\]
% But : Concrétiser la notion de projection à travers des exemples de matrices. Méthode : Vérifier
% la propriété caractéristique des projections et déterminer les sous-espaces associés.

\q Montrer que l'espace vectoriel $E$ se décompose en somme directe de deux sous-espaces :
\[
E = \mathrm{Im}(p) \oplus \mathrm{Ker}(p)
\]
c'est à dire que tout vecteur $\mathbf{x} \in E$ se décompose de manière unique en :
\[
\mathbf{x} = \mathbf{x}_i + \mathbf{x}_k \quad \text{avec} \quad \mathbf{x}_i \in \mathrm{Im}(p), \quad \mathbf{x}_k \in \mathrm{Ker}(p)
\]
% But : Mettre en évidence la décomposition canonique induite par une projection. Méthode : Utiliser
% les propriétés fondamentales des sous-espaces associés.

\q En déduire qu'il existe une base $\mathscr{C}$ dans laquelle la matrice de $p$ est
\[
\Mat{\mathscr{C}}{p} =
\begin{pmatrix}
\mathbf{Id}_r & 0\\
0 & 0
\end{pmatrix}
\]
où $r$ est le rang de $p$.
% But : Déduire la forme matricielle simple d'une projection en base adaptée. Méthode : Utiliser la
% décomposition en somme directe.

\q En déduire que $\mathrm{rg}(p) = \mathrm{Tr}(p)$.
% But : Relier trace et rang pour les projections. Méthode : Exploiter la forme diagonale de la
% matrice associée.

\bigskip
\textbf{Démonstration du théorème -- Sens direct $(\implies)$}

Soit un endomorphisme $u$ qui s'exprime comme une somme finie de projections :
\[
u = p_1 + \dots + p_m
\]

\q Démontrer que la trace est un opérateur linéaire. En déduire que $\mathrm{Tr}(u) \in \mathbb{N}$.
% But : Déduire l'intégralité de la trace de $u$. Méthode : Utiliser le fait que la trace d'une
% projection est entière.

\q Démontrer que pour deux sous-espaces $F$ et $G$ de $E$ :
\[
\dim(F + G) \leq \dim(F) + \dim(G)
\]
En déduire que $\mathrm{Tr}(u) \geq \mathrm{rg}(u)$.
% But : Relier la trace à un majorant du rang. Méthode : Utiliser les inégalités précédentes sur les
% dimensions.

\q Considérer l'application linéaire de $\mathbb{R}^2$ définie comme somme des deux projections
suivantes :
\begin{itemize}
\item Projection sur $\mathrm{Vect}\left(\begin{pmatrix} 1\\0 \end{pmatrix}\right)$ parallèlement à
$\mathrm{Vect}\left(\begin{pmatrix} 0\\1 \end{pmatrix}\right)$
\item Projection sur $\mathrm{Vect}\left(\begin{pmatrix} 1\\1 \end{pmatrix}\right)$ parallèlement à
$\mathrm{Vect}\left(\begin{pmatrix} 0\\1 \end{pmatrix}\right)$
\end{itemize}
Proposer une interprétation graphique, donner la représentation matricielle de cette application en
base canonique, et vérifier l'inégalité entre sa trace et son rang.
% But : Appliquer concrètement le théorème sur un exemple. Méthode : Calculs explicites et
% interprétation géométrique.

\bigskip
\textbf{Démonstration du théorème -- Sens réciproque $(\impliedby)$}

Soit un endomorphisme $u$ tel que $\mathrm{Tr}(u) \in \mathbb{N}$ et $\mathrm{Tr}(u) \geq
\mathrm{rg}(u) = r$.

\q Justifier que le choix d'entiers positifs $t_1, t_2$ tels que $t_1 + t_2 = \mathrm{Tr}(u)$ est
possible.
% But : Établir une décomposition de la trace. Méthode : Utiliser les propriétés des entiers
% naturels.

\q Justifier qu'il existe une base $\mathscr{B}$ dans laquelle
\[
\Mat{\mathscr{B}}{u} =
\begin{pmatrix}
M_1 & 0\\
* & 0
\end{pmatrix}
\]
avec $M_1$ une matrice $(2 \times 2)$.
% But : Réduire l'étude à un sous-espace supplémentaire du noyau. Méthode : Théorème du rang et
% existence de bases adaptées.

\q Dans le sous-espace $E'$, démontrer qu'il existe une base $\mathscr{B}'$ dans laquelle la matrice
représentative de $w = v - t_1 \mathrm{Id}_{E'}$ est
\[
\begin{pmatrix}
0 & *\\
1 & *
\end{pmatrix}
\]
% But : Trouver une forme simplifiée pour l'endomorphisme $w$. Méthode : Utiliser des propriétés
% élémentaires des endomorphismes.

\q Justifier l'existence d'une base $\mathscr{B}''$ dans laquelle la matrice représentative de $v$
est diagonale avec éléments $t_1$ et $t_2$.
% But : Démontrer l'existence d'une base diagonalisante pour $v$. Méthode : Exploiter la structure
% obtenue pour $w$.

\q Conclure que $u$ est la somme d'un nombre fini de projections.
% But : Décomposer $u$ en projections. Méthode : Construction explicite à partir des éléments
% diagonaux.

\q Étudier le cas où $M_1$ est une homothétie :
\[
M_1 = \lambda I_2
\]
Déterminer les valeurs possibles de $\lambda$.
% But : Étudier les cas particuliers liés à la forme de $M_1$. Méthode : Exploiter les contraintes
% sur trace et rang.

\q Montrer que si $\lambda = 1$, $u$ est une projection.
% But : Vérifier que l'hypothèse conduit directement à une projection. Méthode : Vérification
% directe.

\q Pour $\lambda > 1$, décomposer $u$ en utilisant
\[
P_0 =
\begin{pmatrix}
1 & 0 & 0\\
0 & 0 & 0\\
0 & 0 & 0
\end{pmatrix}
\]
et montrer que $u$ est somme de projections.
% But : Réduction au cas précédent par décomposition. Méthode : Construction explicite et
% justification.

\end{document}

\end{document}
% ==================================================================================================
