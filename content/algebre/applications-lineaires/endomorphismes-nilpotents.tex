% PROBLÈME : Endomorphismes nilpotents
% ==================================================================================================
\documentclass[10pt,a4paper]{article}

\usepackage{amsmath, amssymb}
\usepackage{mathrsfs}

\newcommand{\Mat}[2]{\mathrm{Mat}_{#1}(#2)}

\providecommand{\rootpath}{../../..}
% Fonts
\usepackage[utf8]{inputenc} % for accents
\usepackage[T1]{fontenc} % for accents
\usepackage[french]{babel} % for french language
\usepackage{helvet} % sans serif font family
\renewcommand*\familydefault{\sfdefault} % sans serif font family

% Mathematics
\usepackage{amsmath,amsfonts,amssymb} % for math symbols
\usepackage{array} % for tabular


\usepackage{parskip} % no indent, space between paragraphs

\usepackage{geometry} % margin
\geometry{
    a4paper,
    left=15mm,
    right=15mm,
    top=20mm,
    bottom=20mm
}

\usepackage{circledsteps} % to draw circles around numbers

\usepackage{fancyhdr} % for headers and footers

\usepackage{enumitem} % for customizing lists
\setlist[enumerate]{itemsep=1em} % space between items only in enumerate environment (not itemize)
\setlist[itemize]{label=--} % set itemize label to em-dash

% Command: \customPageLayout{#1}{#2}{#3}
% --------------------------------------
% Description: Custom page layout with header and footer content.
% Arguments:
% #1: Header and footer content
% #2: Left header content
% #3: Right header content
% Example:
% \customPageLayout{Title}{Lycée Henri IV}{2024}
% Required Packages: fancyhdr
\newcommand{\customPageLayout}[3]{
    \pagestyle{fancy} % set page style to fancy (add header and footer)
    \fancyhf{} % clear all header and footer content
    \lhead{#2} % left header content
    \rhead{#3} % right header content
    \chead{\textbf{#1}} % center header content in bold (if needed)
    \rfoot{\thepage} % page number in the footer
}


% Counter: \q
% -----------
% Description: Display a question number in a circle.
% Usage:
% - Create a new question: add \q followed by the question content.
% - Reset the question counter: add \setcounter{q}{0} before the first question.
\newcounter{q}
\setcounter{q}{0} % set initial value of the counter
\newcommand{\q}{
    \bigskip
    \addtocounter{q}{1}
    \par
    \Circled{\textbf{\theq}} \space
}


% Counter: \ql
% ------------
% Description: Display a question letter in a round box with indentation (lowercase and not bold).
% Usage:
% - Create a new question: add \ql followed by the question content.
% - Reset the question counter: add \setcounter{ql}{0} before the first question.
\newcounter{ql}
\setcounter{ql}{0} % set initial value of the counter
\newcommand{\ql}{
    \addtocounter{ql}{1}
    \par
    \hspace{1.5em} % indentation before the circled letter
    \textcolor{gray}{\Circled{\alph{ql}}} \space % gray color
}


\title{Applications linéaires et Matrices -- Endomorphismes nilpotents}
\author{Esther Poniatowski}
\date{2024-2025}

\customPageLayout{Sujets d'interrogation orale}{Lycée Henri IV}{2024}

% ==================================================================================================
\begin{document}

\textbf{Contexte}

\textit{Définition} -- Soit $E$ un espace vectoriel de dimension $n \geq 2$. Un endomorphisme $f \in
\mathrm{End}(E)$ est dit \textbf{nilpotent} s'il existe un entier $p \in \mathbb{N}^*$ tel que
\[
f^p = 0
\]
Le plus petit entier $p$ vérifiant cette propriété est appelé \textbf{indice de nilpotence}. Il
vérifie :
\[
f^{p-1} \neq 0 \quad \text{et} \quad f^p = 0
\]

Les endomorphismes nilpotents sont utiles en algèbre linéaire de simplifier de nombreux calculs, car leurs puissances s'annulent au
bout d'un certain nombre d'itérations. De plus, ces endomorphismes interviennent dans la classification des matrices : tout endomorphisme
est, en un certain sens, la somme d'une partie diagonalisable et d'une partie nilpotente.


\bigskip
\textbf{Objectifs}

Étudier la structure des endomorphismes nilpotents dans le cas de $\mathbb{R}^3$ et dans des espaces
de polynômes. Déterminer des formes matricielles canoniques et illustrer les inclusions entre noyaux
et images.

\bigskip
\textbf{Représentation matricielle des endomorphismes nilpotents}

Soit un espace vectoriel réel de dimension 3, muni d'une base $\mathscr{B} = (e_1, e_2,
e_3)$.

Soient deux endomorphismes nilpotents : $f$ d'indice de nilpotence $2$, $g$ d'indice de nilpotence
$3$.

\q Démontrer les inclusions suivantes :
\[
\{0\} \subset \mathrm{Ker}(f) \subset \mathrm{Ker}(f^2) = \mathbb{R}^3
\quad \text{et} \quad
\{0\} \subset \mathrm{Ker}(g) \subset \mathrm{Ker}(g^2) \subset \mathrm{Ker}(g^3) = \mathbb{R}^3
\]
% But : Identifier les inclusions naturelles entre les noyaux des puissances successives d'un
% endomorphisme nilpotent. Méthode : Utiliser les définitions de nilpotence et l'accroissement du
% noyau.

\q Démontrer que :
\[
\mathrm{Im}(f) \subseteq \mathrm{Ker}(f)
\quad \text{et} \quad
\begin{cases}
\mathrm{Im}(g) \subseteq \mathrm{Ker}(g^2)\\
\mathrm{Im}(g^2) \subseteq \mathrm{Ker}(g)
\end{cases}
\]
% But : Étudier les relations entre images et noyaux des puissances successives. Méthode : Analyser
% les compositions successives et leur effet sur les vecteurs.

\q Déterminer les dimensions des noyaux et images :
\[
\dim \mathrm{Ker}(f), \quad \dim \mathrm{Im}(f), \quad
\dim \mathrm{Ker}(g), \quad \dim \mathrm{Ker}(g^2), \quad
\dim \mathrm{Im}(g), \quad \dim \mathrm{Im}(g^2)
\]
% But : Connaître la croissance des noyaux et la décroissance des images. Méthode : Utiliser la
% relation $\dim \mathrm{Ker}(f^k) + \dim \mathrm{Im}(f^k) = \dim E$.

\q Proposer une interprétation géométrique de la structure de $f$ et $g$ à partir des résultats
précédents.
% But : Donner un sens géométrique aux inclusions et dimensions. Méthode : Représentation intuitive
% par sous-espaces emboîtés dans $\mathbb{R}^3$.

\q Montrer qu'il existe une base $\mathscr{B}' = (\mathbf{u}, \mathbf{v}, \mathbf{w})$ de
$\mathbb{R}^3$ dans laquelle la matrice de $f$ est :
\[
\Mat{\mathscr{B}'}{f} =
\begin{pmatrix}
0 & 0 & 1\\
0 & 0 & 0\\
0 & 0 & 0
\end{pmatrix}
\]
% But : Identifier une forme canonique de Jordan pour un nilpotent d'indice 2. Méthode : Construire
% une chaîne de vecteurs emboîtés.

\q Montrer qu'il existe une base $\mathscr{B}'' = (\mathbf{x}, \mathbf{y}, \mathbf{z})$ dans
laquelle la matrice de $g$ est :
\[
\Mat{\mathscr{B}''}{g} =
\begin{pmatrix}
0 & 1 & 0\\
0 & 0 & 1\\
0 & 0 & 0
\end{pmatrix}
\]
% But : Identifier une forme canonique de Jordan pour un nilpotent d'indice 3. Méthode : Construire
% une chaîne de vecteurs emboîtés de longueur 3.

\bigskip
\textbf{Exemples d'endomorphismes nilpotents sur $\mathbb{R}^3$}

Soient les matrices des endomorphismes $f$ et $g$ dans la base canonique $\mathscr{B}$ :
\[
\Mat{\mathscr{B}}{f} =
\begin{pmatrix}
2 & -1 & 1\\
-2 & 1 & -1\\
-6 & 3 & -3
\end{pmatrix}
\qquad
\Mat{\mathscr{B}}{g} =
\begin{pmatrix}
-4 & 2 & -3\\
-5 & 1 & -6\\
3 & -1 & 3
\end{pmatrix}
\]

\q Calculer les matrices $f^2$, $f^3$, $g^2$, $g^3$.
% But : Vérifier la nilpotence effective par calcul direct. Méthode : Calcul matriciel explicite.

\q Déterminer $\mathrm{Ker}(f)$, $\mathrm{Im}(f)$, $\mathrm{Ker}(f^2)$, $\mathrm{Im}(f^2)$. Vérifier
que $\mathrm{Im}(f) \subseteq \mathrm{Ker}(f)$.
% But : Identifier les sous-espaces invariants associés à $f$. Méthode : Résolution de systèmes
% linéaires associés.

\q Déterminer $\mathrm{Ker}(g)$, $\mathrm{Im}(g)$, $\mathrm{Ker}(g^2)$, $\mathrm{Im}(g^2)$. Vérifier
que $\mathrm{Im}(g) = \mathrm{Ker}(g^2)$ et $\mathrm{Im}(g^2) = \mathrm{Ker}(g)$.
% But : Identifier les inclusions et égalités structurelles. Méthode : Calculs matriciels et
% résolution de systèmes.

\q Déterminer des bases $(\mathbf{u}, \mathbf{v}, \mathbf{w})$ et $(\mathbf{x}, \mathbf{y},
\mathbf{z})$ dans lesquelles $f$ et $g$ ont pour matrices celles obtenues précédemment.
% But : Retrouver les bases adaptées à la forme de Jordan. Méthode : Construction de chaînes de
% Jordan.

\q Représenter graphiquement ces transformations dans l'espace euclidien tridimensionnel.
% But : Visualiser l'effet de la nilpotence dans $\mathbb{R}^3$. Méthode : Interprétation
% géométrique à l'aide de plans et directions.

\bigskip
\textbf{Exemples d'endomorphismes nilpotents sur $\mathbb{R}_n[X]$}

Soit l'espace vectoriel $\mathbb{R}_n[X]$ muni de la base canonique $\mathscr{B} = (1, X,
X^2, \dots, X^n)$.

Soient deux endomorphismes polynomiaux :
\begin{itemize}
\item Différence : $\Delta : P \mapsto \Delta(P)(X) = P(X+1) - P(X)$
\item Dérivation : $D : P \mapsto D(P) = P'$
\end{itemize}

\q Déterminer le degré de $\Delta(P)$ en fonction du degré de $P$.
% But : Analyser l'effet de $\Delta$ sur le degré. Méthode : Calcul direct sur un monôme.

\q Déterminer une base de $\mathrm{Ker}(\Delta)$ et une base de $\mathrm{Im}(\Delta)$.
% But : Identifier les sous-espaces stables. Méthode : Étudier l'effet de $\Delta$ sur les polynômes
% constants et non constants.

\q Montrer que $\Delta$ est nilpotent d'indice $n+1$.
% But : Vérifier que toute suite $\Delta^k(P)$ s'annule pour $k > n$. Méthode : Utiliser la
% réduction successive du degré.

\q Justifier que $D$ est nilpotent.
% But : Identifier la dérivation comme un endomorphisme nilpotent. Méthode : Connaissance du
% comportement de la dérivée des polynômes.

\q Écrire la matrice de $D$ dans la base canonique $\mathscr{B}$.
% But : Mettre en évidence la matrice triangulaire associée à $D$. Méthode : Action de $D$ sur les
% monômes.

\q Déterminer une base $\mathscr{B}' = (P_0, P_1, \dots, P_n)$ dans laquelle $D$ a pour matrice :
\[
\begin{pmatrix}
0 & 1 & 0 & \cdots & 0\\
0 & 0 & 1 & \cdots & 0\\
\vdots & \vdots & \ddots & \ddots & \vdots\\
0 & 0 & \cdots & 0 & 1\\
0 & 0 & \cdots & 0 & 0
\end{pmatrix}
\]
% But : Diagonaliser $D$ en forme de Jordan. Méthode : Construire une base de vecteurs emboîtés dans
% le noyau de $D^k$.

\end{document}
% ==================================================================================================
