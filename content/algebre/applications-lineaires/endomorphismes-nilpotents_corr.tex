% CORRECTION : Endomorphismes nilpotents
% ==================================================================================================

\documentclass[10pt,a4paper]{article}

% Set the root path
\providecommand{\rootpath}{../../..}
% Fonts
\usepackage[utf8]{inputenc} % for accents
\usepackage[T1]{fontenc} % for accents
\usepackage[french]{babel} % for french language
\usepackage{helvet} % sans serif font family
\renewcommand*\familydefault{\sfdefault} % sans serif font family

% Mathematics
\usepackage{amsmath,amsfonts,amssymb} % for math symbols
\usepackage{array} % for tabular


\usepackage{parskip} % no indent, space between paragraphs

\usepackage{geometry} % margin
\geometry{
    a4paper,
    left=15mm,
    right=15mm,
    top=20mm,
    bottom=20mm
}

\usepackage{circledsteps} % to draw circles around numbers

\usepackage{fancyhdr} % for headers and footers

\usepackage{enumitem} % for customizing lists
\setlist[enumerate]{itemsep=1em} % space between items only in enumerate environment (not itemize)
\setlist[itemize]{label=--} % set itemize label to em-dash

% Command: \customPageLayout{#1}{#2}{#3}
% --------------------------------------
% Description: Custom page layout with header and footer content.
% Arguments:
% #1: Header and footer content
% #2: Left header content
% #3: Right header content
% Example:
% \customPageLayout{Title}{Lycée Henri IV}{2024}
% Required Packages: fancyhdr
\newcommand{\customPageLayout}[3]{
    \pagestyle{fancy} % set page style to fancy (add header and footer)
    \fancyhf{} % clear all header and footer content
    \lhead{#2} % left header content
    \rhead{#3} % right header content
    \chead{\textbf{#1}} % center header content in bold (if needed)
    \rfoot{\thepage} % page number in the footer
}


% Counter: \q
% -----------
% Description: Display a question number in a circle.
% Usage:
% - Create a new question: add \q followed by the question content.
% - Reset the question counter: add \setcounter{q}{0} before the first question.
\newcounter{q}
\setcounter{q}{0} % set initial value of the counter
\newcommand{\q}{
    \bigskip
    \addtocounter{q}{1}
    \par
    \Circled{\textbf{\theq}} \space
}


% Counter: \ql
% ------------
% Description: Display a question letter in a round box with indentation (lowercase and not bold).
% Usage:
% - Create a new question: add \ql followed by the question content.
% - Reset the question counter: add \setcounter{ql}{0} before the first question.
\newcounter{ql}
\setcounter{ql}{0} % set initial value of the counter
\newcommand{\ql}{
    \addtocounter{ql}{1}
    \par
    \hspace{1.5em} % indentation before the circled letter
    \textcolor{gray}{\Circled{\alph{ql}}} \space % gray color
}


\title{Applications linéaires et Matrices - Endomorphismes nilpotents}
\author{Esther Poniatowski}
\date{2024-2025}

\customPageLayout{Correction}{Lycée Henri IV}{2024}

% ==================================================================================================
\begin{document}

\textbf{Représentation matricielle des endomorphismes nilpotents}

% --------------------------------------------------------------------------------------------------
\q Inclusions des noyaux :

Endomorphisme nilpotent d'indice 2 :\fbox{\(\{0\} \subset \ker(f) \subset \ker(f^2) =
\mathbb{R}^3\)}

\emph{Inclusions ou Egalités}
\begin{itemize}
    \item \(\{0\} \subseteq \ker(f)\) : Par propriété d'une application linéaire.
    \item \(\ker(f^2) = \mathbb{R}^3\) : Par définition d'un endomorphisme nilpotent d'indice 2, qui
    impose \(f^2 = 0\).
    \item \(\ker(f) \subseteq \ker(f^2)\) : Soit \(x \in \ker(f)\), alors \(f(x) = 0 \implies f^2(x)
    = f(f(x)) = f(0) = 0\), donc \(x \in \ker(f^2)\).\\
\end{itemize}

\emph{Non égalités}
\begin{itemize}
    \item Par l'absurde, supposons \(\ker(f) = \{0\}\). Alors \(f\) est injective, donc \(f^2\)
    aussi, donc a fortiori ce n'est pas l'application linéaire nulle. Ceci contredit la définition
    d'un endomorphisme nilpotent d'indice 2, qui impose \(f^2 = 0\).
    \item Par l'absurde, supposons \(\ker(f) = \ker(f^2) = \mathbb{R}^3\). Alors \(\forall x \in
    \mathbb{R}^3, \; f(x) = 0\), donc \(f\) est l'application linéaire nulle. Ceci contredit la
    définition d'un endomorphisme nilpotent d'indice 2, qui impose \(f \neq 0\).\\
\end{itemize}

Endomorphisme nilpotent d'indice 3 :\fbox{\(\{0\} \subset \ker(g) \subset \ker(g^2) \subset
\ker(g^3) = \mathbb{R}^3\)}

 \begin{itemize}
    \item \(\{0\} \subset \ker(g) \subseteq \ker(g^2) \subset \ker(g^3) = \mathbb{R}^3\) : Comme
    précédemment.
    \item Par l'absurde, supposons que \(\ker(g) = \ker(g^2)\). Par définition d'un endomorphisme
    nilpotent d'indice 3 : \(\forall x\in\mathbb{R}^3, \; g^3(x) = 0 = g^2(g(x))\), ce qui implique
    \(g(x) \in \ker(g^2)\). \\
    Or si \(\ker(g) = \ker(g^2)\), alors \(g(x) \in \ker(g)\) aussi, ce qui signifie que \(g(g(x)) =
    0 = g^2(x)\), donc \(\ker(g^2) = \mathbb{R}^3\). Ceci contredit l'hypothèse d'un endomorphisme
    nilpotent d'indice 3, qui impose \(g^2 \neq 0\).\\
\end{itemize}

% --------------------------------------------------------------------------------------------------
\q Inclusions images - noyaux :

Endomorphisme nilpotent d'indice 2 :

\fbox{\(\mathrm{Im}(f)\subseteq\ker(f)\)} : Soit \(x \in \mathrm{Im}(f)\), alors il existe \(y \in
\mathbb{R}^3\) tel que \(f(y) = x\), donc \(f(x) = f(f(y)) = 0\) (car \(f^2 = 0\)).\\

Endomorphisme nilpotent d'indice 3 :

\fbox{\(\mathrm{Im}(g)\subseteq\ker(g^2)\)} : Soit \(x \in \mathrm{Im}(g)\), alors il existe \(y \in
\mathbb{R}^3\) tel que \(g(y) = x\), donc \(g^2(x) = g^2(g(y)) = 0\) (car \(g^3 = 0\)).\\
\fbox{\(\mathrm{Im}(g^2)\subseteq\ker(g)\)} : Soit \(x \in \mathrm{Im}(g^2)\), alors il existe \(y
\in \mathbb{R}^3\) tel que \(g^2(y) = x\), donc \(g(x) = g(g^2(y)) = 0\) (car \(g^3 = 0\)).

% --------------------------------------------------------------------------------------------------
\q Dimensions de l'image et du noyau :

Endomorphisme nilpotent d'indice 2 :
\[
\begin{cases}
    0 < \dim(\mathrm{Im}(f)) \leq \dim(\ker(f)) < 3 \; \text{par inclusion} \\
    3 = \dim(\mathrm{Im}(f)) + \dim(\ker(f)) \; \text{par le théorème du rang}
\end{cases}
\]

La seule possibilité est :
\[
\begin{cases}
    \dim(\ker(f)) = 2\\
    \dim(\mathrm{Im}(f)) = 1
\end{cases}
\]

Endomorphisme nilpotent d'indice 3 :
\[
\begin{cases} 0 < \dim(\ker(g)) < \dim(\ker(g^2)) < 3 \; \text{par inclusions}\\
    \dim(\mathrm{Im}(g)) = 3 - \dim(\ker(g)) \; \text{par le théorème du rang}\\
    \dim(\mathrm{Im}(g^2)) = 3- \dim(\ker(g^2)) \; \text{par le théorème du rang}
\end{cases}
\]

La seule possibilité est :
\[
\begin{cases} \dim(\ker(g)) = 1\\
\dim(\mathrm{Im}(g)) = 2\\
\dim(\ker(g^2)) = 2\\
\dim(\mathrm{Im}(g^2)) = 1 \end{cases}
\]

% --------------------------------------------------------------------------------------------------
\q Interprétations géométriques :

Endomorphisme nilpotent d'indice 2 : Le noyau de \(f\) est un plan de \(\mathbb{R}^3\),
l'image de \(f\) est une droite vectorielle incluse dans ce plan.

Endomorphisme nilpotent d'indice 3 : L'image de \(g\) est un plan de \(\mathbb{R}^3\), le noyau de
\(g\) est une droite vectorielle incluse dans ce plan. Le noyau de \(g^2\) est un plan de
\(\mathbb{R}^3\) et l'image de \(g^2\) est une droite vectorielle incluse dans ce plan.

L'image de \(g\) est un plan inclus dans le noyau de \(g^2\), donc ils sont égaux. L'image de
\(g^2\) est une droite incluse dans le noyau de \(g\), donc ils sont égaux.

% --------------------------------------------------------------------------------------------------
\q Base adaptée pour un endomorphisme nilpotent d'indice 2 :

\textit{Analyse} -- La base recherchée comporte :
\begin{itemize}
 \item deux vecteurs \(u\) et \(v\) tels que \( f(u) = f(v) = 0\),
 \item un vecteur \(w \) tel que \(f(w) = u\).
\end{itemize}
Il suffit donc de choisir  :
\begin{itemize}
 \item \(w \notin \ker(f)\), ce qui est possible car \(\dim(\ker(f)) = 2 < 3\) selon les dimensions
 des noyaux et images.
 \item \(u = f(w)\), qui est non nul puisque \(w \notin \ker(f)\), et qui appartient
 à \(\ker(f)\) car \(\mathrm{Im}(f) \subset \ker(f)\) selon les relations d'inclusions.
 \item \(v \in \ker(f)\) tel que \((u,v)\) est une famille libre
 (base), ce qui est possible car \(\dim(\ker(f)) = 2\) selon les dimensions.
 \end{itemize}

\medskip

\textit{Synthèse} -- La famille \((u,v, w)\) est bien une base de \(\mathbb{R}^3\) :

D'une part, elle comporte 3 vecteurs dans un espace de dimension 3.

D'autre part, il s'agit d'une famille libre : Soient \((\lambda_1, \lambda_2, \lambda_3) \in \mathbb{R}^3\) tels que \(\lambda_1 u + \lambda_2 v
+ \lambda_3 w = 0\).

En appliquant \(f\) :
\[
\begin{aligned}
    \lambda_1 f(u) + \lambda_2 f(v) + \lambda_3 f(w) & = 0\\
    0 + 0 + \lambda_3 f(w) & = 0 \implies \lambda_3 = 0 \; \text{car \(w \notin \ker(f)\)}
\end{aligned}
\]
Alors \(\lambda_1 u + \lambda_2 v = 0\), or \((u,v)\) est
une base de \(\ker(f)\), donc une famille libre, ce qui implique \(\lambda_1 = \lambda_2 = 0\).

Conclusion : Dans \(\mathcal{B}' = (u,v, w)\) telle que
\[
\begin{cases}
w \notin \ker(f)\\
u = f(w)\\
v \in \ker(f) \setminus \mathrm{Im}(f)
\end{cases}
\] alors la matrice de \(f\) est :
\[
\mathrm{Mat}_{\mathcal{B}'}(f) =
\begin{pmatrix}
0 & 0 & 1 \\
0 & 0 & 0 \\
0 & 0 & 0
\end{pmatrix}
\]

% ------------------------------------------------------------------------------------------------
\q Base adaptée pour un endomorphisme nilpotent d'indice 3 :

\textit{Analyse} -- La base recherchée comporte :
\begin{itemize}
 \item un vecteur \(x\) tel que \(g(x) = 0\),
 \item un vecteur \(y\) tel que \(g(y) = x\),
 \item un vecteur \(z\) tel que \(g(z) = y\).
\end{itemize}
Il suffit donc de choisir  :
\begin{itemize}
 \item \(z \notin \ker(g^2)\), ce qui est possible car \(\dim(\ker(g^2)) = 2 < 3\) selon les dimensions
 \item \(y = g(z)\), qui est non nul puisque \(z \notin \ker(g)\)
 \item \(x = g(y)\), qui est non nul puisque \(y \notin \ker(g)\) (car \(z
 \notin \ker(g^2)\)).
\end{itemize}

\medskip
\textit{Synthèse} -- Comme précédemment, \((x,y, z)\) est bien une base de
\(\mathbb{R}^3\).

Conclusion : Dans \(\mathcal{B}'' = (x,y, z)\) telle que
\[
\begin{cases}
z \notin \ker(g^2)\\
y = g(z)\\
x = g(y)
\end{cases}
\]
alors la matrice de \(g\) est :
\[
\mathrm{Mat}_{\mathcal{B}''}(g) =
\begin{pmatrix}
0 & 1 & 0 \\
0 & 0 & 1 \\
0 & 0 & 0
\end{pmatrix}
\]

% --------------------------------------------------------------------------------------------------
\q Matrices des itérés

Endomorphisme nilpotent d'indice 2 :
\[
A^2 =
\begin{pmatrix}
 0 & 0 & 0\\
 0 & 0 & 0\\
 0 & 0 & 0
\end{pmatrix}
\]


Endomorphisme nilpotent d'indice 3 :
\[
B^2 =
\begin{pmatrix}
 -3 & -3 & -9\\
 -3 & -3 & -9\\
 2 & 2 & 6
\end{pmatrix},
B^3 =
\begin{pmatrix}
 0 & 0 & 0\\
 0 & 0 & 0\\
 0 & 0 & 0
\end{pmatrix}
\]

% --------------------------------------------------------------------------------------------------

\q Noyaux et Images de \(f\)

\textit{Noyau} -- En résolvant le système \(f(x) = 0\):
\[
\begin{pmatrix}x \\ y \\ z\end{pmatrix} \in \ker(f) \iff \begin{pmatrix}
 2 & -1 & 1\\
 -2 & 1 & -1\\
 -6 & 3 & -3
\end{pmatrix}
\begin{pmatrix}x \\ y \\ z\end{pmatrix} = \begin{pmatrix}0 \\ 0 \\ 0\end{pmatrix} \iff 2x - y + z = 0
\]
Le noyau est défini par une seule contrainte (plan vectoriel). Pour trouver deux vecteurs
directeurs, il est possible de fixer un degré de liberté à la fois :
\[
\begin{cases}
x = 0 \implies y = z \implies \begin{pmatrix}0 \\ 1 \\ 1\end{pmatrix} \in \ker(f)\\
z = 0 \implies y = 2x \implies \begin{pmatrix}1 \\ 2 \\ 0\end{pmatrix} \in \ker(f)
\end{cases}
\]

Alternativement, par opérations sur les colonnes :
\[
\begin{cases}
C_1 = -2 C_2 \implies f(e_1) + 2 f(e_2) = 0 \implies e_1 + 2 e_2 \in \ker(f)\\
C_3 = - C_2 \implies f(e_2) + f(e_3) = 0 \implies e_2 + e_3 \in \ker(f)
\end{cases}
\]

Conclusion : \fbox{\(\ker(f)\) est un plan vectoriel dont une base est \((e_1 + 2e_2, e_2 + e_3) = \left(\begin{pmatrix}1 \\ 2 \\ 0\end{pmatrix}, \begin{pmatrix}0 \\ 1 \\ 1\end{pmatrix}\right)\).}\\

\textit{Image} -- L'image de \(f\) est engendrée par les vecteurs :
\[
f(e_1) = \begin{pmatrix}2 \\ -2 \\ -6\end{pmatrix}, \quad
f(e_2) = \begin{pmatrix}-1 \\ 1 \\ 3\end{pmatrix}, \quad
f(e_3) = \begin{pmatrix}1 \\ -1 \\ 3\end{pmatrix}
\]
Ces trois vecteurs sont colinéaires, la matrice de \(f\) est de rang 1.\\

Conclusion : \fbox{\(\text{Im}(f)\) est une droite vectorielle dont une base est \((f(e_2)) = (-e_1 + e_2 + 3 e_3) = \left(\begin{pmatrix}-1 \\ 1 \\ 3\end{pmatrix}\right)\).}\\

\textit{Inclusion} -- Les coordonnées de \((-e_1 + e_2 + 3 e_3)\) vérifient l'équation \(2x - y + z = 0 \) de \(\ker(f)\), donc \fbox{\(\text{Im}(f) \subseteq \ker(f)\)}.

\q Noyaux et images de \(g\) :

\textit{Noyau} -- En résolvant le système \(g(x) = 0\):
\[
\begin{pmatrix}x \\ y \\ z\end{pmatrix} \in \ker(g) \iff
\begin{cases}
x - y = 0 \\
2x + 3 z =0
\end{cases} \iff \begin{cases}
x = y \\
z = - \frac{2}{3}x
\end{cases}
\]

Alternativement, par opérations sur les colonnes :
\[
\begin{cases}
    C_2 \leftarrow C_1 + 2 C_2\\
    C_3 \leftarrow -3 C_3 + 4 C_2
 \end{cases}
\mathbf{B} \iff \mathbf{B'} =
\begin{pmatrix}
     -4 & 0 & 0 \\
     -5 & -3 & -9 \\
     3 & 1 & 3
\end{pmatrix}
\]

Cette matrice \(\mathbf{B'}\) est de rang 2, car \(C_3' = 3 C_2'\).

En revenant aux colonnes de \( \mathbf{B}\) par les opérations inverses :
\[
(-3 C_3 + 4 C_2) = 3(C_1 + 2 C_2) \implies 3 C_1 + 3 C_2 - 2 C_3 = 0
\]
Ainsi :
\[
3 g(e_1) + 3 g(e_2) - 2 g(e_3) = 0 \implies 3 e_1 + 3 e_2 - 2 e_3 \in \ker(g)
\]
La matrice \(\mathbf{B}\) est de rang au moins 2 (car les trois colonnes ne sont pas proportionnelles), le noyau est donc de dimension au
 plus 1, donc ces deux vecteurs suffisent à former une base du noyau.

Conclusion : \fbox{\(\ker(g)\) a pour base \((e_1 + e_2 - \frac{2}{3}e_3)\), ou encore \((3e_1 + 3e_2 - 2 e_3)\)}.

\textit{Noyau du carré} -- Par la même approche : \(x + y + 3z = 0\).

Conclusion : \fbox{\(\ker (g^2)\) est un plan vectoriel de base \((e_1 - e_2, 3 e_1 - e_3)\)}.

\textit{Image} -- L'image est de dimension 2 (car le noyau est de dimension 1) et engendrée par
 \(g(e_1) = -4 e_1 - 5 e_2  + 3 e_3\) et \(g(e_2) = 2 e_1 + e_2 - e_3\).

Conclusion : \fbox{Une base de \(\text{Im}(g)\) est \((-4 e_1 - 5 e_2  + 3 e_3, 2 e_1 + e_2 -
e_3)\)}.

\textit{Image du carré} -- \(\text{Im}(g^2)\) est de dimension 1, a pour base \((g^2(e_1))\).

Conclusion : \fbox{Une base de \(\text{Im}(g^2)\) est \((-3 e_1 - 3 e_2 + 2 e_3)\)}.

\textit{Egalités} -- Les coordonnées des deux vecteurs de base de \(\text{Im}(g)\) vérifient l'équation de \(\ker(g^2)\) \(x + y + 3z = 0\), donc \fbox{\(\text{Im}(g) = \ker(g^2)\)}.\\
Le vecteur de base de \(\text{Im}(g^2)\) est aussi celui de \(\ker(g)\), donc \fbox{\(\text{Im}(g^2) = \ker(g)\)}.

% ------------------------------------------------------------------------------------------------
\q Bases adaptées :

Endomorphisme \(f\) -- Il suffit de choisir :
\[
\begin{cases}
\mathbf{w} \notin \ker(f)\\
\mathbf{u} = f(\mathbf{w})\\
\mathbf{v} \in \ker(f) \setminus \mathrm{Im}(f)
\end{cases}
\]
Endomorphisme \(g\) -- Il suffit de choisir :
\[
\begin{cases}
\mathbf{z} \notin \ker(g^2)\\
\mathbf{y} = g(\mathbf{z})\\
\mathbf{x} = g(\mathbf{y})
\end{cases}
\]

% ------------------------------------------------------------------------------------------------
\q \emph{Illustrations}

% --------------------------------------------------------------------------------------------------

\bigskip
\textbf{Exemples d'endomorphismes nilpotents dans un espace de polynômes : différence et dérivation}

\q Degré de \(\Delta(P)\) :

Si \( P = 0\), alors \(\Delta(P) = 0\) donc \fbox{\(\deg P = -\infty \implies \deg \Delta(P) = - \infty\)}.

Si \( P\) est constant, alors \(\Delta(P) = 0\) donc  \fbox{\(\deg P = 0 \implies \deg \Delta(P) = - \infty\)}.

Si \fbox{\(\deg P = m > 0\)}, il existe \((a_0, ..., a_m)\) avec \(a_m \neq 0\) tels que :
\[
\begin{aligned}
 P(X) & = a_m X^m + a_{m-1} X^{m-1} + ... + a_1 X + a_0\\
 P(X+1) & = a_m (X+1)^m + a_{m-1} (X+1)^{m-1} + ... + a_1 (X+1) + a_0\\
 & = a_m X^m + (a_{m-1} + m a_m)X^{m-1} + ... + a_0
\end{aligned}
\]
Alors \(\Delta(P) = m a_m X^{m-1} + ...\) donc \fbox{\(\deg \Delta(P) = m-1\)}.

% ------------------------------------------------------------------------------------------------
\q Base du noyau et base de l'image :

Base de \(\ker(\Delta)\) -- Les seuls polynômes dont l'image par \(\Delta\) est le polynôme nul sont les polynômes de degré 1
(et le polynôme nul), donc \(\ker(\Delta) = \mathrm{Span}(1)\).

\fbox{Une base de \(\ker(\Delta)\) est \((1)\).}

Base de \(\mathrm{Im}(\Delta)\) -- Tout polynôme de degré \(m \geq n\) a une image par \(\Delta\)
de degré \(m-1\), donc l'image de \(\Delta\) est l'ensemble des polynômes de degré au plus \(n-1\).

Formellement :
\begin{itemize}
\item \(\mathrm{Im}(\Delta)\) est un sous-espace vectoriel de \(\mathbb{R}_{n-1}[X]\) : \(\mathrm{Im}(\Delta) \subseteq \mathbb{R}_{n-1}[X]\).
\item Or par le théorème du rang : \(\dim(\mathrm{Im}(\Delta)) = \dim(\mathbb{R}_n[X]) - \dim(\ker(\Delta)) = (n+1) - 1 = n = \dim(\mathbb{R}_{n-1}[X])\).
\item Donc \(\mathrm{Im}(\Delta) = \mathbb{R}_{n-1}[X]\)
\end{itemize}

\fbox{Une base de \(\mathrm{Im}(\Delta)\) est \((1, X, ..., X^{n-1})\).}

% ------------------------------------------------------------------------------------------------

\q Indice de nilpotence de \(\Delta\) :

Montrons que \(\Delta^{n+1} = 0_{\mathrm{Lin}(\mathbb{R}_n[X])}\).

Soit \(P \in \mathbb{R}_n[X]\), le degré de \(P\) est donc au plus \(n\). Une utilisation répétée du
résultat de la question montre que \(\deg \Delta^k(P) \leq n - k, \; \forall k \in \mathbb{N}\).

En particulier, \(\deg \Delta^{n+1}(P) < 0\), donc  \(\Delta^{n+1}(P)\) est le polynôme nul. \\
Donc \(\Delta^{n+1} = 0_{\mathrm{Lin}(\mathbb{R}_n[X])}\).\\
Montrons que \(\Delta^n \neq 0_{\mathrm{Lin}(E)}\).\\
Considérons le polynôme \(P = X^n\), de degré exactement égal à \(n\).\\
En appliquant \(n\) fois le résultat de la question, on obtient que \(\deg\Delta^{n}(X^n) = 0\), donc  \(\Delta^{n}(X^n)\) n'est pas le polynôme nul.\\
Donc \(\Delta^n \neq 0_{\mathrm{Lin}(E)}\).\\
Conclusion : \fbox{\(\Delta\) est un endomorphisme nilpotent d'indice \(n+1\).}

% ------------------------------------------------------------------------------------------------

\q Nilpotence de \(D\) :

Dériver \(n+1\) fois un polynôme de degré au plus \(n\), conduit au polynôme nul.

\q Matrice dans la base canonique :

\(\forall k \in \{ 1, ..., n \}, \; D(X^k) = k X^{k-1}\), donc :\\
\[
\mathrm{Mat}_{\mathcal{B}}(D) = \begin{pmatrix}
0 & 1 & 0 & \cdots & 0 & 0 & 0 \\
0 & 0 & 2 & \cdots & 0 & 0 & 0 \\
0 & 0 & 0 & \ddots & \vdots & 0 & 0 \\
\vdots & \vdots & \vdots &  & n-2 & \vdots & \vdots \\
0 & 0 & 0 &  & 0 & n-1 & 0 \\
0 & 0 & 0 &  & 0 & 0 & n \\
0 & 0 & 0 &  & 0 & 0 & 0
\end{pmatrix}
\]

% ------------------------------------------------------------------------------------------------
\q Base adaptée :

On cherche une base \(\mathcal{B}' = (P_0, P_1, ..., P_n)\) telle que :\\
\[
\begin{cases}
    D(P_n) = P_{n-1}\\
    D(P_{n-1}) = P_{n-2}\\
    \vdots \\
    D(P_1) = P_0\\
    D(P_0) = 0
 \end{cases}
\]
Or
\[
\begin{cases}
    D(X^n) = n X^{n-1}\\
    D(n X^{n-1}) = n(n-1)X^{n-2}\\
    \vdots \\
    D(n(n-1)...2 X) = n!\\
    D(n!) = 0
 \end{cases}
\]
\fbox{\(\mathcal{B}' = (n!, n(n-1)...2 X, ..., n(n-1)X^{n-2}, n X^{n-1}, X^n)\)}

\end{document}
% ==================================================================================================
