\documentclass[10pt,a4paper]{article}

% Set the root path
\providecommand{\rootpath}{../../..}
% Fonts
\usepackage[utf8]{inputenc} % for accents
\usepackage[T1]{fontenc} % for accents
\usepackage[french]{babel} % for french language
\usepackage{helvet} % sans serif font family
\renewcommand*\familydefault{\sfdefault} % sans serif font family

% Mathematics
\usepackage{amsmath,amsfonts,amssymb} % for math symbols
\usepackage{array} % for tabular


\usepackage{parskip} % no indent, space between paragraphs

\usepackage{geometry} % margin
\geometry{
    a4paper,
    left=15mm,
    right=15mm,
    top=20mm,
    bottom=20mm
}

\usepackage{circledsteps} % to draw circles around numbers

\usepackage{fancyhdr} % for headers and footers

\usepackage{enumitem} % for customizing lists
\setlist[enumerate]{itemsep=1em} % space between items only in enumerate environment (not itemize)
\setlist[itemize]{label=--} % set itemize label to em-dash


\title{Correction - Matrices}
\author{}
\date{}

\begin{document}
\maketitle

\section*{Question de cours}

\subsection*{Critères d'inversibilité d'une matrice 2x2}

\begin{itemize}
    \item $\det(A) \neq 0$. \\
    Justification: Le déterminant non nul garantit l'existence d'une solution unique pour $AX = I$.
    \item $A$ ne possède pas de ligne (ou colonne) nulle.\\
    Justification: Une ligne nulle implique un déterminant nul.
    \item Les lignes (ou colonnes) de $A$ ne sont pas proportionnelles.\\
    Justification: Des lignes proportionnelles impliquent un déterminant nul.
\end{itemize}

Le déterminant de $A = \begin{pmatrix} a & b \\ c & d \end{pmatrix}$ est donné par :
$ \det(A) = ad - bc $

Si $\det(A) = 0$, alors : $ ad = bc $

Cette égalité implique une relation de proportionnalité entre les lignes ou les colonnes de la matrice :

\textbf{Proportionnalité des lignes :}

Si $a \neq 0$, on peut écrire :
$$ \frac{c}{a} = \frac{d}{b} = k $$

Cela signifie que la deuxième ligne est proportionnelle à la première :
$$ (c,d) = k(a,b) $$

\textbf{Proportionnalité des colonnes :}

De manière similaire, si $a \neq 0$, on peut écrire :
$$ \frac{b}{a} = \frac{d}{c} = k $$

Cela signifie que la deuxième colonne est proportionnelle à la première :
$$ \begin{pmatrix} b \\ d \end{pmatrix} = k\begin{pmatrix} a \\ c \end{pmatrix} $$

\textbf{Interprétation géométrique :}

Géométriquement, cela signifie que les vecteurs colonnes (ou lignes) de la matrice sont colinéaires.
Ils ne forment donc pas une base de $\mathbb{R}^2$, ce qui explique pourquoi la transformation linéaire associée à cette matrice n'est pas inversible et "écrase" l'espace en une droite ou un point.


\subsection*{Expression de l'inverse}

Si $A = \begin{pmatrix} a & b \\ c & d \end{pmatrix}$ est inversible, son inverse est donné par:

$$A^{-1} = \frac{1}{\det(A)} \begin{pmatrix} d & -b \\ -c & a \end{pmatrix}$$

où $\det(A) = ad - bc$

\section*{Exercices}

\subsection*{Inversibilité}

\begin{enumerate}
    \item Pour $A = \begin{pmatrix} 2 & 1 \\ 4 & 2 \end{pmatrix}$:

    $\det(A) = 2 \cdot 2 - 1 \cdot 4 = 0$

    $A$ n'est pas inversible car son déterminant est nul. Les lignes sont proportionnelles: la deuxième est le double de la première.

    \item Calcul de $C^{-1}$ par la méthode du pivot de Gauss:

    Matrice augmentée :
    $$
    \begin{pmatrix}
    6 & 1 & 4 & | & 1 & 0 & 0 \\
    4 & 8 & 4 & | & 0 & 1 & 0 \\
    6 & 3 & 5 & | & 0 & 0 & 1
    \end{pmatrix}
    $$

    Normaliser la première ligne en divisant par 6 :
    $$
    \begin{pmatrix}
    1 & \frac{1}{6} & \frac{2}{3} & | & \frac{1}{6} & 0 & 0 \\
    4 & 8 & 4 & | & 0 & 1 & 0 \\
    6 & 3 & 5 & | & 0 & 0 & 1
    \end{pmatrix}
    $$

    Éliminer le premier élément de la deuxième et troisième ligne :
    $$
    \begin{pmatrix}
    1 & \frac{1}{6} & \frac{2}{3} & | & \frac{1}{6} & 0 & 0 \\
    0 & \frac{23}{3} & \frac{4}{3} & | & -\frac{2}{3} & 1 & 0 \\
    0 & 2 & 1 & | & -1 & 0 & 1
    \end{pmatrix}
    $$

    Normaliser la deuxième ligne en divisant par $\frac{23}{3}$ :
    $$
    \begin{pmatrix}
    1 & \frac{1}{6} & \frac{2}{3} & | & \frac{1}{6} & 0 & 0 \\
    0 & 1 & \frac{4}{23} & | & -\frac{2}{23} & \frac{3}{23} & 0 \\
    0 & 2 & 1 & | & -1 & 0 & 1
    \end{pmatrix}
    $$

    Éliminer le deuxième élément de la première et troisième ligne :
    $$
    \begin{pmatrix}
    1 & 0 & \frac{15}{23} & | & \frac{5}{23} & -\frac{1}{23} & 0 \\
    0 & 1 & \frac{4}{23} & | & -\frac{2}{23} & \frac{3}{23} & 0 \\
    0 & 0 & \frac{15}{23} & | & -\frac{19}{23} & -\frac{6}{23} & 1
    \end{pmatrix}
    $$

    Normaliser la troisième ligne en divisant par $\frac{15}{23}$ :
    $$
    \begin{pmatrix}
    1 & 0 & \frac{15}{23} & | & \frac{5}{23} & -\frac{1}{23} & 0 \\
    0 & 1 & \frac{4}{23} & | & -\frac{2}{23} & \frac{3}{23} & 0 \\
    0 & 0 & 1 & | & -\frac{19}{15} & -\frac{2}{5} & \frac{23}{15}
    \end{pmatrix}
    $$

    Éliminer le troisième élément de la première et deuxième ligne :
    $$
    \begin{pmatrix}
    1 & 0 & 0 & | & 1 & \frac{1}{4} & -1 \\
    0 & 1 & 0 & | & \frac{1}{7} & \frac{3}{14} & -\frac{2}{7} \\
    0 & 0 & 1 & | & -\frac{9}{7} & -\frac{3}{7} & \frac{11}{7}
    \end{pmatrix}
    $$

    La partie droite de la matrice finale correspond à $C^{-1}$.
    En effet, $ C^{-1}\cdot C = I$.

\end{enumerate}

\subsection*{Puissances de matrice}

\begin{enumerate}
    \item Pour $A = \begin{pmatrix} 2 & 0 \\ 0 & 3 \end{pmatrix}$:

    $A^n = \begin{pmatrix} 2^n & 0 \\ 0 & 3^n \end{pmatrix}$ pour tout $n \geq 1$

    Justification: $A$ est diagonale, donc ses puissances sont obtenues en élevant chaque élément diagonal à la puissance $n$.

    \item Pour $B = \begin{pmatrix} 1 & 1 \\ 0 & 1 \end{pmatrix}$:

    $B^2 = \begin{pmatrix} 1 & 2 \\ 0 & 1 \end{pmatrix}$
    $B^3 = \begin{pmatrix} 1 & 3 \\ 0 & 1 \end{pmatrix}$

    Conjecture: $B^n = \begin{pmatrix} 1 & n \\ 0 & 1 \end{pmatrix}$ pour $n \geq 1$

    Démonstration par récurrence:
    \begin{itemize}
        \item Initialization: Vrai pour $n=1$
        \item Hypothèse de récurrence: Supposons vrai pour $n$
        \item Hérédité: $B^{n+1} = B^n \cdot B = \begin{pmatrix} 1 & n \\ 0 & 1 \end{pmatrix} \cdot
        \begin{pmatrix} 1 & 1 \\ 0 & 1 \end{pmatrix}$ par hypothèse de récurrence\\
        $B^{n+1} = \begin{pmatrix} 1 & n+1 \\ 0 & 1 \end{pmatrix}$
        \item Conclusion: La formule est vraie pour tout $n \geq 1$.
    \end{itemize}

    \item Pour $D = \begin{pmatrix} 0 & 1 \\ -1 & 0 \end{pmatrix}$:

    $D^2 = \begin{pmatrix} 0 & 1 \\ -1 & 0 \end{pmatrix} \cdot \begin{pmatrix} 0 & 1 \\ -1 & 0 \end{pmatrix} = \begin{pmatrix} -1 & 0 \\ 0 & -1 \end{pmatrix} = -I$\\
    $D^4 = (D^2)^2 = (-I)^2 = I$, donc les puissances de $D$ suivent un cycle de période 4.\\
    Conclusion: Pour $n \geq 1$,
    \begin{itemize}
        \item Si $n$ est pair: $D^n = (-1)^{n/2} I$
        \item Si $n$ est impair: $D^n = (-1)^{(n-1)/2} D$
    \end{itemize}

    \item Pour $
    A = \begin{pmatrix}
    2 & 1 & 0 \\
    0 & 2 & 1 \\
    0 & 0 & 2
    \end{pmatrix}
    $
    \begin{itemize}
        \item $N = A - 2I_3 = \begin{pmatrix}
        0 & 1 & 0 \\
        0 & 0 & 1 \\
        0 & 0 & 0
        \end{pmatrix}$

        \item $N^2 = \begin{pmatrix}
        0 & 0 & 1 \\
        0 & 0 & 0 \\
        0 & 0 & 0
        \end{pmatrix}$, $N^3 = 0$

        \item Pour $k \geq 3$, $N^k = 0$ car $N^3 = 0$ et les puissances supérieures seront nulles.

        \item Les deux matrices commutent : $(2I_3)N = 2N = N(2I_3)$

        \item Par le binôme de Newton:
        \begin{align*}
        A^n &= (2I_3 + N)^n = \sum_{k=0}^n \binom{n}{k} (2I_3)^{n-k} N^k \\
        &= 2^n I_3 + n 2^{n-1} N + \frac{n(n-1)}{2} 2^{n-2} N^2
        \end{align*}

        Donc,
        $$
        A^n = \begin{pmatrix}
        2^n & n 2^{n-1} & \frac{n(n-1)}{2} 2^{n-2} \\
        0 & 2^n & n 2^{n-1} \\
        0 & 0 & 2^n
        \end{pmatrix}
        $$
    \end{itemize}
\end{enumerate}

\end{document}
