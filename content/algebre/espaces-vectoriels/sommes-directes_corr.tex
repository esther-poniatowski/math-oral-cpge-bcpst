% CORRECTION : Complémentarité et Décompositions en Sommes Directes
% ==================================================================================================

\documentclass[10pt,a4paper]{article}

% Set the root path
\providecommand{\rootpath}{../../..}
% Fonts
\usepackage[utf8]{inputenc} % for accents
\usepackage[T1]{fontenc} % for accents
\usepackage[french]{babel} % for french language
\usepackage{helvet} % sans serif font family
\renewcommand*\familydefault{\sfdefault} % sans serif font family

% Mathematics
\usepackage{amsmath,amsfonts,amssymb} % for math symbols
\usepackage{array} % for tabular


\usepackage{parskip} % no indent, space between paragraphs

\usepackage{geometry} % margin
\geometry{
    a4paper,
    left=15mm,
    right=15mm,
    top=20mm,
    bottom=20mm
}

\usepackage{circledsteps} % to draw circles around numbers

\usepackage{fancyhdr} % for headers and footers

\usepackage{enumitem} % for customizing lists
\setlist[enumerate]{itemsep=1em} % space between items only in enumerate environment (not itemize)
\setlist[itemize]{label=--} % set itemize label to em-dash

% Command: \customPageLayout{#1}{#2}{#3}
% --------------------------------------
% Description: Custom page layout with header and footer content.
% Arguments:
% #1: Header and footer content
% #2: Left header content
% #3: Right header content
% Example:
% \customPageLayout{Title}{Lycée Henri IV}{2024}
% Required Packages: fancyhdr
\newcommand{\customPageLayout}[3]{
    \pagestyle{fancy} % set page style to fancy, i.e. header and footer
    \fancyhf{#1} % set header and footer content
    \lhead{#2} % set left header content
    \rhead{#3} % set right header content
    \fancyfoot{} % clear footer content
    \rfoot{\thepage} % set page number in footer
}

% Counter: \q
% -----------
% Description: Display a question number in a circle.
\newcounter{q}
\setcounter{q}{0} % set initial value of counter
\newcommand{\q}{
    \bigskip
    \addtocounter{q}{1}
    \par
    \Circled{\textbf{\theq}} \space
}


\title{Correction : Complémentarité et Décompositions en Sommes Directes}
\author{Esther Poniatowski}
\date{2024-2025}

\customPageLayout{Correction}{Lycée Henri IV}{2024}

% ==================================================================================================
\begin{document}

\textbf{Somme de sous-espaces vectoriels}

\q $F+G$ est un sous-espace vectoriel de $E$:
\begin{itemize}
    \item $F+G$ est non vide : $0 = 0 + 0 \in F+G$ puisque $F$ et $G$ contiennent l'élément nul.
    \item  $F+G$ est stable par addition : si $u_1 + v_1 \in F+G$ et $u_2 + v_2 \in F+G$, alors :
   \[
   (u_1 + v_1) + (u_2 + v_2) = (u_1 + u_2) + (v_1 + v_2) \in F+G
   \]
   car $F$ et $G$ sont des sous-espaces vectoriels.
   \item $F+G$ est stable par multiplication scalaire : si $\lambda \in \mathbb{K}$ et $u + v \in
   F+G$, alors :
   \[
   \lambda (u + v) = (\lambda u) + (\lambda v) \in F+G
   \]
   car $F$ et $G$ sont des sous-espaces vectoriels.
\end{itemize}


% --------------------------------------------------------------------------------------------------
\q Intersection de deux sous-espaces vectoriels :

Soient $F$ et $G$ deux sous-espaces vectoriels d'un espace vectoriel $E$. Montrons que $F \cap
G$ vérifie les trois propriétés caractéristiques pour un sous-espace vectoriel de $E$:
\begin{itemize}
    \item $F \cap G$ est non-vide : $\vec{0} \in F$ et $\vec{0} \in G$ car $F$ et $G$
    sont des sous-espaces vectoriels, donc $\vec{0} \in F \cap G$.
    \item $F \cap G$ est stable par addition : Soient $\vec{x}, \vec{y} \in F \cap G$. D'une part,
    $\vec{x} \in F$ et $\vec{y} \in F$, donc $\vec{x} + \vec{y} \in F$. De même, $\vec{x} \in G$ et
    $\vec{y} \in G$, donc $\vec{x} + \vec{y} \in G$. Donc $\vec{x} + \vec{y} \in F \cap G$.
    \item $F \cap G$ est stable par multiplication scalaire : Si $\vec{x} \in F \cap G$ et $\lambda
    \in \mathbb{K}$, alors $\lambda\vec{x} \in F \cap G$.
\end{itemize}

Conclusion : $F \cap G$ est bien un sous-espace vectoriel de $E$.

Conséquence sur la dimension :
\begin{itemize}
    \item $F \cap G \subset F \implies \dim(F \cap G) \leq \dim(F)$
    \item $F \cap G \subset G \implies \dim(F \cap G) \leq \dim(G)$
\end{itemize}
Donc $\dim(F \cap G) \leq \min(\dim(F), \dim(G))$.

% --------------------------------------------------------------------------------------------------
\q Formule de Grassmann :

Soient $F$ et $G$ deux sous-espaces vectoriels de dimensions respectives $n$ et $m$. Leur
intersection est un sous-espace vectoriel de dimension $k \leq \min(n, m)$.

Soit une base de $F \cap G$ (i.e. qui engendre tous les vecteurs qui sont à la fois dans $F$ et dans
$G$) :
\[ \mathcal{B}_\cap = \{e_1, \ldots, e_k\} \]

La base $\mathcal{B}_\cap$ peut être complétée en une base de $F$, en ajoutant $n-k$ vecteurs qui ne
sont pas dans $F \cap G$ (et donc dans $G$) :
\[ \mathcal{F} = \{f_1, \ldots, f_{n-k}\} \]
\[ \mathcal{B}_F = \mathcal{B}_\cap \cup \mathcal{F} = \{e_1, \ldots, e_k, f_1, \ldots, f_{n-k}\} \]

De même, la base $\mathcal{B}_\cap$ peut être complétée en une base de $G$, en ajoutant $m-k$
vecteurs qui ne sont pas dans $F \cap G$ (et donc dans $F$) :
\[ \mathcal{G} = \{g_1, \ldots, g_{m-k}\} \]
\[ \mathcal{B}_G = \mathcal{B}_\cap \cup \mathcal{G} = \{e_1, \ldots, e_k, g_1, \ldots, g_{m-k}\} \]

Soit la famille $\mathcal{B}_\cap \cup \mathcal{F} \cup \mathcal{G}$. Montrons qu'elle forme une
base de $F + G$:
\begin{itemize}
    \item Famille génératrice : $\mathcal{B}_\cap \cup \mathcal{F} \cup \mathcal{G}$ engendre $F +
    G$ car elle contient des bases de $F$ et de $G$.
    \item Famille libre : Soit une combinaison linéaire nulle des vecteurs de $\mathcal{B}_\cap \cup
    \mathcal{F} \cup \mathcal{G}$ :
    $$
    \sum_{i=1}^{k} a_i e_i + \sum_{j=1}^{n-k} b_j f_j + \sum_{l=1}^{m-k} c_l g_l = 0
    $$
    Puisque $\mathcal{B}_F$ est une base de $F$, les vecteurs $e_i$ et $f_j$ sont linéairement
    indépendants. De même, puisque $\mathcal{B}_G$ est une base de $G$, les vecteurs $e_i$ et $g_l$
    sont linéairement indépendants.

    \begin{itemize}
        \item Projection sur $F$ :
        $$
        \sum_{i=1}^{k} a_i e_i + \sum_{j=1}^{n-k} b_j f_j = -\sum_{l=1}^{m-k} c_l g_l
        $$
        Puisque les $g_l$ ne sont pas dans $F$ (sauf pour leur composante dans $F \cap G$), les
        coefficients $b_j$ doivent être nuls pour que l'égalité soit vraie, car les $f_j$ sont
        linéairement indépendants des $e_i$.
        \item Projection sur $G$ :
        $$
        \sum_{i=1}^{k} a_i e_i + \sum_{l=1}^{m-k} c_l g_l = -\sum_{j=1}^{n-k} b_j f_j
        $$
        Puisque tous les $f_j$ ne sont pas dans $G$ (sauf ceux qui sont dans $F \cap G$, déjà
        comptés dans $e_i$), les coefficients $c_l$ doivent être nuls pour que l'égalité soit vraie,
        car les $g_l$ sont linéairement indépendants des $e_i$.
        \item Avec les coefficients $b_j$ et $c_l$ établis comme nuls, l'équation originale se
        simplifie en :
        $$
        \sum_{i=1}^{k} a_i e_i = 0
        $$
        Puisque $\mathcal{B}_\cap$ est une base de $F \cap G$, les vecteurs $e_i$ sont linéairement
        indépendants, ce qui implique que tous les coefficients $a_i$ sont également nuls.
    \end{itemize}


\end{itemize}

Comptage des dimensions :
\[ \dim(F + G) = |\mathcal{B}_\cap \cup \mathcal{F} \cup \mathcal{G}| = k + (n-k) + (m-k) = n + m - k \]
\[ \dim(F) = |\mathcal{B}_F| = k + (n-k) = n \]
\[ \dim(G) = |\mathcal{B}_G| = k + (m-k) = m \]
\[ \dim(F \cap G) = |\mathcal{B}_\cap| = k \]

En combinant ces équations:
\[ \dim(F + G) = \dim(F) + \dim(G) - \dim(F \cap G) \]

% --------------------------------------------------------------------------------------------------
\bigskip
\textbf{Sommes directes}

\q Dimension de $F \oplus G$ :

Par définition de la somme directe, $F \cap G = \{0\}$, donc $\dim(F \cap G) = 0$. En appliquant la
formule de Grassmann :
\[\dim(F \oplus G) = \dim F + \dim G \]

% --------------------------------------------------------------------------------------------------

\q Existence d'un supplémentaire :

Soit $n = \dim(E)$ et $p = \dim(F)$, avec $p \leq n$ (car $F$ est un sous-espace de $E$).

Soit $(f_1, f_2, ..., f_p)$ une base de $F$.

D'après le théorème de la base incomplète, cette base peut être complétée pour former une base de
$E$. Soit donc $(f_1, f_2, ..., f_p, e_{p+1}, ..., e_n)$ une base de $E$.

En définissant $G = \text{Vect}(e_{p+1}, ..., e_n)$, alors $F$ et $G$ sont supplémentaires :
\begin{itemize}
    \item $F + G = E$ car tout vecteur de $E$ s'écrit comme combinaison linéaire des vecteurs de la
   base $(f_1, f_2, ..., f_p, e_{p+1}, ..., e_n)$.
    \item $F \cap G = \{0\}$ car si $v \in F \cap G$, alors $v$ s'écrit à la fois comme
   combinaison linéaire des $f_i$ et des $e_j$, ce qui n'est possible que si $v = 0$ (unicité de la
   décomposition dans une base).
    \item $\dim(G) = n - p$, donc $\dim(F) + \dim(G) = p + (n-p) = n = \dim(E)$.
\end{itemize}
D'après les propriétés caractéristiques des sous-espaces supplémentaires, $F$ et $G$ sont
bien supplémentaires.

% --------------------------------------------------------------------------------------------------
\q Exemple de somme directe dans $\mathbb{R}^n$ :

Soit $F = \text{Vect}((1, 0, 0), (0, 1, 0))$ et $G = \text{Vect}((0, 0, 1)$. On a $F \cap G =
\{0\}$, donc $\mathbb{R}^3 = F \oplus G$.

% --------------------------------------------------------------------------------------------------
\bigskip
\textbf{Bases et décompositions}

\q Espaces supplémentaires et unicité de la décomposition :

($\Rightarrow$) Supposons que $F \oplus G = E$. Alors tout $x \in E$ s'écrit $x = u + v$
avec $u \in F$ et $v \in G$. Supposons que $x = u' + v'$ avec $u' \in F$ et $v' \in G$. Alors :
\[u + v = u' + v' \Rightarrow (u - u') = (v' - v) \in F \cap G. \] Or, $F \cap G = \{0\}$, donc $u -
u' = 0$ et $v - v' = 0$, d'où l'unicité.

($\Leftarrow$) Réciproquement, supposons que toute décomposition est unique. Par l'absurde,
supposons qu'il existe un élément non nul dans l'intersection: $w \in F \cap G$, $w \neq 0$. Une
décomposition est obtenue en écrivant $w = w + 0$, avec $w \in F$ et $0 \in G$, et $w = 0 + w$, avec
$0 \in F$ et $w \in G$. Ces deux décompositions sont distinctes, ce qui contredit l'hypothèse
d'unicité. Donc $F \cap G = \{0\}$.

\q Projections sur $F$ et $G$ :

Tout $x \in E$ peut s'écrire de manière unique comme $x = u + v$ avec $u \in F$ et $v \in G$.

On cherche à exprimer $x$ dans la base des vecteurs de $F$ et de $G$:
\[x = \alpha_1 f_1 + \alpha_2 f_2 + \cdots + \alpha_p f_p + \beta_1 g_1 + \beta_2 g_2 + \cdots +
\beta_q g_q\]
où les inconnues sont les scalaires $\alpha_i$ et $\beta_j$, à exprimer en fonction des coordonnées
de $x$ dans la base canonique de $E$ et des coordonnées des vecteurs de $F$ et de $G$ dans la base
canonique de $E$.

D'une part, $x$ s'exprime dans la base de $E$:
\[x = x_1 e_1 + x_2 e_2 + \cdots + x_n e_n\]

D'autre part, chaque vecteur $f_i$ et chaque vecteur $g_j$ peut s'écrire comme combinaison linéaire
des vecteurs de la base de $E$:
\[f_i = \sum_{k=1}^n a_{ik} e_k \quad \text{et} \quad g_j = \sum_{k=1}^n b_{jk} e_k\]

Ainsi, le vecteur $x$ se décompose dans la base de $E$ comme:
\[x = \sum_{k=1}^n c_k e_k\]
avec $c_k = \alpha_1 a_{1k} + \alpha_2 a_{2k} + \cdots + \beta_1 b_{1k} + \beta_2 b_{2k}$.

Par unicité de la décomposition, on en déduit que les coordonnées de $x$ dans la base de $E$ sont
les mêmes que celles obtenues en exprimant $x$ dans la base de $F$ et de $G$, ce qui impose:
\[\begin{cases}
x_1 = \alpha_1 a_{11} + \alpha_2 a_{21} + \cdots + \beta_1 b_{11} + \beta_2 b_{21} \\
x_2 = \alpha_1 a_{12} + \alpha_2 a_{22} + \cdots + \beta_1 b_{12} + \beta_2 b_{22} \\
\vdots \\
x_n = \alpha_1 a_{1n} + \alpha_2 a_{2n} + \cdots + \beta_1 b_{1n} + \beta_2 b_{2n}
\end{cases}\]

Il s'agit d'un système linéaire de $n$ équations à $n$ inconnues, qui admet une unique solution si
et seulement si le déterminant de la matrice des coefficients est non nul:
\[\begin{vmatrix}
a_{11} & a_{21} & \cdots & b_{11} & b_{21} \\
a_{12} & a_{22} & \cdots & b_{12} & b_{22} \\
\vdots & \vdots & \ddots & \vdots & \vdots \\
a_{1n} & a_{2n} & \cdots & b_{1n} & b_{2n}
\end{vmatrix} \neq 0\]

Coodonnées de \(u\) dans la base canonique de \(E\):
\[[u]_E = A[\alpha]\]
avec \(A\) la matrice \(n \times p\) dont les colonnes sont les coordonnées des vecteurs \(f_1, \ldots, f_p\)
dans la base canonique de \(E\), et \([\alpha] = (\alpha_1, \ldots, \alpha_p)^T\)

Ainsi : \([x]_E = [u]_E + [v]_E = A[\alpha] + [v]_E\)

Produit scalaire: Multiplions les deux côtés de l'équation par \(A^T\):
\[A^T[x]_E = A^T A[\alpha] + A^T[v]_E\]

Orthogonalité: Comme \(v \in G\) et \(G\) est le supplémentaire de \(F\), \(v\) est orthogonal à tous les
vecteurs de \(F\). Donc \(A^T[v]_E = 0\).

Résolution pour \([\alpha]\):
\[A^T[x]_E = A^T A[\alpha]\]
\[[\alpha] = (A^T A)^{-1}A^T[x]_E\]

Expression finale de \([u]_E\): En substituant cette expression de \([\alpha]\) dans l'équation
précédente :
\[ [u]_E = A[\alpha] = A(A^T A)^{-1}A^T[x]_E \]

Pour \([v]_E\), on peut simplement utiliser la relation \([v]_E = [x]_E - [u]_E\), ce qui donne:
\[
[v]_E = [x]_E - A(A^T A)^{-1}A^T[x]_E = (I - A(A^T A)^{-1}A^T)[x]_E
\]

Conclusion : Les projections sur \(F\) et \(G\) sont respectivement les applications linéaires \(p_F\) et
\(p_G\) définies par les matrices \(A(A^T A)^{-1}A^T\) et \(I - A(A^T A)^{-1}A^T\).

% --------------------------------------------------------------------------------------------------
\bigskip
\textbf{Application : Sommes directes dans $\mathbb{R}^4$}

\q Dimensions de \(F\) et \(G\):

\[ F = \{ (x, y, z, t) \in \mathbb{R}^4 : x - y = 0, \; z - t = 0 \} \]
\[ G = \{ (x, y, z, t) \in \mathbb{R}^4 : x - z = 0, \; y - t = 0 \} \]

\( \dim(F) = 2 \) car $F$ est défini par deux équations indépendantes (deux équations imposent les relations entre $x$ et
$y$ d'une part et $z$ et $t$ d'autre part).

\( \dim(G) = 2 \) pour les mêmes raisons.

% --------------------------------------------------------------------------------------------------
\q Vérification que $F$ et $G$ ne sont pas en somme directe :

L'intersection de \( F \) et \( G \) n'est pas réduite à \(\{0\}\) : le vecteur \( v = (1, 1, 1, 1)
\) vérifie \( v \in F \cap G \) :\\
\( v \in F \) car \( v = (1, 1, 0, 0) + (0, 0, 1, 1) \)\\
\( v \in G \) car \( v = (1, 0, 1, 0) + (0, 1, 0, 1) \)

Conclusion : \( F \) et \( G \) ne sont pas en somme directe.

% --------------------------------------------------------------------------------------------------
\q Modification de \( G \) pour obtenir un supplémentaire \( G' \) de \( F \)

Un sous-espace \( G' \) est un supplémentaire de \( F \) dans \( \mathbb{R}^4 \) si et seulement si
:
\begin{itemize}
    \item \( \mathbb{R}^4 = F \oplus G' \), c'est-à-dire que chaque vecteur de \( \mathbb{R}^4 \)
    s'écrit de manière unique comme la somme d'un vecteur de \( F \) et d'un vecteur de \( G' \).
    \item \( \dim G' = 4 - \dim F = 4 - 2 = 2 \).
    \item \( F \cap G' = \{0\} \), c'est-à-dire que les seules solutions de \( v \in F \cap G' \)
    sont \( v = 0 \).
\end{itemize}

La méthode consiste à compléter une base de \( F \) en une base de \( \mathbb{R}^4 \) et de former
une base de \( G' \) à partir de ces vecteurs supplémentaires.

Base de \( F \) :
\[
F = \text{Vect} \left( \begin{bmatrix} 1 \\ 1 \\ 0 \\ 0 \end{bmatrix}, \begin{bmatrix} 0 \\ 0 \\ 1 \\ 1 \end{bmatrix} \right)
\]

Base de \( \mathbb{R}^4 \) : On cherche deux vecteurs supplémentaires \( v_3 \) et \( v_4 \)
linéairement indépendants de ceux de \( F \), c'est-à-dire qui ne peuvent pas être exprimés comme
combinaison linéaire des deux vecteurs de \( F \). Par exemple :
\[
v_3 = \begin{bmatrix} 1 \\ -1 \\ 0 \\ 0 \end{bmatrix}, \quad v_4 = \begin{bmatrix} 0 \\ 0 \\ 1 \\ -1 \end{bmatrix}
\]

La famille \( \{ v_1, v_2, v_3, v_4 \} \) est bien libre et forme une base de \( \mathbb{R}^4 \).

Définition de \( G' \) :
\[
G' = \text{Vect} \left( \begin{bmatrix} 1 \\ -1 \\ 0 \\ 0 \end{bmatrix}, \begin{bmatrix} 0 \\ 0 \\ 1 \\ -1 \end{bmatrix} \right)
\]

Ce sous-espace \( G' \) est un supplémentaire de \( F \) car :
\begin{itemize}
    \item \( \dim G' = 2 \).
    \item \( F \cap G' = \{0\} \) (on peut vérifier que les seuls coefficients permettant d'écrire
    un vecteur de \( G' \) comme combinaison linéaire des vecteurs de \( F \) sont nuls).
    \item La réunion des bases de \( F \) et \( G' \) forme une base de \( \mathbb{R}^4 \), assurant
    que \( \mathbb{R}^4 = F \oplus G' \).
\end{itemize}

% --------------------------------------------------------------------------------------------------
\q Coordonnées dans la décomposition en somme directe :

Soit un vecteur \( x \in \mathbb{R}^4 \). Il se décompose sous la forme :
\[
x = u + v
\]
avec \( u \in F \) et \( v \in G' \);

Le vecteur \( u \in F \) s'écrit sous la forme :
\[
u = \lambda \begin{bmatrix} 1 \\ 1 \\ 0 \\ 0 \end{bmatrix} + \mu \begin{bmatrix} 0 \\ 0 \\ 1 \\ 1 \end{bmatrix}
= \begin{bmatrix} \lambda \\ \lambda \\ \mu \\ \mu \end{bmatrix}
\]

Le vecteur \( v \in G' \) s'écrit sous la forme :
\[
v = \alpha \begin{bmatrix} 1 \\ -1 \\ 0 \\ 0 \end{bmatrix} + \beta \begin{bmatrix} 0 \\ 0 \\ 1 \\ -1 \end{bmatrix}
= \begin{bmatrix} \alpha \\ -\alpha \\ \beta \\ -\beta \end{bmatrix}
\]

Les inconnues \( \lambda, \mu, \alpha, \beta \) telles que \( x = u + v \) conduisent au système :
\[
\begin{bmatrix} x_1 \\ x_2 \\ x_3 \\ x_4 \end{bmatrix}
=
\begin{bmatrix} \lambda \\ \lambda \\ \mu \\ \mu \end{bmatrix}
+
\begin{bmatrix} \alpha \\ -\alpha \\ \beta \\ -\beta \end{bmatrix}
\implies
\begin{cases}
x_1 = \lambda + \alpha \\
x_2 = \lambda - \alpha \\
x_3 = \mu + \beta \\
x_4 = \mu - \beta
\end{cases}
\]

En sommant les deux premières équations :
\[
x_1 + x_2 = \lambda + \alpha + \lambda - \alpha = 2\lambda
\implies
\lambda = \frac{x_1 + x_2}{2}
\]

En soustrayant la deuxième équation de la première :
\[
x_1 - x_2 = \lambda + \alpha - (\lambda - \alpha) = 2\alpha
\implies
\alpha = \frac{x_1 - x_2}{2}
\]

En procédant de la même manière sur les deux dernières équations :
\[
x_3 + x_4 = 2\mu \quad \Rightarrow \quad \mu = \frac{x_3 + x_4}{2}
\]
\[
x_3 - x_4 = 2\beta \quad \Rightarrow \quad \beta = \frac{x_3 - x_4}{2}
\]

Conclusion :
\[
u = \frac{x_1 + x_2}{2} \begin{bmatrix} 1 \\ 1 \\ 0 \\ 0 \end{bmatrix} + \frac{x_3 + x_4}{2} \begin{bmatrix} 0 \\ 0 \\ 1 \\ 1 \end{bmatrix}
\]
\[
v = \frac{x_1 - x_2}{2} \begin{bmatrix} 1 \\ -1 \\ 0 \\ 0 \end{bmatrix} + \frac{x_3 - x_4}{2} \begin{bmatrix} 0 \\ 0 \\ 1 \\ -1 \end{bmatrix}
\]

Les coordonnées de \( x \) dans la décomposition \( F \oplus G' \) sont :
\[
(\lambda, \mu, \alpha, \beta) = \left( \frac{x_1 + x_2}{2}, \frac{x_3 + x_4}{2}, \frac{x_1 - x_2}{2}, \frac{x_3 - x_4}{2} \right)
\]


\end{document}
% ==================================================================================================
