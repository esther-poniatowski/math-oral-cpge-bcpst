% CORRECTION : Systèmes d'Équations Linéaires et Espaces Vectoriels
% ==================================================================================================

\documentclass[10pt,a4paper]{article}

% Set the root path
\providecommand{\rootpath}{../../..}
% Fonts
\usepackage[utf8]{inputenc} % for accents
\usepackage[T1]{fontenc} % for accents
\usepackage[french]{babel} % for french language
\usepackage{helvet} % sans serif font family
\renewcommand*\familydefault{\sfdefault} % sans serif font family

% Mathematics
\usepackage{amsmath,amsfonts,amssymb} % for math symbols
\usepackage{array} % for tabular


\usepackage{parskip} % no indent, space between paragraphs

\usepackage{geometry} % margin
\geometry{
    a4paper,
    left=15mm,
    right=15mm,
    top=20mm,
    bottom=20mm
}

\usepackage{circledsteps} % to draw circles around numbers

\usepackage{fancyhdr} % for headers and footers

\usepackage{enumitem} % for customizing lists
\setlist[enumerate]{itemsep=1em} % space between items only in enumerate environment (not itemize)
\setlist[itemize]{label=--} % set itemize label to em-dash

% Command: \customPageLayout{#1}{#2}{#3}
% --------------------------------------
% Description: Custom page layout with header and footer content.
% Arguments:
% #1: Header and footer content
% #2: Left header content
% #3: Right header content
% Example:
% \customPageLayout{Title}{Lycée Henri IV}{2024}
% Required Packages: fancyhdr
\newcommand{\customPageLayout}[3]{
    \pagestyle{fancy} % set page style to fancy (add header and footer)
    \fancyhf{} % clear all header and footer content
    \lhead{#2} % left header content
    \rhead{#3} % right header content
    \chead{\textbf{#1}} % center header content in bold (if needed)
    \rfoot{\thepage} % page number in the footer
}


% Counter: \q
% -----------
% Description: Display a question number in a circle.
% Usage:
% - Create a new question: add \q followed by the question content.
% - Reset the question counter: add \setcounter{q}{0} before the first question.
\newcounter{q}
\setcounter{q}{0} % set initial value of the counter
\newcommand{\q}{
    \bigskip
    \addtocounter{q}{1}
    \par
    \Circled{\textbf{\theq}} \space
}


% Counter: \ql
% ------------
% Description: Display a question letter in a round box with indentation (lowercase and not bold).
% Usage:
% - Create a new question: add \ql followed by the question content.
% - Reset the question counter: add \setcounter{ql}{0} before the first question.
\newcounter{ql}
\setcounter{ql}{0} % set initial value of the counter
\newcommand{\ql}{
    \addtocounter{ql}{1}
    \par
    \hspace{1.5em} % indentation before the circled letter
    \textcolor{gray}{\Circled{\alph{ql}}} \space % gray color
}


\title{Problème : Systèmes d'Équations Linéaires et Espaces Vectoriels}
\author{Esther Poniatowski}
\date{2024-2025}

\customPageLayout{Correction}{Lycée Henri IV}{2024}

\begin{document}
% ==================================================================================================
\textbf{Représentation vectorielle des systèmes linéaires}

\q Écriture du système linéaire sous forme matricielle :
\[
AX = B,
\]
avec :
\[
A =
\begin{bmatrix}
    a_{11} & a_{12} & \cdots & a_{1n} \\
    a_{21} & a_{22} & \cdots & a_{2n} \\
    \vdots & \vdots & \ddots & \vdots \\
    a_{m1} & a_{m2} & \cdots & a_{mn}
\end{bmatrix},
\quad
X = \begin{bmatrix} x_1 \\ x_2 \\ \vdots \\ x_n \end{bmatrix},
\quad
B = \begin{bmatrix} b_1 \\ b_2 \\ \vdots \\ b_m \end{bmatrix}
\]

Dimensions :

\begin{itemize}
    \item \( A \) : matrice \( m \times n \)
    \item \( X \) : vecteur colonne \( n \times 1 \)
    \item \( B \) : vecteur colonne \( m \times 1 \)
\end{itemize}

% --------------------------------------------------------------------------------------------------
\bigskip
\textbf{Rôle du système homogène associé}

\q L'ensemble des solutions du système homogène $AX = 0$ est un sous-espace vectoriel de
$\mathbb{R}^n$ :
\begin{itemize}
    \item Il contient le vecteur nul (trivialement, $X = 0$ est solution).
    \item Il est stable par combinaison linéaire : si $X_1$ et $X_2$ sont solutions, alors pour tout
    $\lambda, \mu \in \mathbb{R}$, $A(\lambda X_1 + \mu X_2) = \lambda AX_1 + \mu AX_2 = 0$.
\end{itemize}

% --------------------------------------------------------------------------------------------------
\q L'ensemble des solutions du système $AX = B$ est de la forme $S = X_p + S_0$ avec $X_p$ une
solution particulière du système :

\begin{itemize}
    \item  Soit $H \in S_0$ une solution du système homogène. Alors $X_p + H$ est solution du
    système : $$ A(X_p + H) = AX_p + AH = B + 0 = B $$
    \item Réciproquement, soit $X$ une solution du système. Alors $X = X_p + (X - X_p)$ et $(X -
    X_p)$ est solution du système homogène (donc appartient à $S_0$) : $$ A(X - X_p) = AX - AX_p = B
    - B = 0 $$
\end{itemize}

% --------------------------------------------------------------------------------------------------
\bigskip
\textbf{Compatibilité du système}

\q Compatibilité du système :

Sens direct : Supposons que le système \( AX = B \) soit compatible, c'est-à-dire qu'il existe un
vecteur \( X \in \mathbb{R}^n \) tel que \( AX = B \).

Par définition du produit matriciel, \( AX \) est une combinaison linéaire des colonnes de \( A \),
chaque colonne de \( A \) étant pondérée par les coefficients de \( X \). Ainsi, \( B \) s'écrit :
\[
B = x_1 A_1 + x_2 A_2 + \dots + x_n A_n,
\]
où \( A_1, A_2, \dots, A_n \) sont les colonnes de \( A \) et \( x_1, x_2, \dots, x_n \) sont les
composantes du vecteur \( X \). Cela prouve que \( B \) est une combinaison linéaire des colonnes de
\( A \).

Sens réciproque : Supposons que \( B \) soit une combinaison linéaire des colonnes de \( A \),
c'est-à-dire qu'il existe des scalaires \( x_1, x_2, \dots, x_n \) tels que :
\[
B = x_1 A_1 + x_2 A_2 + \dots + x_n A_n
\]
Le vecteur \( X = (x_1, x_2, \dots, x_n)^T \) est une solution du système \( AX = B \), donc le
système est compatible.

Conclusion : Les deux implications étant démontrées, le système \( AX = B \) est compatible si et
seulement si \( B \) est une combinaison linéaire des colonnes de \( A \).

% --------------------------------------------------------------------------------------------------
\q Existence d'une solution pour le système homogène :

L'ensemble des solutions du système homogène \( AX = 0 \) est toujours non vide, car il contient au
moins la solution triviale \( X = 0 \) :
\[
A \cdot 0 = 0
\]

% --------------------------------------------------------------------------------------------------
\bigskip
\textbf{Dimension de l'espace des solutions}

Pour démontrer l'égalité des dimensions :
$$\dim S = n - \text{rang}(A) \iff n = \dim S + \text{rang}(A)$$ une approche consiste à construire
une base de $\mathbb{R}^n$ en juxtaposant deux bases dont les dimensions correspondent aux deux
termes de la relations : une base de $S_0$ et d'une base d'antécédents de $B$ par $A$.

% --------------------------------------------------------------------------------------------------
\q Dimension de l'espace des solutions et dimension de l'espace des solutions du système homogène :

Soit \( S \) l'ensemble des solutions du système \( AX = B \) et \( S_0 \) l'ensemble des solutions
du système homogène \( AX = 0 \).

Tout d'abord, $\dim S = \dim S_0 $ car $S$ est une translation de $S_0$.

% --------------------------------------------------------------------------------------------------
\q Construction d'une base de $\mathbb{R}^n$ :

\begin{itemize}
    \item Soit $\mathcal{V} = \{v_1, \ldots, v_k\}$ une base de $S_0$, comprenant $k = \dim(S_0)$
    vecteurs.
    \item Soit $\mathcal{W} = \{w_1, \ldots, w_l\}$ une famille obtenue en complétant la base
    $\mathcal{V}$ pour former une base de $\mathbb{R}^n$, comprenant $l = n - k$ vecteurs.
\end{itemize}

% --------------------------------------------------------------------------------------------------
\q Base de $\text{Vect}(A)$ avec $\{Aw_1, \ldots, Aw_l\}$ :

\begin{itemize}
    \item Famille génératrice : Soit $Y \in \text{Vect}(A)$. Alors $\exists X \in \mathbb{R}^n$ tel
    que $AX = Y$. En effet, par définition de $\text{Vect}(A)$, $Y$ est une combinaison linéaire des
    colonnes de $A$ :
        $$Y = y_1 A_1 + y_2 A_2 + \cdots + y_n A_n$$ avec $A_i$ les colonnes de $A$ et $y_1, y_2,
        \ldots, y_n$ des scalaires.

        En définissant $X = [y_1, y_2, \ldots, y_n]^T$, par définition du produit matrice-vecteur :
        $$AX = y_1 A_1 + y_2 A_2 + \cdots + y_n A_n = Y$$

    Le vecteur $X$ se décompose dans la base initiale :
    $$X = \sum_{i=1}^k \alpha_i v_i + \sum_{j=1}^l \beta_j w_j$$ Donc : $$Y = AX = A
    \left(\sum_{i=1}^k \alpha_i v_i + \sum_{j=1}^l \beta_j w_j \right) = \sum_{i=1}^k \alpha_i Av_i
    + \sum_{j=1}^l \beta_j Aw_j = \sum_{j=1}^l \beta_j Aw_j$$ car $Av_i = 0$ pour tout $i$.

    \item Indépendance linéaire :

    Supposons qu'il existe $\gamma_1, \ldots, \gamma_l$ tels qu'il existe une combinaison linéaire
    nulle des vecteurs considérés. Alors :
    $$ \sum_{j=1}^l \gamma_j Aw_j = 0 \implies A \left(\sum_{j=1}^l \gamma_j w_j \right) = 0
    \implies \sum_{j=1}^l \gamma_j w_j \in S_0 $$

    Or, puisque $\{v_1, \ldots, v_k, w_1, \ldots, w_l\}$ est une base de $\mathbb{R}^n$, les
    vecteurs $w_j$ sont linéairement indépendants de $S_0$. Donc la seule possibilité est que
    $\gamma_j = 0$ pour tout $j$.
\end{itemize}

% --------------------------------------------------------------------------------------------------
\q Conclusion :

La décomposition de $\mathbb{R}^n$ en deux bases $\mathcal{V}$ et $\mathcal{W}$ conduit à l'égalité
des dimensions :
$$ n = k + l $$ avec :
\begin{itemize}
    \item $k = \dim S_0$
    \item $l = n - k$
\end{itemize}

Puisque $\{Aw_1, \ldots, Aw_l\}$ est une base de $\text{Vect}(A)$, alors $l = \dim(\text{Vect}(A)) =
\text{rang}(A)$.

Conclusion : $n = \dim S + \text{rang}(A)$, ce qui est équivalent à $\dim S = n - \text{rang}(A)$.

% --------------------------------------------------------------------------------------------------
\q Conditions d'existence et d'unicité des solutions :

Le système $AX = B$ admet :
\begin{itemize}
    \item Une unique solution si l'espace des solution est réduit à un point, c'est-à-dire si $\dim
    S = 0$ ($S_0 = \{0\}$), ce qui revient à $\text{rang}(A) = n$.
    \item Une infinité de solutions si $\dim S > 0$, c'est-à-dire si $S_0 \neq \{0\}$, ce qui
    revient à $\text{rang}(A) < n$.
    \item Aucune solution si $B$ n'appartient pas à l'image de $A$, c'est-à-dire si $B \notin
    \text{Vect}(A)$.
\end{itemize}


- Un système admet une unique solution si et seulement si \( \text{rang}(A)
   = n \) (matrice des coefficients de taille \( m \times n \) de rang maximal) et \( \text{rang}(A)
   = \text{rang}([A|B]) \) (matrice augmentée). - Un système admet une infinité de solutions si \(
   \text{rang}(A) < n \) et \( \text{rang}(A) = \text{rang}([A|B]) \). - Un système n'admet aucune
   solution si \( \text{rang}(A) < \text{rang}([A|B]) \).

% ==================================================================================================
\bigskip
\textbf{Application à des exemples concrets}

Pour déterminer la nature de l'ensemble des solutions des systèmes \((E_1)\), \((E_2)\) et \((E_3)\)
sans les résoudre explicitement, il faut :
\begin{itemize}
    \item Déterminer la dimension de l'espace des solutions homogènes et calculant le rang de la
    matrice des coefficients.
    \item Vérifier si le système est bien compatible en examinant si le vecteur \( B \) est une
    combinaison linéaire des colonnes de la matrice des coefficients.
\end{itemize}

Système \((E_1)\) :
\[
A =
\begin{bmatrix}
1 & 1 & 1 \\
2 & -1 & 1 \\
2 & -1 & -1
\end{bmatrix}
\quad
B = \begin{bmatrix} 3 \\ 2 \\ 0 \end{bmatrix}
\]

Rang de \( A \) :  La matrice \( A \) est une matrice \( 3 \times 3 \) dont les colonnes sont
indépendantes. En effet, supposons que la troisième colonne puisse s'écrire comme combinaison
linéaire des deux premières : \( A_3 = \alpha A_1 + \beta A_2 \). Alors, les coefficients imposent
les égalités :
\[
\begin{cases}
1 = \alpha + \beta \\
1 = 2\alpha - \beta \\
-1 = 2\alpha - \beta
\end{cases}
\]
Or les deux dernières lignes sont incompatibles.
Donc \(\text{rang}(A) = 3\).

Compatibilité : Le vecteur \( B \) est une combinaison linéaire des colonnes de \( A \)
puisqu'il est égal à \( A_1 + A_2 + A_3 \).

Conclusion : \( (E_1) \) est un système compatible déterminé.

Système \((E_2)\) :
\[
A =
\begin{bmatrix}
1 & 1 & 2 \\
2 & -1 & 1 \\
1 & 2 & 3
\end{bmatrix}
\quad
B = \begin{bmatrix} 2 \\ 1 \\ 3 \end{bmatrix}
\]

Rang de \( A \) : La troisième colonne est obtenue comme la somme des deux premières : \( A_3 = A_1
+ A_2 \). Cependant, les deux premières colonnes sont linéairement indépendantes, donc \(
\text{rang}(A) = 2 \).

Compatibilité : Le vecteur \( B \) est une combinaison linéaire des colonnes de \( A \) puisqu'il
est égal à \( A_3 \).

Conclusion : \( (E_2) \) est un système compatible indéterminé.

Système \((E_3)\) :
\[
A =
\begin{bmatrix}
1 & 1 & 1 \\
2 & 2 & 2 \\
1 & 1 & 1
\end{bmatrix}
\quad
B = \begin{bmatrix} 1 \\ 2 \\  3 \end{bmatrix}
\]

Rang de \( A \) : Les trois colonnes sont identiques, donc \(\text{rang}(A) = 1\).

Compatibilité : Pour que $B$ soit une combinaison linéaire des colonnes de $A$, il faut que $B$ soit
proportionnel à $(1,1,1)$, ce qui n'est pas le cas.

Conclusion : \( (E_3) \) est un système incompatible (aucune solution).


\q Solution paramétrique du système \((E_2)\) :

Le système \((E_2)\) est compatible indéterminé avec \( \text{rang}(A) = 2 \), donc l'espace des
solutions du système homogène associé a dimension \( n - \text{rang}(A) = 3 - 2 = 1 \). Il existe
donc une solution particulière et une direction de solution homogène :
\[
S = \left\{ \begin{pmatrix} s_1 \\ s_2 \\ s_3 \end{pmatrix} + t \begin{pmatrix} v_1 \\ v_2 \\ v_3
\end{pmatrix} \mid t \in \mathbb{R} \right\}
\]
Interprétation géométrique :
\begin{itemize}
    \item Le premier vecteur \( (s_1, s_2, s_3) \) est une solution particulière du système.
    \item Le second vecteur \( (v_1, v_2, v_3) \) est un vecteur directeur de l'espace des solutions
    homogènes \( AX = 0 \).
    \item L'ensemble des solutions est une droite paramétrée par \( t \), qui représente l'unique
    direction de liberté dans l'espace des solutions.
\end{itemize}

Choisissons \( z \) comme paramètre libre, notons-le \( t \), et réécrivons le système en fonction
de \( z = t \). Le but est d'écrire les deux autres variables $x$ et $y$ en fonction de $t$.
\[
\begin{cases}
x + y + 2t = 2 \quad \Rightarrow \quad x + y = 2 - 2t \\
2x - y + t = 1 \quad \Rightarrow \quad 2x - y = 1 - t \\
x + 2y + 3t = 3 \quad \Rightarrow \quad x + 2y = 3 - 3t
\end{cases}
\]

Exprimons \( y \) à partir de la seconde équation :
\[
y = 2x - 1 + t
\]
Remplaçons dans la première équation pour exprimer \( x \) en fonction de \( t \) uniquement :
\[
x + (2x - 1 + t) = 2 - 2t \quad \Rightarrow \quad 3x = 3 - 3t \quad \Rightarrow \quad x = 1 - t
\]
En remplaçant $x$ dans l'équation pour $y$ :
\[
y = 2(1 - t) - 1 + t = 1 - 2t
\]

Solution générale sous forme paramétrique :
\[
\begin{cases}
x = 1 - t \\
y = 1 - 2t \\
z = t
\end{cases}
\]

Soit, sous forme vectorielle :
\[
\begin{bmatrix} x \\ y \\ z \end{bmatrix}
=
\begin{bmatrix} 1 \\ 1 \\ 0 \end{bmatrix}
+ t
\begin{bmatrix} -1 \\ -2 \\ 1 \end{bmatrix}, \quad t \in \mathbb{R}.
\]

% ==================================================================================================
\bigskip
\textbf{Interprétation géométrique des méthodes de résolution}

\q La soustraction de la seconde équation à la première donne une équation triangulaire, sans modifier l'intersection des droites : la transformation
correspond à une réduction de base dans $\mathbb{R}^2$.

\end{document}
