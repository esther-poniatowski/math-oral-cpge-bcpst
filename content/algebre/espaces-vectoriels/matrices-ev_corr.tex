% CORRECTION : Espaces vectoriels de matrices
% ==================================================================================================

\documentclass[10pt,a4paper]{article}

% Set the root path
\providecommand{\rootpath}{../../..}
% Fonts
\usepackage[utf8]{inputenc} % for accents
\usepackage[T1]{fontenc} % for accents
\usepackage[french]{babel} % for french language
\usepackage{helvet} % sans serif font family
\renewcommand*\familydefault{\sfdefault} % sans serif font family

% Mathematics
\usepackage{amsmath,amsfonts,amssymb} % for math symbols
\usepackage{array} % for tabular


\usepackage{parskip} % no indent, space between paragraphs

\usepackage{geometry} % margin
\geometry{
    a4paper,
    left=15mm,
    right=15mm,
    top=20mm,
    bottom=20mm
}

\usepackage{circledsteps} % to draw circles around numbers

\usepackage{fancyhdr} % for headers and footers

\usepackage{enumitem} % for customizing lists
\setlist[enumerate]{itemsep=1em} % space between items only in enumerate environment (not itemize)
\setlist[itemize]{label=--} % set itemize label to em-dash

% Command: \customPageLayout{#1}{#2}{#3}
% --------------------------------------
% Description: Custom page layout with header and footer content.
% Arguments:
% #1: Header and footer content
% #2: Left header content
% #3: Right header content
% Example:
% \customPageLayout{Title}{Lycée Henri IV}{2024}
% Required Packages: fancyhdr
\newcommand{\customPageLayout}[3]{
    \pagestyle{fancy} % set page style to fancy (add header and footer)
    \fancyhf{} % clear all header and footer content
    \lhead{#2} % left header content
    \rhead{#3} % right header content
    \chead{\textbf{#1}} % center header content in bold (if needed)
    \rfoot{\thepage} % page number in the footer
}


% Counter: \q
% -----------
% Description: Display a question number in a circle.
% Usage:
% - Create a new question: add \q followed by the question content.
% - Reset the question counter: add \setcounter{q}{0} before the first question.
\newcounter{q}
\setcounter{q}{0} % set initial value of the counter
\newcommand{\q}{
    \bigskip
    \addtocounter{q}{1}
    \par
    \Circled{\textbf{\theq}} \space
}


% Counter: \ql
% ------------
% Description: Display a question letter in a round box with indentation (lowercase and not bold).
% Usage:
% - Create a new question: add \ql followed by the question content.
% - Reset the question counter: add \setcounter{ql}{0} before the first question.
\newcounter{ql}
\setcounter{ql}{0} % set initial value of the counter
\newcommand{\ql}{
    \addtocounter{ql}{1}
    \par
    \hspace{1.5em} % indentation before the circled letter
    \textcolor{gray}{\Circled{\alph{ql}}} \space % gray color
}


\title{Correction : Espaces vectoriels de matrices}
\author{Esther Poniatowski}
\date{2024-2025}

\customPageLayout{Correction}{Lycée Henri IV}{2024}
% ==================================================================================================
\begin{document}

\textbf{Structure vectorielle des espaces de matrices}

\q Preuve que $\mathcal{M}_{m,n}(\mathbb{R})$ vérifie les propriétés d'un sous-espace vectoriel de
$\mathbb{R}^{m \times n}$ :

\begin{itemize}
    \item Non-vide : La matrice nulle $0_{m,n}$ appartient à $\mathcal{M}_{m,n}(\mathbb{R})$.
    \item Stabilité par combinaisons linéaires : Soient $A, B \in \mathcal{M}_{m,n}(\mathbb{R})$ et
    $\lambda \in \mathbb{R}$. Alors $\lambda A + B$ est une matrice de même taille, donc appartient
    à $\mathcal{M}_{m,n}(\mathbb{R})$.
\end{itemize}

% --------------------------------------------------------------------------------------------------
\q Base canonique de $\mathcal{M}_{m,n}(\mathbb{R})$ :

Soient les matrices élémentaires $E_{ij}$ définies par :
\[ (E_{ij})_{kl} = \begin{cases} 1 & \text{si } (k,l) = (i,j) \\ 0 & \text{sinon} \end{cases} \]

Ces matrices forment une famille libre et génératrice de $\mathcal{M}_{m,n}(\mathbb{R})$,
constituant ainsi une base :
\begin{itemize}
    \item Famille génératrice : Toute matrice $A \in \mathcal{M}_{m,n}(\mathbb{R})$ peut s'écrire
    comme une combinaison linéaire des $E_{ij}$ en prenant : $$A = \sum_{i=1}^m \sum_{j=1}^n A_{ij}
    E_{ij}$$
    \item Famille libre : Supposons qu'il existe $\lambda_{ij} \in \mathbb{R}$ tels que :
    $$\sum_{i=1}^m \sum_{j=1}^n \lambda_{ij} E_{ij} = 0$$
    Le coefficient en position $(k,l)$ de cette somme est $\lambda_{kl}$, donc $\lambda_{kl} = 0$
    pour tout $(k,l)$ par unicité de l'écriture de la matrice nulle.
\end{itemize}


% --------------------------------------------------------------------------------------------------
\q Dimension de $\mathcal{M}_{m,n}(\mathbb{R})$ :

La base canonique contient $m \times n$ éléments, donc la dimension de
$\mathcal{M}_{m,n}(\mathbb{R})$ est $m \times n$.

% --------------------------------------------------------------------------------------------------
\bigskip
\textbf{Sous-espaces vectoriels de matrices}

\q Justification que $T_n$ est un sous-espace vectoriel de $\mathcal{M}_n(\mathbb{R})$ :
\begin{itemize}
    \item Non-vide : La matrice nulle $0_n$ est triangulaire supérieure car tous les coefficients
    sont nuls pour $i > j$.
    \item Stabilité par combinaisons linéaires : La somme de deux matrices triangulaires supérieures reste triangulaire supérieure,
    et un multiple d'une telle matrice l'est aussi.
\end{itemize}

Justification que $S_n$ est un sous-espace vectoriel de $\mathcal{M}_n(\mathbb{R})$ :
\begin{itemize}
    \item Non vide : La matrice nulle $0_n$ est symétrique, car elle respecte la relation $0_n =
    0_n^T$.
    \item Stabilité par combinaisons linéaires : Soient $A, B \in S_n$ et $\lambda \in \mathbb{R}$.
    Alors $(\lambda A + B)^T = \lambda A^T + B^T = \lambda A + B = \lambda A + B$.
\end{itemize}

% --------------------------------------------------------------------------------------------------
\bigskip
\textbf{Matrices symétriques et antisymétriques}

\q Base de $S_n$ : Matrices symétriques élémentaires $S_{ij} = \frac{1}{2} (E_{ij} + E_{ji}) \quad 1
\leq i \leq j \leq n$, telles que tous les coefficients sont nuls sauf ceux en position $(i,j)$ et
$(j,i)$ (égaux à 1).
\begin{itemize}
    \item Famille génératrice : Toute matrice symétrique $A$ peut s'écrire comme une combinaison
    linéaire des $S_{ij}$ en prenant : $$A = \sum_{i=1}^n \sum_{j=1}^n A_{ij} S_{ij}$$
    \item Famille libre : Comme pour la base canonique.
\end{itemize}

Dimension de $S_n$ : $ \dim S_n = \frac{n(n+1)}{2}$, obtenue en comptant les couples $(i,j)$ tels que
$1 \leq i \leq j \leq n$.

\bigskip
Base de $K_n$ : Matrices antisymétriques élémentaires : $K_{ij} = \frac{1}{2} (E_{ij} - E_{ji}) \quad 1
\leq i < j \leq n$, telles que tous les coefficients sont nuls sauf ceux en position $(i,j)$ et $(j,i)$
(égaux à 1).
\begin{itemize}
    \item Famille génératrice : Toute matrice antisymétrique $A$ peut s'écrire comme une combinaison
    linéaire des $K_{ij}$ en prenant : $$A = \sum_{i=1}^{n-1} \sum_{j=i+1}^n A_{ij} K_{ij}$$
    \item Famille libre : Comme pour la base canonique.
\end{itemize}

Dimension de $K_n$ : $ \dim K_n = \frac{n(n-1)}{2}$, obtenue en comptant les couples $(i,j)$ tels que
$1 \leq i < j \leq n$.

% --------------------------------------------------------------------------------------------------
\q Appartenance aux matrices symétriques :

$(S(A))^T = (\frac{1}{2} (A + A^T))^T = \frac{1}{2} (A^T + A) = S(A)$ donc $S(A) \in S_n$.

Appartenance aux matrices antisymétriques :

$ (K(A))^T = (\frac{1}{2} (A - A^T))^T = \frac{1}{2} (A^T - A) = -K(A)$ donc $K(A) \in K_n$.

% --------------------------------------------------------------------------------------------------
\q Nouvelle base de $\mathcal{M}_n(\mathbb{R})$ :

Soit la famille obtenue en juxtaposant les bases de $S_n$ et $K_n$ :
$$\mathcal{F} = \{S_{ij}, K_{ij} \mid 1 \leq i \leq j \leq n, 1 \leq i < j \leq n\}$$

Cette famille est une base de $\mathcal{M}_n(\mathbb{R})$ :
\begin{itemize}
    \item Famille génératrice : Toute matrice $A \in \mathcal{M}_n(\mathbb{R})$ peut s'écrire comme une
    combinaison linéaire d'un élément de $S_n$ et d'un élément de $K_n$ :
    $$ S(A) + K(A) = \frac{1}{2} (A + A^T) + \frac{1}{2} (A - A^T) = A $$
    \item Argument de dimensions : La famille $\mathcal{F}$ comprend $n^2$ éléments, qui est la
    dimension de $\mathcal{M}_n(\mathbb{R})$.
\end{itemize}

% --------------------------------------------------------------------------------------------------
\q Coordonnées dans la nouvelle base :

Soit $A \in \mathcal{M}_n(\mathbb{R})$. Cette matrice se décompose dans la base $\mathcal{F}$ :

$$ A = \sum_{i=1}^n \sum_{j=1}^n s_{ij} S_{ij} + \sum_{k=1}^{n-1} \sum_{l=k+1}^n k_{kl} K_{kl} $$

D'une part :

$$ S(A) = \sum_{i=1}^n \sum_{j=1}^n s_{ij} S_{ij} = \frac{1}{2} (A + A^T) $$

La matrice élémentaire $S_{ij}$ génère le coefficient de $S(A)$ en position $(i,j)$, donc :
$$s_{ij} = \frac{1}{2} (a_{ij} + a_{ji})$$

D'autre part :

$$ K(A) = \sum_{k=1}^{n-1} \sum_{l=k+1}^n k_{kl} K_{kl} = \frac{1}{2} (A - A^T) $$

La matrice élémentaire $K_{kl}$ génère le coefficient de $K(A)$ en position $(k,l)$, donc :
$$k_{kl} = \frac{1}{2} (a_{kl} - a_{kl})$$


% --------------------------------------------------------------------------------------------------
\bigskip
\textbf{Matrices triangulaires}

\q Intersections entre sous-espaces :

\begin{itemize}
    \item $T_n \cap S_n = D_n$ : Les matrices triangulaires supérieures symétriques sont diagonales.
    \item $T_n \cap D_n = D_n$ : Les matrices triangulaires supérieures diagonales sont des matrices
    diagonales.
    \item $T_n \cap L_n = D_n$ : Les matrices triangulaires supérieures et inférieures imposent que
    tous les coefficients non diagonaux soient nuls.
\end{itemize}

% --------------------------------------------------------------------------------------------------
\q Justification que $UT_n$ est un sous-espace vectoriel de $T_n$ :
\begin{itemize}
    \item Non-vide : La matrice nulle $0_n$ est une matrice triangulaire supérieure stricte.
    \item Stabilité par combinaisons linéaires : La somme de deux matrices triangulaires supérieures
    strictes reste triangulaire supérieure stricte, et un multiple d'une telle matrice l'est aussi.
\end{itemize}

% --------------------------------------------------------------------------------------------------
\q Décomposition d'une matrice en somme de matrices triangulaires :

Toute matrice $A$ peut être décomposée :

$$ A = L + U $$

Avec $ L $ la matrice triangulaire inférieure obtenue en annulant les coefficients au-dessus de la
diagonale, et $ U $ la matrice triangulaire supérieure stricte obtenue en annulant les coefficients
sur la diagonale et en-dessous.

% --------------------------------------------------------------------------------------------------
\q Juxtapositions de bases :

Soit $\mathcal{F}$ la famille obtenue en juxtaposant une base de $T_n$ et une base de $L_n$.

Les dimensions de $T_n$ et de $L_n$ sont toutes deux de $n(n+1)/2$, donc la dimension de la famille
$\mathcal{F}$ est $2 \times n(n+1)/2 = n(n+1)$.

La dimension de $\mathcal{M}_n(\mathbb{R})$ étant $n^2$, la famille $\mathcal{F}$ n'est pas une base
de $\mathcal{M}_n(\mathbb{R})$ car elle ne peut pas être libre.

\end{document}
