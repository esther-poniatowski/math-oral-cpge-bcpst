% PROBLÈME : Systèmes d'Équations Linéaires et Espaces Vectoriels
% ==================================================================================================
\documentclass[10pt,a4paper]{article}

% Set the root path
\providecommand{\rootpath}{../../..}
% Fonts
\usepackage[utf8]{inputenc} % for accents
\usepackage[T1]{fontenc} % for accents
\usepackage[french]{babel} % for french language
\usepackage{helvet} % sans serif font family
\renewcommand*\familydefault{\sfdefault} % sans serif font family

% Mathematics
\usepackage{amsmath,amsfonts,amssymb} % for math symbols
\usepackage{array} % for tabular


\usepackage{parskip} % no indent, space between paragraphs

\usepackage{geometry} % margin
\geometry{
    a4paper,
    left=15mm,
    right=15mm,
    top=20mm,
    bottom=20mm
}

\usepackage{circledsteps} % to draw circles around numbers

\usepackage{fancyhdr} % for headers and footers

\usepackage{enumitem} % for customizing lists
\setlist[enumerate]{itemsep=1em} % space between items only in enumerate environment (not itemize)
\setlist[itemize]{label=--} % set itemize label to em-dash

% Command: \customPageLayout{#1}{#2}{#3}
% --------------------------------------
% Description: Custom page layout with header and footer content.
% Arguments:
% #1: Header and footer content
% #2: Left header content
% #3: Right header content
% Example:
% \customPageLayout{Title}{Lycée Henri IV}{2024}
% Required Packages: fancyhdr
\newcommand{\customPageLayout}[3]{
    \pagestyle{fancy} % set page style to fancy, i.e. header and footer
    \fancyhf{#1} % set header and footer content
    \lhead{#2} % set left header content
    \rhead{#3} % set right header content
    \fancyfoot{} % clear footer content
    \rfoot{\thepage} % set page number in footer
}

% Counter: \q
% -----------
% Description: Display a question number in a circle.
\newcounter{q}
\setcounter{q}{0} % set initial value of counter
\newcommand{\q}{
    \bigskip
    \addtocounter{q}{1}
    \par
    \Circled{\textbf{\theq}} \space
}


\title{Problème : Systèmes d'Équations Linéaires et Espaces Vectoriels}
\author{Esther Poniatowski}
\date{2025}

\customPageLayout{Sujets d'interrogation orale}{Lycée Henri IV}{2024-2025}

% ==================================================================================================
\begin{document}

\textbf{Contexte}

La résolution des systèmes d'équations linéaires repose sur deux perspectives complémentaires :

D'une part, l'approche \textit{algébrique} consiste à manipuler ces systèmes sous forme matricielle,
afin d'utiliser des techniques issues de l'algèbre linéaire. La structure des solutions est alors
décrite en termes de sous-espaces vectoriels et d'additivité des solutions.

D'autre part, l'approche \textit{géométrique} offre une interprétation intuitive des équations
linéaires : chaque équation définit un hyperplan dans un espace vectoriel de dimension finie, et
l'ensemble des solutions du système correspond à l'intersection de ces hyperplans. Cette perspective
permet de visualiser l'existence, l'unicité ou l'infinité des solutions en fonction de la position
relative des hyperplans définis par les équations.

\bigskip
\textbf{Objectifs}

Exploiter ces deux approches afin de caractériser l'ensemble des solutions d'un système linéaire et
d'en comprendre la structure vectorielle.

\bigskip
\textbf{Représentation vectorielle des systèmes linéaires}

Soit le système linéaire suivant à $n$ inconnues et $m$ équations :

$$
\begin{cases}
a_{11}x_1 + a_{12}x_2 + \cdots + a_{1n}x_n = b_1 \\
a_{21}x_1 + a_{22}x_2 + \cdots + a_{2n}x_n = b_2 \\
\vdots \\
a_{m1}x_1 + a_{m2}x_2 + \cdots + a_{mn}x_n = b_m
\end{cases}
$$

\q Justifier que ce système peut s'écrire sous la forme matricielle $AX = B$, où $A$ est une
matrice, $X$ un vecteur colonne, et $B$ un vecteur colonne, dont les dimensions sont à préciser.
% But : Introduire la représentation matricielle du système linéaire.
% Démarche :Identifier les coefficients de la matrice A, les composantes du vecteur X (inconnues) et
% du vecteur B (termes constants).

\bigskip
\textbf{Role du système homogène associé}

\bigskip
Soit le système \textit{homogène} associé : $AX = 0$.

\q Montrer que l'ensemble $S_0$ des solutions de ce système homogène est un sous-espace vectoriel de
$\mathbb{R}^n$.
% But : Démontrer une propriété fondamentale des systèmes homogènes.
% Démarche : Vérifier que S_0 est non vide (contient le vecteur nul) et est stable par combinaison
% linéaire.

\bigskip
Soit $X_p$ une solution \textit{particulière} du système initial $AX = B$.

\q Montrer que l'ensemble $S$ des solutions de $AX = B$ peut s'écrire sous la forme $S = X_p + S_0$.
% But : Établir la structure de l'ensemble des solutions du système non homogène.
% Démarche : Montrer que tout élément de S est de la forme X_p + v avec v dans S_0, et
% réciproquement.

\bigskip
\textbf{Compatibilité du système}

Un système linéaire est dit \textit{compatible} s'il admet au moins une solution.

\q Montrer que le système $AX = B$ est compatible si et seulement si $B$ est une combinaison
linéaire des colonnes de $A$.
% But : Caractériser la compatibilité du système en termes de combinaisons linéaires.
% Démarche : Montrer que si B est une combinaison linéaire des colonnes de A, alors il existe une
% solution, et réciproquement.

\q En déduire que le système homogène $AX = 0$ admet toujours une solution.
% But : Caractériser la compatibilité du système homogène.
% Démarche : Montrer que le vecteur nul est une solution du système homogène, donc il est toujours
% compatible.

\bigskip
\textbf{Dimension de l'espace des solutions}

Le but de cette partie est de démontrer la relation :
$$ \dim S = n - \text{rang}(A) $$
avec
\begin{itemize}
    \item $\dim S$ la dimension de l'espace des solutions,
    \item $n$ le nombre d'inconnues $n$,
    \item $\text{rang}(A) = \dim \text{Vect}(A)$ la dimension de l'espace vectoriel généré par les
    colonnes de $A$ (noté $\text{Vect}(A)$ ici).
\end{itemize}

\q Justifier que la dimension de $S$ est égale à celle de $S_0$.
% But : Comprendre la relation entre la dimension de S et celle de S_0.
% Démarche : Utiliser la structure de S = X_p + S_0 pour montrer que dim S = dim S_0.

\q Justifier qu'il est possible de construire une base de $\mathbb{R}^n$ à partir d'une base de
$S_0$ :
$$ \mathcal{B} = \{v_1, \ldots, v_k, w_1, \ldots, w_{l}\} $$
Préciser les dimensions $k$ et $l$.
% But : Établir une base de R^n à partir de S_0.
% Démarche : Construire une base de S_0, puis compléter cette base pour former une base de R^n.

\q Montrer que $\{Aw_1, \ldots, Aw_l\}$ forme une base de $\text{Vect}(A)$.
% But : Caractériser une base de l'espace colonne de A pour identifier l avec le rang de A.
% Démarche : Montrer que tout élément de Vect(A) est une combinaison linéaire des colonnes de A, et
% réciproquement.

\q En déduire la relation $\dim S = n - \text{rang}(A)$.

\q Résumer les conditions nécessaires et suffisantes pour que le système $AX = B$ admette :
\begin{itemize}
    \item[(a)] Une unique solution (système compatible déterminé)
    \item[(b)] Une infinité de solutions (système compatible indéterminé)
    \item[(c)] Aucune solution (système incompatible)
\end{itemize}
% But : Caractériser complètement les différents cas de résolution du système.
% Démarche : Étudier les cas S_0 = 0, dim S_0 > 0, et comparer rang(A) et n pour la compatibilité du
% système.

\bigskip
\textbf{Application à un exemple concret}

Soient les systèmes suivants :

$$ (E_1) : \;
\begin{cases}
x + y + z = 3 \\
2x - y + z = 2 \\
2x - y - z = 0
\end{cases}
\quad
(E_2) : \;
\begin{cases}
x + y + 2z = 2 \\
2x - y + z = 1 \\
x + 2y + 3z = 3
\end{cases}
\quad
(E_3) : \;
\begin{cases}
x + y + z = 1 \\
2x + 2y + 2z = 2 \\
x + y + z = 3
\end{cases}
$$

\q Sans résoudre les systèmes, déterminer la nature de l'ensemble des solutions.
% But : Utiliser les critères théoriques pour caractériser les solutions sans résolution explicite.
% Démarche : Calculer le rang de A en considérant les combinaisons linéaires potentielles entre les
% colonnes, puis comparer le rang de A avec le nombre de variables et utiliser les conditions
% établies précédemment.

\q Proposer une écriture paramétrique pour la solution générale du système $(E_2)$.
% But : Proposer une écriture paramétrique de la solution.
% Démarche : Utiliser l'analogie avec les espaces vectoriels pour exprimer la solution générale
% lorsqu'il existe une infinité de solutions.

\bigskip
\textbf{Interprétation géométrique des méthodes de résolution}

Le \textit{pivot de Gauss} peut être interprété géométriquement comme une série de transformations
des hyperplans représentés par les équations du système.

Soit le système à deux équations et deux inconnues suivant :

$$
\begin{cases}
2x + y = 3 \\
x + y = 2
\end{cases}
$$

\q Interprétation géométrique : Justifier que chaque équation du système définit une droite de
$\mathbb{R}^2$. Proposer une interprétation géométrique à l'ensemble des solutions du système.
% But : Établir le lien entre les équations linéaires et les hyperplans, et interpréter
% géométriquement l'ensemble des solutions.
% Démarche : Reconnaître que chaque équation est de la forme < a_i, x > = b_i où a_i est la i-ème
% ligne de A, puis interpréter l'ensemble des solutions comme l'intersection de ces hyperplans.

\q Représenter graphiquement les droites initiales. Montrer comment la soustraction de la deuxième
équation à la première modifie l'orientation d'une des droites sans changer leur point
d'intersection. Quelle propriété de l'espace vectoriel $\mathbb{R}^2$ est illustrée par cette
opération ?
% But : Comprendre l'interprétation géométrique du pivot de Gauss et son lien avec les propriétés
% des espaces vectoriels.
% Démarche : Représenter les droites, effectuer l'opération élémentaire, et montrer que le nouveau
% système engendre le même sous-espace affine que le système initial.

\end{document}
% ==================================================================================================


\bigskip
Il existe d'autres méthodes de résolution qui ont une interprétation plus directement géométrique.

La méthode de \textit{projection orthogonale} est à la base de nombreuses techniques d'optimisation,
notamment dans le cas des systèmes \textit{surdéterminés}, pour trouver la solution qui minimise
l'erreur.

Soit le système $AX = B$ où $A$ est une matrice $m \times n$ avec $m > n$ (système surdéterminé).

Soit $E$ le sous-espace vectoriel de $\mathbb{R}^m$ engendré par les colonnes de $A$. La méthode
revient à chercher le point de $E$ le plus proche de $B$.

Cette solution "optimale" est donnée par $X^* = (A^TA)^{-1}A^TB$ lorsque $A^TA$ est inversible.

\q En déduire que la solution optimale $X^*$ vérifie l'équation $A^T(AX^* - B) = 0$. Interpréter
géométriquement cette équation.
% But : Établir les équations normales et leur interprétation géométrique.
% Démarche : Utiliser la caractérisation de la projection orthogonale en termes d'orthogonalité du
% vecteur résiduel avec E.

\q Interpréter géométriquement cette solution comme la projection orthogonale de $B$ sur l'espace
colonne de $A$.

\bigskip
La méthode de \textit{relaxation} est une méthode \textit{itérative} pour résoudre des systèmes linéaires :
\begin{itemize}
    \item Partir d'une approximation initiale $X^{(0)}$ de la solution.
    \item Mettre à jour chaque composante $x_i$ en fonction des autres composantes et des équations
    du système, jusqu'à convergence.
\end{itemize}
\[
x_i^{(k+1)} = \frac{1}{a_{ii}} \left( b_i - \sum_{j \neq i} a_{ij} x_j^{(k)} \right)
\]

\q Soit $X^{(k)}$ une approximation de la solution à l'étape $k$. Montrer que l'équation $a_{ii}x_i
+ \sum_{j\neq i} a_{ij}x_j = b_i$ définit un hyperplan $H_i$ dans $\mathbb{R}^n$.

\q Interpréter géométriquement le passage de $X^{(k)}$ à $X^{(k+1)}$ comme un déplacement dans
l'espace des solutions, i.e. une projection sur les hyperplans définis par chaque équation.
% But : Interpréter géométriquement la méthode de Jacobi.
% Démarche : Montrer que chaque composante est mise à jour en projetant sur l'hyperplan correspondant.
