% PROBLÈME : Complémentarité et Décompositions en Sommes Directes
% ==================================================================================================
\documentclass[10pt,a4paper]{article}

% Set the root path
\providecommand{\rootpath}{../../..}
% Fonts
\usepackage[utf8]{inputenc} % for accents
\usepackage[T1]{fontenc} % for accents
\usepackage[french]{babel} % for french language
\usepackage{helvet} % sans serif font family
\renewcommand*\familydefault{\sfdefault} % sans serif font family

% Mathematics
\usepackage{amsmath,amsfonts,amssymb} % for math symbols
\usepackage{array} % for tabular


\usepackage{parskip} % no indent, space between paragraphs

\usepackage{geometry} % margin
\geometry{
    a4paper,
    left=15mm,
    right=15mm,
    top=20mm,
    bottom=20mm
}

\usepackage{circledsteps} % to draw circles around numbers

\usepackage{fancyhdr} % for headers and footers

\usepackage{enumitem} % for customizing lists
\setlist[enumerate]{itemsep=1em} % space between items only in enumerate environment (not itemize)
\setlist[itemize]{label=--} % set itemize label to em-dash

% Command: \customPageLayout{#1}{#2}{#3}
% --------------------------------------
% Description: Custom page layout with header and footer content.
% Arguments:
% #1: Header and footer content
% #2: Left header content
% #3: Right header content
% Example:
% \customPageLayout{Title}{Lycée Henri IV}{2024}
% Required Packages: fancyhdr
\newcommand{\customPageLayout}[3]{
    \pagestyle{fancy} % set page style to fancy (add header and footer)
    \fancyhf{} % clear all header and footer content
    \lhead{#2} % left header content
    \rhead{#3} % right header content
    \chead{\textbf{#1}} % center header content in bold (if needed)
    \rfoot{\thepage} % page number in the footer
}


% Counter: \q
% -----------
% Description: Display a question number in a circle.
% Usage:
% - Create a new question: add \q followed by the question content.
% - Reset the question counter: add \setcounter{q}{0} before the first question.
\newcounter{q}
\setcounter{q}{0} % set initial value of the counter
\newcommand{\q}{
    \bigskip
    \addtocounter{q}{1}
    \par
    \Circled{\textbf{\theq}} \space
}


% Counter: \ql
% ------------
% Description: Display a question letter in a round box with indentation (lowercase and not bold).
% Usage:
% - Create a new question: add \ql followed by the question content.
% - Reset the question counter: add \setcounter{ql}{0} before the first question.
\newcounter{ql}
\setcounter{ql}{0} % set initial value of the counter
\newcommand{\ql}{
    \addtocounter{ql}{1}
    \par
    \hspace{1.5em} % indentation before the circled letter
    \textcolor{gray}{\Circled{\alph{ql}}} \space % gray color
}


\title{Problème : Complémentarité et Décompositions en Sommes Directes}
\author{Esther Poniatowski}
\date{2025}

\customPageLayout{Sujets d'interrogation orale}{Lycée Henri IV}{2024-2025}

% ==================================================================================================
\begin{document}

\textbf{Contexte}

Dans de nombreux domaines des mathématiques et de la physique, il est souvent utile de décomposer un
espace complexe en parties plus simples et compréhensibles.


\bigskip
\textbf{Objectifs}

Étudier le concept de somme directe de sous-espaces vectoriels et son implication en termes de
bases et décompositions.

\bigskip
\textbf{Somme de sous-espaces vectoriels}

Soient $E$ un espace vectoriel et $F, G$ deux sous-espaces vectoriels de $E$.

La \textit{somme} de deux sous-espaces vectoriels est définie par :
\[
F + G = \{ u + v \mid u \in F, v \in G \}
\]

\q Montrer que $F+G$ est un sous-espace vectoriel de $E$.
% But : Introduire la notion de somme de sous-espaces et montrer qu'elle forme encore un sous-espace
% vectoriel.
% Méthode : Vérifier la stabilité par addition et multiplication scalaire.

\q Montrer que l'intersection de deux sous-espaces vectoriels est un sous-espace vectoriel, et
donner une contrainte sur sa dimension.
% But : Étudier les propriétés de l'intersection de sous-espaces en prévision de la formule de
% Grassmann.

\q Démontrer la formule de Grassmann :
\[
\dim (F+G) = \dim F + \dim G - \dim(F \cap G)
\]
% But : Établir une relation entre les dimensions des sous-espaces et leur intersection, utilisée
% plus tard pour caractériser les sommes directes en termes de dimension.
% Méthode : Utiliser la caractérisation d'une base de $F+G$ en termes des bases de $F$ et $G$.

\textbf{Sommes directes}

La somme \textit{directe} est une notion plus forte que la somme :

\[
F \oplus G = \{ u + v \mid u \in F, v \in G \} \text{ avec } F \cap G = \{0\}
\]

Deux sous-espaces sont dits \textit{supplémentaires} si leur somme est directe.

\q Indiquer la dimension de $F \oplus G$ en fonction de celles de $F$ et $G$.

\q Justifier qu'un sous-espace $F$ admet toujours un supplémentaire dans un espace de dimension finie.
% But : Étudier l'existence d'un complémentaire à un sous-espace.
% Méthode : Utiliser la dimension et la construction d'une base adaptée.

\q Donner un exemple explicite de sommes directes dans $\mathbb{R}^n$, en considérant des
sous-espaces formés par des vecteurs de la base canonique.
% But : Illustrer la somme directe à l'aide d'exemples concrets.
% Méthode : Trouver des sous-espaces dont l'intersection est réduite à 0.

\bigskip
\textbf{Bases et décompositions}

\q Démontrer que $F$ et $G$ sont supplémentaires ssi chaque élément de $E$ s'écrit de manière unique
comme la somme d'un élément de $F$ et d'un élément de $G$.
% But : Introduire les critères d'unicité et de somme directe.
% Méthode : Utiliser la définition d'unicité de la décomposition.

\bigskip
Puisque tout élément $x \in E$ possède une décomposition unique $x = u + v$ avec $u \in F$, $v \in
G$, il est possible de définir des applications $p_F$ et $p_G$ qui associent à chaque $x$ ses
composantes $u$ et $v$, appelées \textit{projections} sur $F$ et $G$.

\q Exprimer $u$ et $v$ en fonction de $x$ en fonction des coordonnées dans la base canonique de $E$.
% But : Introduire la notion de projection associée à une somme directe.
% Méthode : Exprimer les coordonnées de u dans la base de F puis dans celle de E.

\bigskip
\textbf{Application : Sommes directes dans $\mathbb{R}^4$}

Soit $E = \mathbb{R}^4$ muni de sa base canonique $(e_1, e_2, e_3, e_4)$.

On considère les sous-espaces vectoriels $F$ et $G$ définis par:
\[ F = \{ (x, y, z, t) \in \mathbb{R}^4 : x - y = 0, \; z - t = 0 \} \]
\[ G = \{ (x, y, z, t) \in \mathbb{R}^4 : x - z = 0, \; y - t = 0 \} \]

\q Vérifier que les dimensions de $F$ et $G$ sont compatibles avec une somme directe.

\q Montrer que toutefois, $F$ et $G$ ne sont pas en somme directe.

\q Proposer un sous-espace $G'$ supplémentaire de $F$.

\q Déterminer les coordonnées de tout vecteur de \( \mathbb{R}^4 \) dans cette décomposition en
somme directe.
% But : Appliquer la notion de somme directe à un cas concret.
% Méthode : Trouver une base explicite de F, vérifier que l'union avec une base de G est une base de
% R^3, puis exprimer tout vecteur sous cette forme.

\end{document}
% ==================================================================================================

\bigskip
\textbf{Application : Décomposition de fonctions}

Soit $C([1])$ l'espace vectoriel des fonctions continues sur l'intervalle $[0, 1]$, muni de la norme
uniforme $\|f\|_{\infty} = \sup_{x \in[1]} |f(x)|$. On considère les sous-espaces suivants :

\begin{itemize}
    \item $F = \{f \in C([1]) : f(0) = f(1)\}$ (fonctions continues périodiques)
    \item $G = \{g \in C([1]) : g(x) = ax + b, a,b \in \mathbb{R}\}$ (fonctions affines)
\end{itemize}

\q Montrer que $F$ et $G$ sont des sous-espaces vectoriels de $C([1])$.

\q Déterminer $F \cap G$.

\q Montrer que pour toute fonction $h \in C([1])$, il existe une unique décomposition $h = f + g$
avec $f \in F$ et $g \in G$.

\q Conclure que $C([1]) = F \oplus G$.

% But : Appliquer les concepts de somme directe à un espace fonctionnel, plus abstrait que R^3.
% Méthode :
% a) Vérifier la stabilité par combinaison linéaire pour F et G.
% b) Caractériser les fonctions appartenant à la fois à F et G.
% c) Construire explicitement la décomposition et prouver son unicité.
% d) Utiliser les résultats précédents pour conclure sur la somme directe.
