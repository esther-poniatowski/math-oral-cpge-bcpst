% PROBLÈME : Espaces Vectoriels de Matrices
% ==================================================================================================
\documentclass[10pt,a4paper]{article}

% Set the root path
\providecommand{\rootpath}{../../..}
% Fonts
\usepackage[utf8]{inputenc} % for accents
\usepackage[T1]{fontenc} % for accents
\usepackage[french]{babel} % for french language
\usepackage{helvet} % sans serif font family
\renewcommand*\familydefault{\sfdefault} % sans serif font family

% Mathematics
\usepackage{amsmath,amsfonts,amssymb} % for math symbols
\usepackage{array} % for tabular


\usepackage{parskip} % no indent, space between paragraphs

\usepackage{geometry} % margin
\geometry{
    a4paper,
    left=15mm,
    right=15mm,
    top=20mm,
    bottom=20mm
}

\usepackage{circledsteps} % to draw circles around numbers

\usepackage{fancyhdr} % for headers and footers

\usepackage{enumitem} % for customizing lists
\setlist[enumerate]{itemsep=1em} % space between items only in enumerate environment (not itemize)
\setlist[itemize]{label=--} % set itemize label to em-dash

% Command: \customPageLayout{#1}{#2}{#3}
% --------------------------------------
% Description: Custom page layout with header and footer content.
% Arguments:
% #1: Header and footer content
% #2: Left header content
% #3: Right header content
% Example:
% \customPageLayout{Title}{Lycée Henri IV}{2024}
% Required Packages: fancyhdr
\newcommand{\customPageLayout}[3]{
    \pagestyle{fancy} % set page style to fancy, i.e. header and footer
    \fancyhf{#1} % set header and footer content
    \lhead{#2} % set left header content
    \rhead{#3} % set right header content
    \fancyfoot{} % clear footer content
    \rfoot{\thepage} % set page number in footer
}

% Counter: \q
% -----------
% Description: Display a question number in a circle.
\newcounter{q}
\setcounter{q}{0} % set initial value of counter
\newcommand{\q}{
    \bigskip
    \addtocounter{q}{1}
    \par
    \Circled{\textbf{\theq}} \space
}


\title{Problème : Espaces Vectoriels de Matrices}
\author{Esther Poniatowski}
\date{2025}

\customPageLayout{Sujets d'interrogation orale}{Lycée Henri IV}{2024-2025}

% ==================================================================================================
\begin{document}

\textbf{Contexte}

Pour manipuler efficacement les matrices, tant en mathématiques et modélisation, il est commun de
travailler dans une base de l'ensemble des matrices. Selon la nature du problème étudié, certaines
bases permettent une représentation plus adaptée que d'autres, facilitant certains calculs et
interprétations.

\bigskip
\textbf{Objectifs}

Étudier la structure de l'ensemble des matrices en tant qu'espace vectoriel, et explorer les
différentes bases possibles des espaces de matrices.

\bigskip
\textbf{Structure vectorielle des espaces de matrices}

\q Soit $m, n \in \mathbb{N}^*$, et $\mathcal{M}_{m,n}(\mathbb{R})$ l'ensemble des matrices à $m$ lignes et
$n$ colonnes à coefficients réels. Montrer qu'il s'agit d'un espace vectoriel sur $\mathbb{R}$.
% But : Établir la structure d'espace vectoriel de \M_{m,n}(R).
% Méthode : Vérifier les propriétés d'un sous-espace vectoriel de R^{mxn}.

\q En s'inspirant de la base canonique de $\mathbb{R}^n$, construire une base pour
$\mathcal{M}_{m,n}(\mathbb{R})$.
% But : Exhiber une base naturelle pour M_{m,n}(R).
% Méthode : Définir les matrices élémentaires E_{ij}, prouver qu'elles forment une famille
% génératrice et libre.

\q Déterminer la dimension de $\mathcal{M}_{m,n}(\mathbb{R})$.
% But : Calculer la dimension de M_{m,n}(R).
% Méthode : Utiliser la base canonique pour conclure.

\bigskip
\textbf{Sous-espaces vectoriels de matrices}

Soit $n \in \mathbb{N}^*$. On considère les ensembles suivants dans $\mathcal{M}_n(\mathbb{R})$ :
\begin{itemize}
    \item $D_n = {A \in \mathcal{M}_n(\mathbb{R}) \mid A_{ij} = 0 \text{ pour } i \neq j}$ (matrices
    diagonales)
    \item $L_n = \{A \in \mathcal{M}_n(\mathbb{R}) \mid A_{ij} = 0 \text{ pour } i < j\}$ (matrices
    triangulaires inférieures)
    \item $T_n = \{A \in \mathcal{M}_n(\mathbb{R}) \mid A_{ij} = 0 \text{ pour } i > j\}$ (matrices
    triangulaires supérieures)
    \item $UT_n = \{A \in T_n \mid A_{ii} = 0 \text{ pour tout } i\}$ (matrices triangulaires
    supérieures strictes)
    \item $S_n = \{A \in \mathcal{M}_n(\mathbb{R}) \mid A = A^T\}$ et dont les coefficients
    vérifient $A_{ij} = A_{ji}$ (matrices symétriques)
    \item $K_n = \{A \in \mathcal{M}_n(\mathbb{R}) \mid A^T = -A\}$ et dont les coefficients
    vérifient $A_{ij} = -A_{ji}$ (matrices antisymétriques)
\end{itemize}

\q Justifier que $T_n$ et $S_n$ sont bien des sous-espaces vectoriels de $\mathcal{M}_n(\mathbb{R})$
(sans exhiber de base).
% But : Identifier des sous-espaces vectoriels importants de M_n(R).
% Méthode : Pour chaque ensemble, vérifier qu'il contient la matrice nulle et qu'il est stable par
% combinaison linéaire.

\bigskip
\textbf{Matrices symétriques et antisymétriques}

Les matrices symétriques et antisymétriques apparaissent dans de nombreux problèmes en algèbre et en
physique. En particulier, dans les systèmes dynamiques, les matrices antisymétriques sont associées
aux rotations et aux lois de conservation de l'énergie.

\q Déterminer une base de $S_n$ et une base de $K_n$ et les dimensions de ces sous-espaces.
% Méthode : Construire des familles génératrices, prouver leur liberté, et utiliser la dimension
% pour conclure.
% - Pour S_n, considérer les matrices symétriques élémentaires.
% - Pour K_n, considérer les matrices antisymétriques élémentaires.

\q Soit $A \in \mathcal{M}_n(\mathbb{R})$. Montrer que $S(A) = \frac{1}{2}(A + A^T) \in S_n$ et
$K(A) = \frac{1}{2}(A - A^T) \in K_n$.
% But : Question intermédiaire pour obtenir une décomposition de toute matrice A dans les deux
% sous-espaces considérés.

\q En déduire une base de $\mathcal{M}_n(\mathbb{R})$ différente de la base canonique.
% But : Construire une base alternative pour M_n(R).
% Méthode :
% - Former une base de M_n(R) en joignant les bases de S_n et K_n.
% - Montrer que toute matrice A peut s'écrire A = S(A) + K(A) (famille génératrice).
% - Conclure en utilisant la dimension de la famille obtenue pour montrer que c'est une base.

\q Expliciter les coordonnées de tout élément $A \in \mathcal{M}_n(\mathbb{R})$ dans cette base.

\bigskip
\textbf{Matrices triangulaires}

Les matrices triangulaires sont impliquées dans des décompositions matricielles telles que la
factorisation LU, qui simplifie la résolution des systèmes linéaires et l'inversion de matrices.

\q Déterminer $T_n \cap S_n$, $T_n \cap D_n$, $T_n \cap L_n$.

\q Justifier que $UT_n$ est un sous-espace vectoriel de $T_n$.

\q Montrer que pour toute matrice $A \in \mathcal{M}_n(\mathbb{R})$, il existe une matrice $T
\in T_n$ et une matrice $L \in L_n$ telles que $A = T + L$.

\q Soit $\mathcal{F}$ la famille obtenue en juxtaposant une base de $T_n$ et une base de $L_n$.
Cette famille forme-t-elle une base de $\mathcal{M}_n(\mathbb{R})$ ?


\end{document}
% ==================================================================================================
