\documentclass[10pt,a4paper]{article}

% Fonts
\usepackage[utf8]{inputenc} % for accents
\usepackage[T1]{fontenc} % for accents
\usepackage[french]{babel} % for french language
\usepackage{helvet} % sans serif font family
\renewcommand*\familydefault{\sfdefault} % sans serif font family

% Mathematics
\usepackage{amsmath,amsfonts,amssymb} % for math symbols
\usepackage{array} % for tabular


\usepackage{parskip} % no indent, space between paragraphs

\usepackage{geometry} % margin
\geometry{
    a4paper,
    left=15mm,
    right=15mm,
    top=20mm,
    bottom=20mm
}

\usepackage{circledsteps} % to draw circles around numbers

\usepackage{fancyhdr} % for headers and footers

\usepackage{enumitem} % for customizing lists
\setlist[enumerate]{itemsep=1em} % space between items only in enumerate environment (not itemize)
\setlist[itemize]{label=--} % set itemize label to em-dash
 % TODO: Include in global search paths


\title{Sujets d'interrogation orale - Matrices}
\author{}
\date{}

\begin{document}
\maketitle

\subsection*{Question de cours}
\begin{itemize}
    \item Énoncer les critères d'inversibilité d'une matrice $2 \times 2$. Justifier chacun des critères.
    \item Lorsque la matrice est inversible, exprimer son l'inverse.
\end{itemize}

\bigskip

\subsection*{Exercices}

\textbf{Inversibilité}
Pour les matrices suivantes, déterminer si elle sont inversibles, jsutifier, et calculer l'inverse
si possible.
\begin{enumerate}
    \item $A = \begin{pmatrix} 2 & 1 \\ 4 & 2 \end{pmatrix}$
    \item $C = \begin{pmatrix}
        6 & 1 & 4 \\
        4 & 8 & 4 \\
        6 & 3 & 5
        \end{pmatrix}$
\end{enumerate}

\bigskip

\textbf{Puissances de matrice}
\begin{enumerate}
    \item Soit $A = \begin{pmatrix} 2 & 0 \\ 0 & 3 \end{pmatrix}$. Déterminer l'expression de $A^n$ pour tout entier $n \geq 1$. Justifier.
    \item Soit $B = \begin{pmatrix} 1 & 1 \\ 0 & 1 \end{pmatrix}$.
    \begin{itemize}
        \item Calculer les premières puissances $B^2$ et $B^3$.
        \item Conjecturer la forme de $B^n$.
        \item Démontrer cette conjecture.
    \end{itemize}
    \item Soit $D = \begin{pmatrix} 0 & 1 \\ -1 & 0 \end{pmatrix}$.
    \begin{itemize}
        \item Montrer que $D^2 = -I$, où $I$ est la matrice identité.
        \item En déduire une expression pour $D^n$ pour tout entier $n \geq 1$.
    \end{itemize}
    \item Soit $A$ une matrice carrée d'ordre 3 définie par :
    $$
    A = \begin{pmatrix}
    2 & 1 & 0 \\
    0 & 2 & 1 \\
    0 & 0 & 2
    \end{pmatrix}
    $$
    \begin{itemize}
        \item Montrer que $A$ peut s'écrire sous la forme $A = 2I_3 + N$, où $I_3$ est la matrice identité d'ordre 3 et $N$ est une matrice à déterminer.
        \item Calculer $N^2$ et $N^3$.
        \item Montrer que $N^k = 0$ pour tout $k \geq 3$.
        \item En déduire $A^n$ en fonction de $n$, $I_3$, $N$ et $N^2$ pour tout entier naturel $n$.
    \end{itemize}
\end{enumerate}

\end{document}
