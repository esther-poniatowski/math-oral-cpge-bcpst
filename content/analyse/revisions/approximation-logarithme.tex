% PROBLÈME : Approximation du Logarithme et Suites associées
% ==================================================================================================
%
% Buts
% ----
% - Méthodes de résolution approchée d'équations
% - Propriétés de convergence de suites récurrentes.
% Réviser les notions de :
% - équivalents
% - développements limités
% - dérivées
% - théorèmes des accroissements finis
% - suites réelles
% - méthodes de Newton
% ==================================================================================================

\documentclass[10pt,a4paper]{article}

% Set the root path
\providecommand{\rootpath}{../../..}
% Fonts
\usepackage[utf8]{inputenc} % for accents
\usepackage[T1]{fontenc} % for accents
\usepackage[french]{babel} % for french language
\usepackage{helvet} % sans serif font family
\renewcommand*\familydefault{\sfdefault} % sans serif font family

% Mathematics
\usepackage{amsmath,amsfonts,amssymb} % for math symbols
\usepackage{array} % for tabular


\usepackage{parskip} % no indent, space between paragraphs

\usepackage{geometry} % margin
\geometry{
    a4paper,
    left=15mm,
    right=15mm,
    top=20mm,
    bottom=20mm
}

\usepackage{circledsteps} % to draw circles around numbers

\usepackage{fancyhdr} % for headers and footers

\usepackage{enumitem} % for customizing lists
\setlist[enumerate]{itemsep=1em} % space between items only in enumerate environment (not itemize)
\setlist[itemize]{label=--} % set itemize label to em-dash

% Command: \customPageLayout{#1}{#2}{#3}
% --------------------------------------
% Description: Custom page layout with header and footer content.
% Arguments:
% #1: Header and footer content
% #2: Left header content
% #3: Right header content
% Example:
% \customPageLayout{Title}{Lycée Henri IV}{2024}
% Required Packages: fancyhdr
\newcommand{\customPageLayout}[3]{
    \pagestyle{fancy} % set page style to fancy, i.e. header and footer
    \fancyhf{#1} % set header and footer content
    \lhead{#2} % set left header content
    \rhead{#3} % set right header content
    \fancyfoot{} % clear footer content
    \rfoot{\thepage} % set page number in footer
}

% Counter: \q
% -----------
% Description: Display a question number in a circle.
\newcounter{q}
\setcounter{q}{0} % set initial value of counter
\newcommand{\q}{
    \bigskip
    \addtocounter{q}{1}
    \par
    \Circled{\textbf{\theq}} \space
}


\title{Analyse - Approximation du Logarithme et Suites associées}
\author{Esther Poniatowski}
\date{2024-2025}

\customPageLayout{Sujets d'interrogation orale}{Lycée Henri IV}{2024}

% ==================================================================================================
\begin{document}

\textbf{Contexte}

Pour développer des méthodes de calcul approchées des fonctions logarithmiques, il est nécessaire
d'étudier finement leur comportement asymptotique et leurs approximations.

\bigskip
\textbf{Objectifs}

Étudier des fonctions logarithmiques sous plusieurs aspects : équivalents, approximations,
résolution approchée d'équations, suites récurrentes.

% --------------------------------------------------------------------------------------------------
\vspace{0.5cm}
\textbf{Étude asymptotique de la fonction logarithme translatée}

Soit la fonction \( f : x \mapsto \ln(1+x) \).

\q Montrer que lorsque \( x \to 0 \), on a \( \ln(1+x) \sim x \).
% But : trouver un équivalent de ln(1+x) en 0.
% Méthode : développement limité ou équivalent.

\q Donner le développement limité à l'ordre 2 de \( \ln(1+x) \) au voisinage de 0.
% But : trouver un développement limité d'ordre 2 pour affiner l'équivalent.
% Méthode : développement limité en 0.

\q Montrer que lorsque \( x \to +\infty \), on a \( \ln(1+x) \sim \ln x \).
% But : équivalent en l'infini.
% Méthode : changement de variable, comportement logarithme à l'infini.

\vspace{0.5cm}

\textbf{Comparaison avec une approximation affine}

Soit la fonction \( g : x \mapsto \ln(1+x) - \frac{x}{2} \).

\q Étudier les variations de \( g(x) = \ln(1+x) - \frac{x}{2} \) sur \( [0, +\infty[ \).
% But : étude de signe, comportement de ln(1+x) par rapport à x/2.
% Méthode : étude de fonction par la dérivée.

\q Montrer qu'il existe un unique \( c > 0 \) tel que \( \ln(1+c) = \frac{c}{2} \).
% But : existence et unicité d'un zéro de g, interprétation graphique.
% Méthode : théorème des valeurs intermédiaires.

\q Donner un encadrement de \( c \) entre deux réels faciles à calculer.
% But : encadrement de c.
% Méthode : étude de g en des points précis et signe.

% --------------------------------------------------------------------------------------------------
\vspace{0.5cm}
\textbf{Méthode de Newton pour une résolution approchée}

Soit l'équation \( \ln(1+x) = \frac{x}{2} \).

\q Donner la formule de récurrence générale pour la méthode de Newton permettant de résoudre cette
équation.
% But : poser et interpréter la méthode de Newton pour cette équation.
% Méthode : dérivation, mise en forme de la méthode.

\q En déduire la suite \( (u_n) \) définie par la méthode de Newton pour une valeur initiale \( u_0
> 0 \), et donner son expression explicite :
\[
u_{n+1} = u_n - \frac{\ln(1+u_n) - \frac{u_n}{2}}{\frac{1}{1+u_n} - \frac{1}{2}}.
\]
% But : expliciter la suite pour notre fonction.
% Méthode : calcul explicite du quotient différentiel.

\q Simplifier cette expression pour obtenir une forme plus simple, en fonction de \( u_n \)
uniquement.
% But : simplification de la formule pour la rendre exploitable.
% Méthode : calcul algébrique.

% --------------------------------------------------------------------------------------------------
\vspace{0.5cm}
\textbf{Convergence de la méthode de Newton}

Admis : La fonction \( x \mapsto \ln(1+x) - \frac{x}{2} \) est strictement croissante sur \( [0,
+\infty[ \), et sa dérivée ne s'annule pas sur cet intervalle.

\q Montrer que la suite \( (u_n) \), définie par la méthode de Newton, converge vers \( c \) pour
tout \( u_0 > 0 \).
% But : convergence de la suite vers c, point fixe.
% Méthode : étude qualitative de la méthode de Newton et propriétés de la fonction.

\q Expliquer pourquoi la méthode de Newton permet d'obtenir une approximation de \( c \) avec une
précision rapide (quadratique), et estimer, à l'aide d'un raisonnement qualitatif, le nombre
d'itérations nécessaires pour approcher \( c \) à \( 10^{-3} \) près.
% But : vitesse de convergence, approximation du nombre de décimales.
% Méthode : étude qualitative, majoration de l'erreur.

% --------------------------------------------------------------------------------------------------
\vspace{0.5cm}
\textbf{Étude asymptotique locale de la suite et lien avec l'équation}

\q Montrer que si \( u_0 > c \), alors \( (u_n) \) est décroissante et minorée par \( c \), et
converge vers \( c \).
% But : montrer que la suite décroît vers c si u_0 > c.
% Méthode : étude du comportement local, signe du résidu.

\q Montrer que si \( u_n \) est proche de \( c \), alors \( u_{n+1} - c \sim K (u_n - c)^2 \) pour
une constante \( K > 0 \), ce qui illustre la convergence quadratique de la méthode de Newton.
% But : équivalent de l'erreur.
% Méthode : développement limité et linéarisation locale.

\end{document}
% ==================================================================================================
