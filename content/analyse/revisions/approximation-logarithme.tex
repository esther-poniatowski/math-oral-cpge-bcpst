% PROBLÈME : Résolution Approchée d'Equations Logarithmiques
% ==================================================================================================
%
% Buts
% ----
% - Méthodes de résolution approchée d'équations
% - Propriétés de convergence de suites récurrentes.
% Réviser les notions de :
% - équivalents
% - développements limités
% - dérivées
% - théorèmes des accroissements finis
% - suites réelles
% - méthodes de Newton
% ==================================================================================================

\documentclass[10pt,a4paper]{article}

% Set the root path
\providecommand{\rootpath}{../../..}
% Fonts
\usepackage[utf8]{inputenc} % for accents
\usepackage[T1]{fontenc} % for accents
\usepackage[french]{babel} % for french language
\usepackage{helvet} % sans serif font family
\renewcommand*\familydefault{\sfdefault} % sans serif font family

% Mathematics
\usepackage{amsmath,amsfonts,amssymb} % for math symbols
\usepackage{array} % for tabular


\usepackage{parskip} % no indent, space between paragraphs

\usepackage{geometry} % margin
\geometry{
    a4paper,
    left=15mm,
    right=15mm,
    top=20mm,
    bottom=20mm
}

\usepackage{circledsteps} % to draw circles around numbers

\usepackage{fancyhdr} % for headers and footers

\usepackage{enumitem} % for customizing lists
\setlist[enumerate]{itemsep=1em} % space between items only in enumerate environment (not itemize)
\setlist[itemize]{label=--} % set itemize label to em-dash

% Command: \customPageLayout{#1}{#2}{#3}
% --------------------------------------
% Description: Custom page layout with header and footer content.
% Arguments:
% #1: Header and footer content
% #2: Left header content
% #3: Right header content
% Example:
% \customPageLayout{Title}{Lycée Henri IV}{2024}
% Required Packages: fancyhdr
\newcommand{\customPageLayout}[3]{
    \pagestyle{fancy} % set page style to fancy, i.e. header and footer
    \fancyhf{#1} % set header and footer content
    \lhead{#2} % set left header content
    \rhead{#3} % set right header content
    \fancyfoot{} % clear footer content
    \rfoot{\thepage} % set page number in footer
}

% Counter: \q
% -----------
% Description: Display a question number in a circle.
\newcounter{q}
\setcounter{q}{0} % set initial value of counter
\newcommand{\q}{
    \bigskip
    \addtocounter{q}{1}
    \par
    \Circled{\textbf{\theq}} \space
}


\title{Analyse - Approximation du Logarithme et Suites associées}
\author{Esther Poniatowski}
\date{2024-2025}

\customPageLayout{Sujets d'interrogation orale}{Lycée Henri IV}{2024}

% ==================================================================================================
\begin{document}

\textbf{Contexte}

Pour développer des méthodes de calcul approchées des fonctions logarithmiques, il est nécessaire
d'étudier finement leur comportement asymptotique et leurs approximations.

\bigskip
\textbf{Objectifs}

Étudier des fonctions logarithmiques sous plusieurs aspects : équivalents, approximations,
résolution approchée d'équations, suites récurrentes.

Soit la fonction \( g : x \mapsto \ln(1+x) - \frac{x}{2} \).

% --------------------------------------------------------------------------------------------------
\vspace{0.5cm}
\textbf{Étude asymptotique d'une fonction logarithmique}

\q Donner un équivalent de  \( \ln(1+x) \) lorsque \( x \to 0 \).
% But : trouver un équivalent de ln(1+x) en 0.
% Méthode : développement limité ou équivalent.

\q Montrer l'équivalent suivant lorsque \( x \to +\infty \) : \( \ln(1+x) \sim \ln x \).
% But : équivalent en l'infini.
% Méthode : changement de variable, comportement logarithme à l'infini.

\q Représenter graphiquement les fonctions \( x \mapsto \ln(x) \), \( x \mapsto \ln(1+x) \) et \( x
\mapsto \frac{x}{2} \).

\vspace{0.5cm}

\textbf{Existence des zéros de la fonction}

\q Étudier les variations de \( g \) sur \( ]-1, +\infty[ \).
% But : étude de signe, comportement de ln(1+x) par rapport à x/2.
% Méthode : étude de fonction par la dérivée.

\q Justifier que \( g \) admet un maximum sur \( ]0, +\infty[ \) et qu'il est positif.

\q En déduire qu'il existe un unique \( c > 0 \) tel que \( \ln(1+c) = \frac{c}{2} \).
% But : existence et unicité d'un zéro de g, interprétation graphique.
% Méthode : théorème des valeurs intermédiaires.

% --------------------------------------------------------------------------------------------------
\vspace{0.5cm}
\textbf{Concavité de la fonction}

\q Étudier le signe de la dérivée seconde \( g''(x) \) sur \( ]0, +\infty[ \).
% But : Déterminer la convexité de g(x) afin d'analyser la forme locale de la fonction et son
% impact sur la méthode de Newton.
% Méthode : Calculer g''(x) explicitement et étudier son signe.

\q Montrer que pour tout \( x \) et \( x_0 \), il existe un point \( \xi \) entre \( x_0 \) et \( x
\) et une constante \( \theta \in [0,1] \) tels que :
\[
g(x) = g(x_0) + g'(x_0)(x - x_0) + \theta g''(\xi) (x - x_0)^2
\]

\q Vérifier que pour tout \( x > 0 \), la fonction \( g(x) \) est toujours inférieure ou égale à sa
tangente en un point donné \( x_0 \).
% But : Guider les élèves vers l'idée que g(x) est située sous sa tangente en tout point en
% utilisant uniquement des outils élémentaires.
% Méthode : Utiliser le développement de Taylor à l'ordre 2 autour de x_0 et le signe de g''(x) pour
% comparer g(x) et son approximation affine.

% --------------------------------------------------------------------------------------------------
\vspace{0.5cm}
\textbf{Méthode de Newton pour une résolution approchée}

Soit l'équation \( \ln(1+x) = \frac{x}{2} \).

Soit \( (u_n) \) la suite des approximations de \( c \) obtenues par la méthode de Newton, pour une
valeur initiale \( u_0 > 0 \).

\q Rappeler la formule de récurrence générale pour la méthode de Newton.
% But : poser et interpréter la méthode de Newton pour cette équation.
% Méthode : dérivation, mise en forme de la méthode.

\q Appliquer cette formule au problème étudié.
% But : expliciter la suite pour la fonction étudiée.
% Méthode : calcul explicite du quotient différentiel.

\q Proposer une intuition sur l'évolution de la suite en exploitant la concavité de la fonction \( g
\).
% But : Exploiter la concavité pour établir une intuition sur l'évolution de la suite (u_n).
% Méthode : Utiliser le signe de g''(x) pour conclure et illustrer la situation par une
% interprétation géométrique de la tangente dans la méthode de Newton.

\q Identifier un intervalle sur lequel la suite \( (u_n) \) est décroissante.
% But : Montrer que (u_n) est une suite décroissante, ce qui constitue une première étape vers
% la démonstration de sa convergence.
% Méthode : Comparer les signes de g(u_n) et g'(u_n) pour déterminer le signe du quotient dans la
% méthode de Newton.

\q Montrer que l'intervalle sur lequel la suite est décroissante est stable par la méthode de
Newton, en exploitant la concavité de la fonction \( g \) montrée précédemment.
% But : Montrer que l'intervalle sur lequel la suite est décroissante est stable pour préparer
% l'argument de la minoration.
% Méthode : Utiliser la concavité de g pour montrer que l'intervalle est stable, et la décroissance
% de la fonction pour conclure.

\q En déduire que la suite est convergente.
% But : Conclure rigoureusement à la convergence de la suite en exploitant sa décroissance et sa
% minoration.
% Méthode : Prouver que la suite est minorée, puis utiliser le théorème de la convergence des suites
% monotones.

\q Démontrer que la suite \( (u_n) \) converge vers \( c \).
% But : Montrer que la limite est bien le zéro de la fonction g
% Méthode : Utiliser la convergence de la suite et le fait que g est continue pour conclure.

\end{document}
% ==================================================================================================


Point Fixe de la Méthode de Newton

\q Montrer que si un=cun=c, alors un+1=cun+1=c, où cc est le zéro de f(x)f(x).
% But : identifier le point fixe.
% Méthode : substitution dans la formule de récurrence.

Si un=cun=c, alors f(c)=0f(c)=0, et donc :
un+1=c-f(c)f'(c)=c-0f'(c)=c
un+1=c-f'(c)f(c)=c-f'(c)0=c

Analyse de la Convergence Locale

\q Étudier la convergence locale de la méthode de Newton autour du point fixe cc.
% But : analyser la stabilité locale.
% Méthode : développement limité ou linéarisation.

Pour étudier la convergence locale, on peut linéariser la formule de récurrence autour de cc en utilisant un développement limité ou en analysant la fonction g(u)=u−f(u)f′(u)g(u)=u−f′(u)f(u).

Démonstration de la Convergence Quadratique

\q Démontrer que la méthode de Newton converge quadratiquement vers cc si u0u0 est suffisamment proche de cc.
% But : prouver la convergence quadratique.
% Méthode : utilisation du théorème des accroissements finis et propriétés des dérivées.

Pour démontrer la convergence quadratique, on peut utiliser le fait que la dérivée seconde de
f(x)f(x) est continue autour de cc et que la méthode de Newton peut être exprimée sous forme de
série de Taylor. Cela implique que l'erreur à chaque étape diminue quadratiquement.
