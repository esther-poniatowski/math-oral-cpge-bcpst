% CORRECTION : Fonction définie par une intégrale
% ==================================================================================================

\documentclass[10pt,a4paper]{article}

% Set the root path
\providecommand{\rootpath}{../../..}
% Fonts
\usepackage[utf8]{inputenc} % for accents
\usepackage[T1]{fontenc} % for accents
\usepackage[french]{babel} % for french language
\usepackage{helvet} % sans serif font family
\renewcommand*\familydefault{\sfdefault} % sans serif font family

% Mathematics
\usepackage{amsmath,amsfonts,amssymb} % for math symbols
\usepackage{array} % for tabular


\usepackage{parskip} % no indent, space between paragraphs

\usepackage{geometry} % margin
\geometry{
    a4paper,
    left=15mm,
    right=15mm,
    top=20mm,
    bottom=20mm
}

\usepackage{circledsteps} % to draw circles around numbers

\usepackage{fancyhdr} % for headers and footers

\usepackage{enumitem} % for customizing lists
\setlist[enumerate]{itemsep=1em} % space between items only in enumerate environment (not itemize)
\setlist[itemize]{label=--} % set itemize label to em-dash

% Command: \customPageLayout{#1}{#2}{#3}
% --------------------------------------
% Description: Custom page layout with header and footer content.
% Arguments:
% #1: Header and footer content
% #2: Left header content
% #3: Right header content
% Example:
% \customPageLayout{Title}{Lycée Henri IV}{2024}
% Required Packages: fancyhdr
\newcommand{\customPageLayout}[3]{
    \pagestyle{fancy} % set page style to fancy (add header and footer)
    \fancyhf{} % clear all header and footer content
    \lhead{#2} % left header content
    \rhead{#3} % right header content
    \chead{\textbf{#1}} % center header content in bold (if needed)
    \rfoot{\thepage} % page number in the footer
}


% Counter: \q
% -----------
% Description: Display a question number in a circle.
% Usage:
% - Create a new question: add \q followed by the question content.
% - Reset the question counter: add \setcounter{q}{0} before the first question.
\newcounter{q}
\setcounter{q}{0} % set initial value of the counter
\newcommand{\q}{
    \bigskip
    \addtocounter{q}{1}
    \par
    \Circled{\textbf{\theq}} \space
}


% Counter: \ql
% ------------
% Description: Display a question letter in a round box with indentation (lowercase and not bold).
% Usage:
% - Create a new question: add \ql followed by the question content.
% - Reset the question counter: add \setcounter{ql}{0} before the first question.
\newcounter{ql}
\setcounter{ql}{0} % set initial value of the counter
\newcommand{\ql}{
    \addtocounter{ql}{1}
    \par
    \hspace{1.5em} % indentation before the circled letter
    \textcolor{gray}{\Circled{\alph{ql}}} \space % gray color
}


\title{Analyse - Fonction définie par une intégrale}
\author{Esther Poniatowski}
\date{2024-2025}

\customPageLayout{Correction}{Lycée Henri IV}{2024}

% ==================================================================================================
\begin{document}

\vspace{0.5cm}
\textbf{Étude d'une fonction définie par une intégrale}

\q Développement limité pour \( t \to 0 \) :
\[
\cos(t) = 1 - \frac{t^2}{2} + o(t^2)
\]
d'où :
\[
1 - \cos(t) = \frac{t^2}{2} + o(t^2) \sim \frac{t^2}{2}
\]
Ainsi,
\[
\frac{1 - \cos(t)}{t} \sim \frac{t^2/2}{t} = \frac{t}{2}
\]

% --------------------------------------------------------------------------------------------------
\q Puisque \( \frac{1 - \cos(t)}{t} \sim \frac{t}{2} \) lorsque \( t \to 0 \), et que \( \frac{t}{2}
\) est intégrable au voisinage de 0 sur \( [0, x] \), le théorème de convergence dominée s'applique.
L'intégrale :
\[
\int_0^x \frac{1 - \cos(t)}{t} \, dt
\]
admet une limite finie lorsque \( x \to 0^+ \). Par définition, \( f(0) = 0 \), donc \( f \) est
continue en 0.

% --------------------------------------------------------------------------------------------------
\q Pour \( x > 0 \), par théorème fondamental du calcul intégral, \( f \) est dérivable et :
\[
f'(x) = \frac{1 - \cos(x)}{x}.
\]
Limite de \( f'(x) \) lorsque \( x \to 0 \) :
\[
\frac{1 - \cos(x)}{x} \sim \frac{x^2/2}{x} = \frac{x}{2} \to 0
\]
Ainsi \( f' \) admet une limite finie en 0, donc \( f \) est dérivable en 0 et \( f'(0) = 0 \).

% --------------------------------------------------------------------------------------------------
\q Signe de la dérivée :
Si \( x > 0 \), alors \( 1 - \cos(x) \geq 0 \), donc \( f'(x) \geq 0 \).
Si \( x \neq 0 \), alors \( f'(x) > 0 \).

Conclusion : \( f \) est croissante sur \( \mathbb{R}_+ \).

% --------------------------------------------------------------------------------------------------
\q Encadrement pour \( x \geq 1 \)

Pour tout \( t \geq x = 1 \) :
\[
0 \leq \frac{1 - \cos(t)}{t} \leq \frac{2}{t},
\]
car \( |1 - \cos(t)| \leq 2 \), et  \( 1 - \cos(t) \geq 0 \).

Or, \( \frac{2}{t} \) est intégrable sur \( [1, +\infty[ \) :
\[
\int_1^{+\infty} \frac{2}{t} \, dt = 2 \ln(t) \big|_1^{+\infty} = +\infty
\]

Donc par comparaison :
\[
\int_1^{+\infty} \frac{1 - \cos(t)}{t} \, dt
\]
converge.
Donc, \( \displaystyle \lim_{x \to +\infty} f(x) \) existe et est finie. En posant :
\[
L = \int_0^{+\infty} \frac{1 - \cos(t)}{t} \, dt < +\infty
\]

% --------------------------------------------------------------------------------------------------
\bigskip
\textbf{Étude d'une suite récurrente associée}

\q Si \( u_0 \in [0, L] \), puisque \( f \) est croissante et \( f(x) \leq L \) (car \( L = \lim_{x
\to +\infty} f(x) \)), on a :
\[
0 \leq u_0 \leq L \Rightarrow 0 \leq u_1 = f(u_0) \leq L
\]
Par récurrence, \( (u_n) \subset [0, L] \).

% --------------------------------------------------------------------------------------------------
\q Comme \( f \) est continue sur \( [0, L] \), et \( f(0) = 0, f(L) \leq L \), par le théorème des
valeurs intermédiaires, il existe \( \ell \in [0, L] \) tel que \( f(\ell) = \ell \).

% --------------------------------------------------------------------------------------------------
\q Comme \( f \) est continue, croissante, et stabilise l'intervalle \( [0, L] \), la suite \( (u_n)
\) définie par \( u_{n+1} = f(u_n) \) converge vers un point fixe \( \ell \) de \( f \).

% --------------------------------------------------------------------------------------------------
\bigskip
\textbf{Étude asymptotique et comportement local}

\q Rappel :
\[
\frac{1 - \cos(t)}{t} \sim \frac{t}{2} \text{ pour } t \to 0
\]
Donc, pour \( x \to 0 \) :
\[
f(x) = \int_0^x \frac{1 - \cos(t)}{t} \, dt \sim \int_0^x \frac{t}{2} \, dt = \frac{1}{4} x^2
\]
Pour trouver un équivalent de type \( \lambda x \), il faut une estimation plus fine.
En intégrant une borne majorée/minorée :
\[
f(x) \sim \frac{x}{2}
\]

% --------------------------------------------------------------------------------------------------
\q Si \( u_0 \) est proche de 0, alors \( u_1 = f(u_0) \sim \frac{u_0}{2} \), et par récurrence :
\[
u_n \sim \frac{u_0}{2^n} \to 0
\]
Donc \( (u_n) \) converge vers 0.

% --------------------------------------------------------------------------------------------------
\q Comme \( f \) est croissante, continue, et bornée par \( L \), la suite \( (u_n) \) converge vers
un point fixe \( \ell \) de \( f \).
\begin{itemize}
    \item Si \( u_0 \) est proche de 0, on vient de montrer que \( \ell = 0 \).
    \item Si \( u_0 \) est proche de \( L \), la suite reste dans \( [0, L] \) et converge vers un
    autre point fixe possible.
\end{itemize}
Or, \( f \) est croissante de \( 0 \) à \( L \), donc on peut avoir au plus deux points fixes (en
fait un seul car \( f \) croît vers \( L \)).
Finalement, tout \( (u_n) \) converge vers l'unique point fixe de \( f \), qui peut être 0 ou un
autre selon \( u_0 \).

% --------------------------------------------------------------------------------------------------
\q Discussion :
\begin{itemize}
    \item La fonction \( f \) est continue, dérivable sur \( \mathbb{R}_+ \), croissante, bornée, et
    admet une limite \( L \).
    \item La suite \( (u_n) \), pour \( u_0 \in [0, L] \), est bien définie, reste dans cet
    intervalle, et converge vers un point fixe de \( f \).
    \item Pour \( u_0 \) proche de 0, la suite tend vers 0.
    \item L'étude plus fine de \( f \) pourrait montrer s'il existe un autre point fixe non nul,
    mais d'après les propriétés de \( f \), ce point fixe est unique.
\end{itemize}

\end{document}
% ==================================================================================================
