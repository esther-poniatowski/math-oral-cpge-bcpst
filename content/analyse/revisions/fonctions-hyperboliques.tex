% PROBLÈME : Fonctions Hyperboliques
% ==================================================================================================

% ==================================================================================================

\documentclass[10pt,a4paper]{article}

% Set the root path
\providecommand{\rootpath}{../../..}
% Fonts
\usepackage[utf8]{inputenc} % for accents
\usepackage[T1]{fontenc} % for accents
\usepackage[french]{babel} % for french language
\usepackage{helvet} % sans serif font family
\renewcommand*\familydefault{\sfdefault} % sans serif font family

% Mathematics
\usepackage{amsmath,amsfonts,amssymb} % for math symbols
\usepackage{array} % for tabular


\usepackage{parskip} % no indent, space between paragraphs

\usepackage{geometry} % margin
\geometry{
    a4paper,
    left=15mm,
    right=15mm,
    top=20mm,
    bottom=20mm
}

\usepackage{circledsteps} % to draw circles around numbers

\usepackage{fancyhdr} % for headers and footers

\usepackage{enumitem} % for customizing lists
\setlist[enumerate]{itemsep=1em} % space between items only in enumerate environment (not itemize)
\setlist[itemize]{label=--} % set itemize label to em-dash

% Command: \customPageLayout{#1}{#2}{#3}
% --------------------------------------
% Description: Custom page layout with header and footer content.
% Arguments:
% #1: Header and footer content
% #2: Left header content
% #3: Right header content
% Example:
% \customPageLayout{Title}{Lycée Henri IV}{2024}
% Required Packages: fancyhdr
\newcommand{\customPageLayout}[3]{
    \pagestyle{fancy} % set page style to fancy (add header and footer)
    \fancyhf{} % clear all header and footer content
    \lhead{#2} % left header content
    \rhead{#3} % right header content
    \chead{\textbf{#1}} % center header content in bold (if needed)
    \rfoot{\thepage} % page number in the footer
}


% Counter: \q
% -----------
% Description: Display a question number in a circle.
% Usage:
% - Create a new question: add \q followed by the question content.
% - Reset the question counter: add \setcounter{q}{0} before the first question.
\newcounter{q}
\setcounter{q}{0} % set initial value of the counter
\newcommand{\q}{
    \bigskip
    \addtocounter{q}{1}
    \par
    \Circled{\textbf{\theq}} \space
}


% Counter: \ql
% ------------
% Description: Display a question letter in a round box with indentation (lowercase and not bold).
% Usage:
% - Create a new question: add \ql followed by the question content.
% - Reset the question counter: add \setcounter{ql}{0} before the first question.
\newcounter{ql}
\setcounter{ql}{0} % set initial value of the counter
\newcommand{\ql}{
    \addtocounter{ql}{1}
    \par
    \hspace{1.5em} % indentation before the circled letter
    \textcolor{gray}{\Circled{\alph{ql}}} \space % gray color
}


\title{Analyse - Fonctions Hyperboliques}
\author{Esther Poniatowski}
\date{2024-2025}

\customPageLayout{Sujets d'interrogation orale}{Lycée Henri IV}{2024}

% ==================================================================================================
\begin{document}

\textbf{Contexte}

Trois fonctions hyperboliques sont définies par des combinaisons linéaires de fonctions
exponentielles. Elles sont analogues aux fonctions trigonométriques circulaires, et interviennent
dans des problèmes de physique, notamment en relativité.

\bigskip
\textbf{Objectifs}

Mener une étude analytique des fonctions hyperboliques et de leurs réciproques.

\bigskip
\textbf{Définitions}

Sinus hyperbolique :
\[
\sinh : x \in \mathbb{R} \mapsto \frac{e^x - e^{-x}}{2}
\]
Cosinus hyperbolique :
\[
\cosh : x \in \mathbb{R} \mapsto \frac{e^x + e^{-x}}{2}
\]
Tangente hyperbolique :
\[
\tanh : x \in \mathbb{R} \mapsto \frac{\sinh(x)}{\cosh(x)} = \frac{e^x - e^{-x}}{e^x + e^{-x}}
\]

% --------------------------------------------------------------------------------------------------
\bigskip
\textbf{Étude qualitative des fonctions hyperboliques}

\q Représenter graphiquement l'allure des fonctions $\sinh$, $\cosh$ et $\tanh$, en se fondant sur
les valeurs en certains points et les limites en $+\infty$ et $-\infty$.

\q Calculer les dérivées des fonctions $\sinh$, $\cosh$ et $\tanh$.

\q Déterminer des intervalles maximums sur lesquels les fonctions $\sinh$, $\cosh$ et $\tanh$
établissent des bijections.

% --------------------------------------------------------------------------------------------------
\bigskip
\textbf{Résultats généraux : Dérivées et Inverses}

L'objectif de cette partie est de d'exprimer la dérivée de la réciproque $(f^{-1})'$ en fonction de
la dérivée $f'$ et de la réciproque $f^{-1}$ (en supposant que \( f \) est une fonction dérivable et
bijective sur son domaine).

\q Soit \( f : I \to J \) une fonction strictement monotone et dérivable sur un intervalle \( I \),
admettant une fonction réciproque \( f^{-1} : J \to I \). Justifier que, pour tout \( x \in J \),
\[
f(f^{-1}(x)) = x
\]
% But : Amener les élèves à exprimer la propriété fondamentale des fonctions réciproques.
% Méthode : Utilisation directe de la définition de la réciproque et application de \( f \) à \(
% f^{-1}(x) \).

\q En supposant que \( f^{-1} \) est dérivable, dériver l'égalité \( f(f^{-1}(x)) = x \) par rapport
à \( x \).
% But : Introduire implicitement la dérivée de la fonction réciproque.
% Méthode : Utilisation de la dérivation en chaîne en posant \( u = f^{-1}(x) \).

\q En déduire l'expression de \( (f^{-1})'(x) \) et préciser sur quel ensemble elle est valable.
% But : Finaliser la démonstration en précisant les hypothèses nécessaires.
% Méthode : Synthèse des conditions établies dans les questions précédentes.

% --------------------------------------------------------------------------------------------------
\bigskip
\textbf{Réciproques des fonctions hyperboliques}

\q Représenter graphiquement les fonctions $\sinh^{-1}$, $\cosh^{-1}$ et $\tanh^{-1}$.

\q Etablir l'identité $\cosh^2(z) - \sinh^2(z) = 1,\; \forall z \in \mathbb{R}$.

\q Justifier que les fonctions $\sinh^{-1}$, $\cosh^{-1}$ et $\tanh^{-1}$ sont dérivables sur des
intervalles à préciser, et calculer leurs dérivées (sans exprimer $\sinh^{-1}$, $\cosh^{-1}$ et
$\tanh^{-1}$, et en utilisant l'identité précédente).

\q Justifier que $\sinh^{-1}$ admet un développement limité à l'ordre $4$ en $0$, et le déterminer.

\q Trouver une expression de $\sinh^{-1}$, $\cosh^{-1}$ et $\tanh^{-1}$ et retrouver leurs
dérivées.\\
\emph{\small Indication : Pour $\sinh^{-1}$ et $\cosh^{-1}$, résoudre un polynôme du second degré en
posant $X = e^x$.}

\end{document}
% ==================================================================================================
