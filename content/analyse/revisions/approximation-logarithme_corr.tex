% CORRECTION : Approximation du Logarithme et Suites associées
% ==================================================================================================

\documentclass[10pt,a4paper]{article}

% Set the root path
\providecommand{\rootpath}{../../..}
% Fonts
\usepackage[utf8]{inputenc} % for accents
\usepackage[T1]{fontenc} % for accents
\usepackage[french]{babel} % for french language
\usepackage{helvet} % sans serif font family
\renewcommand*\familydefault{\sfdefault} % sans serif font family

% Mathematics
\usepackage{amsmath,amsfonts,amssymb} % for math symbols
\usepackage{array} % for tabular


\usepackage{parskip} % no indent, space between paragraphs

\usepackage{geometry} % margin
\geometry{
    a4paper,
    left=15mm,
    right=15mm,
    top=20mm,
    bottom=20mm
}

\usepackage{circledsteps} % to draw circles around numbers

\usepackage{fancyhdr} % for headers and footers

\usepackage{enumitem} % for customizing lists
\setlist[enumerate]{itemsep=1em} % space between items only in enumerate environment (not itemize)
\setlist[itemize]{label=--} % set itemize label to em-dash

% Command: \customPageLayout{#1}{#2}{#3}
% --------------------------------------
% Description: Custom page layout with header and footer content.
% Arguments:
% #1: Header and footer content
% #2: Left header content
% #3: Right header content
% Example:
% \customPageLayout{Title}{Lycée Henri IV}{2024}
% Required Packages: fancyhdr
\newcommand{\customPageLayout}[3]{
    \pagestyle{fancy} % set page style to fancy (add header and footer)
    \fancyhf{} % clear all header and footer content
    \lhead{#2} % left header content
    \rhead{#3} % right header content
    \chead{\textbf{#1}} % center header content in bold (if needed)
    \rfoot{\thepage} % page number in the footer
}


% Counter: \q
% -----------
% Description: Display a question number in a circle.
% Usage:
% - Create a new question: add \q followed by the question content.
% - Reset the question counter: add \setcounter{q}{0} before the first question.
\newcounter{q}
\setcounter{q}{0} % set initial value of the counter
\newcommand{\q}{
    \bigskip
    \addtocounter{q}{1}
    \par
    \Circled{\textbf{\theq}} \space
}


% Counter: \ql
% ------------
% Description: Display a question letter in a round box with indentation (lowercase and not bold).
% Usage:
% - Create a new question: add \ql followed by the question content.
% - Reset the question counter: add \setcounter{ql}{0} before the first question.
\newcounter{ql}
\setcounter{ql}{0} % set initial value of the counter
\newcommand{\ql}{
    \addtocounter{ql}{1}
    \par
    \hspace{1.5em} % indentation before the circled letter
    \textcolor{gray}{\Circled{\alph{ql}}} \space % gray color
}


\title{Correction - Approximation du Logarithme et Suites associées}
\author{Esther Poniatowski}
\date{2024-2025}

\customPageLayout{Correction}{Lyc\'ee Henri IV}{2024}

% ==================================================================================================
\begin{document}

\textbf{Étude asymptotique de la fonction logarithme translatée}

\q Démonstration de l'équivalent de \( \ln(1+x) \) en 0.

Le développement limité de \( \ln(1+x) \) en 0 donne :
\[
\ln(1+x) = x - \frac{x^2}{2} + o(x^2)
\]
Ainsi, \( \ln(1+x) \sim x \) lorsque \( x \to 0 \).

% ------

\q Développement limité à l'ordre 2 de \( \ln(1+x) \) en 0.

Le développement limité à l'ordre 2 est :
\[
\ln(1+x) = x - \frac{x^2}{2} + o(x^2)
\]

% ------

\q Démonstration de \( \ln(1+x) \sim \ln x \) en \( +\infty \) :

Par changement de variable \( x = e^y -1 \) :
\[
\ln(1+x) = \ln(e^y) = y = \ln x + o(1).
\]
Ainsi, \( \ln(1+x) \sim \ln x \) lorsque \( x \to +\infty \)

% --------------------------------------------------------------------------------------------------
\bigskip
\textbf{Comparaison avec une approximation affine}

\q Étude des variations de \( g(x) = \ln(1+x) - \frac{x}{2} \) :

La dérivée de \( g \) est :
\[
 g'(x) = \frac{1}{1+x} - \frac{1}{2}
\]
Elle s'annule pour \( x = c \), où \( c \) est la solution unique de \( \ln(1+c) = \frac{c}{2} \).

% ------

\q Existence et unicité de \( c \) :

La fonction \( g(x) \) est strictement croissante, \( g(0) < 0 \) et \( \lim_{x \to +\infty} g(x) =
+\infty \). Par le théorème des valeurs intermédiaires, il existe un unique \( c > 0 \) tel que \(
g(c) = 0 \).

% ------

\q Encadrement de \( c \) :

En évaluant \( g(x) \) pour quelques valeurs :
\[ g(0.5) \approx -0.097, \quad g(1) \approx 0.193. \]

Ainsi : \( 0.5 < c < 1 \).

\bigskip
\textbf{Méthode de Newton pour une résolution approchée}

\q Formule de récurrence de Newton :

La méthode de Newton appliquée à \( h(x) = \ln(1+x) - \frac{x}{2} \) donne :
\[
 u_{n+1} = u_n - \frac{h(u_n)}{h'(u_n)}
\]

% ------

\q Explicitation de la suite :

\[
 u_{n+1} = u_n - \frac{\ln(1+u_n) - \frac{u_n}{2}}{\frac{1}{1+u_n} - \frac{1}{2}}
\]

% ------

\q Simplification de l'expression :

En réarrangeant :
\[
 u_{n+1} = u_n - \frac{(2\ln(1+u_n) - u_n)(1+u_n)}{2 - (1+u_n)}
\]

% --------------------------------------------------------------------------------------------------
\bigskip
\textbf{Convergence de la méthode de Newton}

\q Convergence de la suite \( (u_n) \).

La fonction \( h(x) \) est strictement croissante et dérivable sur \( [0, +\infty[ \). La méthode de
Newton converge donc vers \( c \) pour tout \( u_0 > 0 \).

% ------

\q Estimation du nombre d'itérations :

La méthode de Newton a une convergence quadratique : l'erreur est de l'ordre de \( O((u_n - c)^2)
\). On peut approximer le nombre d'itérations nécessaires par :
\[
 n \approx \log_2 \log_{10}(\frac{1}{10^{-3}}) = \log_2 3 \approx 1.6.
\]
Donc, en pratique, 2 à 3 itérations suffisent.

\bigskip
\textbf{Étude asymptotique locale de la suite}

\q Monotonie et convergence :

Si \( u_0 > c \), alors \( (u_n) \) est décroissante et minorée par \( c \), donc convergente vers
\( c \).

% ------

\q Convergence quadratique :

Un développement limité autour de \( c \) donne :
\[
 u_{n+1} - c \approx K (u_n - c)^2.
\]
Cela confirme la convergence quadratique.

\end{document}
% ==================================================================================================
