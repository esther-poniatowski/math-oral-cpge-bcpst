% CORRECTION : Résolution Approchée d'Equations Logarithmiques
% ==================================================================================================

\documentclass[10pt,a4paper]{article}

\usepackage{tikz}
\usepackage{pgfplots}

% Set the root path
\providecommand{\rootpath}{../../..}
% Fonts
\usepackage[utf8]{inputenc} % for accents
\usepackage[T1]{fontenc} % for accents
\usepackage[french]{babel} % for french language
\usepackage{helvet} % sans serif font family
\renewcommand*\familydefault{\sfdefault} % sans serif font family

% Mathematics
\usepackage{amsmath,amsfonts,amssymb} % for math symbols
\usepackage{array} % for tabular


\usepackage{parskip} % no indent, space between paragraphs

\usepackage{geometry} % margin
\geometry{
    a4paper,
    left=15mm,
    right=15mm,
    top=20mm,
    bottom=20mm
}

\usepackage{circledsteps} % to draw circles around numbers

\usepackage{fancyhdr} % for headers and footers

\usepackage{enumitem} % for customizing lists
\setlist[enumerate]{itemsep=1em} % space between items only in enumerate environment (not itemize)
\setlist[itemize]{label=--} % set itemize label to em-dash

% Command: \customPageLayout{#1}{#2}{#3}
% --------------------------------------
% Description: Custom page layout with header and footer content.
% Arguments:
% #1: Header and footer content
% #2: Left header content
% #3: Right header content
% Example:
% \customPageLayout{Title}{Lycée Henri IV}{2024}
% Required Packages: fancyhdr
\newcommand{\customPageLayout}[3]{
    \pagestyle{fancy} % set page style to fancy, i.e. header and footer
    \fancyhf{#1} % set header and footer content
    \lhead{#2} % set left header content
    \rhead{#3} % set right header content
    \fancyfoot{} % clear footer content
    \rfoot{\thepage} % set page number in footer
}

% Counter: \q
% -----------
% Description: Display a question number in a circle.
\newcounter{q}
\setcounter{q}{0} % set initial value of counter
\newcommand{\q}{
    \bigskip
    \addtocounter{q}{1}
    \par
    \Circled{\textbf{\theq}} \space
}


\title{Correction - Approximation du Logarithme et Suites associées}
\author{Esther Poniatowski}
\date{2024-2025}

\customPageLayout{Correction}{Lyc\'ee Henri IV}{2024}

% ==================================================================================================
\begin{document}

\textbf{Étude asymptotique d'une fonction logarithmique}

\q Equivalent de \( \ln(1+x) \) en 0 :

Le développement limité de \( \ln(1+x) \) en 0 donne :
\[
\ln(1+x) = x - \frac{x^2}{2} + o(x^2)
\]
Ainsi, \( \ln(1+x) \sim x \) lorsque \( x \to 0 \).

% ------

\q Démonstration de \( \ln(1+x) \sim \ln x \) en \( +\infty \) :

Par changement de variable \( x = e^y -1 \) :
\[
\ln(1+x) = \ln(e^y) = y = \ln x + o(1).
\]
Ainsi, \( \ln(1+x) \sim \ln x \) lorsque \( x \to +\infty \)

% ------

\q Représentations graphiques :

\begin{center}
\begin{tikzpicture}
\begin{axis}[
    axis lines = middle,
    xlabel = \( x \),
    ylabel = \( y \),
    xmin = -1,
    xmax = 5,
    ymin = -2,
    ymax = 2,
    xtick = {0, 1, 2, 3, 4},
    ytick = {-1, 0, 1},
    legend pos = south east,
]
\addplot [domain=0:5, samples=100, color=blue,]{ln(x)};
\addlegendentry{\( \ln(x) \)}
\addplot [domain=-1:5, samples=100, color=red,]{ln(1+x)};
\addlegendentry{\( \ln(1+x) \)}
\addplot [domain=-1:5, samples=100, color=orange,]{x/2};
\addlegendentry{\( \frac{x}{2} \)}
\end{axis}
\end{tikzpicture}
\end{center}

\begin{center}
    \begin{tikzpicture}
    \begin{axis}[
        axis lines = middle,
        xlabel = \( x \),
        ylabel = \( y \),
        xmin = -1,
        xmax = 5,
        ymin = -2,
        ymax = 2,
        xtick = {0, 1, 2, 3, 4},
        ytick = {-1, 0, 1},
        legend pos = south east,
    ]
    \addplot [domain=-1:5, samples=100, color=black,]{ln(1+x)-x/2};
    \addlegendentry{\( \ln(1+x) - \frac{x}{2} \)}
    \end{axis}
    \end{tikzpicture}
    \end{center}

% --------------------------------------------------------------------------------------------------
\bigskip
\textbf{Existence des zéros de la fonction}

\q Étude des variations de \( g \) :

Dérivée de \( g \) sur \( ]-1, +\infty[ \) :
\[
 g'(x) = \frac{1}{1+x} - \frac{1}{2} = \frac{2 - (1+x)}{2(1+x)} = \frac{1 - x}{2(1+x)}
\]

Signe de la dérivée :

Le dénominateur \( 2(1+x) \) est strictement positif sur \( ]-1, +\infty[ \).
Le numérateur \( 1-x \) s'annule pour \( x = 1 \) et change de signe en ce point :
\begin{itemize}
    \item \( 1-x > 0 \) si \( x < 1 \) (positif)
    \item \( 1-x = 0 \) si \( x = 1 \) (nul)
    \item \( 1-x < 0 \) si \( x > 1 \) (négatif)
\end{itemize}

Valeurs particulières de la fonction \( g \) :
\begin{itemize}
    \item \( \lim\limits_{x \to -1^+} g(x) = \ln(0) - \frac{-1}{2} = -\infty \).
    \item \( \lim\limits_{x \to +\infty} g(x) = \ln(1+\infty) - \frac{\infty}{2} = -\infty \).
    \item \( g(1) = \ln(2) - \frac{1}{2} \)
\end{itemize}

\[
\begin{array}{c|ccccc}
x & -1 &  & 1 &  & +\infty \\
\hline
g'(x) &  & + & 0 & - &  \\
\hline
g(x) & -\infty & \nearrow & \ln(2) - \frac{1}{2} & \searrow & -\infty
\end{array}
\]

% ------

\q Maximum sur \( ]0, +\infty[ \) :

La fonction \( g \) admet un maximum en \( 1 \), qui vaut \( \ln(2) - \frac{1}{2} \).

Or, \( g(0) = 0 \) et \( g \) est croissante sur \( ]-1, 1] \), donc \( g(1) > g(0) = 0 \).

% ------

\q Existence et unicité de \( c > 0 \) :

Sur  \( ]-1, 1[ \), la fonction \( g \) est strictement positive.

Sur  \( ]1, +\infty[ \), la fonction \( g \) est strictement décroissante, \( g(1) > 0 \) et \(
\lim_{x \to +\infty} g(x) = -\infty \). Par le théorème des valeurs intermédiaires, il existe un
unique \( c > 0 \) tel que \( g(c) = 0 \).

% --------------------------------------------------------------------------------------------------
\bigskip
\textbf{Concavité de la fonction}

\q Signe de la dérivée seconde \( g''(x) \) sur \( ]0, +\infty[ \):

En dérivant à nouveau la dérivée :
\[
g''(x) = -\frac{1}{(1+x)^2}
\]
Le terme \( -\frac{1}{(1+x)^2} \) est strictement négatif sur \( ]0, +\infty[ \), ce qui prouve que
\( g(x) \) est strictement concave sur cet intervalle.

% ------
\q Développement limité à l'odre 2 et théorème des accroissements finis :

Application du théorème des accroissements finis à \( g \) sur \( [x_0, x] \) (avec \( x > x_0 \),
le raisonnement est identique pour \( x < x_0 \)) :
\[
\exists \eta \in ]x_0, x[ \quad \text{tel que} \quad g(x) - g(x_0) = g'(\eta) (x - x_0)
\]
Cette relation exprime que la variation de \( g \) entre \( x_0 \) et \( x \) est donnée par la
valeur de sa dérivée en un point intermédiaire \( \eta \).

Application du théorème des accroissements finis à \( g'(x) \) sur \( [x_0, x] \) (où elle est
dérivable) :
\[
\exists \xi \in ]x_0, x[ \quad \text{tel que} \quad g'(\eta) - g'(x_0) = g''(\xi) (x - x_0)
\]
Cette relation exprime que la variation de \( g' \) est contrôlée par sa dérivée seconde \( g'' \)
évaluée en un point intermédiaire \( \xi \).

Substitution dans la première équation :
\[
g(x) - g(x_0) = \big( g'(x_0) + g''(\xi)(\eta - x_0) \big) (x - x_0)
\]
Or, \( \eta \) étant un point intermédiaire entre \( x_0 \) et \( x \), il peut s'écrire sous la
forme :
\[
\eta = x_0 + \theta (x - x_0) \quad \theta \in ]0,1[
\]
En substituant cette expression :
\[
g(x) - g(x_0) = g'(x_0)(x - x_0) + g''(\xi) \theta (x - x_0)^2
\]

\q Concavité :

La tangente à \( g(x) \) en \( x_0 \) est donnée par :
\[
T_{x_0}(x) = g(x_0) + g'(x_0)(x - x_0)
\]

L'objectif est de montrer que :
\[
g(x) \leq T_{x_0}(x) \quad \text{pour tout } x > 0.
\]

D'après la question précédente, \( g''(x) \) est strictement négatif sur \( ]0, +\infty[ \), donc \(
g''(\xi) < 0 \). De plus, le facteur \( \theta \) est compris entre 0 et 1. Par conséquent :
\[
\theta g''(\xi) (x - x_0)^2 \leq 0
\]

Il en résulte que :
\[
g(x) \leq g(x_0) + g'(x_0)(x - x_0) = T_{x_0}(x)
\]
Ainsi, la fonction \( g(x) \) est toujours située sous sa tangente en tout point \( x_0 \).

% --------------------------------------------------------------------------------------------------
\bigskip
\textbf{Méthode de Newton pour une résolution approchée}

\q Formule de récurrence de Newton :

La méthode de Newton consiste à approcher une solution d'une équation de la forme \( f(x) = 0 \) itérativement.

Pour chaque pas d'itération, la nouvelle estimation $x_{n+1}$ est l'abscisse où la tangente au point
actuel \( x_n \) coupe l'axe des abscisses.
\begin{itemize}
    \item Equation de la tangente : \( y = f'(x_n)(x - x_n) + f(x_n) \).
    \item Intersection avec l'axe des abscisses : \( f'(x_n)(x_{n+1} - x_n) + f(x_n) = 0 \)
\end{itemize}
Ainsi:
\[
x_{n+1} = x_n - \frac{f(x_n)}{f'(x_n)}
\]

% ------

\q Application au problème étudié :
\begin{itemize}
    \item Fonction d'intérêt : \( g(x) = \ln(1+x) - \frac{x}{2} \).
    \item Dérivée : \( g'(x) = \frac{1}{1+x} - \frac{1}{2} \).
\end{itemize}
Ainsi :
\[
 u_{n+1} = u_n - \frac{h(u_n)}{h'(u_n)} = u_n - \frac{\ln(1+u_n) - \frac{u_n}{2}}{\frac{1}{1+u_n} - \frac{1}{2}}
\]
Simplification de l'expression :
\[
 u_{n+1} = u_n - \frac{(2\ln(1+u_n) - u_n)(1+u_n)}{2 - (1+u_n)}
\]


% --------------------------------------------------------------------------------------------------
\bigskip
\textbf{Convergence de la méthode de Newton}

\q Décroissance de la suite :

La suite est décroissante dès lors que \( g(u_n) \) et \( g'(u_n) \) ont le même signe.

Signe de \( g(u_n) = \ln(1+u_n) - \frac{u_n}{2} \) (selon le tableau de variations établi
précédemment) :
\begin{itemize}
    \item \( g(x) > 0 \) pour \( 0 < x < c \)
    \item \( g(c) = 0 \)
    \item \( g(x) < 0 \) pour \( x > c \)
\end{itemize}

Signe de \( g'(u_n) = \frac{1}{1+u_n} - \frac{1}{2} \) (selon l'étude de la dérivée de \( g \)
établie précédemment) :
\begin{itemize}
    \item \( g'(x) > 0 \) pour \( 0 < x < 1 \)
    \item \( g'(x) < 0 \) pour \( x > 1 \)
\end{itemize}

Tableau de signes pour le quotient :
\[
\begin{array}{c|ccccccc}
x & 0 &  & 1 &  & c & & +\infty \\
\hline
g(x) & & + &  & + & 0 & - &  \\
\hline
g'(x) &  & + & 0 & - &  & - &  \\
\hline
\frac{g(x)}{g'(x)} & & + & N/A & - & 0 & + &  \\
\end{array}
\]

Ainsi, un intervalle dans lequel la suite \( (u_n) \) est décroissante est \( [c , +\infty[ \).

Dans ce cas :
\[
\frac{g(u_n)}{g'(u_n)} > 0
\]
\[
u_{n+1} = u_n - \frac{g(u_n)}{g'(u_n)} < u_n
\]
ce qui prouve que \( (u_n) \) est bien décroissante.

\q Stabilité de l'intervalle d'étude :

L'objectif est de prouver que si \( u_n \geq c \), alors \( u_{n+1} \geq c \).

Or, \( u_{n+1} \) est défini par le point d'intersection de la tangente en  \( u_n \) avec l'axe des
abscisses. Puisque la fonction \( g \) est concave, elle est toujours située sous sa tangente :
\[
g(u_{n+1}) \leq 0
\]
Or, la fonction \( g \) s'annule en \( c \) et est strictement décroissante sur \( [1, +\infty[ \),
ce qui implique :
\[
g(u_{n+1}) \leq 0 \implies u_{n+1} \geq c
\]

Conclusion : L'intervalle \( [c, +\infty[ \) est stable pour la méthode de Newton.

% -----

\q Convergence de la suite :

D'une part, la suite \( (u_n) \) est minorée par \( c \) car l'intervalle d'étude est stable par la
méthode de Newton.

D'autre part, la suite est décroissante sur cet intervalle.

Par le théorème des suites monotones, la suite \( (u_n) \) est convergente.

% ------

\q Limite de la suite :

D'après la question précédente, la suite \( (u_n) \) converge vers une limite \( \ell \geq c \).

En passant à la limite dans la relation de récurrence :
\[
\lim_{n \to \infty} u_{n+1} = \lim_{n \to \infty} \left( u_n - \frac{g(u_n)}{g'(u_n)} \right)
\]

Soit \( \ell = \lim_{n \to \infty} u_n \). Comme \( g \) est continue, il est légitime de passer à
la limite dans la relation de récurrence :
\[
\ell = \ell - \frac{g(\ell)}{g'(\ell)}
\]
D'où :
\[
\frac{g(\ell)}{g'(\ell)} = 0
\]

Or, \( g(x) = 0 \) uniquement pour \( x = c \), donc nécessairement \( \ell = c \), ce qui prouve
que la suite \( (u_n) \) converge vers \( c \).


\end{document}
% ==================================================================================================

% ------

\q Vitesse de convergence et interprétation asymptotique

Le but est de trouver une expression itérative de l'erreur \( e_n = u_n - c \) pour évaluer la
vitesse de convergence de la méthode de Newton.

Relation de récurrence pour l'erreur :
\[
e_{n+1} = u_{n+1} - c = u_n - c - \frac{g(u_n)}{g'(u_n)} = e_n - \frac{g(u_n)}{g'(u_n)}
\]

Le numérateur peut être estimé en fonction de l'erreur \( e_n \) par un développement limité de \(
g(u_n) \) au voisinage de \( c \) :
\[
g(u_n) = g(c) + g'(c) (u_n - c) + \frac{1}{2} g''(\xi_n) (u_n - c)^2
\]
Comme \( g(c) = 0 \) :
\[
g(u_n) = g'(c) e_n + \frac{1}{2} g''(\xi_n) e_n^2
\]

En remplaçant \( g(u_n) \) par son développement :
\[
e_{n+1} = e_n - \frac{g'(c) e_n + \frac{1}{2} g''(\xi_n) e_n^2}{g'(u_n)}
\]

De même, le dénominateur peut être approximé par un développement limité de \( g'(u_n) \) au voisinage
de \( c \) :
\[
g'(u_n) = g'(c) + g''(\xi_n) (u_n - c)
\]
En remplaçant le dénominateur :
\[
e_{n+1} = e_n - \frac{g'(c) e_n + \frac{1}{2} g''(\xi_n) e_n^2}{g'(c) + g''(\xi_n) e_n}
\]

TODO: Comment conclure ?
