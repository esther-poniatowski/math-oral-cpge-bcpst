% PROBLÈME : Fonctions Hyperboliques
% ==================================================================================================

% ==================================================================================================

\documentclass[10pt,a4paper]{article}

\usepackage{tikz}
\usepackage{pgfplots}

% Set the root path
\providecommand{\rootpath}{../../..}
% Fonts
\usepackage[utf8]{inputenc} % for accents
\usepackage[T1]{fontenc} % for accents
\usepackage[french]{babel} % for french language
\usepackage{helvet} % sans serif font family
\renewcommand*\familydefault{\sfdefault} % sans serif font family

% Mathematics
\usepackage{amsmath,amsfonts,amssymb} % for math symbols
\usepackage{array} % for tabular


\usepackage{parskip} % no indent, space between paragraphs

\usepackage{geometry} % margin
\geometry{
    a4paper,
    left=15mm,
    right=15mm,
    top=20mm,
    bottom=20mm
}

\usepackage{circledsteps} % to draw circles around numbers

\usepackage{fancyhdr} % for headers and footers

\usepackage{enumitem} % for customizing lists
\setlist[enumerate]{itemsep=1em} % space between items only in enumerate environment (not itemize)
\setlist[itemize]{label=--} % set itemize label to em-dash

% Command: \customPageLayout{#1}{#2}{#3}
% --------------------------------------
% Description: Custom page layout with header and footer content.
% Arguments:
% #1: Header and footer content
% #2: Left header content
% #3: Right header content
% Example:
% \customPageLayout{Title}{Lycée Henri IV}{2024}
% Required Packages: fancyhdr
\newcommand{\customPageLayout}[3]{
    \pagestyle{fancy} % set page style to fancy (add header and footer)
    \fancyhf{} % clear all header and footer content
    \lhead{#2} % left header content
    \rhead{#3} % right header content
    \chead{\textbf{#1}} % center header content in bold (if needed)
    \rfoot{\thepage} % page number in the footer
}


% Counter: \q
% -----------
% Description: Display a question number in a circle.
% Usage:
% - Create a new question: add \q followed by the question content.
% - Reset the question counter: add \setcounter{q}{0} before the first question.
\newcounter{q}
\setcounter{q}{0} % set initial value of the counter
\newcommand{\q}{
    \bigskip
    \addtocounter{q}{1}
    \par
    \Circled{\textbf{\theq}} \space
}


% Counter: \ql
% ------------
% Description: Display a question letter in a round box with indentation (lowercase and not bold).
% Usage:
% - Create a new question: add \ql followed by the question content.
% - Reset the question counter: add \setcounter{ql}{0} before the first question.
\newcounter{ql}
\setcounter{ql}{0} % set initial value of the counter
\newcommand{\ql}{
    \addtocounter{ql}{1}
    \par
    \hspace{1.5em} % indentation before the circled letter
    \textcolor{gray}{\Circled{\alph{ql}}} \space % gray color
}


\title{Analyse - Fonctions Hyperboliques}
\author{Esther Poniatowski}
\date{2024-2025}

\customPageLayout{Correction}{Lycée Henri IV}{2024}

% Colors
\definecolor{Turquoise_perso}{RGB}{ 27, 196, 189 }
\definecolor{Orange_perso}{RGB}{229, 96, 29}
\definecolor{Blue_perso}{RGB}{29, 89, 229}
\definecolor{Blue0_perso}{RGB}{3, 37, 76}
\definecolor{Blue1_perso}{RGB}{17, 103, 177}
\definecolor{Blue2_perso}{RGB}{24, 123, 205}
\definecolor{Blue3_perso}{RGB}{42, 157, 244}
\definecolor{Blue4_perso}{RGB}{208, 239, 255}


% ==================================================================================================
\begin{document}

% --------------------------------------------------------------------------------------------------
\bigskip
\textbf{Étude qualitative des fonctions hyperboliques}

\q Représentations graphiques :

\begin{figure}[htb]
\resizebox{\textwidth}{!}{%

\begin{tikzpicture}
\begin{axis}[xmin=-3, xmax=3, ymin=-4, ymax=4, axis y line=center, axis x line=middle, gray]
\addplot[smooth, Blue_perso, line width=1.5]
{cosh(x)}; \draw(1.25,3.5) node[Blue_perso]{cosh(x)};
\addplot[smooth, Orange_perso, line width=1.5]
{1/2*exp(-x)}; \draw(1.3,0.6) node[Orange_perso]{$\frac{1}{2}$e$^{-x}$}; \addplot[smooth,
Turquoise_perso, line width=1.5]{1/2*exp(x)}; \draw(-1.3,0.6)
node[Turquoise_perso]{$\frac{1}{2}$e$^x$};
\end{axis}
\end{tikzpicture}
\hskip 10pt

\begin{tikzpicture}
\begin{axis}[xmin=-3, xmax=3, ymin=-4, ymax=4, axis y line=center, axis x line=middle, gray]
\addplot[smooth, Blue_perso, line width=1.5]
{sinh(x)}; \draw(1.25,3.5) node[Blue_perso]{sinh(x)};
\addplot[smooth, Orange_perso, line width=1.5]
{-1/2*exp(-x)}; \draw(1.3,0.6) node[Orange_perso]{$-\frac{1}{2}$e$^{-x}$}; \addplot[smooth,
Turquoise_perso, line width=1.5]{1/2*exp(x)}; \draw(-1.3,0.6)
node[Turquoise_perso]{$\frac{1}{2}$e$^x$};
\end{axis}
\end{tikzpicture}
\hskip 10pt

\begin{tikzpicture}
\begin{axis}[xmin=-3, xmax=3, ymin=-4, ymax=4, axis y line=center, axis x line=middle, gray]
\addplot[smooth, Blue_perso, line width=1.5]
{tanh(x)}; \draw(2,2) node[Blue_perso]{tanh(x)}; \addplot[smooth, Orange_perso, line width=1.5,
dashed]{-1}; \addplot[smooth, Orange_perso, line width=1.5, dashed]{1};
\end{axis}
\end{tikzpicture}

}%
\end{figure}

% -----

\q Dérivées :

Les trois fonctions sont dérivables sur $\mathbb{R}$ comme composées d'exponentielles, et pour
$\tanh$ comme produit de fonctions dont le dénominateur ne s'annulle pas.\\
\begin{itemize}
    \item $\sinh'(x) = \frac{e^x + e^{-x}}{2}$ donc \fbox{$\sinh'(x) = \cosh(x)$}
    \item $\cosh'(x) = \frac{e^x - e^{-x}}{2}$ donc \fbox{$\cosh'(x) = \sinh(x)$}
    \item $\tanh'(x) = \frac{\sinh'(x)\cosh(x)- \sinh(x)\cosh'(x)}{\cosh^2(x)} = \frac{\cosh^2(x)-
    \sinh^2(x)}{\cosh^2(x)}$ donc \fbox{$\tanh'(x) = 1 - \tanh^2(x)$}
\end{itemize}

\q Bijections :

\emph{Méthode}
\begin{itemize}
 \item Existence d'un anticédent : garantie par le théorème des valeurs intermédiaires.
 \item Unicité de l'antécédent : garantie par la monotonie.
\end{itemize}

\bigskip
\fbox{$\sinh$ : bijection de $\mathbb{R}$ dans $\mathbb{R}$}
\begin{itemize}
 \item Existence : $\sinh$ est continue (car dérivable) sur $\mathbb{R}$, et $\begin{cases} \lim_{x
 \to +\infty}\sinh(x) = +\infty\\
   \lim_{x \to -\infty}\sinh(x) = -\infty\\
  \end{cases}$, donc par le théorème des valeurs intérmédiaires, tout élément de $\mathbb{R}$ admet
  au moins un antécédent.
 \item Unicité : $\sinh'(x) = \cosh(x) > 0 , \; \forall x \in \mathbb{R}$, donc $\sinh$ est
 strictement croissante, cet antcédent est unique.
\end{itemize}

\bigskip
\fbox{$\cosh$ : bijection de $\mathbb{R}_+$ dans $[1, + \infty[$}
\begin{itemize}
 \item Existence : $\cosh$ est continue (car dérivable) sur $\mathbb{R}_+$, atteint son minimum en
 $0$, et $ \begin{cases} \cosh(0) = 1\\
\cosh(x) \xrightarrow[x \to +\infty]{} +\infty \end{cases}$ donc par le théorème des valeurs
 intérmédiaires, tout élément de $[1, + \infty[$ admet au moins un antécédent.
 \item Unicité : $\cosh'(x) = \sinh(x) > 0 \; \forall x \in \mathbb{R}_+^*$ et ne s'annulle que
 ponctuellement en $0$, donc $\cosh$ est strictement croissante sur $\mathbb{R}_+^*$, donc cet
 antécédent est unique.
\end{itemize}

\bigskip
\fbox{$\tanh$ : bijection de $\mathbb{R}$ dans $]-1,1[$} (idem).

% -----

\q Identité :

\begin{align*}
\cosh^2(x) - \sinh^2(x) & = (\cosh(x) + \sinh(x))(\cosh(x) - \sinh(x)) & \text{(identité remarquable)}\\
& = \frac{1}{2}(e^x + e^{-x} + e^x - e^{-x})\frac{1}{2}(e^x + e^{-x} - e^x + e^{-x})\\
& = \frac{1}{2}2 e^x \frac{1}{2}2e^{-x} & \text{donc \fbox{$\cosh^2(x) - \sinh^2(x) = 1 \; \forall x \in \mathbb{R}$}}
\end{align*}

% --------------------------------------------------------------------------------------------------
\bigskip
\textbf{Résultats généraux : Dérivées et Inverses}

\q Equation caractéristique :

Soit \( f : I \to J \) une fonction strictement monotone et dérivable sur un intervalle \( I \),
admettant une fonction réciproque \( f^{-1} : J \to I \).

Par définition d'une fonction réciproque, pour tout \( x \in J \),
\[
f(f^{-1}(x)) = x
\]

% -----

\q Différentiation de l'identité \( f(f^{-1}(x)) = x \)

Soit \( u \) la fonction définie par :
\[
u = f^{-1}(x)
\]

En appliquant la règle de dérivation en chaîne à \( f(u) = x \) :
\[
\frac{d}{dx} f(u) = \frac{d}{dx} x
\]

Or, par la dérivation en chaîne :
\[
f'(u) \cdot \frac{du}{dx} = 1
\]

Ce qui s'écrit, en remplaçant \( u \) par \( f^{-1}(x) \) :
\[
f'(f^{-1}(x)) \cdot (f^{-1})'(x) = 1
\]

% -----

\q Conclusion :

En supposant que \( f \) est une fonction bijective et dérivable sur un intervalle \( I \), et que
sa dérivée est non nulle, la fonction réciproque \( f^{-1} \) est dérivable et vérifie :
\[
(f^{-1})'(x) = \frac{1}{f'(f^{-1}(x))}
\]

Cette expression est valable pour tout \( x \) appartenant au domaine de \( f^{-1} \) sous la
condition que \( f'(f^{-1}(x)) \neq 0 \).

% --------------------------------------------------------------------------------------------------
\bigskip
\textbf{Réciproques des fonctions hyperboliques}

\q Dérivées des fonctions réciproques :

\emph{Méthode}
\begin{itemize}
    \item Domaine de dérivabilité : $ x \in \mathcal{D}_{f^{-1}}, \; f'(x) \neq 0 $
    \item $(f^{-1})'(x) = \frac{1}{f'(f^{-1}(x))}$
\end{itemize}

\bigskip
\fbox{$\sinh^{-1}$}

$\sinh$ est bijective de $\mathbb{R}$ dans $\mathbb{R}$, dérivable sur $\mathbb{R}$ et sa dérivée ne
s'y annule pas ($\sinh'(x) = \cosh(x) > 0 \; \forall x \in \mathbb{R}$), donc $\sinh^{-1}$ est
dérivable sur $\mathbb{R}$.\\
$$\forall y \in \mathbb{R}, \; (\sinh^{-1})'(y) = \frac{1}{\cosh(\sinh^{-1}(y))}$$
Or $\cosh^2(\sinh^{-1}(y)) = 1 + \sinh^2(\sinh^{-1}(y)) = 1 + y^2 \iff \cosh(\sinh^{-1}(y)) = \pm
\sqrt{1 + y^2}$,\\ et comme $\cosh(x) > 0 \; \forall x \in \mathbb{R}$, seule la solution positive
subsiste : $\cosh(\sinh^{-1}(y)) = \sqrt{1 + y^2}$.\\
Donc : $$\forall y \in \mathbb{R}, \; (\sinh^{-1})'(y) = \frac{1}{\sqrt{1 + y^2}}$$

\bigskip
\fbox{$\cosh^{-1}$}

$\sinh$ est bijective de $\mathbb{R}_+$ dans $[1,+\infty[$, dérivable sur $\mathbb{R}_+$, mais sa
dérivée s'annulle en 0 ($\cosh'(0) = \sinh(0) = 0$), et $0 = \cosh^{-1}(1)$, donc $\cosh^{-1}$ est
dérivable sur $]1,+\infty[$.\\
$$\forall y \in ]1,+\infty[, \; (\cosh^{-1})'(y) = \frac{1}{\sinh(\cosh^{-1}(y))}$$
En appliquant l'identité comme précédemment : $\sinh(\cosh^{-1}(y)) = \pm \sqrt{y^2-1}$,\\ et comme
$\cosh^{-1}(y) > 0 \; \forall y \in ]1, +\infty[$, alors $\sinh(\cosh^{-1}(y)) > 0$, donc seule la
solution positive subsiste : $\sinh(\cosh^{-1}(y)) = \sqrt{y^2-1}$.\\
$$\forall y \in ]1,+\infty[, \; (\cosh^{-1})'(y) = \frac{1}{\sqrt{y^2-1}}$$

\bigskip
\fbox{$\tanh^{-1}$}

De même, $\tanh^{-1}$ est dérivable sur $]-1,1[$.\\
$$\forall y \in ]-1,1[, \; (\tanh^{-1})'(y) = \frac{1}{1 - \tanh^2(\tanh^{-1}(y))}$$
$$\forall y \in ]-1,1[, \; (\tanh^{-1})'(y) = \frac{1}{1 - y^2}$$

\q \emph{Développement limité de $\sinh^{-1}$ à l'ordre $4$ en $0$}\\
Cette fonction est impaire, donc son développement limité ne contient que des termes impairs :\\
$\sinh^{-1}(h) =  (\sinh^{-1})'(0) h +\frac{(\sinh^{-1})^{(3)}(0)}{3 !}h^3 +  o(h^4), \; \forall h
\in \mathbb{R}$. \\
Dérivées successives : \\
$\forall y \in \mathbb{R}, \; (\sinh^{-1})'(y) = \frac{1}{\sqrt{1 + y^2}} = (1 +
y^2)^{-\frac{1}{2}}$\\
$\forall y \in \mathbb{R}, \; (\sinh^{-1})''(y) = -\frac{1}{2}2y(1 + y^2)^{-\frac{3}{2}} = -y(1 +
y^2)^{-\frac{3}{2}}$\\
$\forall y \in \mathbb{R}, \; (\sinh^{-1})^{(3)}(y) = (-y)\left(-\frac{3}{2}2y(1 +
y^2)^{-\frac{5}{2}}\right) + (-1)\left((1 + y^2)^{-\frac{3}{2}}\right) = 3y^2(1 +
y^2)^{-\frac{5}{2}} - (1 + y^2)^{-\frac{3}{2}}$\\
Termes impairs evalués en $0$ :\\
$(\sinh^{-1})'(0) = \frac{1}{\sqrt{1 + 0}} = 1$.\\
$(\sinh^{-1})^{(3)}(0) = 0 - (1 + 0)^{-\frac{3}{2}} = -1$\\
Ainsi, au voisinage de $0$ :
$$\sinh^{-1}(h) = h - \frac{1}{6}h^3 + o(h^4), \; \forall h \in
\mathbb{R}$$

\q Expression des réciproques :

\bigskip
\fbox{$\sinh^{-1}$}

Pour $y \in \mathbb{R}$ fixé, on résout l'équation $\sinh(x) = y$ d'inconnue $x \in \mathbb{R}$. On
se ramène à une écriture polynômiale :\\
$\frac{e^x - e^{-x}}{2} = y \iff e^x - e^{-x} = 2y \iff e^{2x} - 1 = 2y e^x \iff e^{2x} - 2y e^x - 1
 = 0 \iff X^2 - 2y X - 1 = 0$ en posant $X = e^x$.\\
 $\Delta = 4 y^2 + 4 > 0$, donc les racines sont : $ \begin{cases} y + \sqrt{y^2 + 1} > 0 \; \forall
 y \in \mathbb{R}\\
 y - \sqrt{y^2 + 1} < 0 \; \forall y \in \mathbb{R} \end{cases}$.\\
 Or $X = e^x > 0$, donc l'unique solution est : $e^x = y + \sqrt{y^2 + 1}$.\\
 D'où :
 $$\forall y \in \mathbb{R},\; \sinh^{-1}(y) = \ln(y + \sqrt{y^2+1})$$

\bigskip
\fbox{$\cosh^{-1}$}

Pour $y \in [1,+\infty[$ fixé, on résout l'équation $\cosh(x) = y$ d'inconnue $x \in
\mathbb{R}_+$.\\
En procédant de même, on obtient pour racines :
$ \begin{cases} y + \sqrt{y^2 - 1} \geq 1 \; \forall
 y \geq 1 \\
 y - \sqrt{y^2 - 1} \leq 1 \; \forall y \geq 1 \end{cases}$.\\
 Or $X = e^x \geq 1$ (car $x \geq 0$), donc l'unique solution est : $e^x = y + \sqrt{y^2 - 1}$.\\
 D'où :
$$\forall y \in \mathbb{R},\; \cosh^{-1}(y) = \ln(y + \sqrt{y^2-1})$$

\bigskip
\fbox{$\tanh^{-1}$}

Pour $y \in ]-1,1[$ fixé, on résout l'équation $\tanh(x) = y$ d'inconnue $x \in \mathbb{R}$.\\
$\frac{e^x - e^{-x}}{e^x + e^{-x}} = y \iff e^x - e^{-x} = y(e^x + e^{-x}) \iff e^{2x} - 1 = y
 (e^{2x} + 1) \iff e^{2x}(1 - y) = 1 + y \iff e^{2x} = \frac{1 + y}{1 - y} \iff x =
 \frac{1}{2}\ln\left(\frac{1 + y}{1 - y}\right)$.\\
 D'où :
 $$\forall y \in ]-1,1[\; \tanh^{-1}(y) = \frac{1}{2}(\ln(1+y) - \ln(1-y)$$

\q Représentations graphiques des réciproques :

Construire les symétriques de $\sinh$, $\cosh$ et $\tanh$ par rapport à la droite d'équation $y =
x$.

\begin{figure}[h!]
\resizebox{\textwidth}{!}{%

\begin{tikzpicture}
\begin{axis}[xmin=-3, xmax=3, ymin=-4, ymax=4, axis y line=center, axis x line=middle, gray]
\addplot[smooth, gray, line width=1.5, dashed]{x}; \addplot[smooth, Blue_perso, line width=1.5,
domain=0:4]{cosh(x)}; \draw(1.2,3.5) node[Blue_perso]{cosh(x)}; \addplot[smooth, Orange_perso, line
width=1.5, domain=1:4]{ln(x+sqrt(x^2-1))}; \draw(2,0.6) node[Orange_perso]{cosh$^{-1}$(x)};
\end{axis}
\end{tikzpicture}
\hskip 10pt

\begin{tikzpicture}
\begin{axis}[xmin=-3, xmax=3, ymin=-4, ymax=4, axis y line=center, axis x line=middle, gray]
\addplot[smooth, gray, line width=1.5, dashed]
{x};
\addplot[smooth, Blue_perso, line width=1.5]
{sinh(x)}; \draw(1.2,3.5) node[Blue_perso]{sinh(x)};
\addplot[smooth, Orange_perso, line width=1.5]
{ln(x+sqrt(x^2+1))}; \draw(2,0.6) node[Orange_perso]{sinh$^{-1}$(x)};
\end{axis}
\end{tikzpicture}
\hskip 10pt

\begin{tikzpicture}
\begin{axis}[xmin=-3, xmax=3, ymin=-4, ymax=4, axis y line=center, axis x line=middle, gray]
\addplot[smooth, gray, line width=1.5, dashed]
{x};
\addplot[smooth, Blue_perso, line width=1.5]
{tanh(x)}; \draw(2.3,1.3) node[Blue_perso]{tanh(x)};
\addplot[smooth, Orange_perso, line width=1.5, domain=-1:1, samples=100]
{1/2*(ln(1+x)-ln(1-x))}; \draw[dashed, Orange_perso, line width=1.5](-1,-4) -- (-1,4); \draw[dashed,
Orange_perso, line width=1.5](1,-4) -- (1,4); \draw(1,3.5) node[Orange_perso]{tanh$^{-1}$(x)};
\end{axis}
\end{tikzpicture}

}%
\end{figure}


\end{document}
% ==================================================================================================
