% PROBLÈME : Fonction définie par une intégrale et suites associées
% ==================================================================================================
%
% But
% ---
% Réviser les notions de :
% - limite
% - continuité
% - dérivabilité
% - étude de fonction
% - suites définies par récurrence
% - point fixe
% - stabilité,
% - comportement asymptotique
% ==================================================================================================

\documentclass[10pt,a4paper]{article}

% Set the root path
\providecommand{\rootpath}{../../..}
% Fonts
\usepackage[utf8]{inputenc} % for accents
\usepackage[T1]{fontenc} % for accents
\usepackage[french]{babel} % for french language
\usepackage{helvet} % sans serif font family
\renewcommand*\familydefault{\sfdefault} % sans serif font family

% Mathematics
\usepackage{amsmath,amsfonts,amssymb} % for math symbols
\usepackage{array} % for tabular


\usepackage{parskip} % no indent, space between paragraphs

\usepackage{geometry} % margin
\geometry{
    a4paper,
    left=15mm,
    right=15mm,
    top=20mm,
    bottom=20mm
}

\usepackage{circledsteps} % to draw circles around numbers

\usepackage{fancyhdr} % for headers and footers

\usepackage{enumitem} % for customizing lists
\setlist[enumerate]{itemsep=1em} % space between items only in enumerate environment (not itemize)
\setlist[itemize]{label=--} % set itemize label to em-dash

% Command: \customPageLayout{#1}{#2}{#3}
% --------------------------------------
% Description: Custom page layout with header and footer content.
% Arguments:
% #1: Header and footer content
% #2: Left header content
% #3: Right header content
% Example:
% \customPageLayout{Title}{Lycée Henri IV}{2024}
% Required Packages: fancyhdr
\newcommand{\customPageLayout}[3]{
    \pagestyle{fancy} % set page style to fancy, i.e. header and footer
    \fancyhf{#1} % set header and footer content
    \lhead{#2} % set left header content
    \rhead{#3} % set right header content
    \fancyfoot{} % clear footer content
    \rfoot{\thepage} % set page number in footer
}

% Counter: \q
% -----------
% Description: Display a question number in a circle.
\newcounter{q}
\setcounter{q}{0} % set initial value of counter
\newcommand{\q}{
    \bigskip
    \addtocounter{q}{1}
    \par
    \Circled{\textbf{\theq}} \space
}


\title{Analyse - Fonction définie par une intégrale}
\author{Esther Poniatowski}
\date{2024-2025}

\customPageLayout{Sujets d'interrogation orale}{Lycée Henri IV}{2024}

% ==================================================================================================
\begin{document}

\textbf{Contexte}

De nombreux problèmes font intervenir une fonction définie par une intégrale à paramètre. Par
exemple, des fonctions de la forme \( \displaystyle F(x) = \int_0^x \frac{\varphi(t)}{t^\alpha} \,
dt \) apparaissent dans l'étude de certaines équations différentielles.

De telles fonctions peuvent présenter des difficultés au voisinage d'un point, généralement en 0 ou
à l'infini, qui proviennent d'une singularité de l'intégrande.

\bigskip
\textbf{Objectifs}

Étudier les propriétés analytiques d'une fonction définie par une intégrale : limites, continuité,
dérivabilité, variations...

Construire une suite récurrente définie à partir de cette fonction et en étudier son comportement
asymptotique.

\bigskip

% --------------------------------------------------------------------------------------------------
\textbf{Étude d'une fonction définie par une intégrale}

Soit la fonction définie sur \( \mathbb{R}_+ \) par :
\[
f(x) =
\begin{cases}
\displaystyle \int_0^x \frac{1 - \cos(t)}{t} \, dt & \text{si } x > 0, \\
0 & \text{si } x = 0
\end{cases}
\]

\q Pour \( t \to 0 \), établir l'équivalent \( 1 - \cos(t) \sim \frac{t^2}{2} \). En déduire un
équivalent de \( \frac{1 - \cos(t)}{t} \) lorsque \( t \to 0 \).
% But : étudier la limite en 0 pour montrer la continuité, via équivalent de la fonction intégrée.
% Méthode : équivalent, passage à la limite.

\q En déduire que l'intégrale définissant \( f(x) \) admet une limite finie lorsque \( x \to 0^+ \),
et que \( f \) est continue en \( 0 \).
% But : existence de la limite de l'intégrale et valeur de f(0).
% Méthode : théorème de convergence dominée ou intégrabilité de l'équivalent.

\q Montrer que \( f \) est dérivable sur \( \mathbb{R}_+^* \) et que, pour tout \( x > 0 \),
\[
f'(x) = \frac{1 - \cos(x)}{x}.
\]
Étudier la limite de \( f'(x) \) lorsque \( x \to 0 \), et conclure sur la dérivabilité de \( f \)
en \( 0 \).
% But : déterminer une expression de la dérivée de f sur R+*, étudier la dérivabilité sur R+.
% Méthode : théorème fondamental du calcul intégral, étude locale.

\q Étudier le signe de \( f'(x) \) sur \( \mathbb{R}_+^* \), puis en déduire les variations de \( f
\) sur \( \mathbb{R}_+ \).
% But : étudier le signe de f', donc les variations de f.
% Méthode : étude de fonction, variation de la fonction dérivée.

\q Montrer que \( f(x) \) admet une limite finie lorsque \( x \to +\infty \), notée \( L
\). Donner une majoration permettant de justifier cette convergence.
% But : limite à l'infini.
% Méthode : étude d'équivalent ou encadrement intégral.

% --------------------------------------------------------------------------------------------------
\vspace{0.5cm}

\textbf{Étude d'une suite récurrente associée}

Soit la suite \( (u_n) \) définie par :
\[
\begin{cases}
    u_0 \in [0, +\infty[ \\
    \forall n \in \mathbb{N}, \quad u_{n+1} = f(u_n)
\end{cases}
\]

\q Montrer que si \( u_0 \in [0, L] \), alors \( (u_n) \) reste dans \( [0, L] \) pour tout \( n \).
% But : définir et justifier la suite, domaine de définition, intervalle stable.
% Méthode : définition d'un intervalle stable.

\q Montrer que \( f \) admet au moins un point fixe sur \( [0, L] \), c'est-à-dire un réel \( \ell
\) tel que \( f(\ell) = \ell \).
% But : existence des points fixes de la fonction f.
% Méthode : théorème des valeurs intermédiaires.

\q Admettre que \( f \) est continue et croissante sur \( \left[0, +\infty \right[ \). En déduire
que si \( u_0 \in [0, L] \), alors \( (u_n) \) converge vers un point fixe \( \ell \) de \( f \).
% But : convergence de la suite, lien avec point fixe.
% Méthode : démonstration de convergence via théorème des suites récurrentes et étude de f.

% --------------------------------------------------------------------------------------------------
\vspace{0.5cm}

\textbf{Étude asymptotique et comportement local}

\q Montrer que lorsque \( x \to 0 \) :
\[
f(x) \mathop{\sim}\limits_{ x \to 0 } \frac{x}{2}
\]
% But : équivalent de f(x) en 0 pour affiner l'étude de la suite.
% Méthode : équivalent, développement limité.

\q En déduire que si \( u_0 \) est suffisamment proche de 0, alors \( (u_n) \) converge vers 0.
% But : convergence vers 0 selon u_0.
% Méthode : comparaison, comportement asymptotique de la suite.

\q Discuter de la convergence de \( (u_n) \) selon la valeur initiale \( u_0 \), en particulier
lorsque \( u_0 \) est proche de \( L \).
% But : lien entre point fixe et comportement asymptotique.
% Méthode : réflexion finale.

\end{document}
% ==================================================================================================
