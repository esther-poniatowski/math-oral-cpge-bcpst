% CORRECTION : Modèles de Population par des Suites Récurrentes
% ==================================================================================================

\documentclass[10pt,a4paper]{article}

\providecommand{\rootpath}{../../..}
% Fonts
\usepackage[utf8]{inputenc} % for accents
\usepackage[T1]{fontenc} % for accents
\usepackage[french]{babel} % for french language
\usepackage{helvet} % sans serif font family
\renewcommand*\familydefault{\sfdefault} % sans serif font family

% Mathematics
\usepackage{amsmath,amsfonts,amssymb} % for math symbols
\usepackage{array} % for tabular


\usepackage{parskip} % no indent, space between paragraphs

\usepackage{geometry} % margin
\geometry{
    a4paper,
    left=15mm,
    right=15mm,
    top=20mm,
    bottom=20mm
}

\usepackage{circledsteps} % to draw circles around numbers

\usepackage{fancyhdr} % for headers and footers

\usepackage{enumitem} % for customizing lists
\setlist[enumerate]{itemsep=1em} % space between items only in enumerate environment (not itemize)
\setlist[itemize]{label=--} % set itemize label to em-dash

% Command: \customPageLayout{#1}{#2}{#3}
% --------------------------------------
% Description: Custom page layout with header and footer content.
% Arguments:
% #1: Header and footer content
% #2: Left header content
% #3: Right header content
% Example:
% \customPageLayout{Title}{Lycée Henri IV}{2024}
% Required Packages: fancyhdr
\newcommand{\customPageLayout}[3]{
    \pagestyle{fancy} % set page style to fancy, i.e. header and footer
    \fancyhf{#1} % set header and footer content
    \lhead{#2} % set left header content
    \rhead{#3} % set right header content
    \fancyfoot{} % clear footer content
    \rfoot{\thepage} % set page number in footer
}

% Counter: \q
% -----------
% Description: Display a question number in a circle.
\newcounter{q}
\setcounter{q}{0} % set initial value of counter
\newcommand{\q}{
    \bigskip
    \addtocounter{q}{1}
    \par
    \Circled{\textbf{\theq}} \space
}


\title{Correction -- Modèles de Population par des Suites Récurrentes}
\author{Esther Poniatowski}
\date{2024-2025}

\customPageLayout{Correction des sujets d'interrogation orale}{Lycée Henri IV}{2024}

% ==================================================================================================
\begin{document}

\bigskip
\textbf{Étude de la fonction logistique}

\q Démonstration de la continuité et de l'image de \( f \).

La fonction \( f(x) = r x (1 - x) \) est polynomiale en \( x \), donc continue sur \( \mathbb{R} \)
a fortiori sur \( [0,1] \).

Pour \( x \in [0,1] \), il s'agit de montrer que \( f(x) \in [0,1] \). En effet, \( x(1 - x) \in
[0,1/4] \) :

Soit la fonction auxiliaire, définie sur \( \mathbb{R} \) :
\[
g(x) = x(1 - x) = -x^2 + x
\]
Elle s'annule aux points \( x = 0 \) et \( x = 1 \).

Cette fonction quadratique concave (car le coefficient de \( x^2 \) est négatif) admet un maximum,
qui est atteint au sommet de la parabole, dont l'abscisse est donnée par :
\[
x_s = -\frac{b}{2a} = -\frac{1}{2(-1)} = \frac{1}{2}
\]
La valeur de \( g(x) \) en \( x_s = \frac{1}{2} \) est :
\[
f\left(\frac{1}{2}\right) = \frac{1}{2} \left(1 - \frac{1}{2} \right) = \frac{1}{2} \times \frac{1}{2} = \frac{1}{4}
\]

Comme \( r \in (0,4) \), il s'ensuit que :
\[
0 \leq f(x) \leq r \times \frac{1}{4} < 1
\]
d'où \( f([0,1]) \subset [0,1] \).

% ------

\q Calcul de la dérivée et étude des variations :

La fonction \( f \) est dérivable sur \( \mathbb{R} \) et :
\[
f'(x) = r (1 - 2x)
\]
Signes de la dérivée :
\begin{itemize}
    \item \( f'(x) = 0 \) pour \( x = 1/2 \),
    \item \( f'(x) > 0 \) pour \( x < 1/2 \),
    \item \( f'(x) < 0 \) pour \( x > 1/2 \).
\end{itemize}

Ainsi, \( f \) est croissante sur \( [0,1/2] \),
décroissante sur \( [1/2, 1] \), et atteint un maximum en \( x = 1/2 \).

\[
f\left( \frac{1}{2} \right) = r \times \frac{1}{2} \times \frac{1}{2} = \frac{r}{4}
\]

Tableau de signes et de variations :
\[
\begin{array}{c|ccccc}
x & 0 &  & 1/2 &  & 1 \\
\hline
f'(x) &  & + & 0 & - &  \\
\hline
f(x) & & \nearrow & \frac{r}{4} & \searrow &
\end{array}
\]

% ------

\q Détermination des points fixes :

Recherche de \( x \) tel que \( f(x) = x \), soit :
\[
r x (1 - x) = x
\]
\[
r x (1 - x) - x = 0
\]
\[
x (r (1 - x) - 1) = 0
\]
\[
x = 0 \quad \text{ou} \quad r (1 - x) = 1
\]
\[
x = 0 \quad \text{ou} \quad x = 1 - \frac{1}{r}
\]
Conclusion :
\begin{itemize}
    \item Pour \( r > 1 \), il existe deux points fixes : \( 0 \) et \( 1 - \frac{1}{r} \).
    \item Pour \( r \leq 1 \), seul \( 0 \) appartient à \( [0,1] \).
\end{itemize}

% --------------------------------------------------------------------------------------------------
\bigskip
\textbf{Étude qualitative de la suite}

\q Invariance de l'intervalle \( [0,1] \) :

Pour \( u_0 \in [0,1] \), \( u_1 = f(u_0) \in [0,1] \), car \( f([0,1]) \subset [0,1] \).

Par récurrence immédiate, \( u_n \in [0,1] \) pour tout \( n \in \mathbb{N} \).

% ------

\q Limite et point fixe :

Si \( (u_n) \) converge vers \( \ell \), alors par passage à la limite dans \( u_{n+1} = f(u_n) \),
il vient \( \ell = f(\ell) \), donc \( \ell \) est un point fixe.

% ------

\q Interprétation graphique :

Le point fixe \( \ell \) est l'intersection de la courbe de \( f \) avec la droite d'équation \( y =
x \). En effet, \( f(x) = x \) équivaut à \( r x (1 - x) = x \), soit \( r x (1 - x) - x = 0 \), ce
qui correspond à l'équation de la courbe de \( f \).

% --------------------------------------------------------------------------------------------------
\bigskip
\textbf{Stabilité des points fixes}

\q Contractivité et Stabilité :

Par définition de la contractivité : $|u_{n+1} - c| = |g(u_n) - g(c)| < k|u_{n} - c|$, qui implique
par récurrence :
$$|u_{n+1} - c| < k^n |u_0 - c|$$

Par comparaison avec une suite géométrique de raison $k < 1$, l'écart $|u_{n+1} - c| \to 0$ donc la
suite converge vers le point fixe.

% -----

\q Illustration graphique :

La contractivité de \( f \) est visualisée par le cône de sommet \( (c, c) \) et de base \( (u_0,
f(u_0)) \), qui se rétrécit à chaque itération.

% -----

\q Contractivité et Stabilité locales :

Puisque \( f \) est de classe \( \mathcal{C}^1 \), alors sa dérivée est continue.

Si \( |f'(c)| < 1 \), alors il existe un intervalle autour de \( c \), par exemple \( [c - h, c +
h] \), dans lequel la dérivée est bornée par une constante \( k < 1 \).

Ainsi, la fonction \( f \) est contractante sur cet intervalle, et le point fixe \( c \) est stable.

% -----

\q Stabilité des points fixes selon la valeur du paramètre \( r \) :

\begin{itemize}
    \item Pour \( r \leq 1 \), seul \( 0 \) est stable.
    \item Pour \( r \in (1,3) \), \( 0 \) est instable et \( 1 - \frac{1}{r} \) est stable.
    \item Pour \( r \in (3,4) \), les deux points fixes sont instables.
\end{itemize}

\end{document}

% ==================================================================================================

VERSION ANTERIEURE
\textbf{Stabilité des points fixes}

\q Expression de l'écart par rapport au point fixe :

En écrivant \( u_n = \ell + \epsilon_n \), où \( \epsilon_n \) est l'écart par rapport à \( \ell \)
:
\[
u_{n+1} = f(u_n) = f(\ell + \epsilon_n) = f(\ell) + f'(\ell) \epsilon_n + o(\epsilon_n)
\]
Par définition du point fixe, \( f(\ell) = \ell \), donc :
\[
u_{n+1} = \ell + f'(\ell) \epsilon_n + o(\epsilon_n)
\]
\[
u_{n+1} - \ell = f'(\ell) \epsilon_n + o(\epsilon_n)
\]
Conclusion :
\[
\epsilon_{n+1} = f'(\ell) \epsilon_n + o(\epsilon_n)
\]

% ------

\q Condition de stabilité : L'objectif est de démontrer que si \( |f'(\ell)| < 1 \), alors \(
\epsilon_n \to 0 \).

Pour cela, montrons que cette suite est inférieure à une suite géométrique de raison \( 0 < q < 1
\), i.e. :
\[
\forall n \in \mathbb{N},\ |\epsilon_n| \leq q^n |\epsilon_0|
\]

Choix de la raison : Puisque $|f'(\ell)| < 1$, il est possible de choisir un paramètre \( k \) tel
que \( q = |f'(\ell)| + k < 1 \).

Initialisation :

Pour \( n = 0 \), \( |\epsilon_0| \leq q^0 |\epsilon_0| \) est vérifié.

Hérédité :

Supposons que \( |\epsilon_n| \leq q^n |\epsilon_0| \) pour un certain \( n \geq 0 \). Alors :
\[
|\epsilon_{n+1}| = |f'(\ell) \epsilon_n + o(\epsilon_n)| \leq |f'(\ell)| |\epsilon_n| + |o(\epsilon_n)| \leq |f'(\ell)| |\epsilon_n| + k |\epsilon_n| = (|f'(\ell)| + k) |\epsilon_n| = q |\epsilon_n|
\]
Contrôle des termes de négligeabilité : Par définition,
\[
\lim_{\epsilon_n \to 0} \frac{o(\epsilon_n)}{\epsilon_n} = 0
\]

Ainsi, pour tout seuil \( k \) arbitrairement petit, il existe \( \delta_n > 0 \) tel que :
\[
 |\epsilon_n| < \delta_n \implies |o(\epsilon_n)| < k |\epsilon_n|
\]
Et dans ce cas :
\begin{align*}
    |\epsilon_{n+1}| &= |f'(\ell) \epsilon_n + o(\epsilon_n)| \\
    &\leq |f'(\ell)| |\epsilon_n| + |o(\epsilon_n)| \\
    &< |f'(\ell)| |\epsilon_n| + k |\epsilon_n| \\
    &= (|f'(\ell)| + k) |\epsilon_n| \\
\end{align*}

Puisque \( |f'(\ell)| < 1 \), il est possible de choisir \( k \) suffisamment petit pour que \(
|f'(\ell)| + k < 1 \). Ainsi :
\[
|\epsilon_{n+1}| < (|f'(\ell)| + k) |\epsilon_n| < |\epsilon_n|
\]

Récurrence :

Supposons que \( u_0 \) soit suffisamment proche de \( \ell \) pour que \( |\epsilon_0| < \delta
\). Alors, par récurrence, on montre que \( |\epsilon_n| < |\epsilon_{n-1}| \) pour tout \( n \geq 1
\), ce qui implique que \( \epsilon_n \) est une suite décroissante et bornée inférieurement par 0
(ou par \( -\delta \) si on considère des valeurs négatives).

Puisque \( \epsilon_n \) est décroissante et bornée, elle converge vers une limite \( L \). Nous
devons montrer que \( L = 0 \).

Supposons que \( L \neq 0 \). Alors, pour \( n \) suffisamment grand, \( |\epsilon_n| \) est proche
de \( L \) et donc \( |\epsilon_n| > L/2 \) pour un certain \( L/2 > 0 \). Cependant, puisque \(
|\epsilon_{n+1}| < (|f'(\ell)| + k) |\epsilon_n| \) et que \( |f'(\ell)| + k < 1 \), on a :
\[
|\epsilon_{n+1}| < |\epsilon_n|
\]

Cela implique que \( |\epsilon_n| \) diminue à chaque itération, ce qui contredit l'hypothèse que \(
L \neq 0 \) si \( L > 0 \). Par conséquent, \( L = 0 \), ce qui signifie que \( \epsilon_n \to 0 \)
lorsque \( n \to \infty \).


% -----

\q Condition d'instabilité : cas \( |f'(\ell)| > 1 \)

Lorsque \( |f'(\ell)| > 1 \), l'itération de la relation :
\[
\epsilon_n = (f'(\ell))^n \epsilon_0 + \sum_{k=0}^{n-1} (f'(\ell))^{n-1-k} o(\epsilon_k)
\]
montre que le premier terme \( (f'(\ell))^n \epsilon_0 \) diverge exponentiellement, car \(
|f'(\ell)|^n \to \infty \) si \( |f'(\ell)| > 1 \).

Même en tenant compte du terme \( o(\epsilon_n) \), celui-ci reste négligeable par rapport à la
croissance exponentielle de \( (f'(\ell))^n \), ce qui entraîne :
\[
\epsilon_n \to \infty.
\]
Ainsi, la suite \( (u_n) \) s'éloigne de \( \ell \), ce qui établit son instabilité.

% ------

\q Stabilité locale des points fixes :

Le critère est \( |f'(\ell)| < 1 \) pour la stabilité. Or :
\[
f'(x) = r (1 - 2x)
\]
\[
f'(0) = r, \quad f'\left(1 - \frac{1}{r}\right) = r \left(1 - 2 + \frac{2}{r}\right) = r \left( -1 + \frac{2}{r} \right) = 2 - r
\]
Ainsi :
\[
|f'(0)| = |r| = r, \quad |f'(\ell)| = |2 - r| = 2 - r
\]
Le point fixe \( 0 \) est stable si \( r < 1 \), \( 1 - \frac{1}{r} \) est stable si \( 1 < r < 3
\).

% --------------------------------------------------------------------------------------------------
\bigskip
\textbf{Étude asymptotique en fonction du paramètre}

\textit{Cas \( r \in (0,1) \) : Extinction.}

\q Pour \( r \in (0,1) \), \( 0 \) est unique point fixe dans \( [0,1] \), avec \( |f'(0)| = r < 1
\), donc stable. Ainsi, pour tout \( u_0 \in (0,1) \), \( u_n \to 0 \).

% ------

\textit{Cas \( r \in (1,2) \) : Stabilisation.}

\q Pour \( r \in (1,2) \), \( 0 \) est instable (car \( |f'(0)| = r > 1 \)) et \( \ell = 1 -
\frac{1}{r} \) est stable car \( |2 - r| < 1 \). Ainsi, \( (u_n) \) converge vers \( \ell \), pour
\( u_0 \in (0,1) \setminus \{0\} \).

% ------

\textit{Cas \( r \in (2,3) \) : Étude fine.}

\q Le point fixe \( \ell = 1 - \frac{1}{r} \) existe toujours. De plus :
\[
|f'(\ell)| = |2 - r|
\]
Lorsque \( r \to 3^- \), \( |f'(\ell)| \to 1 \). La stabilité diminue, et pour \( r \) proche de 3,
la convergence vers \( \ell \) devient lente.

\q Conjecture : pour \( r \in (2,3) \), \( (u_n) \) converge vers \( \ell \), bien que la vitesse de
convergence diminue à mesure que \( r \to 3^- \).

% ------

\textit{Cas \( r > 3 \) : Comportement chaotique.}

\q Lorsque \( |f'(\ell)| > 1 \), le point fixe \( \ell \) devient instable, ce qui ouvre la
possibilité de cycles de période 2, puis de comportements plus complexes (bifurcations, chaos). En
effet, la suite peut osciller entre plusieurs valeurs ou devenir apériodique, ce qui correspond à
des phénomènes de bifurcation ou de chaos déterministe.
