% CORRECTION : Modèles de Population par des Suites Récurrentes
% ==================================================================================================

\documentclass[10pt,a4paper]{article}

\providecommand{\rootpath}{../../..}
% Fonts
\usepackage[utf8]{inputenc} % for accents
\usepackage[T1]{fontenc} % for accents
\usepackage[french]{babel} % for french language
\usepackage{helvet} % sans serif font family
\renewcommand*\familydefault{\sfdefault} % sans serif font family

% Mathematics
\usepackage{amsmath,amsfonts,amssymb} % for math symbols
\usepackage{array} % for tabular


\usepackage{parskip} % no indent, space between paragraphs

\usepackage{geometry} % margin
\geometry{
    a4paper,
    left=15mm,
    right=15mm,
    top=20mm,
    bottom=20mm
}

\usepackage{circledsteps} % to draw circles around numbers

\usepackage{fancyhdr} % for headers and footers

\usepackage{enumitem} % for customizing lists
\setlist[enumerate]{itemsep=1em} % space between items only in enumerate environment (not itemize)
\setlist[itemize]{label=--} % set itemize label to em-dash

% Command: \customPageLayout{#1}{#2}{#3}
% --------------------------------------
% Description: Custom page layout with header and footer content.
% Arguments:
% #1: Header and footer content
% #2: Left header content
% #3: Right header content
% Example:
% \customPageLayout{Title}{Lycée Henri IV}{2024}
% Required Packages: fancyhdr
\newcommand{\customPageLayout}[3]{
    \pagestyle{fancy} % set page style to fancy (add header and footer)
    \fancyhf{} % clear all header and footer content
    \lhead{#2} % left header content
    \rhead{#3} % right header content
    \chead{\textbf{#1}} % center header content in bold (if needed)
    \rfoot{\thepage} % page number in the footer
}


% Counter: \q
% -----------
% Description: Display a question number in a circle.
% Usage:
% - Create a new question: add \q followed by the question content.
% - Reset the question counter: add \setcounter{q}{0} before the first question.
\newcounter{q}
\setcounter{q}{0} % set initial value of the counter
\newcommand{\q}{
    \bigskip
    \addtocounter{q}{1}
    \par
    \Circled{\textbf{\theq}} \space
}


% Counter: \ql
% ------------
% Description: Display a question letter in a round box with indentation (lowercase and not bold).
% Usage:
% - Create a new question: add \ql followed by the question content.
% - Reset the question counter: add \setcounter{ql}{0} before the first question.
\newcounter{ql}
\setcounter{ql}{0} % set initial value of the counter
\newcommand{\ql}{
    \addtocounter{ql}{1}
    \par
    \hspace{1.5em} % indentation before the circled letter
    \textcolor{gray}{\Circled{\alph{ql}}} \space % gray color
}


\title{Correction -- Modèles de Population par des Suites Récurrentes}
\author{Esther Poniatowski}
\date{2024-2025}

\customPageLayout{Correction des sujets d'interrogation orale}{Lycée Henri IV}{2024}

% ==================================================================================================
\begin{document}

\bigskip
\textbf{Étude de la fonction logistique}

\q Démonstration de la continuité et de l'image de \( f \).

La fonction \( f(x) = r x (1 - x) \) est polynomiale en \( x \), donc continue sur \( \mathbb{R} \)
a fortiori sur \( [0,1] \). Pour \( x \in [0,1] \), il s'agit de montrer que \( f(x) \in [0,1] \).
En effet, \( x(1 - x) \in [0,1/4] \) et comme \( r \in (0,4) \), il s'ensuit que
\[
0 \leq f(x) \leq r \times \frac{1}{4} < 1
\]
d'où \( f([0,1]) \subset [0,1] \).

% ------

\q Calcul de la dérivée et étude des variations :

La fonction \( f \) est dérivable sur \( \mathbb{R} \) et :
\[
f'(x) = r (1 - 2x)
\]
L'annulation de \( f'(x) \) donne \( x = \frac{1}{2} \). Pour \( x < \frac{1}{2} \), \( f'(x) > 0
\), et pour \( x > \frac{1}{2} \), \( f'(x) < 0 \). Ainsi, \( f \) est croissante sur \( [0,1/2] \),
décroissante sur \( [1/2, 1] \), et atteint un maximum en \( x = 1/2 \).

\[
f\left( \frac{1}{2} \right) = r \times \frac{1}{2} \times \frac{1}{2} = \frac{r}{4}
\]

% ------

\q Détermination des points fixes :

Recherche de \( x \) tel que \( f(x) = x \), soit :
\[
r x (1 - x) = x
\]
\[
r x (1 - x) - x = 0
\]
\[
x (r (1 - x) - 1) = 0
\]
\[
x = 0 \quad \text{ou} \quad r (1 - x) = 1
\]
\[
x = 0 \quad \text{ou} \quad x = 1 - \frac{1}{r}
\]
Ainsi, pour \( r > 1 \), il existe deux points fixes : \( 0 \) et \( 1 - \frac{1}{r} \). Pour \( r
\leq 1 \), seul \( 0 \) appartient à \( [0,1] \).

% ------

\bigskip
\textbf{Étude qualitative de la suite}

\q Invariance de l'intervalle \( [0,1] \).

Pour \( u_0 \in [0,1] \), \( u_1 = f(u_0) \in [0,1] \), car \( f([0,1]) \subset [0,1] \). Par
récurrence, \( u_n \in [0,1] \) pour tout \( n \in \mathbb{N} \).

% ------

\q Limite de la suite : point fixe.

Si \( (u_n) \) converge vers \( \ell \), alors par passage à la limite dans \( u_{n+1} = f(u_n) \),
il vient \( \ell = f(\ell) \), donc \( \ell \) est un point fixe.

% ------

\q Stabilité locale des points fixes :

Le critère est \( |f'(\ell)| < 1 \) pour la stabilité. Or :
\[
f'(x) = r (1 - 2x)
\]
\[
f'(0) = r, \quad f'\left(1 - \frac{1}{r}\right) = r \left(1 - 2 + \frac{2}{r}\right) = r \left( -1 + \frac{2}{r} \right) = 2 - r
\]
Ainsi :
\[
|f'(0)| = |r| = r, \quad |f'(\ell)| = |2 - r| = 2 - r
\]
Le point fixe \( 0 \) est stable si \( r < 1 \), \( 1 - \frac{1}{r} \) est stable si \( 1 < r < 3
\).

% --------------------------------------------------------------------------------------------------
\bigskip
\textbf{Étude asymptotique en fonction du paramètre}

\textit{Cas \( r \in (0,1) \) : Extinction.}

\q Pour \( r \in (0,1) \), \( 0 \) est unique point fixe dans \( [0,1] \), avec \( |f'(0)| = r < 1
\), donc stable. Ainsi, pour tout \( u_0 \in (0,1) \), \( u_n \to 0 \).

% ------

\textit{Cas \( r \in (1,2) \) : Stabilisation.}

\q Pour \( r \in (1,2) \), \( 0 \) est instable (car \( |f'(0)| = r > 1 \)) et \( \ell = 1 -
\frac{1}{r} \) est stable car \( |2 - r| < 1 \). Ainsi, \( (u_n) \) converge vers \( \ell \), pour
\( u_0 \in (0,1) \setminus \{0\} \).

% ------

\textit{Cas \( r \in (2,3) \) : Étude fine.}

\q Le point fixe \( \ell = 1 - \frac{1}{r} \) existe toujours. De plus :
\[
|f'(\ell)| = |2 - r|
\]
Lorsque \( r \to 3^- \), \( |f'(\ell)| \to 1 \). La stabilité diminue, et pour \( r \) proche de 3,
la convergence vers \( \ell \) devient lente.

\q Conjecture : pour \( r \in (2,3) \), \( (u_n) \) converge vers \( \ell \), bien que la vitesse de
convergence diminue à mesure que \( r \to 3^- \).

% ------

\textit{Cas \( r > 3 \) : Comportement chaotique.}

\q Lorsque \( |f'(\ell)| > 1 \), le point fixe \( \ell \) devient instable, ce qui ouvre la
possibilité de cycles de période 2, puis de comportements plus complexes (bifurcations, chaos). En
effet, la suite peut osciller entre plusieurs valeurs ou devenir apériodique, ce qui correspond à
des phénomènes de bifurcation ou de chaos déterministe.


\end{document}
% ==================================================================================================
