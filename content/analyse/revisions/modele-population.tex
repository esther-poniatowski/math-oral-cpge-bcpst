% PROBLÈME : Modèles de Population par des Suites Récurrentes
% ==================================================================================================
%
% But
% ---
% Étudier une suite récurrente classique :
% - analyse de la fonction associée
% - points fixes
% - stabilité
% - convergence
% - étude asymptotique Ouvrir vers les comportements plus complexes : bifurcations, chaos.
%   ==================================================================================================

\documentclass[10pt,a4paper]{article}

% Set the root path
\providecommand{\rootpath}{../../..}
% Fonts
\usepackage[utf8]{inputenc} % for accents
\usepackage[T1]{fontenc} % for accents
\usepackage[french]{babel} % for french language
\usepackage{helvet} % sans serif font family
\renewcommand*\familydefault{\sfdefault} % sans serif font family

% Mathematics
\usepackage{amsmath,amsfonts,amssymb} % for math symbols
\usepackage{array} % for tabular


\usepackage{parskip} % no indent, space between paragraphs

\usepackage{geometry} % margin
\geometry{
    a4paper,
    left=15mm,
    right=15mm,
    top=20mm,
    bottom=20mm
}

\usepackage{circledsteps} % to draw circles around numbers

\usepackage{fancyhdr} % for headers and footers

\usepackage{enumitem} % for customizing lists
\setlist[enumerate]{itemsep=1em} % space between items only in enumerate environment (not itemize)
\setlist[itemize]{label=--} % set itemize label to em-dash

% Command: \customPageLayout{#1}{#2}{#3}
% --------------------------------------
% Description: Custom page layout with header and footer content.
% Arguments:
% #1: Header and footer content
% #2: Left header content
% #3: Right header content
% Example:
% \customPageLayout{Title}{Lycée Henri IV}{2024}
% Required Packages: fancyhdr
\newcommand{\customPageLayout}[3]{
    \pagestyle{fancy} % set page style to fancy, i.e. header and footer
    \fancyhf{#1} % set header and footer content
    \lhead{#2} % set left header content
    \rhead{#3} % set right header content
    \fancyfoot{} % clear footer content
    \rfoot{\thepage} % set page number in footer
}

% Counter: \q
% -----------
% Description: Display a question number in a circle.
\newcounter{q}
\setcounter{q}{0} % set initial value of counter
\newcommand{\q}{
    \bigskip
    \addtocounter{q}{1}
    \par
    \Circled{\textbf{\theq}} \space
}


\title{Analyse - Modèles de Population par des Suites Récurrentes}
\author{Esther Poniatowski}
\date{2024-2025}

\customPageLayout{Sujets d'interrogation orale}{Lycée Henri IV}{2024}

% ==================================================================================================
\begin{document}

\textbf{Contexte}

Les suites récurrentes sont adaptées à  la modélisation de l'évolution démographique, lorsque le
nombre d'individus à une génération dépend de la génération précédente. L'un des modèles classiques
est la suite logistique, qui prend en compte une croissance proportionnelle à la population
présente, ainsi qu'une limitation due aux ressources finies.

\bigskip
\textbf{Objectifs}

Étudier le comportement de la suite logistique selon les valeurs d'un paramètre.

\bigskip
La suite logistique est définie par la relation :
\[
u_{n+1} = r u_n (1 - u_n), \quad u_0 \in [0,1], \quad r \in [0,4].
\]

Le paramètre \( r \) est appelé \emph{taux de croissance} et détermine le comportement de la suite.

% --------------------------------------------------------------------------------------------------
\vspace{0.5cm}
\textbf{Étude de la fonction logistique}

Soit la fonction \( f \) définie sur \( [0,1] \) par \( f(x) = r x (1 - x) \).

\q Justifier que pour \( r \in [0,4] \), \( f \) est une fonction continue de \( [0,1] \) dans \(
[0,1] \).

\q Étudier la dérivabilité et les variations de \( f \) sur \( [0,1] \).
% But : étude complète de la fonction associée pour préparer l'étude de la suite. Méthode : calcul,
% variations)

\q Déterminer les points fixes de \( f \), c'est-à-dire les solutions de \( f(x) = x \), en fonction
de \( r \).
% But : identification des points fixes, étude de leur dépendance en r. Méthode : résolution
% d'équation quadratique.

% --------------------------------------------------------------------------------------------------
\vspace{0.5cm}
\textbf{Étude qualitative de la suite}

\q Montrer que si \( u_0 \in [0,1] \), alors \( \forall n \in \mathbb{N},\ u_n \in [0,1] \).
% But : montrer que l'intervalle [0,1] est stable par f. Méthode : étude du signe et encadrement.

\q En supposant que \( (u_n) \) converge vers une limite \( \ell \), montrer que \( \ell \) est
nécessairement un point fixe de \( f \).
% But : utiliser le théorème de convergence des suites récurrentes. Méthode : continuité, étude des
% points fixes et comportement local.

\q Interpréter ce résultat graphiquement en considérant l'intersection de la courbe de \( f \) avec
la droite d'équation \( y = x \).
% But : Introduire la définition de point fixe de manière élémentaire. Méthode : Résolution d'une
% équation et interprétation graphique.

% --------------------------------------------------------------------------------------------------
\vspace{0.5cm}
\textbf{Stabilité des points fixes}

Les points fixes d'une fonction sont caractérisés par leur \textit{stabilité} : un point fixe \( c \) est dit
\textit{stable} (ou \textit{attractif}) si pour tout \( u_0 \) "proche" de \( c \), la suite \(
(u_n) \) définie par \( u_{n+1} = f(u_n) \) converge vers \( c \).

La stabilité peut être garantie par la propriété de \textit{contractivité} de la fonction :

Une fonction \( g \) est dite \textit{contractante} sur un intervalle \( I \) s'il existe une
constante \( k \in [0,1[ \) telle que :
\[
\forall x, y \in I, \quad |g(x) - g(y)| \leq k |x - y|
\]

\q Soit une fonction contractante \( g \) sur un intervalle \( I \), admettant un unique point fixe
\( c \) sur cet intervalle. Soit \( \epsilon_n = u_n - c \), l'écart par rapport au point fixe.
Étudier le comportement de cet écart, et en déduire la stabilité du point fixe \( c \).
% But : démontrer la convergence d'une suite contractante.

\q Illustrer graphiquement ce résultat.
% But : visualiser la propriété de contractivité dans un cône et l'escalier de la convergence.

\q Soit une fonction \( f \) de classe \( \mathcal{C}^1 \) sur un
intervalle \( I \), admettant un unique point fixe \( c \). Justifier que si \( |f'(c)| < 1 \),
alors il existe un intervalle sur lequel la fonction \( f \) est contractante. Conclure sur la
stabilité de \( c \).

\q En déduire, selon les valeurs de \( r \), quels sont les points fixes stables.


% --------------------------------------------------------------------------------------------------

\end{document}

% ==================================================================================================


VERSION ANTERIEURE

\vspace{0.5cm}
\textbf{Stabilité des points fixes}

Les points fixes sont caractérisés par leur \textit{stabilité} : un point fixe \( \ell \) est dit
\textit{stable} si pour tout \( u_0 \) "proche" de \( \ell \), la suite \( (u_n) \) converge vers \(
\ell \).

\q Soit une valeur initiale \( u_0 = \ell +  \epsilon_0  \) où \( \epsilon_0 \) représente l'écart
par rapport à \( \ell \).
Exprimer l'écart \( \epsilon_{n+1} = u_{n+1} - \ell \) en fonction de \( \epsilon_{n} \), en
utilisant une approximation linéaire.
% But : Traduire l'évolution des écarts par une relation de récurrence.
% Méthode : Développement de f(u_n) et f(ell) autour de l.

\q Condition de stabilité : En déduire que si \( |f'(\ell)| < 1 \), alors \( \epsilon_n
\to 0 \) et la suite \( (u_n) \) converge vers \( \ell \).
Interpréter ce résultat en expliquant pourquoi la suite reste proche de \( \ell \).
% But : Établir un critère rigoureux de stabilité basé sur |f'(l)|.
% Méthode : Étudier l'évolution de l'écart selon les valeurs de f'(l).

\q Condition d'instabilité : Montrer que si \( |f'(\ell)| > 1 \), alors la suite \( (u_n) \)
s'éloigne de \( \ell \).
% But : Montrer que la dérivée supérieure à 1 implique l'instabilité du point fixe.
% Méthode : Étudier le comportement de l'écart pour ∣f'(l)∣>1.

\q En déduire, selon les valeurs de \( r \), quels sont les points fixes stables.
% But : étude de la stabilité des points fixes selon r.
% Méthode : dérivée, condition de stabilité.


\vspace{0.5cm}
\textbf{Étude asymptotique en fonction du paramètre}

\textit{Cas \( r \in (0,1) \) : Extinction}
\q Montrer que \( 0 \) est l'unique point fixe, stable, et que pour tout \( u_0 \in (0,1) \), \(
(u_n) \) converge vers \( 0 \).
% But : cas r < 1 : extinction de la population. Méthode : étude des points fixes et de leur
% stabilité.

\textit{Cas \( r \in (1,2) \) : Stabilisation}

\q Montrer que \( 0 \) est instable, et que l'autre point fixe \( \ell = 1 - \frac{1}{r} \) est
stable. En déduire que \( (u_n) \) converge vers \( \ell \) pour tout \( u_0 \in (0,1) \setminus
\{0\} \).
% But : cas 1 < r < 2 : convergence vers un point fixe positif. Méthode : même approche.

\textit{Cas \( r \in (2,3) \) : Etude fine}

\q Montrer que \( \ell = 1 - \frac{1}{r} \) reste un point fixe, mais que la valeur de \( |f'(\ell)|
\) approche 1 lorsque \( r \to 3^- \).
% But : étude de la stabilité locale du point fixe positif. Méthode : calcul de la dérivée en l,
% critère de stabilité.

\ Proposer une conjecture sur la convergence de \( (u_n) \) pour ces valeurs de \( r \).

\textit{Cas \( r > 3 \) : Comportement chaotique}

Admis : Pour \( r > 3 \), la suite \( (u_n) \) peut ne plus converger vers un point fixe et adopter
un comportement périodique ou chaotique.

\q Expliquer pourquoi la perte de stabilité du point fixe \( \ell \) (lorsque \( |f'(\ell)| > 1 \))
rend ce phénomène possible.
% But : ouvrir vers la complexité pour r > 3. Méthode : interprétation graphique, lien avec la
% stabilité.
