% CORRECTION : Sujet 4 AgroVeto 2018
% Exercice donné par Guillaume Roux
% ==================================================================================================

\documentclass[10pt,a4paper]{article}

\usepackage{tikz}
\usepackage{pgfplots}
\pgfplotsset{compat=1.18}

% Set the root path
\providecommand{\rootpath}{../../..}
% Fonts
\usepackage[utf8]{inputenc} % for accents
\usepackage[T1]{fontenc} % for accents
\usepackage[french]{babel} % for french language
\usepackage{helvet} % sans serif font family
\renewcommand*\familydefault{\sfdefault} % sans serif font family

% Mathematics
\usepackage{amsmath,amsfonts,amssymb} % for math symbols
\usepackage{array} % for tabular


\usepackage{parskip} % no indent, space between paragraphs

\usepackage{geometry} % margin
\geometry{
    a4paper,
    left=15mm,
    right=15mm,
    top=20mm,
    bottom=20mm
}

\usepackage{circledsteps} % to draw circles around numbers

\usepackage{fancyhdr} % for headers and footers

\usepackage{enumitem} % for customizing lists
\setlist[enumerate]{itemsep=1em} % space between items only in enumerate environment (not itemize)
\setlist[itemize]{label=--} % set itemize label to em-dash

% Command: \customPageLayout{#1}{#2}{#3}
% --------------------------------------
% Description: Custom page layout with header and footer content.
% Arguments:
% #1: Header and footer content
% #2: Left header content
% #3: Right header content
% Example:
% \customPageLayout{Title}{Lycée Henri IV}{2024}
% Required Packages: fancyhdr
\newcommand{\customPageLayout}[3]{
    \pagestyle{fancy} % set page style to fancy (add header and footer)
    \fancyhf{} % clear all header and footer content
    \lhead{#2} % left header content
    \rhead{#3} % right header content
    \chead{\textbf{#1}} % center header content in bold (if needed)
    \rfoot{\thepage} % page number in the footer
}


% Counter: \q
% -----------
% Description: Display a question number in a circle.
% Usage:
% - Create a new question: add \q followed by the question content.
% - Reset the question counter: add \setcounter{q}{0} before the first question.
\newcounter{q}
\setcounter{q}{0} % set initial value of the counter
\newcommand{\q}{
    \bigskip
    \addtocounter{q}{1}
    \par
    \Circled{\textbf{\theq}} \space
}


% Counter: \ql
% ------------
% Description: Display a question letter in a round box with indentation (lowercase and not bold).
% Usage:
% - Create a new question: add \ql followed by the question content.
% - Reset the question counter: add \setcounter{ql}{0} before the first question.
\newcounter{ql}
\setcounter{ql}{0} % set initial value of the counter
\newcommand{\ql}{
    \addtocounter{ql}{1}
    \par
    \hspace{1.5em} % indentation before the circled letter
    \textcolor{gray}{\Circled{\alph{ql}}} \space % gray color
}


\title{Analyse - Fonctions and Suites}
\author{Esther Poniatowski}
\date{2024-2025}

\customPageLayout{Correction - Sujet 4 AgroVeto 2018}{Lycée Henri IV}{2024}

% ==================================================================================================
\begin{document}

Soit l'équation définie pour \( n \in \mathbb{N}^* \) et \( x \in \mathbb{R}^* \) par :
\[
   (E_n) : \frac{\ln(x)^2}{x} = \frac{1}{n}
\]

Soit \( f \) la fonction définie sur \( [1; +\infty[ \) par :
\[
   f(x) = \frac{\ln(x)^2}{x}
\]

% --------------------------------------------------------------------------------------------------
\bigskip
\textbf{1. (a) Tableau de variations de \( f \) sur son ensemble de définition}

Dérivée pour \( x \in [1; +\infty[ \) :

\[
f'(x) = \frac{x \cdot \frac{d}{dx}(\ln(x)^2) - \ln(x)^2 \cdot \frac{d}{dx}(x)}{x^2}
\]

\[
f'(x) = \frac{x \cdot 2\ln(x) \cdot \frac{1}{x} - \ln(x)^2 \cdot 1}{x^2}
\]

\[
f'(x) = \frac{2\ln(x) - \ln(x)^2}{x^2}
\]

\[
f'(x) = \frac{\ln(x)(2 - \ln(x))}{x^2}
\]

Pour \( x \geq 1 \), le facteur \( \ln(x) \geq 0 \), donc le signe dépend du facteur \( (2 - \ln(x))
\) :
\begin{itemize}
\item Si \( \ln(x) < 2 \), alors \( 2 - \ln(x) > 0 \), donc \( f'(x) > 0 \)
\item Si \( \ln(x) = 2 \), alors \( 2 - \ln(x) = 0 \), donc \( f'(x) = 0 \)
\item Si \( \ln(x) > 2 \), alors \( 2 - \ln(x) < 0 \), donc \( f'(x) < 0 \) \\
\end{itemize}

La condition \( \ln(x) = 2 \) équivaut à \( x = e^2 \).

Ainsi, \( f \) est croissante sur \( [1; e^2] \) et décroissante sur \( [e^2; +\infty[ \).

Valeurs remarquables de \( f \) :

\[
f(1) = \frac{\ln(1)^2}{1} = \frac{0^2}{1} = 0
\]

\[
f(e^2) = \frac{\ln(e^2)^2}{e^2} = \frac{2^2}{e^2} = \frac{4}{e^2}
\]

\[
\lim_{x \to +\infty} f(x) = \lim_{x \to +\infty} \frac{\ln(x)^2}{x} = 0
\]

Tableau de variations de \( f \) :

\[
\begin{array}{c|ccccc}
x & 1 &  & e^2 &  & +\infty \\
\hline
f'(x) &  & + & 0 & - &  \\
\hline
f(x) & 0 & \nearrow & \frac{4}{e^2} & \searrow & 0
\end{array}
\]

% --------------------------------------------------------------------------------------------------
\bigskip
\textbf{1. (b) Absence de solution pour l'équation \( (E_1) \)}

L'équation \( (E_1) \) s'écrit :
\[
\frac{\ln(x)^2}{x} = 1
\]

D'après le tableau de variations, \( f \) atteint son maximum en \( x = e^2 \) avec la valeur \( f(e^2) =
\frac{4}{e^2} \).

Cette valeur maximale est inférieure à 1. En effet:
\[
e > 2.7 > 2 \implies e^2 > 2^2 = 4 \implies \frac{4}{e^2} < 1
\]


Ainsi, pour tout \( x \in [1; +\infty[ \) :
\[
f(x) < 1
\]

Conclusion : L'équation \( (E_1) \) n'admet pas de solution.

% --------------------------------------------------------------------------------------------------
\bigskip
\textbf{1. (c) Solutions de l'équation \( (E_n) \) pour \( n \geq 2 \)}

L'équation \( (E_n) \) s'écrit :
\[
\frac{\ln(x)^2}{x} = \frac{1}{n}
\]

\medskip
Dans l'intervalle \( [1; e^2] \):
\begin{itemize}
\item \( f(1) = 0 < \frac{1}{n} \) pour tout \( n \geq 1 \).
\item \( f(e^2) = \frac{4}{e^2} \geq \frac{1}{n} \), car
      \( \frac{1}{n} \geq \frac{1}{2} \; (\forall n \geq 2) \),
      et \( e < 2.5 = \frac{5}{2} \implies e^2 < \frac{25}{4} \implies \frac{4}{e^2} > \frac{16}{25}
      > \frac{16}{32} > \frac{1}{2} \).
\item \( f \) est strictement croissante et continue.
\end{itemize}

Par le théorème des valeurs intermédiaires, il existe un unique \( \alpha_n \in ]1; e^2[ \) tel que \( f(\alpha_n) =
\frac{1}{n} \).

\medskip
Dans l'intervalle \( [e^2; +\infty[ \):
\begin{itemize}
\item \( \lim_{x \to +\infty} f(x) = 0 < \frac{1}{n} \)
\item \( \frac{1}{n} \leq \frac{1}{2} < f(e^2) \) (comme précédemment)
\item \( f \) est strictement décroissante et continue.
\end{itemize}

Par le théorème des valeurs intermédiaires, il existe un unique \( \beta_n \in ]e^2; +\infty[ \) tel que \( f(\beta_n) =
\frac{1}{n} \).

\medskip
Conclusion : Pour \( n \geq 2 \), l'équation \( (E_n) \) admet exactement deux solutions \( \alpha_n
\) et \( \beta_n \) vérifiant :
\[
1 \leq \alpha_n \leq e^2 \leq \beta_n
\]

% --------------------------------------------------------------------------------------------------
\bigskip
\textbf{2. Représentation graphique}

La courbe de \( f \) présente les caractéristiques suivantes :
\begin{itemize}
\item Elle passe par l'origine \( (1, 0) \) car \( f(1) = 0 \)
\item Elle atteint son maximum en \( x = e^2 \) avec la valeur \( f(e^2) = \frac{4}{e^2} \approx
0.54 \)
\item Elle est croissante sur \( [1; e^2] \) et décroissante sur \( [e^2; +\infty[ \)
\item Elle tend vers \( 0 \) quand \( x \) tend vers \( +\infty \)
\end{itemize}

\medskip
Les droites horizontales \( D_i : y = \frac{1}{i} \) coupent la courbe en deux points exactement
lorsque \( i \geq 2 \), ces points d'intersection correspondant aux solutions \( \alpha_i \) et \(
\beta_i \) de l'équation \( (E_i) \).


\begin{figure}[htbp]
\centering
\begin{tikzpicture}
\begin{axis}[
    width=12cm, height=8cm,
    xlabel={$x$},
    ylabel={$y$},
    axis lines=middle,
    xmin=0, xmax=25,
    ymin=0, ymax=0.6,
    xtick={1, 7.389, 10, 15, 20},
    xticklabels={$1$, $e^2$, $10$, $15$, $20$},
    ytick={0, 0.25, 0.3333, 0.5, 0.5413},
    yticklabels={$0$, $\frac{1}{4}$, $\frac{1}{3}$, $\frac{1}{2}$, $\frac{4}{e^2}$},
    domain=1:25,
    samples=300,
    legend pos=north east,
    grid=both,
]

% Fonction f(x) = ln(x)²/x
\addplot[blue, thick] {ln(x)^2/x};
\addlegendentry{$f(x) = \frac{\ln(x)^2}{x}$}

% Droites horizontales Di: y = 1/i
\addplot[red, dashed] {0.5};
\addlegendentry{$D_2: y = \frac{1}{2}$}

\addplot[orange, dashed] {0.3333};
\addlegendentry{$D_3: y = \frac{1}{3}$}

\addplot[green, dashed] {0.25};
\addlegendentry{$D_4: y = \frac{1}{4}$}

\addplot[purple, dashed] {0.2};
\addlegendentry{$D_5: y = \frac{1}{5}$}

% Ligne verticale à x = e^2
\draw[red, dotted] (7.389,0) -- (7.389,0.54);

% Points spéciaux
\node[circle, fill=blue, inner sep=1.5pt] at (1,0) {};
\node[circle, fill=blue, inner sep=1.5pt] at (7.389,0.54) {};

\end{axis}
\end{tikzpicture}
\caption{Représentation de $f(x) = \frac{\ln(x)^2}{x}$ et des droites $D_i: y = \frac{1}{i}$ pour $2 \leq i \leq 6$}
\end{figure}

% --------------------------------------------------------------------------------------------------
\bigskip
\textbf{3. Conjectures sur les suites \( (\alpha_n) \) et \( (\beta_n) \) }

\begin{itemize}
\item La suite \( (\alpha_n) \) semble être décroissante et converger vers \( 1 \) quand \( n \)
tend vers \( +\infty \).
\item La suite \( (\beta_n) \) semble être croissante et tendre vers \( +\infty \) quand \( n \)
tend vers \( +\infty \).
\end{itemize}

\medskip
Justification graphique : Lorsque \( n \) augmente,
\begin{itemize}
\item Les droites \( D_n : y = \frac{1}{n} \) se rapprochent de l'axe des abscisses.
\item Pour la suite \( (\alpha_n) \), les points d'intersection parcourent la branche de \( f \) du
point maximum de \( (e^2, 4/e^2) \) vers le point \( (1, 0) \), car la fonction \( f \) croît sur \(
[1; e^2] \). Les abscisses des points d'intersection se rapprochent de \( 1 \).
\item Pour \( (\beta_n) \), les points d'intersection parcourent la branche de \( f \) du point
maximum de \( (e^2, 4/e^2) \) vers le point "limite" \( (+\infty, 0) \), car la fonction \( f \)
décroît sur \( [e^2; +\infty[ \). Les abscisses des points d'intersection s'éloignent de \( e^2 \)
et tendent vers \( +\infty \).
\end{itemize}

% --------------------------------------------------------------------------------------------------
\bigskip
\textbf{4. Etude de la suite \( (\beta_n) \)}

(a) \textbf{Monotonie de la suite \((\beta_n)\)}

Montrons que la suite \((\beta_n)\) est strictement croissante, c'est-à-dire que pour tout \(n \geq
2\), \(\beta_{n+1} > \beta_n\).

Soit \(n \geq 2\). Comparons les équations \((E_n)\) et \((E_{n+1})\) :
\begin{align}
f(\beta_n) &= \frac{1}{n} \\
f(\beta_{n+1}) & = \frac{1}{n+1}
\end{align}

Or, la fonction \( f \) est strictement décroissante sur \([e^2; +\infty[\), donc :

\[
\frac{1}{n+1} < \frac{1}{n} \implies f(\beta_{n+1}) > f(\beta_n)
\]

Conclusion : La suite \((\beta_n)\) est strictement croissante.

% --------------------------------------------------------------------------------------------------
\bigskip
(b) \textbf{Limite de la suite \((\beta_n)\)}

Montrons que \(\lim_{n \to +\infty} \beta_n = +\infty\).

Supposons par l'absurde que \((\beta_n)\) admet une limite finie \(L \in \mathbb{R}, \).

En passant à la limite quand \(n \to +\infty\) et par continuité de \(f\):
\[
\lim_{n \to +\infty} f(\beta_n) = f(L) = \frac{\ln(L)^2}{L} = \lim_{n \to +\infty} \frac{1}{n} = 0
\]

Ce qui impliquerait \(f(L) = \frac{\ln(L)^2}{L} = 0\). Or, pour tout \(x > 1\), \(\ln(x) > 0\), donc
\(f(x) > 0\), ce qui aboutit à une contradiction.

Par conséquent, \(\lim_{n \to +\infty} \beta_n = +\infty\).

% --------------------------------------------------------------------------------------------------
\bigskip
(c) \textbf{Étude de la suite \((u_n)\) définie par \(u_n = \frac{\beta_n}{n}\)}

Montrons que \(u_n \sim \ln^2(n)\) quand \(n \to +\infty\).

Par définition de \(\beta_n\) :
\[
\frac{\ln(\beta_n)^2}{\beta_n} = \frac{1}{n} \implies
u_n := \frac{\beta_n}{n} = \ln(\beta_n)^2
\]

Comme \(\beta_n = n u_n\) :
\begin{align*}
\ln(\beta_n) &= \ln(n u_n)\\
&= \ln(n) + \ln(u_n) \\
&= \ln(n) + o(\ln(n)) \quad \text{(car \(\ln(u_n) = o(\ln(n))\) par hypothèse de l'énoncé)}\\
&\sim \ln(n)
\end{align*}

Or, \(u_n = \ln(\beta_n)^2\), donc quand \(n \to +\infty\) :
\[
u_n \sim \ln(n)^2
\]

% --------------------------------------------------------------------------------------------------
\bigskip
(d) \textbf{Équivalent de \(\beta_n\) quand \(n \to +\infty\)}

Par la question précédente : \(u_n = \frac{\beta_n}{n} \sim \ln^2(n)\).

Ce qui équivaut directement à :
\[
\beta_n \sim n \ln^2(n)
\]

Ce qui confirme la croissance rapide de \(\beta_n\) et justifie la conjecture formulée précédemment.


\end{document}
% ==================================================================================================
