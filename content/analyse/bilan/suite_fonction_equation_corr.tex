% CORRECTION : Fonctions et Suites - DM 3 9 février
% Exercice donné par Guillaume Roux
% ==================================================================================================

\documentclass[10pt,a4paper]{article}

% Set the root path
\providecommand{\rootpath}{../../..}
% Fonts
\usepackage[utf8]{inputenc} % for accents
\usepackage[T1]{fontenc} % for accents
\usepackage[french]{babel} % for french language
\usepackage{helvet} % sans serif font family
\renewcommand*\familydefault{\sfdefault} % sans serif font family

% Mathematics
\usepackage{amsmath,amsfonts,amssymb} % for math symbols
\usepackage{array} % for tabular


\usepackage{parskip} % no indent, space between paragraphs

\usepackage{geometry} % margin
\geometry{
    a4paper,
    left=15mm,
    right=15mm,
    top=20mm,
    bottom=20mm
}

\usepackage{circledsteps} % to draw circles around numbers

\usepackage{fancyhdr} % for headers and footers

\usepackage{enumitem} % for customizing lists
\setlist[enumerate]{itemsep=1em} % space between items only in enumerate environment (not itemize)
\setlist[itemize]{label=--} % set itemize label to em-dash

% Command: \customPageLayout{#1}{#2}{#3}
% --------------------------------------
% Description: Custom page layout with header and footer content.
% Arguments:
% #1: Header and footer content
% #2: Left header content
% #3: Right header content
% Example:
% \customPageLayout{Title}{Lycée Henri IV}{2024}
% Required Packages: fancyhdr
\newcommand{\customPageLayout}[3]{
    \pagestyle{fancy} % set page style to fancy, i.e. header and footer
    \fancyhf{#1} % set header and footer content
    \lhead{#2} % set left header content
    \rhead{#3} % set right header content
    \fancyfoot{} % clear footer content
    \rfoot{\thepage} % set page number in footer
}

% Counter: \q
% -----------
% Description: Display a question number in a circle.
\newcounter{q}
\setcounter{q}{0} % set initial value of counter
\newcommand{\q}{
    \bigskip
    \addtocounter{q}{1}
    \par
    \Circled{\textbf{\theq}} \space
}


\title{Analyse - Fonctions and Suites}
\author{Esther Poniatowski}
\date{2024-2025}

\customPageLayout{Correction - DM3 9 février}{Lycée Henri IV}{2024}

% ==================================================================================================
\begin{document}

Soit la fonction \( f_n \) définie pour tout \( x > 0 \) et \( n \geq 2 \) par :

\[
    f_n(x) = x^n - \ln(x) - n
\]

\textbf{1. Tableau de variations de \( f_n \)}

Dérivée pour \( x > 0 \) :

\[
    f_n'(x) = n x^{n-1} - \frac{1}{x}
\]

Solutions de \( f_n'(x) = 0 \) :

\[
    n x^{n} = 1 \implies x = \left(\frac{1}{n}\right)^{\frac{1}{n}}
\]

Signes de la dérivée et variations de la fonction :

\begin{itemize}
    \item Pour \( x < \left(\frac{1}{n}\right)^{\frac{1}{n}} \), \( f_n'(x) < 0 \),
          donc \( f_n \) décroît sur \( ]0, \left(\frac{1}{n}\right)^{\frac{1}{n}}[ \).
    \item Pour \( x > \left(\frac{1}{n}\right)^{\frac{1}{n}} \), \( f_n'(x) > 0 \),
          donc \( f_n \) croît sur \( ]\left(\frac{1}{n}\right)^{\frac{1}{n}}, +\infty[ \).
\end{itemize}

La fonction admet un minimum en \( \left(\frac{1}{n}\right)^{\frac{1}{n}} \), qui vaut :

\[
    f_n\left(\left(\frac{1}{n}\right)^{\frac{1}{n}}\right) = \left(\left(\frac{1}{n}\right)^{\frac{1}{n}}\right)^{n} - \ln\left(\left(\frac{1}{n}\right)^{\frac{1}{n}}\right) - n
\]
\[
    = \frac{1}{n} - \frac{1}{n} \ln\left(\frac{1}{n}\right) - n
\]
\[
    = \frac{1}{n} + \frac{1}{n} \ln(n) - n
\]

\[
\begin{array}{c|ccccc}
x & 0 &  & \left(\frac{1}{n}\right)^{\frac{1}{n}} &  & +\infty \\
\hline
f_n'(x) &  & - & 0 & + &  \\
\hline
f_n(x) & -\infty & \searrow & \frac{1}{n} + \frac{1}{n} \ln(n) - n & \nearrow & +\infty
\end{array}
\]

% --------------------------------------------------------------------------------------------------
\bigskip
\textbf{2. Preuve de l'inégalité \( \ln(x) \leq x - 1 \) pour tout \( x > 0 \)}

Soit la fonction \( g(x) = x - 1 - \ln(x) \). Sa dérivée pour \( x > 0 \) est :

\[
    g'(x) = 1 - \frac{1}{x}
\]

\begin{itemize}
    \item Si \( x > 1 \), alors \( g'(x) > 0 \), dont \( g \) croît.
    \item Si \( 0 < x < 1 \), alors \( g'(x) < 0 \), donc \( g \) décroît.
\end{itemize}

La fonction \( g \) admet un minimum en \( x = 1 \), où \( g(1) = 0 \), ainsi :

\[
    \ln(x) \leq x - 1, \quad \forall x > 0
\]

% --------------------------------------------------------------------------------------------------
\bigskip
\textbf{3. Existence and Unicité de \( u_n \) et \( v_n \)}

Soient les solutions de l'équation \( f_n(x) = 0 \), si elles existent. La fonction \( f_n \) étant
strictement décroissante puis croissante, elle possède exactement deux racines si et seulement si
son minimum est strictement négatif (théorème des valeurs intermédiaires).

Le minimum est majoré par (en utilisant l'inégalité de la question 2) :
\[
    \frac{1+\ln(n)}{n} - n < \frac{1 + (n-1)}{n}  - n = 1 - n
\]

Cette différence est négative pour \( n > 1 \).

Ainsi, pour \( n \geq 2 \), le minimum est négatif, donc \( f_n \) admet deux racines distinctes :
\begin{itemize}
    \item \( u_n \) dans \( ]0, \left(\frac{1}{n}\right)^{\frac{1}{n}}[ \)
    \item \( v_n \) dans \( ]\left(\frac{1}{n}\right)^{\frac{1}{n}}, +\infty[ \) \\
\end{itemize}

L'unicité est assurée par la stricte monotonicité de \( f_n \) sur chaque intervalle.

\bigskip
\textbf{4. Étude de la suite \( v_n \)}

\textbf{4.a. Borne de \( v_n \)}

Montrons que \( f_n \left((2n)^{\frac{1}{n}}\right) > 0 \) pour tout \( n \geq 2 \).

L'expression de la fonction en ce point s'écrit :

\[
    f_n \left((2n)^{\frac{1}{n}}\right) = \left((2n)^{\frac{1}{n}}\right)^n - \ln \left((2n)^{\frac{1}{n}}\right) - n
\]
\[
    = 2n - \frac{\ln(2n)}{n} - n
\]
\[
    = n - \frac{\ln(2n)}{n}
\]
\[
    = \frac{n^2 - \ln(2n)}{n}
\]

Ce quotient est positif si et seulement si le numérateur est positif :
\[
    n^2 - \ln(2n) > 0
\]
Or, \( \ln(2n) < (2n - 1)\) pour \( n \geq 2 \) (inégalité de la question 2), donc :
\[
    n^2 - \ln(2n) > n^2 - (2n - 1) = n^2 - 2n + 1 = (n-1)^2 > 0
\]

Ainsi, pour tout \( n \geq 2 \) :
\[
\begin{cases}
    f_n \left((2n)^{\frac{1}{n}}\right) > 0 \\
    f_n(v_n) = 0
\end{cases}
\implies f_n(v_n) < f_n \left((2n)^{\frac{1}{n}}\right)
\]

Par stricte croissance de \( f_n \) sur \( ]\left(\frac{1}{n}\right)^{\frac{1}{n}}, +\infty[ \) :
\[
    v_n < (2n)^{\frac{1}{n}}
\]


% --------------------------------------------------------------------------------------------------
\bigskip
\textbf{4.b. Convergence et limite de \( v_n \)}

Montrons que la suite \( v_n \) converge vers une limite \( \ell \), et déterminons cette limite.

La suite \( v_n \) est encadrée par :

\[
    \left(\frac{1}{n}\right)^{\frac{1}{n}} < v_n < (2n)^{\frac{1}{n}}
\]

Ces bornes convergent vers 1 lorsque \( n \) tend vers \( +\infty \) :

\[
\begin{cases}
\left(\frac{1}{n}\right)^{\frac{1}{n}} = e^{-\frac{\ln(n)}{n}} \underset{n \to +\infty}{\longrightarrow} 1 \\
(2n)^{\frac{1}{n}} = e^{\frac{\ln(2n)}{n}} \underset{n \to +\infty}{\longrightarrow} 1
\end{cases}
\]

Par le théorème des gendarmes :

\[
    v_n \underset{n \to +\infty}{\longrightarrow} 1
\]


% --------------------------------------------------------------------------------------------------
\bigskip
\textbf{4.c. Démonstration de l'équivalence logarithmique}

Par définition, \( a_n \sim b_n \) signifie que :

\[
    a_n = b_n (1 + \varepsilon_n), \quad \varepsilon_n \underset{n \to +\infty}{\longrightarrow} 0
\]

et \( \ln(a_n) \sim \ln(b_n) \) signifie que :
\[
    \ln(a_n) = \ln(b_n) + \delta_n, \quad \delta_n \underset{n \to +\infty}{\longrightarrow} 0
\]

\medskip
En replaçant l'expression de \( a_n \) dans le logarithme :

\[
    \ln(a_n) = \ln \left[b_n (1 + \varepsilon_n)\right]
             = \ln(b_n) + \ln(1 + \varepsilon_n)
\]

Or, le développement asymptotique du logarithme pour \( \varepsilon_n \to 0 \) donne :

\[
    \ln(1 + \varepsilon_n) \sim \varepsilon_n
\]

Ainsi, le logarithme de \( a_n \) est équivalent à :

\[
    \ln(a_n) \sim \ln(b_n) + \varepsilon_n
\]

Or, par hypothèse :
\[
\begin{cases}
    \varepsilon_n \underset{n \to +\infty}{\longrightarrow} 0 \\
    \lim b_n = +\infty
\end{cases}
\]

Donc le terme \( \varepsilon_n \) devient négligeable devant \( \ln(b_n) \). Ainsi :

\[
    \ln(a_n) \sim \ln(b_n)
\]

Conclusion :
Si \( a_n \sim b_n \) et \( \lim b_n = +\infty \), alors \( \ln(a_n) \sim \ln(b_n) \).

% --------------------------------------------------------------------------------------------------
\bigskip
\textbf{4.d. Equivalence asymptotique de \( v_n - \ell \)}

% Méthode: Etudier quantitativement la différence entre une suite et sa limite.
% 1. Utiliser la définition même de v_n comme racine de l'équation f_n(x) = 0, qui définit
%    implicitement la valeur de v_n. C'est le point de départ naturel puisque c'est la seule
%    relation dont on dispose. L'idée est d'exploiter cette équation pour exprimer le comportement
%    de v_n​ autour de sa limite. Le choix de cette méthode repose sur le fait que l'équation est
%    exacte, donc toute approximation en découlera de manière rigoureuse.
% 2. Identifier le comportement dominant.
% 3. Linéariser les termes pour faire apparaître la différence entre la suite et sa limite. Ici,
%    cela revient à composer par le logarithme (pour linéariser le terme exponentiel) puis à
%    utiliser un développement limité du logarithme. Ainsi, tous les termes en v_n​ sont linéarisés,
%    ce qui fait apparaître la différence avec sa limite.

Avec \( \ell = 1 \), le but est de déterminer un équivalent asymptotique de \( v_n - 1 \).

Par l'équation \( f_n(v_n) = 0 \) :

\[
v_n^n - \ln(v_n) - n = 0 \quad \implies \quad v_n^n = n + \ln(v_n) \sim n
\]

Par l'équivalence logarithmique (question 3.c) avec \( a_n = v_n^n \sim b_n = n \) :

\[
\ln(v_n^n) \sim \ln(n)
\]

\[
n \ln(v_n) \sim \ln(n)
\]

Comme \( \ln(v_n) \sim v_n - 1 \) (développement limité en \( v_n \to 1 \)), il vient :

\[
n (v_n - 1) \sim \ln(n) \quad \implies \quad v_n - 1 \sim \frac{\ln(n)}{n}
\]

Conclusion :

\[
\boxed{v_n - 1 \sim \frac{\ln(n)}{n}}
\]


% --------------------------------------------------------------------------------------------------
\bigskip
\textbf{4. Étude de la suite \( u_n \)}

% --------------------------------------------------------------------------------------------------
\bigskip
\textit{a. Variations de \( u_n \)}

Etudions le signe de \( f_n(u_{n+1}) \) :

???

On en déduit les variations de \( u_n \) :

???

% --------------------------------------------------------------------------------------------------
\bigskip
\textit{b. Limite de \( u_n^{n} \)}

% --------------------------------------------------------------------------------------------------
\bigskip
\textit{c. Équivalence asymptotique de \( u_n \)}

Montrons que :

\[
    u_n \sim e^{-n}
\]

\end{document}
% ==================================================================================================
