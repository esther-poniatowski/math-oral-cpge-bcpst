% PROBLÈME : Barycentre
% ==================================================================================================
%
% But
% ---
% Étudier le barycentre d'un système de points pondérés, en exploitant ses propriétés analytiques et
% géométriques:
% - Relations avec les projections orthogonales
% - Relations avec le produit scalaire et l'orthogonalité des droites et plans.
% Etude dans R^2 puis extension dans R^3.
%
% Objectifs spécifiques
% ---------------------
% -
% ==================================================================================================

\documentclass[10pt,a4paper]{article}

% Set the root path
\providecommand{\rootpath}{../..}
% Fonts
\usepackage[utf8]{inputenc} % for accents
\usepackage[T1]{fontenc} % for accents
\usepackage[french]{babel} % for french language
\usepackage{helvet} % sans serif font family
\renewcommand*\familydefault{\sfdefault} % sans serif font family

% Mathematics
\usepackage{amsmath,amsfonts,amssymb} % for math symbols
\usepackage{array} % for tabular


\usepackage{parskip} % no indent, space between paragraphs

\usepackage{geometry} % margin
\geometry{
    a4paper,
    left=15mm,
    right=15mm,
    top=20mm,
    bottom=20mm
}

\usepackage{circledsteps} % to draw circles around numbers

\usepackage{fancyhdr} % for headers and footers

\usepackage{enumitem} % for customizing lists
\setlist[enumerate]{itemsep=1em} % space between items only in enumerate environment (not itemize)
\setlist[itemize]{label=--} % set itemize label to em-dash

% Command: \customPageLayout{#1}{#2}{#3}
% --------------------------------------
% Description: Custom page layout with header and footer content.
% Arguments:
% #1: Header and footer content
% #2: Left header content
% #3: Right header content
% Example:
% \customPageLayout{Title}{Lycée Henri IV}{2024}
% Required Packages: fancyhdr
\newcommand{\customPageLayout}[3]{
    \pagestyle{fancy} % set page style to fancy, i.e. header and footer
    \fancyhf{#1} % set header and footer content
    \lhead{#2} % set left header content
    \rhead{#3} % set right header content
    \fancyfoot{} % clear footer content
    \rfoot{\thepage} % set page number in footer
}

% Counter: \q
% -----------
% Description: Display a question number in a circle.
\newcounter{q}
\setcounter{q}{0} % set initial value of counter
\newcommand{\q}{
    \bigskip
    \addtocounter{q}{1}
    \par
    \Circled{\textbf{\theq}} \space
}


\title{Géométrie - Droites et Plans - Barycentre}
\author{Esther Poniatowski}
\date{2024-2025}

\customPageLayout{Sujets d'interrogation orale}{Lycée Henri IV}{2024}

\begin{document}

\textbf{Objectifs}

Étudier les propriétés analytiques et géométriques du \textbf{barycentre} d'un système.

Soient trois points distincts $A(x_1, y_1)$, $B(x_2, y_2)$, $C(x_3, y_3)$ dans le plan
$\mathbb{R}^2$ et trois réels positifs $m_1, m_2, m_3$ représentant des coefficients de pondération
(ou masses).

Le barycentre $G$ de ce système est défini par :
$$
G \left( \frac{m_1 x_1 + m_2 x_2 + m_3 x_3}{m_1 + m_2 + m_3},
         \frac{m_1 y_1 + m_2 y_2 + m_3 y_3}{m_1 + m_2 + m_3} \right)
$$


\section{Barycentre et Centre de Gravité d'un Triangle}


Le \textbf{centre de gravité} d'un triangle est défini comme l'unique point d'intersection des trois
médianes du triangle.
Une \textbf{médiane} d'un triangle est une droite passant par un sommet et le milieu du côté
opposé. On note $M_A, M_B, M_C$ les milieux respectifs des côtés $[BC], [AC], [AB]$.

% QUESTION 1
% ----------
% Compétences mobilisées :
% - Manipuler les droites et les plans dans R^2 (point de vue cartésien).
\q Établir que les trois médianes sont concourantes en un unique point, et déterminer ses
coordonnées en fonction des coordonnées des points $A, B, C$.

% QUESTION
% ----------
% Compétences mobilisées :
% - Manipuler les droites et les plans dans R^2 (point de vue cartésien).
\q En déduire le lien entre le centre de gravité d'un triangle et le barycentre de ses sommets.

% QUESTION
% ----------
% Compétences mobilisées :
% - Manipuler les droites dans R^2 sous plusieurs points de vue.
\q Démontrer que dans ce cas, $G$ est situé aux deux tiers de chaque médiane en partant du sommet.


\section{Barycentre et Projeté sur une droite}

On considère une droite $D$ d'équation cartésienne :
$$ax + by + c = 0$$

% QUESTION 4
% ----------
% Compétences mobilisées :
% - Définition du projeté orthogonal d'un point sur une droite.
% - Définition de la distance d'un point à une droite comme étant égale à la distance entre ce point
%   et son projeté orthogonal.
\q Déterminer analytiquement les coordonnées du projeté orthogonal $G'$ de $G$ sur $D$ en fonction
de $a, b, c, x_G, y_G$.

% QUESTION 5
% ----------
% Compétences mobilisées :
% - Caractérisation de la distance d'un point à une droite comme la plus petite distance séparant ce
%   point d'un point de la droite.
\q Notons $A', B', C'$ les projetés orthogonaux respectifs de $A, B, C$ sur $D$. Montrer que le
projeté orthogonal $G'$ du barycentre $G$ sur $D$ est aussi le barycentre des projections
orthogonales $A', B', C'$ des trois points pondérés avec les mêmes coefficients.


\end{document}
% ==================================================================================================
\section{Barycentre et Produit Scalaire}

% QUESTION 6
% ----------
% Compétences mobilisées :
% - Bilinéarité du produit scalaire.
\q Lien entre le barycentre et les relations vectorielles
\setcounter{ql}{0}

\ql Montrer que le barycentre vérifie la relation :
   $$
   m_1 \overrightarrow{GA} + m_2 \overrightarrow{GB} + m_3 \overrightarrow{GC} = \overrightarrow{0}
   $$

\ql Interpréter cette relation en termes de combinaison linéaire des vecteurs position.

% QUESTION 7
% ----------
% Compétences mobilisées :
% - Connaître les différentes expressions du produit scalaire.
% - Utilisation de la bilinéarité du produit scalaire.
\q Utilisation du produit scalaire pour démontrer une propriété fondamentale
\setcounter{ql}{0}

\ql Montrer que pour tout point $M(x, y)$ du plan, la somme pondérée des carrés des distances est
   minimisée au point $G$, c'est-à-dire que :
   $$
   f(M) = m_1 ||\overrightarrow{MA}||^2 + m_2 ||\overrightarrow{MB}||^2 + m_3 ||\overrightarrow{MC}||^2
   $$
   est minimale lorsque $M = G$.

\ql Déduire une interprétation en mécanique (centre de masse d'un système de points pondérés).

% QUESTION 8
% ----------
% Compétences mobilisées :
% - Définition du projeté orthogonal d'un point sur une droite.
\q Lien entre barycentre et projections orthogonales
\setcounter{ql}{0}

\ql Démontrer que $G'$, projeté orthogonal de $G$ sur une droite $D$, est le barycentre des
   projections orthogonales de $A, B, C $ sur $D$, avec les mêmes coefficients de pondération.

\ql Interpréter ce résultat géométriquement et analytiquement.


\section{Extension dans $\mathbb{R}^3$}

On considère trois points $A(x_1, y_1, z_1)$, $B(x_2, y_2, z_2)$, $C(x_3, y_3, z_3)$
dans l'espace, ainsi qu'un plan $P$ d'équation :
$$Ax + By + Cz + D = 0$$

% QUESTION 9
% ----------
% Compétences mobilisées :
% - Caractériser l'orthogonalité entre droites et plans.
\q Barycentre et distance à un plan
\setcounter{ql}{0}

\ql Montrer que le projeté orthogonal $G'$ du barycentre $G $ sur $P$ est le barycentre des
  projections orthogonales des points $A, B, C$ sur $P $.

\ql Déterminer analytiquement les coordonnées du projeté orthogonal $G'$ de $G$ sur $P$.
