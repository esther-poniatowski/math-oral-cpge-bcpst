% CORRECTION : Étude d'un tétraèdre régulier
% ==================================================================================================

\documentclass[10pt,a4paper]{article}

% Set the root path
\providecommand{\rootpath}{../..}
% Fonts
\usepackage[utf8]{inputenc} % for accents
\usepackage[T1]{fontenc} % for accents
\usepackage[french]{babel} % for french language
\usepackage{helvet} % sans serif font family
\renewcommand*\familydefault{\sfdefault} % sans serif font family

% Mathematics
\usepackage{amsmath,amsfonts,amssymb} % for math symbols
\usepackage{array} % for tabular


\usepackage{parskip} % no indent, space between paragraphs

\usepackage{geometry} % margin
\geometry{
    a4paper,
    left=15mm,
    right=15mm,
    top=20mm,
    bottom=20mm
}

\usepackage{circledsteps} % to draw circles around numbers

\usepackage{fancyhdr} % for headers and footers

\usepackage{enumitem} % for customizing lists
\setlist[enumerate]{itemsep=1em} % space between items only in enumerate environment (not itemize)
\setlist[itemize]{label=--} % set itemize label to em-dash

% Command: \customPageLayout{#1}{#2}{#3}
% --------------------------------------
% Description: Custom page layout with header and footer content.
% Arguments:
% #1: Header and footer content
% #2: Left header content
% #3: Right header content
% Example:
% \customPageLayout{Title}{Lycée Henri IV}{2024}
% Required Packages: fancyhdr
\newcommand{\customPageLayout}[3]{
    \pagestyle{fancy} % set page style to fancy (add header and footer)
    \fancyhf{} % clear all header and footer content
    \lhead{#2} % left header content
    \rhead{#3} % right header content
    \chead{\textbf{#1}} % center header content in bold (if needed)
    \rfoot{\thepage} % page number in the footer
}


% Counter: \q
% -----------
% Description: Display a question number in a circle.
% Usage:
% - Create a new question: add \q followed by the question content.
% - Reset the question counter: add \setcounter{q}{0} before the first question.
\newcounter{q}
\setcounter{q}{0} % set initial value of the counter
\newcommand{\q}{
    \bigskip
    \addtocounter{q}{1}
    \par
    \Circled{\textbf{\theq}} \space
}


% Counter: \ql
% ------------
% Description: Display a question letter in a round box with indentation (lowercase and not bold).
% Usage:
% - Create a new question: add \ql followed by the question content.
% - Reset the question counter: add \setcounter{ql}{0} before the first question.
\newcounter{ql}
\setcounter{ql}{0} % set initial value of the counter
\newcommand{\ql}{
    \addtocounter{ql}{1}
    \par
    \hspace{1.5em} % indentation before the circled letter
    \textcolor{gray}{\Circled{\alph{ql}}} \space % gray color
}



\title{Géométrie - Droites et Plans - Étude d'un tétraèdre régulier}
\author{}
\date{2024}

\customPageLayout{Correction}{Lycée Henri IV}{2024}

\begin{document}

\section{Coordonnées et distances dans le tétraèdre}

\q \textbf{Coordonnées des sommets du tétraèdre}

\[
A(1,0,0), \quad B(0,1,0), \quad C(0,0,1),  \quad D(1,1,1)
\]


\q \textbf{Calcul de la longueur d'une arête}

$$
AB = \sqrt{(1-0)^2 + (0-1)^2 + (0-0)^2} = \sqrt{1^2 + (-1)^2 + 0^2} = \sqrt{2}
$$

Cette longueur est la même pour toutes les arêtes du tétraèdre.


\q \textbf{Équation cartésienne des plans des faces}

Voici le raisonnement formatté en LaTeX :

\textbf{Méthode générale :}
Pour chaque face, nous utilisons l'équation générale d'un plan $ax + by + cz + d = 0$, et déterminons les coefficients $a$, $b$, $c$, et $d$ en utilisant les coordonnées des trois points de chaque face.

\textbf{Face ABC :}
Points : $A(1,0,0)$, $B(0,1,0)$, $C(0,0,1)$

Substituons ces points dans l'équation générale :
\begin{align*}
A &: a(1) + b(0) + c(0) + d = 0 \implies a + d = 0 \\
B &: a(0) + b(1) + c(0) + d = 0 \implies b + d = 0 \\
C &: a(0) + b(0) + c(1) + d = 0 \implies c + d = 0
\end{align*}

On en déduit que $a = b = c = -d$
En choisissant $a = b = c = 1$, on obtient $d = -1$

Équation : $x + y + z - 1 = 0$, soit $x + y + z = 1$

\textbf{Face ABD :}
Points : $A(1,0,0)$, $B(0,1,0)$, $D(1,1,1)$

\begin{align*}
A &: a(1) + b(0) + c(0) + d = 0 \implies a + d = 0 \\
B &: a(0) + b(1) + c(0) + d = 0 \implies b + d = 0 \\
D &: a(1) + b(1) + c(1) + d = 0 \implies a + b + c + d = 0
\end{align*}

En soustrayant les deux premières équations de la troisième :
$c = 0 - (a + d) - (b + d) = -2d$

Choisissons $a = b = 1$, $c = -1$, $d = -1$

Équation : $x + y - z - 1 = 0$, soit $x + y - z = 1$

\textbf{Face ACD :}
Points : $A(1,0,0)$, $C(0,0,1)$, $D(1,1,1)$

\begin{align*}
A &: a(1) + b(0) + c(0) + d = 0 \implies a + d = 0 \\
C &: a(0) + b(0) + c(1) + d = 0 \implies c + d = 0 \\
D &: a(1) + b(1) + c(1) + d = 0 \implies a + b + c + d = 0
\end{align*}

En soustrayant la première et la deuxième équation de la troisième :
$b = 0$

Choisissons $a = -1$, $b = 1$, $c = -1$, $d = 1$

Équation : $-x + y - z + 1 = 0$, soit $-x + y - z = -1$

\subsection*{4. Face BCD :}
Points : $B(0,1,0)$, $C(0,0,1)$, $D(1,1,1)$

\begin{align*}
B &: a(0) + b(1) + c(0) + d = 0 \implies b + d = 0 \\
C &: a(0) + b(0) + c(1) + d = 0 \implies c + d = 0 \\
D &: a(1) + b(1) + c(1) + d = 0 \implies a + b + c + d = 0
\end{align*}

En soustrayant les deux premières équations de la troisième :
$a = 0 - (b + d) - (c + d) = -2d$

Choisissons $a = -1$, $b = 1$, $c = 1$, $d = -1$

Équation : $-x + y + z - 1 = 0$, soit $-x + y + z = 1$

Ainsi, nous obtenons les équations cartésiennes de toutes les faces du tétraèdre ABCD.


\q \textbf{Distance du point \(D\) au plan \((ABC)\)}

Cette distance est définie comme la longueur entre \( D \) et le point du plan \( (ABC) \) le plus
proche de \( D \). Il s'agit de sont son projeté orthogonal \( H \) sur le plan. Autrement dit, la
distance recherchée est la longueur du vecteur \( \overrightarrow{DH} \).

Déterminons les coordonnées du point $H$, projection orthogonale de $D$ sur le plan $(ABC)$.
Le point $H$ est sur la droite perpendiculaire au plan $(ABC)$ passant par $D$.

Comme le plan $(ABC)$ a pour équation $x + y + z = 1$, un vecteur normal à ce plan est
$\vec{n}(1,1,1)$.

L'équation paramétrique de la droite $(DH)$ est donc :
   $$ \begin{cases}
   x = 1 - t \\
   y = 1 - t \\
   z = 1 - t
   \end{cases} $$
où $t$ est un paramètre réel.

Le point $H$ est l'intersection de cette droite avec le plan $(ABC)$.
Ses coordonnées vérifient donc :
$$ (1-t) + (1-t) + (1-t) = 1 $$

Résolvons cette équation :
$$ 3 - 3t = 1 $$
$$ -3t = -2 $$
$$ t = \frac{2}{3} $$

Les coordonnées de H sont donc :
$$ H(\frac{1}{3}, \frac{1}{3}, \frac{1}{3}) $$

La distance $DH$ est la norme du vecteur :
$$ \overrightarrow{DH} = (\frac{1}{3} - 1, \frac{1}{3} - 1, \frac{1}{3} - 1) = (-\frac{2}{3}, -\frac{2}{3}, -\frac{2}{3}) $$
$$ DH = \sqrt{(\frac{2}{3})^2 + (\frac{2}{3})^2 + (\frac{2}{3})^2} = \frac{2}{3}\sqrt{3} = \frac{2}{\sqrt{3}} $$

La distance du point $D$ au plan $(ABC)$ est $\frac{2}{\sqrt{3}}$.


\q \textbf{Hauteur du tétraèdre}

La hauteur du tétraèdre est la distance de \(D\) au plan \((ABC)\).


\q \textbf{Démonstration que \(H\) est le centre de gravité de \(ABC\)}

Nous avons déjà trouvé que H a pour coordonnées $(\frac{1}{3}, \frac{1}{3}, \frac{1}{3})$.

Pour montrer que H est situé à $\frac{2}{3}$ de la distance de chaque sommet au milieu du côté
opposé, prenons l'exemple du sommet A :

- Le milieu M de BC a pour coordonnées $(0, \frac{1}{2}, \frac{1}{2})$

- $\overrightarrow{AM} = (-1, \frac{1}{2}, \frac{1}{2})$

- $\overrightarrow{AH} = (-\frac{2}{3}, \frac{1}{3}, \frac{1}{3}) = \frac{2}{3}\overrightarrow{AM}$

Ceci confirme que H est situé aux $\frac{2}{3}$ de AM, et on peut faire le même raisonnement pour
les autres sommets.


\section{Orthogonalité dans le tétraèdre}

\q \textbf{Orthogonalité des arêtes [AB] et [CD]}

Déterminons les vecteurs directeurs de ces arêtes :

$\overrightarrow{AB} = (0-1, 1-0, 0-0) = (-1, 1, 0)$

$\overrightarrow{CD} = (1-0, 1-0, 1-1) = (1, 1, 0)$

Le produit scalaire de ces vecteurs est :
$\overrightarrow{AB} \cdot \overrightarrow{CD} = (-1)(1) + (1)(1) + (0)(0) = 0$

Comme le produit scalaire est nul, les vecteurs sont orthogonaux, donc les arêtes [AB] et [CD] sont
orthogonales.


\q \textbf{Equation paramétrique de la droite passant par le milieu de [AB] et le milieu de [CD]}

Milieu de [AB] : $M_{AB} = (\frac{1+0}{2}, \frac{0+1}{2}, \frac{0+0}{2}) = (\frac{1}{2}, \frac{1}{2}, 0)$

Milieu de [CD] : $M_{CD} = (\frac{0+1}{2}, \frac{0+1}{2}, \frac{1+1}{2}) = (\frac{1}{2}, \frac{1}{2}, 1)$

Vecteur directeur de la droite : $\overrightarrow{M_{AB}M_{CD}} = (0, 0, 1)$

Équation paramétrique de la droite :
$$ \begin{cases}
x = \frac{1}{2} \\
y = \frac{1}{2} \\
z = t
\end{cases} \quad t \in  \mathbb{R}$$


\q \textbf{Vérification de la perpendicularité de cette droite avec \([AB]\) et \([CD]\)}

Vecteur directeur de la droite : $\vec{u} = (0, 0, 1)$

$\overrightarrow{AB} = (-1, 1, 0)$

$\overrightarrow{CD} = (1, 1, 0)$

Produits scalaires :
$\vec{u} \cdot \overrightarrow{AB} = (0)(-1) + (0)(1) + (1)(0) = 0$ et
$\vec{u} \cdot \overrightarrow{CD} = (0)(1) + (0)(1) + (1)(0) = 0$

Comme les deux produits scalaires sont nuls, la droite est bien perpendiculaire aux deux arêtes [AB] et [CD].



On calcule les produits scalaires entre le vecteur directeur de la droite et ceux des arêtes
\([AB]\) et \([CD]\).
On trouve qu'ils sont tous nuls, ce qui prouve la perpendicularité.

\end{document}
