% PROBLÈME : Théorème des trois perpendiculaires
% ==================================================================================================
%
% But
% ---
% Analyser une configuration spécifique impliquant un plan et des droites dans l'espace, culminant
% avec la démonstration du théorème des trois perpendiculaires.
%
% Objectifs spécifiques
% ---------------------
% - Manipuler les représentations paramétriques et cartésiennes des droites et des plans.
% - Étudier les positions relatives des droites et des plans dans l'espace.
% - Démontrer la caractérisation de la distance d'un point à un plan comme la plus petite distance
%   entre ce point et tout point du plan.
%   ==================================================================================================

\documentclass[10pt,a4paper]{article}

% Set the root path
\providecommand{\rootpath}{../..}
% Fonts
\usepackage[utf8]{inputenc} % for accents
\usepackage[T1]{fontenc} % for accents
\usepackage[french]{babel} % for french language
\usepackage{helvet} % sans serif font family
\renewcommand*\familydefault{\sfdefault} % sans serif font family

% Mathematics
\usepackage{amsmath,amsfonts,amssymb} % for math symbols
\usepackage{array} % for tabular


\usepackage{parskip} % no indent, space between paragraphs

\usepackage{geometry} % margin
\geometry{
    a4paper,
    left=15mm,
    right=15mm,
    top=20mm,
    bottom=20mm
}

\usepackage{circledsteps} % to draw circles around numbers

\usepackage{fancyhdr} % for headers and footers

\usepackage{enumitem} % for customizing lists
\setlist[enumerate]{itemsep=1em} % space between items only in enumerate environment (not itemize)
\setlist[itemize]{label=--} % set itemize label to em-dash

% Command: \customPageLayout{#1}{#2}{#3}
% --------------------------------------
% Description: Custom page layout with header and footer content.
% Arguments:
% #1: Header and footer content
% #2: Left header content
% #3: Right header content
% Example:
% \customPageLayout{Title}{Lycée Henri IV}{2024}
% Required Packages: fancyhdr
\newcommand{\customPageLayout}[3]{
    \pagestyle{fancy} % set page style to fancy, i.e. header and footer
    \fancyhf{#1} % set header and footer content
    \lhead{#2} % set left header content
    \rhead{#3} % set right header content
    \fancyfoot{} % clear footer content
    \rfoot{\thepage} % set page number in footer
}

% Counter: \q
% -----------
% Description: Display a question number in a circle.
\newcounter{q}
\setcounter{q}{0} % set initial value of counter
\newcommand{\q}{
    \bigskip
    \addtocounter{q}{1}
    \par
    \Circled{\textbf{\theq}} \space
}


\title{Géométrie - Droites et Plans - Théorème des trois perpendiculaires}
\author{Esther Poniatowski}
\date{2024-2025}

\customPageLayout{Sujets d'interrogation orale}{Lycée Henri IV}{2024}

\begin{document}

Ce problème explore une configuration géométrique particulière dans l'espace, mettant en jeu des
droites et des plans. Il aboutit à la démonstration du théorème des trois perpendiculaires, une
propriété géométrique intéressante :

\begin{center}
\textbf{Théorème des trois perpendiculaires}

Soit $P$ un plan de l'espace et $D$ une droite non parallèle au plan $P$. Si une droite du plan $P$ est
perpendiculaire à la projection de $D$ sur $P$, alors elle est perpendiculaire à $D$.
\end{center}

\section{Étude préliminaire d'un plan et d'une droite}

On considère un repère orthonormé $(O, \mathbf{i}, \mathbf{j}, \mathbf{k})$ de l'espace. Soit le
plan $P$ d'équation :
   \[
   x + y + z = 1.
   \]
Soit la droite $D$ passant par le point $A(1, 0, 0)$ et de vecteur directeur
$\mathbf{u} = (1,1,-1)$.

\q Caractérisation de l'orthogonalité entre droites et plans
\setcounter{ql}{0}

\ql Déterminer un vecteur normal au plan $P$.

\ql En déduire deux vecteurs directeurs du plan $P$.

\q Représentation paramétrique et cartésienne de la droite $D$
\setcounter{ql}{0}

\ql Donner une représentation paramétrique de $D$.

\ql Déterminer une équation cartésienne de $D$.

\q Position relative de $D$ et $P$
\setcounter{ql}{0}

\ql Montrer que $D$ et $P$ ne sont pas parallèles.

\ql Calculer les coordonnées du point d'intersection $I$ de $D$ et $P$.

\q Soit le point $M(2, -1, 3)$. Déterminer les coordonnées du projeté orthogonal $H$ de $M$ sur le
plan $P$.

\q Soit $D'$ la droite passant par $M$ et orthogonale au plan $P$.
\setcounter{ql}{0}

\ql Donner une équation paramétrique de $D'$.

\ql Montrer que $H$ appartient à $D'$.


\section{Vérification du théorème}

Soit $\Delta$ la droite d'intersection du plan $P$ et du plan perpendiculaire à $D$ passant par
$H$.

\q Déterminer l'équation cartésienne du plan perpendiculaire à $D$ passant par $H$.

\q Déterminer un vecteur directeur de $\Delta$.

\q Montrer que $\Delta$ est perpendiculaire à $D$.


\section{Démonstration du théorème des trois perpendiculaires}

Soit $D$ une droite de l'espace non parallèle au plan $P$, et soit $\pi(D)$ la projection
orthogonale de $D$ sur $P$.

Soit $d$ une droite contenue dans $P$, perpendiculaire à $\pi(D)$ dans $P$.

On note $\mathbf{v_D}$ et $\mathbf{v_d}$ les vecteurs directeurs des droites.

\q Pour démontrer le théorème des trois perendiculaires, quelle propriété faut-il prouver concernant
la relation $d$ et $D$ ?

\q Traduire cette condition en termes de produits scalaires faisant intervenir les vecteurs
directeurs des droites.

\q Décomposer le vecteur directeur $\mathbf{v_D}$ de la droite $D$ en faisant intervenir le vecteur
normal au plan $\mathbf{n}$.

\q Calculer le produit scalaire établi précédemment et conclure.

\end{document}
