% EXERCICES D'APPLICATION :  Droites et Plans
% ==================================================================================================
%
% But
% ---
% Mobiliser les compétences aux programmes dans divers courts exercices indépendants.
% ==================================================================================================

\documentclass[10pt,a4paper]{article}

% Set the root path
\providecommand{\rootpath}{../..}
% Fonts
\usepackage[utf8]{inputenc} % for accents
\usepackage[T1]{fontenc} % for accents
\usepackage[french]{babel} % for french language
\usepackage{helvet} % sans serif font family
\renewcommand*\familydefault{\sfdefault} % sans serif font family

% Mathematics
\usepackage{amsmath,amsfonts,amssymb} % for math symbols
\usepackage{array} % for tabular


\usepackage{parskip} % no indent, space between paragraphs

\usepackage{geometry} % margin
\geometry{
    a4paper,
    left=15mm,
    right=15mm,
    top=20mm,
    bottom=20mm
}

\usepackage{circledsteps} % to draw circles around numbers

\usepackage{fancyhdr} % for headers and footers

\usepackage{enumitem} % for customizing lists
\setlist[enumerate]{itemsep=1em} % space between items only in enumerate environment (not itemize)
\setlist[itemize]{label=--} % set itemize label to em-dash

% Command: \customPageLayout{#1}{#2}{#3}
% --------------------------------------
% Description: Custom page layout with header and footer content.
% Arguments:
% #1: Header and footer content
% #2: Left header content
% #3: Right header content
% Example:
% \customPageLayout{Title}{Lycée Henri IV}{2024}
% Required Packages: fancyhdr
\newcommand{\customPageLayout}[3]{
    \pagestyle{fancy} % set page style to fancy, i.e. header and footer
    \fancyhf{#1} % set header and footer content
    \lhead{#2} % set left header content
    \rhead{#3} % set right header content
    \fancyfoot{} % clear footer content
    \rfoot{\thepage} % set page number in footer
}

% Counter: \q
% -----------
% Description: Display a question number in a circle.
\newcounter{q}
\setcounter{q}{0} % set initial value of counter
\newcommand{\q}{
    \bigskip
    \addtocounter{q}{1}
    \par
    \Circled{\textbf{\theq}} \space
}



\title{Géométrie - Droites et Plans - Applications de cours}
\author{}
\date{2024}

\customPageLayout{Sujets d'interrogation orale}{Lycée Henri IV}{2024}

\begin{document}
\maketitle

\section*{Question de cours}
\begin{itemize}
    \item Définir la distance d'un point à une droite dans $\mathbb{R}^2$.
    \item Caractériser la distance d'un point à une droite comme la plus petite distance séparant ce point d'un point de la droite.
\end{itemize}

\section*{Exercices}

\textbf{Représentations et équations}
\begin{enumerate}
    \item Dans $\mathbb{R}^2$, soit la droite passant par le point $A(1, 2)$ et de vecteur directeur $\vec{u} = \begin{pmatrix} 3 \\ -1 \end{pmatrix}$.
    \begin{itemize}
        \item Écrire son équation cartésienne et son équation paramétrique.
        \item Vérifier si le point $B(4, 1)$ appartient à cette droite.
    \end{itemize}
    \item Dans $\mathbb{R}^3$, soit la droite définie par $\begin{cases} x - 1 = 2t \\ y = 3t \\ z + 1 = t \end{cases}$.
    \begin{itemize}
        \item Écrire son équation cartésienne.
        \item Vérifier si le point $(3, 0, -2)$ appartient à cette droite.
    \end{itemize}
\end{enumerate}

\textbf{Distances}
\begin{enumerate}
    \item Soit la droite $\Delta : 2x - y + 1 = 0$. Calculer la distance du point $M(3, 4)$ à $\Delta$.
    \item Déterminer les coordonnées du projeté orthogonal de $M$ sur $\Delta$.
\end{enumerate}

\textbf{Positions relatives}\\

Rappeler les différentes positions relatives possibles entre deux droites dans $\mathbb{R}^3$.\\

Déterminer les positions relatives des droites suivantes.\\
Si elles sont sécantes, calculer les coordonnées de leur point d'intersection.

\begin{enumerate}
    \item $D_1 : \begin{cases} x = t \\ y = 2t \\ z = t+1 \end{cases}$ et
           $D_2 : \begin{cases} x = 2s \\ y = s+1 \\ z = s \end{cases}$
    \item $\Delta_1 : x + y - 2 = 0$ et $\Delta_2 : 2x - y + 3 = 0$
\end{enumerate}


\end{document}
