% CORRECTION : Théorème des trois perpendiculaires
% ==================================================================================================
%
% But
% ---
% Fournir une correction détaillée des questions posées dans l'exercice sur le théorème des trois
% perpendiculaires.
%
% ==================================================================================================

\documentclass[10pt,a4paper]{article}

% Set the root path
\providecommand{\rootpath}{../..}
% Fonts
\usepackage[utf8]{inputenc} % for accents
\usepackage[T1]{fontenc} % for accents
\usepackage[french]{babel} % for french language
\usepackage{helvet} % sans serif font family
\renewcommand*\familydefault{\sfdefault} % sans serif font family

% Mathematics
\usepackage{amsmath,amsfonts,amssymb} % for math symbols
\usepackage{array} % for tabular


\usepackage{parskip} % no indent, space between paragraphs

\usepackage{geometry} % margin
\geometry{
    a4paper,
    left=15mm,
    right=15mm,
    top=20mm,
    bottom=20mm
}

\usepackage{circledsteps} % to draw circles around numbers

\usepackage{fancyhdr} % for headers and footers

\usepackage{enumitem} % for customizing lists
\setlist[enumerate]{itemsep=1em} % space between items only in enumerate environment (not itemize)
\setlist[itemize]{label=--} % set itemize label to em-dash

% Command: \customPageLayout{#1}{#2}{#3}
% --------------------------------------
% Description: Custom page layout with header and footer content.
% Arguments:
% #1: Header and footer content
% #2: Left header content
% #3: Right header content
% Example:
% \customPageLayout{Title}{Lycée Henri IV}{2024}
% Required Packages: fancyhdr
\newcommand{\customPageLayout}[3]{
    \pagestyle{fancy} % set page style to fancy, i.e. header and footer
    \fancyhf{#1} % set header and footer content
    \lhead{#2} % set left header content
    \rhead{#3} % set right header content
    \fancyfoot{} % clear footer content
    \rfoot{\thepage} % set page number in footer
}

% Counter: \q
% -----------
% Description: Display a question number in a circle.
\newcounter{q}
\setcounter{q}{0} % set initial value of counter
\newcommand{\q}{
    \bigskip
    \addtocounter{q}{1}
    \par
    \Circled{\textbf{\theq}} \space
}


\title{Géométrie - Droites et Plans - Théorème des trois perpendiculaires}
\author{}
\date{2024}

\customPageLayout{Correction}{Lycée Henri IV}{2024}

\begin{document}

\section{Étude préliminaire du plan et de la droite}

On considère un repère orthonormé $(O, \mathbf{i}, \mathbf{j}, \mathbf{k})$ de l'espace. Soit le
plan $P$ d'équation :
   \[
   x + y + z = 1.
   \]
Soit la droite $D$ passant par le point $A(1, 0, 0)$ et de vecteur directeur $\mathbf{u} =
(1,1,-1)$.

\q Caractérisation de l'orthogonalité entre droites et plans
\setcounter{ql}{0}

\ql \textbf{Détermination d'un vecteur normal au plan $P$}

Un vecteur normal à un plan d'équation $ax + by + cz + d = 0$ est donné par $\mathbf{n} = (a, b,
c)$. Dans notre cas, on identifie :
\[
\mathbf{n} = (1,1,1).
\]

\ql \textbf{Deux vecteurs directeurs du plan $P$}

Un plan est défini par un vecteur normal, et tout vecteur orthogonal à ce vecteur normal est
directeur du plan. On cherche deux vecteurs $\mathbf{v_1}$ et $\mathbf{v_2}$ tels que :
\[
\mathbf{n} \cdot \mathbf{v_1} = 0, \quad \mathbf{n} \cdot \mathbf{v_2} = 0.
\]
On peut prendre, par exemple :
\[
\mathbf{v_1} = (1,-1,0), \quad \mathbf{v_2} = (1,0,-1).
\]

\q Représentation paramétrique et cartésienne de la droite $D$
\setcounter{ql}{0}

\ql \textbf{Représentation paramétrique de $D$}

La droite $D$ passe par $A(1,0,0)$ et est dirigée par $\mathbf{u} = (1,1,-1)$. Son équation
paramétrique est :
\[
\begin{cases}
x = 1 + t \\
y = t \\
z = -t
\end{cases}, \quad t \in \mathbb{R}.
\]

\ql \textbf{Équation cartésienne de $D$}

Éliminons $t$ :
\[
y = x - 1, \quad z = - (x - 1).
\]
On obtient une équation cartésienne sous la forme :
\[
y + z = 0, \quad x - y - 1 = 0.
\]

\q Position relative de $D$ et $P$
\setcounter{ql}{0}

\ql \textbf{Montrer que $D$ et $P$ ne sont pas parallèles}

La droite $D$ est parallèle à $P$ si et seulement si son vecteur directeur $\mathbf{u}$ est
orthogonal au vecteur normal $\mathbf{n}$ du plan :
\[
\mathbf{n} \cdot \mathbf{u} = 1 \cdot 1 + 1 \cdot 1 + 1 \cdot (-1) = 1 \neq 0.
\]
Donc, $D$ et $P$ ne sont pas parallèles.

\ql \textbf{Calcul des coordonnées du point d'intersection $I$ de $D$ et $P$}

On résout :
\[
(1+t) + t + (-t) = 1.
\]
Ce qui donne $t = 0$, donc $I = (1, 0, 0)$.

\q Projeté orthogonal d'un point sur un plan
\setcounter{ql}{0}

\ql \textbf{Calcul des coordonnées du projeté orthogonal $H$ de $M$ sur $P$}

Le projeté orthogonal d'un point $M(x_M, y_M, z_M)$ sur un plan est donné par :
\[
H = M - \lambda \mathbf{n},
\]
où $\lambda$ est déterminé en imposant que $H$ appartient à $P$.

On trouve :
\[
H(1, 0, 0)
\]

\q Droite perpendiculaire à un plan
\setcounter{ql}{0}

\ql \textbf{Équation paramétrique de $D'$}

La droite $D'$ passant par $M$ et orthogonale à $P$ a pour vecteur directeur $\mathbf{n}$. Son
équation paramétrique est donc :
\[
\begin{cases}
x = 2 + t \\
y = -1 + t \\
z = 3 + t
\end{cases}, \quad t \in \mathbb{R}.
\]

\ql \textbf{Vérification que $H$ appartient à $D'$}

En remplaçant $H(1,0,0)$ dans l'équation paramétrique de $D'$, on trouve $t = -1$, donc $H \in D'$.



\section{Vérification du théorème}

\q \textbf{Définition du plan perpendiculaire à $D$ passant par $H$}

Le plan perpendiculaire à $D$ passant par $H$ est défini par un vecteur normal parallèle au vecteur directeur de $D$, soit $\mathbf{u} = (1,1,-1)$.

L'équation cartésienne de ce plan s'écrit sous la forme :
\[
(x - x_H) + (y - y_H) - (z - z_H) = 0.
\]

En remplaçant les coordonnées de $H$, on obtient :
\[
(x - 1) + (y - 0) - (z - 0) = 0.
\]

Soit l'équation du plan perpendiculaire à $D$ :
\[
x + y - z = 1.
\]

\q \textbf{Détermination d'un vecteur directeur de $\Delta$}

Nous disposons déjà du point $H$ qui appartient à $\Delta$. Pour déterminer un vecteur directeur de
$\Delta$, il suffit de trouver un second point $B(x, y, z)$ appartenant à $\Delta$.

Puisque la droite $\Delta$ est l'intersection de deux plans, le point recherché appartient à la fois
au plan $P : x + y + z = 1$ et au plan perpendiculaire à $D$ passant par $H$ est défini par
l'équation  $x + y - z = 1$ :
\[
x + y + z = 1, \quad x + y - z = 1.
\]

En fixant $y = 1$, nous résolvons :
\[
x + 1 + z = 1, \quad x + 1 - z = 1.
\]

\[
x + z = 0, \quad x - z = 0.
\]

En additionnant, on obtient :
\[
2x = 0 \quad \Rightarrow \quad x = 0.
\]

Puis, en remplaçant dans $x + z = 0$, on trouve $z = 0$.

Ainsi, un deuxième point appartenant à $\Delta$ est :
\[
B(0,1,0)
\]

Le vecteur $\overrightarrow{HB}$ est donné par :
\[
\overrightarrow{HB} = (x_B - x_H, y_B - y_H, z_B - z_H)
\]

\[
\overrightarrow{HB} = (0 - 1, 1 - 0, 0 - 0) = (-1,1,0)
\]

Ainsi, un \textbf{vecteur directeur} de $\Delta$ est :
\[
\mathbf{v_{\Delta}} = (-1,1,0).
\]


\q \textbf{Perpendicularité entre $\Delta$ et $D$}

Deux droites sont perpendiculaires si le produit scalaire de leurs vecteurs directeurs est nul.

Les vecteurs directeurs sont :
\[
\mathbf{v_\Delta} = (-1,1,0), \quad \mathbf{u_D} = (1,1,-1).
\]

Calcul du produit scalaire :
\[
\mathbf{v_\Delta} \cdot \mathbf{u_D} = (-1 \times 1) + (1 \times 1) + (0 \times -1).
\]

\[
\mathbf{v_\Delta} \cdot \mathbf{u_D} = -1 + 1 + 0 = 0.
\]

Le produit scalaire est nul, donc $\Delta$ est perpendiculaire à $D$.




\section{Démonstration du théorème des trois perpendiculaires}


\q Nous devons prouver que $d$ est aussi perpendiculaire à $D$.

\q En termes de produits scalaires :
\[
\mathbf{v_d} \cdot \mathbf{v_D} = 0
\]

\q Le vecteur directeur $\mathbf{v_D}$ de la droite $D$ peut être décomposé en une composante
parallèle au plan $P$ et une composante normale :
\[
\mathbf{v_D} = \mathbf{v_{\pi(D)}} + \mathbf{v_n},
\]
où $\mathbf{v_{\pi(D)}}$ est la projection orthogonale de $\mathbf{v_D}$ sur $P$ et $\mathbf{v_n}$
est la composante normale à $P$, parallèle au vecteur normal $\mathbf{n}$ du plan $P$.

Par construction, $\mathbf{v_n}$ est colinéaire à $\mathbf{n}$, donc :
\[
\mathbf{v_n} = \lambda \mathbf{n}, \quad \text{avec } \lambda \in \mathbb{R}
\]

\q Calculons le produit scalaire entre $\mathbf{v_d}$ et $\mathbf{v_D}$ :
\[
\mathbf{v_d} \cdot \mathbf{v_D} = \mathbf{v_d} \cdot (\mathbf{v_{\pi(D)}} + \mathbf{v_n})
\]

En utilisant la linéarité du produit scalaire :
\[
\mathbf{v_d} \cdot \mathbf{v_D} = \mathbf{v_d} \cdot \mathbf{v_{\pi(D)}} + \mathbf{v_d} \cdot \mathbf{v_n}
\]

Or, par hypothèse, $d$ est perpendiculaire à $\pi(D)$ dans $P$, donc :
\[
\mathbf{v_d} \cdot \mathbf{v_{\pi(D)}} = 0
\]

De plus, $\mathbf{v_n}$ est normal au plan $P$, et comme $\mathbf{v_d}$ est un vecteur contenu dans $P$, il est nécessairement orthogonal à $\mathbf{v_n}$ :
\[
\mathbf{v_d} \cdot \mathbf{v_n} = 0
\]

Ainsi, nous obtenons :
\[
\mathbf{v_d} \cdot \mathbf{v_D} = 0 + 0 = 0
\]

\textbf{Conclusion}

Nous avons montré que si une droite contenue dans $P$ est perpendiculaire à la projection de $D$ sur
$P$, alors elle est aussi perpendiculaire à $D$.

\textbf{Interprétation géométrique}

Intuitivement, si une droite est perpendiculaire à la projection d'une droite oblique sur un plan,
elle est également perpendiculaire à cette droite oblique dans l'espace, car l'inclinaison ne
modifie pas l'angle entre les directions perpendiculaires.


\end{document}
