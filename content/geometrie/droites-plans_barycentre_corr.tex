% CORRECTION : Barycentre
% ==================================================================================================

\documentclass[10pt,a4paper]{article}

\providecommand{\rootpath}{../..}
% Fonts
\usepackage[utf8]{inputenc} % for accents
\usepackage[T1]{fontenc} % for accents
\usepackage[french]{babel} % for french language
\usepackage{helvet} % sans serif font family
\renewcommand*\familydefault{\sfdefault} % sans serif font family

% Mathematics
\usepackage{amsmath,amsfonts,amssymb} % for math symbols
\usepackage{array} % for tabular


\usepackage{parskip} % no indent, space between paragraphs

\usepackage{geometry} % margin
\geometry{
    a4paper,
    left=15mm,
    right=15mm,
    top=20mm,
    bottom=20mm
}

\usepackage{circledsteps} % to draw circles around numbers

\usepackage{fancyhdr} % for headers and footers

\usepackage{enumitem} % for customizing lists
\setlist[enumerate]{itemsep=1em} % space between items only in enumerate environment (not itemize)
\setlist[itemize]{label=--} % set itemize label to em-dash

% Command: \customPageLayout{#1}{#2}{#3}
% --------------------------------------
% Description: Custom page layout with header and footer content.
% Arguments:
% #1: Header and footer content
% #2: Left header content
% #3: Right header content
% Example:
% \customPageLayout{Title}{Lycée Henri IV}{2024}
% Required Packages: fancyhdr
\newcommand{\customPageLayout}[3]{
    \pagestyle{fancy} % set page style to fancy (add header and footer)
    \fancyhf{} % clear all header and footer content
    \lhead{#2} % left header content
    \rhead{#3} % right header content
    \chead{\textbf{#1}} % center header content in bold (if needed)
    \rfoot{\thepage} % page number in the footer
}


% Counter: \q
% -----------
% Description: Display a question number in a circle.
% Usage:
% - Create a new question: add \q followed by the question content.
% - Reset the question counter: add \setcounter{q}{0} before the first question.
\newcounter{q}
\setcounter{q}{0} % set initial value of the counter
\newcommand{\q}{
    \bigskip
    \addtocounter{q}{1}
    \par
    \Circled{\textbf{\theq}} \space
}


% Counter: \ql
% ------------
% Description: Display a question letter in a round box with indentation (lowercase and not bold).
% Usage:
% - Create a new question: add \ql followed by the question content.
% - Reset the question counter: add \setcounter{ql}{0} before the first question.
\newcounter{ql}
\setcounter{ql}{0} % set initial value of the counter
\newcommand{\ql}{
    \addtocounter{ql}{1}
    \par
    \hspace{1.5em} % indentation before the circled letter
    \textcolor{gray}{\Circled{\alph{ql}}} \space % gray color
}


\title{Géométrie - Barycentre}
\author{}
\date{2024}

\customPageLayout{Correction}{Lycée Henri IV}{2024}

\begin{document}

\section{Barycentre et Centre de Gravité d'un Triangle}

\q \textbf{Concourrence des médianes}
\setcounter{ql}{0}

\ql Considérons les équations des deux médianes $(AM_A)$ et $(BM_B)$, et montrons qu'elles se
coupent en un unique point. En effet, les trois médianes ne définissent pas trois équations
indépendantes : l'une des trois est automatiquement vérifiée si les deux autres sont
satisfaites. Il est donc suffisant de résoudre le système avec deux équations, puis de vérifier que
le point obtenu appartient bien à la troisième médiane.

\begin{itemize}
    \item Le milieu $M_A$ du segment $[BC]$ a pour coordonnées :
    $$
    M_A \left( \frac{x_2 + x_3}{2}, \frac{y_2 + y_3}{2} \right)
    $$
    \item La droite $(AM_A)$ passe par $A(x_1, y_1)$ et $M_A$, donc son équation est :
    $$
    y - y_1 = \frac{y_2 + y_3 - 2y_1}{x_2 + x_3 - 2x_1} (x - x_1)
    $$
\end{itemize}
En effet, pour obtenir l'équation cartésienne de la droite $(AM_A)$ s'obtient en considérant la
proportionnalité des coordonnées.

Soit un point générique $M(x,y)$ sur la droite $(AM_A)$.

Les vecteurs $\overrightarrow{AM}$ et $\overrightarrow{AM_A}$ sont colinéaires.
La colinéarité implique que les rapports des coordonnées de ces vecteurs sont égaux.

En posant $k$ comme le rapport commun :
$$
x - x_1 = k(\frac{x_2 + x_3}{2} - x_1)
$$
$$
y - y_1 = k(\frac{y_2 + y_3}{2} - y_1)
$$

En divisant la seconde équation par la première :
$$
\frac{y - y_1}{x - x_1} = \frac{\frac{y_2 + y_3}{2} - y_1}{\frac{x_2 + x_3}{2} - x_1}
$$
$$
y - y_1 = \frac{\frac{y_2 + y_3}{2} - y_1}{\frac{x_2 + x_3}{2} - x_1}(x - x_1)
$$
$$
y - y_1 = \frac{y_2 + y_3 - 2y_1}{x_2 + x_3 - 2x_1}(x - x_1)
$$



De même, on peut établir l'équation de la médiane $(BM_B)$.

Résolvons le système comprenant les équations des médianes \((AM_A)\) et \((BM_B)\) :
\begin{align*}
y - y_1 &= \frac{y_2 + y_3 - 2y_1}{x_2 + x_3 - 2x_1} (x - x_1), \\
y - y_2 &= \frac{y_1 + y_3 - 2y_2}{x_1 + x_3 - 2x_2} (x - x_2).
\end{align*}

En résolvant ce système, l'intersection des deux médianes est donnée par :
\[
G \left( \frac{x_1 + x_2 + x_3}{3}, \frac{y_1 + y_2 + y_3}{3} \right).
\]

Vérification avec la troisième médiane \((CM_C)\), qui a pour équation :
\[
y - y_3 = \frac{y_1 + y_2 - 2y_3}{x_1 + x_2 - 2x_3} (x - x_3).
\]

En remplaçant \((x_G, y_G)\) dans cette équation, on vérifie que :
\[
y_G - y_3 = \frac{y_1 + y_2 - 2y_3}{x_1 + x_2 - 2x_3} (x_G - x_3).
\]

Ce qui est bien satisfait. On conclut donc que les trois médianes sont concourantes et que leur
intersection est le point \( G \), centre de gravité du triangle.

\q \textbf{Lien entre barycentre et centre de gravité} : Le centre de gravité correspond au
barycentre des trois sommets du triangle, pondérés avec des masses égales.

\q \textbf{Position du barycentre sur les médianes}

Paramétrisation de la médiane

Le point \( G \) étant situé sur la médiane, il peut être exprimé sous forme paramétrique comme :
\[
G = (1 - k) A + k M_A,
\]
où \( k \) est un paramètre réel.

Le but est de montrer que \( k = \frac{2}{3} \).

Décomposons cette relation :
\[
\begin{cases}
x_G = (1 - k) x_1 + k \frac{x_2 + x_3}{2}, \\
y_G = (1 - k) y_1 + k \frac{y_2 + y_3}{2}
\end{cases}
\]

En utilisant l'expression des coordonnées du centre de gravité :
\[
x_G = \frac{x_1 + x_2 + x_3}{3}, \quad
y_G = \frac{y_1 + y_2 + y_3}{3},
\]
nous obtenons deux équations en \( k \) :
\[
\frac{x_1 + x_2 + x_3}{3} = (1 - k) x_1 + k \frac{x_2 + x_3}{2}
\]

Résolvons cette équation pour \( k \) :
\[
\frac{x_1 + x_2 + x_3}{3} = x_1 - k x_1 + k \frac{x_2 + x_3}{2}
\]

En isolant \( k \) :
\[
\frac{x_1 + x_2 + x_3}{3} - x_1 = -k x_1 + k \frac{x_2 + x_3}{2}
\]

Factorisons \( k \) :
\[
\frac{x_1 + x_2 + x_3}{3} - x_1 = k \left( \frac{x_2 + x_3}{2} - x_1 \right)
\]

En simplifiant, on obtient :
\[
k = \frac{2}{3}
\]


\section{Barycentre et Projeté sur une droite}

\q Projeté orthogonal du barycentre sur $D$

$$
x_{G'} = x_G - \frac{a(ax_G + by_G + c)}{a^2 + b^2}, \quad
y_{G'} = y_G - \frac{b(ax_G + by_G + c)}{a^2 + b^2}
$$

\q Expression de la distance du barycentre à $D$

$$
d(G, D) = \frac{|ax_G + by_G + c|}{\sqrt{a^2 + b^2}}
$$

\q G appartient à D si et seulement si $ax_G + by_G + c = 0$.
\q Projeté orthogonal du barycentre sur $D$
\setcounter{ql}{0}

\q \textbf{Projeté orthogonal $G'$ de $G$ sur $D$}

On considère la droite $D$ d'équation cartésienne :
\[
ax + by + c = 0.
\]
Le projeté orthogonal $G'$ de $G(x_G, y_G)$ sur $D$ est le point de $D$ qui réalise la distance
minimale à $G$.

Le projeté orthogonal vérifie la relation vectorielle suivante :
\[
\overrightarrow{GG'} = \lambda \mathbf{n},
\]
où $\mathbf{n} = (a, b)$ est un vecteur normal à la droite $D$, et $\lambda$ est un réel à
déterminer.

Les coordonnées de $G'$ s'écrivent donc :
\[
x_{G'} = x_G + \lambda a, \quad y_{G'} = y_G + \lambda b.
\]
Puisque $G'$ appartient à $D$, ses coordonnées doivent satisfaire l'équation de la droite :
\[
a x_{G'} + b y_{G'} + c = 0.
\]

En remplaçant $x_{G'}$ et $y_{G'}$ :
\[
a (x_G + \lambda a) + b (y_G + \lambda b) + c = 0.
\]

Développons :
\[
ax_G + \lambda a^2 + by_G + \lambda b^2 + c = 0.
\]

Factorisons $\lambda$ :
\[
\lambda (a^2 + b^2) = - (ax_G + by_G + c).
\]

On en déduit :
\[
\lambda = -\frac{ax_G + by_G + c}{a^2 + b^2}.
\]

En substituant dans les expressions de $x_{G'}$ et $y_{G'}$, on obtient :
\[
x_{G'} = x_G - \frac{a(ax_G + by_G + c)}{a^2 + b^2}, \quad
y_{G'} = y_G - \frac{b(ax_G + by_G + c)}{a^2 + b^2}.
\]


\q \textbf{Interprétation du projeté orthogonal du barycentre en termes de barycentre des projetés}

Nous avons :
\[
x_G = \frac{m_1 x_1 + m_2 x_2 + m_3 x_3}{m_1 + m_2 + m_3}, \quad
y_G = \frac{m_1 y_1 + m_2 y_2 + m_3 y_3}{m_1 + m_2 + m_3}.
\]

On sait que le projeté orthogonal d'un point \( M(x,y) \) sur la droite \( D \) d'équation \( ax + by + c = 0 \) est donné par :
\[
x_{M'} = x - \frac{a(ax + by + c)}{a^2 + b^2}, \quad
y_{M'} = y - \frac{b(ax + by + c)}{a^2 + b^2}.
\]

Appliquons cette formule aux points \( A, B, C \) :
\[
x_{A'} = x_1 - \frac{a(ax_1 + by_1 + c)}{a^2 + b^2}, \quad
y_{A'} = y_1 - \frac{b(ax_1 + by_1 + c)}{a^2 + b^2}.
\]

De même pour \( B' \) et \( C' \).

Prenons maintenant le barycentre des projections :
\[
x_{G'} = \frac{m_1 x_{A'} + m_2 x_{B'} + m_3 x_{C'}}{m_1 + m_2 + m_3},
\]
\[
y_{G'} = \frac{m_1 y_{A'} + m_2 y_{B'} + m_3 y_{C'}}{m_1 + m_2 + m_3}.
\]

En remplaçant \( x_{A'}, x_{B'}, x_{C'} \) par leurs expressions :
\[
x_{G'} = \frac{m_1 \left( x_1 - \frac{a(ax_1 + by_1 + c)}{a^2 + b^2} \right)
+ m_2 \left( x_2 - \frac{a(ax_2 + by_2 + c)}{a^2 + b^2} \right)
+ m_3 \left( x_3 - \frac{a(ax_3 + by_3 + c)}{a^2 + b^2} \right) }
{m_1 + m_2 + m_3}.
\]

En séparant les termes :
\[
x_{G'} = \frac{m_1 x_1 + m_2 x_2 + m_3 x_3}{m_1 + m_2 + m_3}
- \frac{a}{a^2 + b^2} \cdot \frac{m_1 (ax_1 + by_1 + c) + m_2 (ax_2 + by_2 + c) + m_3 (ax_3 + by_3 + c)}{m_1 + m_2 + m_3}.
\]

Mais on sait que :
\[
ax_G + by_G + c = \frac{m_1 (ax_1 + by_1 + c) + m_2 (ax_2 + by_2 + c) + m_3 (ax_3 + by_3 + c)}{m_1 + m_2 + m_3}.
\]

Donc,
\[
x_{G'} = x_G - \frac{a(ax_G + by_G + c)}{a^2 + b^2}.
\]

Un raisonnement identique donne :
\[
y_{G'} = y_G - \frac{b(ax_G + by_G + c)}{a^2 + b^2}.
\]

Ainsi, nous avons bien :
\[
G' = \operatorname{bar}(A', B', C'; m_1, m_2, m_3).
\]

\end{document}
% ==================================================================================================

\section{Barycentre et Produit Scalaire}

\q Lien entre le barycentre et les relations vectorielles
\setcounter{ql}{0}

\ql On a :
\begin{align*}
m_1 \overrightarrow{GA} + m_2 \overrightarrow{GB} + m_3 \overrightarrow{GC} &=
m_1 (\overrightarrow{OA} - \overrightarrow{OG}) + m_2 (\overrightarrow{OB} - \overrightarrow{OG}) + m_3 (\overrightarrow{OC} - \overrightarrow{OG}) \\
&= m_1 \overrightarrow{OA} + m_2 \overrightarrow{OB} + m_3 \overrightarrow{OC} - (m_1 + m_2 + m_3)\overrightarrow{OG} \\
&= (m_1 + m_2 + m_3)\overrightarrow{OG} - (m_1 + m_2 + m_3)\overrightarrow{OG} \\
&= \overrightarrow{0}
\end{align*}

\ql Cette relation signifie que G est le point d'équilibre du système de points pondérés.

\q Utilisation du produit scalaire pour démontrer une propriété fondamentale
\setcounter{ql}{0}

\ql Pour tout point M, on a :
\begin{align*}
f(M) &= m_1 ||\overrightarrow{MA}||^2 + m_2 ||\overrightarrow{MB}||^2 + m_3 ||\overrightarrow{MC}||^2 \\
&= m_1 (\overrightarrow{MA} \cdot \overrightarrow{MA}) + m_2 (\overrightarrow{MB} \cdot \overrightarrow{MB}) + m_3 (\overrightarrow{MC} \cdot \overrightarrow{MC}) \\
&= (m_1 + m_2 + m_3)(\overrightarrow{MG} \cdot \overrightarrow{MG}) + 2\overrightarrow{MG} \cdot (m_1 \overrightarrow{GA} + m_2 \overrightarrow{GB} + m_3 \overrightarrow{GC}) + C
\end{align*}
où C est une constante indépendante de M. Le deuxième terme est nul d'après la question précédente.
Donc f(M) est minimale lorsque $\overrightarrow{MG} = \overrightarrow{0}$, c'est-à-dire lorsque M =
G.

\ql En mécanique, cela signifie que G est le centre de masse du système de points pondérés.

\q Lien entre barycentre et projections orthogonales
\setcounter{ql}{0}

\ql Soit $\overrightarrow{n}$ un vecteur normal à D. Pour tout point P, on a $\overrightarrow{PP'} =
k\overrightarrow{n}$ où P' est la projection de P sur D et k un réel. Donc :
\begin{align*}
m_1 \overrightarrow{G'A'} + m_2 \overrightarrow{G'B'} + m_3 \overrightarrow{G'C'} &=
m_1 (\overrightarrow{GA} - k_1\overrightarrow{n}) + m_2 (\overrightarrow{GB} - k_2\overrightarrow{n}) + m_3 (\overrightarrow{GC} - k_3\overrightarrow{n}) \\
&= (m_1 \overrightarrow{GA} + m_2 \overrightarrow{GB} + m_3 \overrightarrow{GC}) - (m_1k_1 + m_2k_2 + m_3k_3)\overrightarrow{n} \\
&= -(m_1k_1 + m_2k_2 + m_3k_3)\overrightarrow{n} \\
&= k(m_1 + m_2 + m_3)\overrightarrow{n} \\
&= \overrightarrow{0}
\end{align*}
où $k = \frac{m_1k_1 + m_2k_2 + m_3k_3}{m_1 + m_2 + m_3}$. Donc G' est le barycentre de A', B', C'
avec les mêmes coefficients.

\ql Géométriquement, cela signifie que la projection du barycentre est le barycentre des
projections. Analytiquement, cela implique que les coordonnées de G' peuvent être calculées
directement à partir des coordonnées de A', B', C'.

\section{Extension dans $\mathbb{R}^3$}

\q Barycentre et distance à un plan
\setcounter{ql}{0}

\ql La démonstration est similaire à celle de la question 8, en remplaçant la droite D par le plan
P.

\ql Les coordonnées du projeté orthogonal G' de G sur P sont données par :
$$
x_{G'} = x_G - \frac{A(Ax_G + By_G + Cz_G + D)}{A^2 + B^2 + C^2}, \quad
y_{G'} = y_G - \frac{B(Ax_G + By_G + Cz_G + D)}{A^2 + B^2 + C^2}, \quad
z_{G'} = z_G - \frac{C(Ax_G + By_G + Cz_G + D)}{A^2 + B^2 + C^2}
$$

\q Cas où le barycentre appartient au plan
\setcounter{ql}{0}

\ql G appartient à P si et seulement si ses coordonnées satisfont l'équation du plan :
$$
A\left(\frac{m_1 x_1 + m_2 x_2 + m_3 x_3}{m_1 + m_2 + m_3}\right) +
B\left(\frac{m_1 y_1 + m_2 y_2 + m_3 y_3}{m_1 + m_2 + m_3}\right) +
C\left(\frac{m_1 z_1 + m_2 z_2 + m_3 z_3}{m_1 + m_2 + m_3}\right) + D = 0
$$

\ql En mécanique, cela signifie que le centre de masse du système se trouve dans le plan P. En
géométrie, cela implique que le barycentre est coplanaire avec les points A, B, C.

% ==================================================================================================
\setcounter{q}{0}
\section{Calcul du barycentre et interprétation géométrique}


\q Correction du lien entre barycentre et alignement
\setcounter{ql}{0}




\q Correction de la position du barycentre sur les médianes
\setcounter{ql}{0}

\ql Si $m_1 = m_2 = m_3$, on montre que $G$ appartient aux médianes du triangle en écrivant :
   $$
   \overrightarrow{GA} + \overrightarrow{GB} + \overrightarrow{GC} = \overrightarrow{0}.
   $$

\ql On démontre que $G$ est situé aux deux tiers de chaque médiane en paramétrant le segment reliant un sommet à son milieu.

\section{Étude de la distance du barycentre à une droite}

\q Correction du projeté orthogonal du barycentre sur $D$
\setcounter{ql}{0}

\ql Le projeté orthogonal $G'$ de $G$ sur $D$ est défini comme le point de $D$ minimisant :
   $$
   d(G, D) = \frac{|ax_G + by_G + c|}{\sqrt{a^2 + b^2}}.
   $$

\ql En résolvant le système des équations de $D$ et de la droite passant par $G$ de vecteur directeur $\overrightarrow{n} = (a, b)$, on obtient :
   $$
   G' \left( x_G - \lambda a, y_G - \lambda b \right),
   $$
   avec $\lambda$ tel que $G' \in D$.

\q Correction de la distance du barycentre à $D$
\setcounter{ql}{0}

\ql On utilise la formule classique de la distance d'un point à une droite :
   $$
   d(G, D) = \frac{|ax_G + by_G + c|}{\sqrt{a^2 + b^2}}.
   $$

\ql Lorsque $G$ appartient à $D$, on a $ax_G + by_G + c = 0$, donc $d(G, D) = 0$.

\section{Lien entre barycentre et produit scalaire}

\q Correction du lien entre barycentre et relations vectorielles
\setcounter{ql}{0}

\ql On part de la relation :
   $$
   m_1 \overrightarrow{GA} + m_2 \overrightarrow{GB} + m_3 \overrightarrow{GC} = \overrightarrow{0}.
   $$

\ql Cela traduit que $G$ est un équilibre des positions pondérées.

\q Correction de la propriété fondamentale du produit scalaire
\setcounter{ql}{0}

\ql On développe la somme pondérée des carrés des distances :
   $$
   f(M) = m_1 ||\overrightarrow{MA}||^2 + m_2 ||\overrightarrow{MB}||^2 + m_3 ||\overrightarrow{MC}||^2.
   $$

\ql En utilisant la bilinéarité du produit scalaire, on montre que $f(M)$ est minimal lorsque $M = G$.

\q Correction du lien entre barycentre et projections orthogonales
\setcounter{ql}{0}

\ql On montre que $G'$ est le barycentre des projections orthogonales des points $A, B, C$ sur $D$ en projetant chaque terme de l'équation vectorielle du barycentre.

\ql Géométriquement, cela montre que la projection orthogonale préserve la relation barycentrique.

\section{Extension en $\mathbb{R}^3$}

\q Correction du barycentre et distance à un plan
\setcounter{ql}{0}

\ql On applique le même raisonnement que dans $\mathbb{R}^2$ en utilisant le vecteur normal du plan :
   $$
   \overrightarrow{n} = (A, B, C).
   $$

\ql Le projeté orthogonal $G'$ de $G$ sur $P$ est donné par :
   $$
   G' \left( x_G - \lambda A, y_G - \lambda B, z_G - \lambda C \right),
   $$
   avec $\lambda$ déterminé par l'équation du plan.

\q Correction du cas où le barycentre appartient au plan
\setcounter{ql}{0}

\ql $G$ appartient à $P$ si :
   $$
   A x_G + B y_G + C z_G + D = 0.
   $$

\ql Cela signifie que la somme des moments pondérés des coordonnées satisfait l'équation du plan.
