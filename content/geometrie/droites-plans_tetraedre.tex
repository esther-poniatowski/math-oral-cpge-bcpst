% PROBLÈME : Étude d'un tétraèdre régulier
% ==================================================================================================
%
% But
% ---
% Explorer les propriétés géométriques d'un tétraèdre régulier inscrit dans un cube unitaire.
%
% Objectifs spécifiques
% ---------------------
% - Utiliser la géométrie analytique dans l'espace pour étudier une figure tridimensionnelle
%   complexe.
% - Démontrer des propriétés géométriques du tétraèdre régulier, notamment concernant sa hauteur et
%   son centre de gravité.
% - Établir des relations d'orthogonalité entre les éléments du tétraèdre.
% ==================================================================================================

\documentclass[10pt,a4paper]{article}

% Set the root path
\providecommand{\rootpath}{../..}
% Fonts
\usepackage[utf8]{inputenc} % for accents
\usepackage[T1]{fontenc} % for accents
\usepackage[french]{babel} % for french language
\usepackage{helvet} % sans serif font family
\renewcommand*\familydefault{\sfdefault} % sans serif font family

% Mathematics
\usepackage{amsmath,amsfonts,amssymb} % for math symbols
\usepackage{array} % for tabular


\usepackage{parskip} % no indent, space between paragraphs

\usepackage{geometry} % margin
\geometry{
    a4paper,
    left=15mm,
    right=15mm,
    top=20mm,
    bottom=20mm
}

\usepackage{circledsteps} % to draw circles around numbers

\usepackage{fancyhdr} % for headers and footers

\usepackage{enumitem} % for customizing lists
\setlist[enumerate]{itemsep=1em} % space between items only in enumerate environment (not itemize)
\setlist[itemize]{label=--} % set itemize label to em-dash

% Command: \customPageLayout{#1}{#2}{#3}
% --------------------------------------
% Description: Custom page layout with header and footer content.
% Arguments:
% #1: Header and footer content
% #2: Left header content
% #3: Right header content
% Example:
% \customPageLayout{Title}{Lycée Henri IV}{2024}
% Required Packages: fancyhdr
\newcommand{\customPageLayout}[3]{
    \pagestyle{fancy} % set page style to fancy, i.e. header and footer
    \fancyhf{#1} % set header and footer content
    \lhead{#2} % set left header content
    \rhead{#3} % set right header content
    \fancyfoot{} % clear footer content
    \rfoot{\thepage} % set page number in footer
}

% Counter: \q
% -----------
% Description: Display a question number in a circle.
\newcounter{q}
\setcounter{q}{0} % set initial value of counter
\newcommand{\q}{
    \bigskip
    \addtocounter{q}{1}
    \par
    \Circled{\textbf{\theq}} \space
}


\usepackage{tikz}
\usepackage{tikz-3dplot}

\title{Géométrie - Droites et Plans - Étude d'un tétraèdre régulier}
\author{}
\date{2024}

\customPageLayout{Sujets d'interrogation orale}{Lycée Henri IV}{2024}

\begin{document}

\textbf{Objectif}

Étudier les propriétés géométriques d'un tétraèdre régulier, en utilisant les outils de la géométrie
analytique dans l'espace.

Un tétraèdre régulier est un polyèdre à quatre faces triangulaires équilatérales.

On considère un tétraèdre régulier \(ABCD\) inscrit dans un cube d'arête \( a = 1 \).
Les côtés du tétraèdre sont des diagonales du cube.

On se place dans un repère orthonormé \((O, \mathbf{i}, \mathbf{j}, \mathbf{k})\) où l'origine \(O\)
est un sommet du cube.

% Angle de vue
\tdplotsetmaincoords{70}{110}

\begin{tikzpicture}[tdplot_main_coords, scale=4]
    % Cube
    \draw[gray, thin] (0,0,0) -- (1,0,0) -- (1,1,0) -- (0,1,0) -- cycle;
    \draw[gray, thin] (0,0,1) -- (1,0,1) -- (1,1,1) -- (0,1,1) -- cycle;
    \draw[gray, thin] (0,0,0) -- (0,0,1);
    \draw[gray, thin] (1,0,0) -- (1,0,1);
    \draw[gray, thin] (1,1,0) -- (1,1,1);
    \draw[gray, thin] (0,1,0) -- (0,1,1);

    % Tétraèdre
    \draw[red, thick] (1,0,0) -- (0,1,0) -- (0,0,1) -- (1,1,1) -- cycle;
    \draw[red, thick] (1,0,0) -- (0,0,1);
    \draw[red, thick] (0,1,0) -- (1,1,1);

    % Points du tétraèdre
    \fill[blue] (1,0,0) circle (0.05) node[anchor=south west] {A};
    \fill[blue] (0,1,0) circle (0.05) node[anchor=south east] {B};
    \fill[blue] (0,0,1) circle (0.05) node[anchor=south] {C};
    \fill[blue] (1,1,1) circle (0.05) node[anchor=south west] {D};

    % Axes
    \draw[-stealth] (0,0,0) -- (1.5,0,0) node[anchor=north east]{$x$};
    \draw[-stealth] (0,0,0) -- (0,1.5,0) node[anchor=north west]{$y$};
    \draw[-stealth] (0,0,0) -- (0,0,1.5) node[anchor=south]{$z$};

    % Origine
    \fill[black] (0,0,0) circle (0.05) node[anchor=north east] {O};
\end{tikzpicture}



\section{Coordonnées et distances dans le tétraèdre}

\q Déterminer les coordonnées des sommets du tétraèdre \( A, B, C, D \).

\q Calculer la longueur des arêtes du tétraèdre à partir des coordonnées déterminées.

\q Déterminer des équations cartésiennes pour les plans associés aux faces du tétraèdre.

\q Calculer la distance du point \(D\) au plan \((ABC)\).

\q En déduire la hauteur du tétraèdre, définie comme la distance entre un sommet et le plan
formé par la face opposée à ce sommet.

\q Soit \(H\) le projeté orthogonal de \(D\) sur le plan \((ABC)\). Démontrer que \(H\) est le
centre de gravité du triangle \(ABC\), qui est situé à une distance égale à \(\frac{2}{3}\) de la
distance de chaque sommet au milieu du côté opposé.


\section{Orthogonalité dans le tétraèdre}

\q Montrer que les arêtes \([AB]\) et \([CD]\) sont orthogonales.

\q Déterminer l'équation paramétrique de la droite passant par le milieu de \([AB]\) et le milieu de \([CD]\).

\q Vérifier que cette droite est perpendiculaire aux deux arêtes \([AB]\) et \([CD]\).

\end{document}
% ==================================================================================================


Ce problème vise à étudier les propriétés géométriques d'un tétraèdre régulier, en utilisant les
outils de la géométrie analytique dans l'espace.

1. Soit un tétraèdre régulier ABCD inscrit dans un cube d'arête

% QUESTION 1
% ----------
% Compétence: Manipuler les droites et les plans dans R^3
1. Déterminer les coordonnées des sommets du tétraèdre dans un repère orthonormé dont l'origine est
un sommet du cube.

% QUESTION 2
% ----------
% Compétence: Manipuler les droites et les plans dans R^3
2. Calculer la longueur d'une arête du tétraèdre.

% QUESTION 3
% ----------
% Compétences:
% - Déterminer une équation cartésienne du plan (ABC)
% - Manipuler les droites et les plans dans R^3; Caractériser l'orthogonalité entre droites et plans
3. Déterminer une équation cartésienne du plan (ABC).

% QUESTION 4
% ----------
% Compétences:
% - Définition de la distance d'un point à une droite
% - Manipuler les droites et les plans dans R^3
4. Calculer la distance du point D au plan (ABC). En déduire la hauteur du tétraèdre.

% QUESTION 5
% ----------
% Compétences: Définition du projeté orthogonal d'un point sur une droite (à généraliser pour un plan)
5. Soit H le projeté orthogonal de D sur le plan (ABC). Démontrer que H est le centre de gravité du
triangle ABC.

% QUESTION 6
% ----------
% Compétences: Caractériser l'orthogonalité entre droites et plans
6. Montrer que les arêtes [AB] et [CD] sont orthogonales.

% QUESTION 7
% ----------
% Compétences: Manipuler les droites et les plans dans R^3
7. Déterminer l'équation paramétrique de la droite passant par le milieu de [AB] et le milieu de
[CD].

% QUESTION 8
% ----------
% Compétences: Caractériser l'orthogonalité entre droites et plans
8. Démontrer que cette droite est perpendiculaire aux deux arêtes [AB] et [CD].
