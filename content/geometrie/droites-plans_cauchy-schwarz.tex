% PROBLÈME : Démonstration géométrique de l'inégalité de Cauchy-Schwarz
% ==================================================================================================
%
% But
% ---
% Démontrer l'inégalité de Cauchy-Schwarz en utilisant une approche géométrique.
%
% Objectifs spécifiques
% ---------------------
% -
% ==================================================================================================

\documentclass[10pt,a4paper]{article}

% Set the root path
\providecommand{\rootpath}{../..}
% Fonts
\usepackage[utf8]{inputenc} % for accents
\usepackage[T1]{fontenc} % for accents
\usepackage[french]{babel} % for french language
\usepackage{helvet} % sans serif font family
\renewcommand*\familydefault{\sfdefault} % sans serif font family

% Mathematics
\usepackage{amsmath,amsfonts,amssymb} % for math symbols
\usepackage{array} % for tabular


\usepackage{parskip} % no indent, space between paragraphs

\usepackage{geometry} % margin
\geometry{
    a4paper,
    left=15mm,
    right=15mm,
    top=20mm,
    bottom=20mm
}

\usepackage{circledsteps} % to draw circles around numbers

\usepackage{fancyhdr} % for headers and footers

\usepackage{enumitem} % for customizing lists
\setlist[enumerate]{itemsep=1em} % space between items only in enumerate environment (not itemize)
\setlist[itemize]{label=--} % set itemize label to em-dash



\title{Sujets d'interrogation orale - Géométrie : Droites et Plans - Inégalité de Cauchy-Schwarz}
\author{}
\date{2024}

\begin{document}
\maketitle

But : Démontrer géométriquement l'inégalité de Cauchy-Schwarz, une propriété fondamentale du produit scalaire. 1-5. (Garder les questions 1 à 5 du problème original)

% QUESTION 1
% ----------
% Compétence: Manipuler les droites et les plans dans R^2
1. Soit un plan muni d'un repère orthonormé (O, i, j). Considérer deux vecteurs u(a,\ql et v(c, d).

% QUESTION 2
% ----------
% Compétences: Connaître les différentes expressions du produit scalaire
2. Exprimer le produit scalaire u · v en fonction de a, b, c et d.

% QUESTION 3
% ----------
% Compétences: Bilinéarité du produit scalaire
3. Soit t un réel. Considérer le vecteur w(t) = u - tv. Calculer le produit scalaire w(t) · w(t).

% QUESTION 4
% ----------
% Compétences: Bilinéarité du produit scalaire
4. Montrer que w(t) · w(t) est toujours positif ou nul, quel que soit t.

% QUESTION 5
% ----------
% Outil manquant : Notion de forme quadratique
5. En déduire que le discriminant du polynôme w(t) · w(t) est négatif ou nul.

% QUESTION 6
% ----------
% Compétences: Bilinéarité du produit scalaire
6. Développer cette inégalité et montrer qu'elle équivaut à (u · v)^2 ≤ (u · u)(v · v).

% QUESTION 7
% ----------
% Compétences: Définition du projeté orthogonal d'un point sur une droite
7. Interpréter géométriquement cette inégalité en termes de projection orthogonale.

% QUESTION 8
% ----------
% Compétences: Caractérisation de la distance d'un point à une droite comme la plus petite distance
% séparant ce point d'un point de la droite
8. Déterminer les conditions d'égalité dans l'inégalité de Cauchy-Schwarz.

% QUESTION 9
% ----------
% Compétences: Connaître les différentes expressions du produit scalaire
9. Application : Soit un triangle ABC. Démontrer l'inégalité du cosinus : a^2 + b^2 - 2ab cos(C) ≥ 0.

% QUESTION 10
% -----------
% Compétences: Connaître les différentes expressions du produit scalaire
10. Soit θ l'angle entre u et v. Exprimer cos θ en fonction de u·v, ||u|| et ||v||.

% QUESTION 11
% -----------
% Outil manquant : Manipulation algébrique avancée
11. Considérer l'expression (||u|| ||v|| - u·v)(||u|| ||v|| + u·v). Développer cette expression et montrer qu'elle est toujours positive ou nulle.

% QUESTION 12
% -----------
% Compétences: ?
12. En déduire que (u·v)^2 ≤ ||u||^2 ||v||^2. C'est l'inégalité de Cauchy-Schwarz.

% QUESTION 13
% -----------
% Compétences: Définition du projeté orthogonal d'un point sur une droite
13. Interpréter géométriquement cette inégalité en termes de projection orthogonale.

% QUESTION 14
% -----------
% Compétences: Caractérisation de la distance d'un point à une droite comme la plus petite distance
% séparant ce point d'un point de la droite
14. Déterminer les conditions d'égalité dans l'inégalité de Cauchy-Schwarz et les interpréter géométriquement.

\end{document}
