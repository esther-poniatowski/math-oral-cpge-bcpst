% PROBLÈME : Angles et intersections entre droites et plans dans R^3
% ==================================================================================================
%
% But
% ---
% Démontrer une condition analytique reliant l'angle entre une droite et un plan dans R^3 à leurs
% équations et explorer les conséquences géométriques et analytiques de ce résultat.
%
% Objectifs spécifiques
% ---------------------
% -
% ==================================================================================================

\documentclass[10pt,a4paper]{article}

% Set the root path
\providecommand{\rootpath}{../..}
% Fonts
\usepackage[utf8]{inputenc} % for accents
\usepackage[T1]{fontenc} % for accents
\usepackage[french]{babel} % for french language
\usepackage{helvet} % sans serif font family
\renewcommand*\familydefault{\sfdefault} % sans serif font family

% Mathematics
\usepackage{amsmath,amsfonts,amssymb} % for math symbols
\usepackage{array} % for tabular


\usepackage{parskip} % no indent, space between paragraphs

\usepackage{geometry} % margin
\geometry{
    a4paper,
    left=15mm,
    right=15mm,
    top=20mm,
    bottom=20mm
}

\usepackage{circledsteps} % to draw circles around numbers

\usepackage{fancyhdr} % for headers and footers

\usepackage{enumitem} % for customizing lists
\setlist[enumerate]{itemsep=1em} % space between items only in enumerate environment (not itemize)
\setlist[itemize]{label=--} % set itemize label to em-dash

% Command: \customPageLayout{#1}{#2}{#3}
% --------------------------------------
% Description: Custom page layout with header and footer content.
% Arguments:
% #1: Header and footer content
% #2: Left header content
% #3: Right header content
% Example:
% \customPageLayout{Title}{Lycée Henri IV}{2024}
% Required Packages: fancyhdr
\newcommand{\customPageLayout}[3]{
    \pagestyle{fancy} % set page style to fancy, i.e. header and footer
    \fancyhf{#1} % set header and footer content
    \lhead{#2} % set left header content
    \rhead{#3} % set right header content
    \fancyfoot{} % clear footer content
    \rfoot{\thepage} % set page number in footer
}

% Counter: \q
% -----------
% Description: Display a question number in a circle.
\newcounter{q}
\setcounter{q}{0} % set initial value of counter
\newcommand{\q}{
    \bigskip
    \addtocounter{q}{1}
    \par
    \Circled{\textbf{\theq}} \space
}



\title{Géométrie - Droites et Plans - Angles et intersections dans $\mathbb{R}^3$}
\author{}
\date{2024}

\customPageLayout{Sujets d'interrogation orale}{Lycée Henri IV}{2024}

\begin{document}
\maketitle

Dans l'espace $\mathbb{R}^3$, considérons une droite $D$ définie par :
$$
x = x_0 + \lambda u_x, \quad y = y_0 + \lambda u_y, \quad z = z_0 + \lambda u_z
$$
et un plan $P$ d'équation :
$$
Ax + By + Cz + D = 0.
$$
On cherche à exprimer l'angle $\theta$ entre la droite et le plan en fonction de leurs paramètres.

% QUESTION 1
% ----------
% Compétences:
1. Intersection entre une droite et un plan
  \ql Expliquer la méthode permettant de déterminer si $D$ et $P$ se coupent.
  \ql Trouver analytiquement l'éventuel point d'intersection en résolvant un système d'équations.

% QUESTION 2
% ----------
% Compétences:
2. Expression de l'angle entre une droite et un plan
  \ql Justifier que le vecteur $\mathbf{n} = (A,B,C)$ est un vecteur normal au plan.
  \ql Définir l'angle $\theta$ entre la droite et le plan en utilisant le produit scalaire entre $\mathbf{n}$ et le vecteur directeur $\mathbf{u} = (u_x, u_y, u_z)$.
   c) Démontrer la formule :
   $$
   \cos \theta = \frac{|A u_x + B u_y + C u_z|}{\sqrt{A^2 + B^2 + C^2} \cdot \sqrt{u_x^2 + u_y^2 + u_z^2}}.
   $$

% QUESTION 3
% ----------
% Compétences:
3. Cas particuliers et conséquences
  \ql Vérifier l'orthogonalité pour $P : x - 2y + 3z = 5$ et $D : (1+2\lambda, -2+\lambda, 3-\lambda)$.
  \ql Interpréter le cas où $\cos \theta = 0$ en termes d'angles et de positions relatives.

% QUESTION 4
% ----------
% Compétences:
4. Généralisation et applications
  \ql Adapter cette étude au cas de deux plans en $\mathbb{R}^3$ et établir la condition d'orthogonalité entre eux.
  \ql Discuter des applications en physique et ingénierie (réflexion des ondes sur un plan, trajectoire des particules, etc.).

\end{document}
