% PROBLÈME : Étude d'une transformation géométrique
% ==================================================================================================
%
% But
% ---
% Analyser en profondeur une transformation géométrique spécifique du plan.
%
% Objectifs spécifiques
% ---------------------
% - Démontrer la bilinéarité d'une application linéaire.
% - Prouver que la transformation conserve le produit scalaire, les distances et les angles.
% - Étudier l'effet de la transformation sur des objets géométriques (droite, cercle).
% - Caractériser complètement la transformation comme une isométrie du plan.
% ==================================================================================================

\documentclass[10pt,a4paper]{article}

% Set the root path
\providecommand{\rootpath}{../..}
% Fonts
\usepackage[utf8]{inputenc} % for accents
\usepackage[T1]{fontenc} % for accents
\usepackage[french]{babel} % for french language
\usepackage{helvet} % sans serif font family
\renewcommand*\familydefault{\sfdefault} % sans serif font family

% Mathematics
\usepackage{amsmath,amsfonts,amssymb} % for math symbols
\usepackage{array} % for tabular


\usepackage{parskip} % no indent, space between paragraphs

\usepackage{geometry} % margin
\geometry{
    a4paper,
    left=15mm,
    right=15mm,
    top=20mm,
    bottom=20mm
}

\usepackage{circledsteps} % to draw circles around numbers

\usepackage{fancyhdr} % for headers and footers

\usepackage{enumitem} % for customizing lists
\setlist[enumerate]{itemsep=1em} % space between items only in enumerate environment (not itemize)
\setlist[itemize]{label=--} % set itemize label to em-dash

% Command: \customPageLayout{#1}{#2}{#3}
% --------------------------------------
% Description: Custom page layout with header and footer content.
% Arguments:
% #1: Header and footer content
% #2: Left header content
% #3: Right header content
% Example:
% \customPageLayout{Title}{Lycée Henri IV}{2024}
% Required Packages: fancyhdr
\newcommand{\customPageLayout}[3]{
    \pagestyle{fancy} % set page style to fancy, i.e. header and footer
    \fancyhf{#1} % set header and footer content
    \lhead{#2} % set left header content
    \rhead{#3} % set right header content
    \fancyfoot{} % clear footer content
    \rfoot{\thepage} % set page number in footer
}

% Counter: \q
% -----------
% Description: Display a question number in a circle.
\newcounter{q}
\setcounter{q}{0} % set initial value of counter
\newcommand{\q}{
    \bigskip
    \addtocounter{q}{1}
    \par
    \Circled{\textbf{\theq}} \space
}



\title{Géométrie - Droites et Plans - Étude d'une transformation géométrique}
\author{Esther Poniatowski}
\date{2024-2025}

\customPageLayout{Sujets d'interrogation orale}{Lycée Henri IV}{2024}

\begin{document}
\maketitle

Ce problème étudie une transformation géométrique dans le plan, en utilisant les outils de la géométrie analytique et du produit scalaire.

% QUESTION 1
% ----------
% Outil manquant : Notion de transformation géométrique
1. Soit f la transformation du plan qui à tout point M(x, y) associe le point M'(x', y') tel que :
   x' = 2x - y
   y' = x + y

% QUESTION 2
% ----------
% Outil manquant : Notion d'application linéaire et de matrice associée
2. Montrer que f est une application linéaire. Déterminer sa matrice dans la base canonique.

% QUESTION 3
% ----------
% Compétences: Bilinéarité du produit scalaire
3. Soit u(1, 1) et v(-1, 2) deux vecteurs du plan.
  \ql Calculer f(u) et f(v).
  \ql Démontrer la bilinéarité de f en utilisant ces vecteurs.

% QUESTION 4
% ----------
% Compétences: Connaître les différentes expressions du produit scalaire
4. Montrer que f conserve le produit scalaire, c'est-à-dire que pour tous vecteurs a et b, on a :
   f(a) · f(b) = a · b

% QUESTION 5
% ----------
% Outil manquant : Notion de conservation des distances et des angles
5. En déduire que f conserve les distances et les angles.

% QUESTION 6
% ----------
% Compétences: Manipuler les droites et les plans dans R^2; Caractériser l'orthogonalité entre
% droites et plans
6. Soit D la droite d'équation y = 2x + 1.
  \ql Déterminer l'équation de la droite D' image de D par f.
  \ql Montrer que D et D' sont perpendiculaires.

% QUESTION 7
% ----------
% Compétences: Outil manquant : Équation d'une ellipse
7. Soit C le cercle de centre O(0, 0) et de rayon 1.
  \ql Déterminer l'ensemble des points M'(x', y') images des points M(x, y) de C par f.
  \ql Montrer que cet ensemble est une ellipse. Déterminer ses axes et son excentricité.

% QUESTION 8
% ----------
% Compétences: Outil manquant : Notion d'isométrie
8. Démontrer que f est une isométrie du plan. Caractériser complètement cette transformation géométrique.

\end{document}
