% PROBLÈME : Optimization sous contrainte affine
% ==================================================================================================
%
% But
% ---
% Distance minimale entre un point mobile et une famille de droites parallèles (Optimisation sous
% contrainte affine)
% Objectif final : Démontrer analytiquement que la distance entre un point donné et une droite
% contrainte est minimisée par un projeté spécifique et justifier ce résultat dans un cadre plus
% général d'optimisation sous contrainte.
%
% Objectifs spécifiques
% ---------------------
% -
% ==================================================================================================

\documentclass[10pt,a4paper]{article}

% Set the root path
\providecommand{\rootpath}{../..}
% Fonts
\usepackage[utf8]{inputenc} % for accents
\usepackage[T1]{fontenc} % for accents
\usepackage[french]{babel} % for french language
\usepackage{helvet} % sans serif font family
\renewcommand*\familydefault{\sfdefault} % sans serif font family

% Mathematics
\usepackage{amsmath,amsfonts,amssymb} % for math symbols
\usepackage{array} % for tabular


\usepackage{parskip} % no indent, space between paragraphs

\usepackage{geometry} % margin
\geometry{
    a4paper,
    left=15mm,
    right=15mm,
    top=20mm,
    bottom=20mm
}

\usepackage{circledsteps} % to draw circles around numbers

\usepackage{fancyhdr} % for headers and footers

\usepackage{enumitem} % for customizing lists
\setlist[enumerate]{itemsep=1em} % space between items only in enumerate environment (not itemize)
\setlist[itemize]{label=--} % set itemize label to em-dash

% Command: \customPageLayout{#1}{#2}{#3}
% --------------------------------------
% Description: Custom page layout with header and footer content.
% Arguments:
% #1: Header and footer content
% #2: Left header content
% #3: Right header content
% Example:
% \customPageLayout{Title}{Lycée Henri IV}{2024}
% Required Packages: fancyhdr
\newcommand{\customPageLayout}[3]{
    \pagestyle{fancy} % set page style to fancy (add header and footer)
    \fancyhf{} % clear all header and footer content
    \lhead{#2} % left header content
    \rhead{#3} % right header content
    \chead{\textbf{#1}} % center header content in bold (if needed)
    \rfoot{\thepage} % page number in the footer
}


% Counter: \q
% -----------
% Description: Display a question number in a circle.
% Usage:
% - Create a new question: add \q followed by the question content.
% - Reset the question counter: add \setcounter{q}{0} before the first question.
\newcounter{q}
\setcounter{q}{0} % set initial value of the counter
\newcommand{\q}{
    \bigskip
    \addtocounter{q}{1}
    \par
    \Circled{\textbf{\theq}} \space
}


% Counter: \ql
% ------------
% Description: Display a question letter in a round box with indentation (lowercase and not bold).
% Usage:
% - Create a new question: add \ql followed by the question content.
% - Reset the question counter: add \setcounter{ql}{0} before the first question.
\newcounter{ql}
\setcounter{ql}{0} % set initial value of the counter
\newcommand{\ql}{
    \addtocounter{ql}{1}
    \par
    \hspace{1.5em} % indentation before the circled letter
    \textcolor{gray}{\Circled{\alph{ql}}} \space % gray color
}



\title{Géométrie - Droites et Plans - Optimisation sous contrainte affine}
\author{Esther Poniatowski}
\date{2024-2025}

\customPageLayout{Sujets d'interrogation orale}{Lycée Henri IV}{2024}

\begin{document}
\maketitle

On considère un point $M(x_0, y_0)$ dans $\mathbb{R}^2$ et une famille de droites définie par :
$$
D_k : ax + by + k = 0, \quad k \in \mathbb{R}.
$$
Cette famille représente une translation continue de la droite $D_0 : ax + by = 0$.

On cherche à déterminer analytiquement le réel $k$ minimisant la distance entre $M$ et $D_k$.


% QUESTION 1
% ----------
% Compétences:
1. Projeté orthogonal sur une droite
  \ql Justifier pourquoi le projeté orthogonal $P$ de $M$ sur une droite donnée est le point minimisant la distance entre $M$ et la droite.
  \ql Exprimer analytiquement les coordonnées de $P$ sur $D_k$ en fonction de $a, b, x_0, y_0, k$.

% QUESTION 2
% ----------
% Compétences:
2. Optimisation sous contrainte affine
  \ql Montrer que la distance entre $M$ et $D_k$ s'écrit sous la forme :
   $$
   d(k) = \frac{|ax_0 + by_0 + k|}{\sqrt{a^2 + b^2}}.
   $$
  \ql Étudier analytiquement le minimum de $d(k)$ en fonction de $k$ et en déduire la valeur optimale.

% QUESTION 3
% ----------
% Compétences:
3. Application numérique et interprétation
  \ql Soit $M(3,2)$ et la famille de droites $2x - 3y + k = 0$. Déterminer la droite réalisant la distance minimale et la distance elle-même.
  \ql Interpréter ce résultat dans un cadre plus général de géométrie affine et d'optimisation sous contrainte linéaire.

% QUESTION 4
% ----------
% Compétences:
4. Extension au cas $\mathbb{R}^3$
  \ql Adapter les résultats précédents pour minimiser la distance entre un point $M(x_0, y_0, z_0)$ et un plan parallèle dans $\mathbb{R}^3$.
  \ql Expliquer l'intérêt de cette méthode en physique (exemple : minimisation d'énergie potentielle).

\end{document}
